\documentclass[12pt, letterpaper, oneside]{book}
\usepackage[margin={0.6in, 0.75in}]{geometry}
\usepackage{microtype}
% \usepackage{kpfonts}
\usepackage{amsmath, amssymb, amsthm}
\usepackage{parskip}
\usepackage[many]{tcolorbox}
\usepackage{footnote}
\usepackage{cancel}
\usepackage{titlesec}
\usepackage{pgffor}
\usepackage[shortlabels, inline]{enumitem}
\usepackage{hyperref}
\usepackage{tikz-cd}

\usepackage[overload]{textcase}

\renewcommand{\chaptername}{Lecture}
\newtheorem{axiom}{Axiom}[chapter]
\newtheorem{theorem}{Theorem}[chapter]
\newtheorem{prop}{Proposition}[chapter]
\newtheorem{corollary}{Corollary}[theorem]
\newtheorem{lemma}{Lemma}[chapter]
\theoremstyle{definition}
\newtheorem{definition}{Definition}[chapter]
\newtheorem{exercise}{Exercise}[chapter]
\newtheorem{example}{Example}[definition]
\newtheorem*{remark}{Remark}

\tcbset{sharp corners, breakable, enhanced, parbox=false}
\newtcolorbox{mybox}[3][]
{
  colframe = #2!150,
  colback  = #2!5,
  coltitle = #2!0!white,  
  title    = {#3},
  #1,
}

\titleformat{\chapter}[display]
    {\normalfont\huge\bfseries}{\chaptertitlename\ \thechapter}{20pt}{\Huge}
\titlespacing*{\chapter}{0pt}{0pt}{40pt}

\newcommand{\R}{\mathbb{R}}
\newcommand{\N}{\mathbb{N}}
\newcommand{\Z}{\mathbb{Z}}
\newcommand{\C}{\mathbb{C}}
\newcommand{\Q}{\mathbb{Q}}
\newcommand{\F}{\mathbb{F}}
\newcommand{\sphere}{\mathbb{S}}
\newcommand{\ZF}{\mathsf{ZF}}
\newcommand{\ZFC}{\mathsf{ZFC}}
\newcommand{\AC}{\mathsf{AC}}

\newcommand{\T}{\mathcal{T}}
\newcommand{\B}{\mathcal{B}}

\DeclareMathOperator{\Vol}{Vol}
\DeclareMathOperator{\Int}{int}
\DeclareMathOperator{\area}{area}
\DeclareMathOperator{\curl}{curl}
\DeclareMathOperator{\maps}{maps}
\DeclareMathOperator{\id}{id}
\DeclareMathOperator{\con}{\#}

\title{MATH 8803: Nonlinear Dispersive Equations}
\author{Frank Qiang\\Instructor: Gong Chen}
\date{Georgia Institute of Technology\\Spring 2025}

\begin{document}
  \maketitle

  \begingroup
  \let\cleardoublepage\clearpage
  \tableofcontents
  \endgroup

  % \foreach \i in {00, 01, 02, 03, 04, ..., 50} {%
  %   \edef\FileName{lectures/lecture\i.tex}%     The % here are necessary to eliminate any
  %   \IfFileExists{\FileName}{%  spurious spaces that may get inserted
  %      \input{\FileName}%       at these points
  %   }
  % }
  \chapter{Jan.~6 --- Introduction to Dispersion}

\section{Introduction to Dispersion}
\begin{definition}
  An evolution equation is \emph{dispersive} if
  when no boundary conditions are imposed (e.g. on
  $\R^n$), its wave solutions spread out in space
  as they evolve in time.
\end{definition}

\begin{example}
  Two classic examples of dispersive equations are:
  \begin{itemize}
    \item The \emph{Schr\"odinger equation}: $i u_t + \Delta u = 0$.
    \item The \emph{Airy (linearized KdV) equation}:
      $u_t + u_{xxx} = 0$.
  \end{itemize}
\end{example}

\begin{remark}
  Consider the equation
  $u_t + p(\partial_x) u = 0$, where $p$ is a polynomial,
  and a plane-wave solution
  \[
    u(t, x) = e^{i(kx - \omega t)}
    = e^{ik(x - (\omega / k) t)}.
  \]
  Here $k$ is the \emph{wave number} or
  \emph{space frequency}, and $\omega$ is the
  \emph{(time) frequency}. Plugging the plane-wave
  solution into the equation, we obtain the
  relation $\omega(k) = -i p(ik)$, i.e.
  \[
    \frac{\omega(k)}{k} = \frac{1}{ik} p(ik).
  \]
  The above equation is known as the
  \emph{dispersive relation}. This gives the
  traveling speed of the plane-wave solution with
  wave number $k$,
  which is called the \emph{phase velocity}.
\end{remark}

\begin{example}
  The following are some examples of dispersive relations:
  \begin{itemize}
    \item For the \emph{linear advection equation}
      $u_t + c u_x = 0$ with $c \in \R$, one
      can compute that $\omega / k = c$.
    \item For the Schr\"odinger equation
      $i u_t + \frac{1}{2} \Delta u = 0$, we have
      $\omega / k = k / 2 \in \R$.
     
      In this case of the Schr\"odinger equation,
      plane waves with large wave number (large space
      frequency) travel faster than low-frequency waves.
  \end{itemize}
\end{example}

\begin{remark}
  In general, dispersion means that different
  frequency plane waves travel at different speeds.
\end{remark}

\begin{remark}
  Given initial data $u_0$,
  we can write using the Fourier transform that
  \[
    u_0 = \int \widehat{u}_0(k) e^{ikx} \, dk.
  \]
  Then we get the solution $u$ as
  \[
    u(t, x) = \int \widehat{u}_0(k) e^{i k (x - (\omega(k) / k) t)} \, dk.
  \]
\end{remark}

\begin{example}
  In the case of the linear advection equation, we
  obtain the solution as
  \[
    u(t, x) = \int \widehat{u}_0(k) e^{i k (x - ct)} \, dk
    = u_0(x - ct).
  \]
  For the Schr\"odinger equation, we instead have
  the solution
  \[
    u(t, x) = \int \widehat{u}_0(k) e^{i k (x - (k / 2) t)} \, dk.
  \]
  Since different $k$ travels at different speeds,
  the original profile quickly spreads out.
\end{example}

\begin{exercise}
  Calculate the dispersive relation $\omega / k$
  for the linearized KdV equation $u_t + u_{xxx} = 0$.
\end{exercise}

\begin{example}
  The \emph{KdV equation} is given by
  \[
    \partial_t u + \partial_{xxx} u + 6 u \partial_x u = 0.
  \]
  This equation is used to model shallow water
  surfaces, and is a nonlinear dispersive
  equation. Russell observed a great bump of water
  in a channel that traveled for a long time
  and kept its shape. This is due to the
  nonlinear effects in the KdV equation, and these
  effects are called \emph{solitons}.
\end{example}

\begin{definition}
  A \emph{soliton} is a self-reinforcing solitary
  wave (a wave packet or pulse) that maintains
  its shape while traveling at a constant speed.
\end{definition}

\section{Fourier Transform and the Free Schr\"odinger Equation}
Consider the following free Schr\"odinger equation:
\[
  \begin{cases}
    i \partial_t \psi + \frac{1}{2} \Delta \psi = 0, \\
    \psi|_{t = 0} = \psi_0.
  \end{cases}
\]
We will solve this equation using the \emph{Fourier transform}
\[
  \widehat{f}(\xi) = \int_{\R^d} e^{-ix \cdot \xi} f(x)\, dx.
\]
Note that one can recover $f$ from its Fourier
transform via the \emph{inversion formula}
\[
  f(x) = \frac{1}{(2\pi)^d} \int_{\R^d} e^{ix \cdot \xi} \widehat{f}(\xi)\, d\xi.
\]

\begin{exercise}
  Check that $(\partial_{x_j} f)^\wedge = i \xi_j \widehat{f}$.
\end{exercise}

Applying the Fourier transform to the free Schr\"odinger
equation, one has
\[
  i \partial_t \psi + \frac{1}{2} \Delta \psi = 0 \quad
  \xrightarrow{\text{F.T.}} \quad
  i \partial_t \widehat{\psi} - \frac{1}{2} |\xi|^2 \widehat{\psi} = 0
\]
and initial condition $\widehat{\psi}(0, \xi) = \widehat{\psi}_0(\xi)$.
So for fixed $\xi$, we have an ODE, so we
can solve the equation via
\[
  \widehat{\psi}(t, \xi) =
  e^{-i |\xi|^2 t / 2} \widehat{\psi}_0(\xi).
\]
Now by applying the inverse Fourier transform, we obtain
the solution
\[
  \psi(t, x) = \frac{1}{(2\pi)^d} \int e^{ix\xi} \widehat{\psi}(t, \xi)\, d\xi
  = \frac{1}{(2\pi)^d} \int e^{ix\xi} e^{-i |\xi|^2 t / 2} \widehat{\psi}_0(\xi)\, d\xi.
\]
Recalling Plancherel's theorem that
$\|f\|_{L^2} = C\|\widehat{f}\|_{L^2}$ (for a constant
$C$ independent of $f$), we obtain
\[
  \|\psi(t, x)\|_{L^2}
  = C\|\widehat{\psi}(t, \xi)\|_{L^2}
  = C\|\widehat{\psi}(0, \xi)\|_{L^2}
  = \|\psi(0, x)\|_{L^2}
  = \|\psi_0(x)\|_{L^2},
\]
where the second equality is noticing that
$e^{-i |\xi|^2 t / 2}$ is purely imaginary. This is
a rigorous justification that the linear Schr\"odinger
evolution preserves the $L^2$ norm of the solution.

\begin{exercise}
  Compute that
  \[
    \frac{d}{dt} \int_{\R^d} |\psi(t, x)|^2\, dx = 0.
  \]
  This is an alternative way to show that
  the $L^2$ norm of the solution is preserved.
\end{exercise}

\section{Sobolev Spaces}
\begin{definition}
  The \emph{Sobolev spaces} $H^{\gamma} = W^{\gamma, 2}$
  for $\gamma \in \R$
  are defined via the norm
  \[
    \|f\|_{H^{\gamma}}
    = \left(\int_{\R^d} (1 + |\xi|^2)^\gamma |\widehat{f}(\xi)|^2\, d\xi\right)^{1 / 2}.
  \]
  The \emph{homogeneous Sobolev spaces} $\dot{H}^\gamma$
  are defined by the norm
  \[
    \|f\|_{\dot{H}^\gamma}
    = \left(\int_{\R^d} |\xi|^{2\gamma} |\widehat{f}(\xi)|^2\, d\xi\right)^{1 / 2}.
  \]
\end{definition}

\begin{remark}
  If $\gamma \in \N$ and $d = 1$, then
  \[
    \|f\|_{H^{\gamma}}
    \sim \sum_{m = 0}^\gamma \| \partial_x^m f \|_{L^2}.
  \]
  In particular, this means that
  $f \in H^\gamma$ if and only if
  $\partial_x^m f \in L^2$ for all $m \le \gamma$.
\end{remark}

\begin{exercise}
  Check that if $f_\lambda(x) = f(\lambda x)$, then
  $\widehat{f_\lambda}(\xi) = \lambda^{-d} \widehat{f}(\xi / \lambda)$.
\end{exercise}

\begin{remark}
  In the Sobolev spaces, this means that (change
  variables $\eta = \xi / \lambda$ for the last equality)
  \[
    \|f_\lambda\|_{\dot{H}^{\gamma}}
    = \left(\int_{\R^d} |\xi|^{2\gamma} |\widehat{f_\lambda}(\xi)|^2\, d\xi\right)^{1 / 2}
    = \left(\int_{\R^d} |\xi|^{2\gamma} |\lambda^{-d} \widehat{f}(\xi / \lambda)|^2\, d\xi\right)^{1 / 2}
    = \lambda^{\gamma - d / 2} \|f\|_{\dot{H}^{\gamma}}.
  \]
\end{remark}

\begin{lemma}
  In the Schr\"odinger equation, $\|\psi(t)\|_{H^{\gamma}} = \|\psi_0\|_{H^{\gamma}}$
  and $\|\psi(t)\|_{\dot{H}^{\gamma}} = \|\psi_0\|_{\dot{H}^{\gamma}}$
  for all $t$ and $\gamma$.
\end{lemma}

\begin{proof}
  We can compute that
  \[
    \|\psi(t)\|_{\dot{H}^{\gamma}}
    = \int_{\R^d} |\xi|^{2\gamma} |\widehat{\psi}(t, \xi)|^2\, d\xi
    = \int_{\R^d} |\xi|^{2\gamma} |e^{-i |\xi|^2 / 2}\widehat{\psi_0}(\xi)|^2\, d\xi
    = \int_{\R^d} |\xi|^{2\gamma} |\widehat{\psi_0}(\xi)|^2\, d\xi
    = \|\psi_0\|_{\dot{H}^{\gamma}}.
  \]
  The same argument works for the $H^{\gamma}$ case
  after replacing $|\xi|^{2\gamma}$ with
  $(1 + |\xi|^2)^\gamma$.
\end{proof}

  \chapter{Jan.~8 --- Special Solutions}

\section{Special Solutions}
\begin{example}
  The following are special solutions to the
  Schr\"odinger equation:
  \begin{enumerate}
    \item Gaussian: $\psi_0 = e^{-|x|^2 / 2}$.
      One can compute the Fourier transform and get
      \[
        \widehat{\psi}_0(\xi)
        = \int_{\R^d} e^{-ix \cdot \xi} e^{-|x|^2 / 2} \, dx
        = \int_{\R^d} e^{-|x + i\xi|^2 / 2} e^{-|\xi|^2 / 2} \, dx
        = e^{-|\xi|^2 / 2}\int_{\R^d} e^{-|x + i\xi|^2 / 2} \, dx.
      \]
      The last integral is a contour integral in the
      complex plane along $\Im z = \xi$, and
      we can deform the contour via Cauchy's theorem
      to the real axis to obtain (the integrand is
      analytic on $0 \le \Im z \le \xi$)
      \[
        \widehat{\psi}_0(\xi)
        = e^{-|\xi|^2 / 2}\int_{\R^d} e^{-|x|^2 / 2} \, dx
        = (2\pi)^{d / 2} e^{-|\xi|^2 / 2}.
      \]
      Then taking inverse Fourier transforms, we obtain
      the solution
      \begin{align*}
        \psi(t, x)
        = (2\pi)^{-d} \int_{\R^d} e^{i(x \cdot \xi - |\xi|^2 t / 2)} \widehat{\psi}_0(\xi) \, d\xi
        &= (2\pi)^{-d / 2} \int_{\R^d} e^{i(x \cdot \xi - |\xi|^2 t / 2)} e^{-|\xi|^2 / 2} \, d\xi \\
        &= (2\pi)^{-d / 2} \int_{\R^d} e^{-\frac{1}{2}(1 + it)|\xi|^2} e^{ix \cdot \xi} \, d\xi.
      \end{align*}
      Now formally put $\eta = (1 + it)^{1 / 2} \xi$
      to get
      \[
        \psi(t, x) = (2\pi)^{-d / 2} (1 + it)^{-d / 2} \int_{\R^d} e^{-\frac{1}{2}|\eta|^2} e^{ix \eta / (1 + it)^{1 / 2}}\, d\eta.
      \]
      Fill in the details of the above change of
      variables as an exercise (e.g. one has to
      worry about choosing a branch cut when taking
      the square root). Computing the integral explicitly,
      one obtains
      \[
        \psi(t, x) = (1 + it)^{-d / 2} e^{-|x|^2 / (2(1 + it))}.
      \]
      One can from this that $\psi$ has decay in
      time. Furthermore, one can see that
      \[
        |\psi(t, x)|^2 = (1 + t^2)^{-d / 2} e^{-|x|^2 / (1 + t^2)}.
      \]
      From this we can observe an $L^\infty$ decay
      of $\psi$ like $t^{-d / 2}$, and that the influence
      region of the solution grows like order $t$.
      We can also see again from this explicit
      computation that $\|\psi(t)\|_{L^2} = C$.
    \item Modulated Gaussian: $\psi_0 = e^{-|x|^2 / 2} e^{ix \cdot v}$.
      The Fourier transform of this initial data is
      \[
        \widehat{\psi}_0(\xi)
        = (2\pi)^{d / 2} e^{-i|\xi - v|^2 / 2}.
      \]
      So the solution corresponding to this initial
      data is
      \begin{align*}
        \psi(t, x)
        = (2\pi)^{-d} \int_{\R^d} e^{i(x \cdot \xi - |\xi|^2 t / 2)} \widehat{\psi}_0(\xi) \, d\xi
        &= (2\pi)^{-d / 2} \int_{\R^d} e^{i(x \cdot \xi - |\xi|^2 t / 2)} e^{-|\xi - v|^2} \, d\xi \\
        &= e^{ix \cdot v} e^{-|v|^2 t / 2} (2\pi)^{-d / 2}
        \int_{\R^d} e^{i(x - vt) \cdot \xi} e^{-(1 + it) |\xi|^2 / 2}\, d\xi \\
        &= e^{ix \cdot v} e^{-|v|^2 t / 2}
        (1 + it)^{-d / 2}
        \exp\left(-\frac{|x - vt|^2}{2(1 + it)}\right).
      \end{align*}
      From this we can see that the influence region
      of the solution moves with velocity $v$.
    \item Fundamental solution: We want a
      \emph{fundamental solution} $K$ such that
      $K$ solves
      \[
        i \partial_t K + \frac{1}{2} \Delta K = 0
        \quad \text{and} \quad K|_{t = 0} = \delta_0.
      \]
      We will find $K$ by scaling arguments. Suppose
      such a $K$ exists. Then we must have
      \[
        \psi(t, x) = \int_{\R^d} K(t, x - y) \psi_0(y)\, dy \tag{1}
      \]
      since $K|_{t = 0} = \delta_0$.
      Now define the scaling
      $\psi_\lambda(t, x) = \psi(\lambda^2 t, \lambda x)$.
      Then $\psi_\lambda$ also solves
      \[
        i \partial_t \psi_\lambda + \frac{1}{2} \Delta \psi_\lambda = 0
      \]
      and we have the initial condition
      $\psi_\lambda(0, x) = \psi_0(\lambda x)$. Then
      \[
        \psi_\lambda(t, x)
        = \int_{\R^d} K(t, x - y)\psi_0(\lambda y)\, dy
        = \psi(\lambda^2 t, \lambda x).
      \]
      Setting $t' = \lambda^2 t$, $x' = \lambda x$, and
      $y' = \lambda y$, we get
      \[
        \psi(t', x')
        = \frac{1}{\lambda^d} \int_{\R^d} K\left(\frac{t'}{\lambda^2}, \frac{x' - y'}{\lambda}\right)
        \psi_0(y')\, dy'.
        \tag{2}
      \]
      Comparing (1) and (2), we see that we must have
      \[
        K(t, x - y) = \lambda^{-d} K\left(\frac{t}{\lambda^2}, \frac{x - y}{\lambda}\right).
      \]
      Setting $u = x - y$, we get
      \[
        K(t, u) = \lambda^{-d} K\left(\frac{t}{\lambda^2}, \frac{u}{\lambda}\right).
      \]
      Thus we expect
      $K(t, x) = t^{-d / 2} \Phi(|x|^2 / t)$
      for some $\Phi$. Now we use the fact that
      $i \partial_t K + \frac{1}{2} \Delta K = 0$.
      Setting $m = |x|^2 / t$, one can
      plug in the above guess for $K$ to obtain
      (note that $\Delta = \nabla \cdot \nabla$)
      \[
        -\frac{id}{2} t^{-d / 2 - 1} \Phi(m)
        - it^{-d / 2} \Phi'(m) \frac{m}{t}
        + \frac{1}{2} t^{-d / 2} \nabla \cdot \left(\frac{2x}{t} \Phi'(m)\right) = 0.
      \]
      Then we get
      \[
        -i \frac{d}{2} \Phi(m) - im \Phi'(m)
        + d \Phi'(m) + 2m \Phi''(m) = 0,
      \]
      which gives
      \[
        d\left(\Phi'(m) - \frac{i}{2} \Phi(m)\right)
        + 2m \frac{d}{dm} \left(\Phi'(m) - \frac{i}{2}\Phi(m)\right) = 0.
      \]
      Now observe that
      $\Phi(m) = e^{im / 2}$ solves the above equation.
      Since $\Phi(m)$ solves the
      equation, $c \Phi(m)$ also solves the equation for
      any $c \in \C$, and thus we have
      \[
        K(t, x) = c t^{-d / 2} \Phi(|x|^2 / t)
        = c t^{-d / 2} e^{i|x|^2 / 2t}.
      \]
      To determine $c$, we use $K|_{t = 0} = \delta_0$,
      from which one can obtain $c = (2\pi i)^{-d / 2}$. Thus
      \[
        K(t, x) = (2\pi it)^{-d / 2} e^{i|x|^2 / 2t}.
      \]
      The rough computation is that since
      $\widehat{K}(0, \xi) = 1$, we have
      \[
        K = (2\pi)^{-d} \int_{\R^d} e^{i(x \cdot \xi - |\xi|^2 t / 2)} \widehat{K}(0, \xi)\, d\xi
        = (2\pi)^{-d} \int_{\R^d} e^{i(x \cdot \xi - |\xi|^2 t / 2)}\, d\xi.
      \]
      This is not necessarily integrable a priori,
      but one can take limits and obtain
      \begin{align*}
        K
        = (2\pi)^{-d} \lim_{\epsilon \to 0^+}
        \int_{\R^d} e^{ix \cdot \xi} e^{-(\epsilon + it)|\xi|^2 / 2}\, d\xi
        &= \lim_{\epsilon \to 0^+} (\epsilon + it)^{-d / 2} (2\pi)^{-d / 2} e^{-|x|^2 / (2(\epsilon + it))} \\
        &= (2\pi it)^{-d / 2} e^{-|x|^2 / 2it}.
      \end{align*}
      Note that this computation matches the result
      of the previous scaling argument.
  \end{enumerate}
\end{example}

\begin{theorem}
  Let $\psi_0 \in \mathcal{S}(\R^d)$.\footnote{Here $\mathcal{S}(\R^d)$ is the space of \emph{Schwartz functions}.}
  Then there exists a solution to
  \[
    \begin{cases}
      i \partial_t \psi + \frac{1}{2} \Delta \psi = 0, \\
      \psi|_{t = 0} = \psi_0,
    \end{cases}
  \]
  which is unique and given by
  \[
    \psi(t, x) = \int_{\R^d} K(t, x - y) \psi_0(y)\, dy
    = (2\pi it)^{-d / 2} \int_{\R^d} e^{-|x - y|^2 / 2it} \psi_0(y)\, dy.
  \]
\end{theorem}

\begin{proof}
  This theorem is a summary of the results of
  the previous explicit computations.
\end{proof}

\begin{remark}
  Recall that the Schr\"odinger evolution
  preserves the $L^2$ norm of a solution, i.e.
  \[
    \|\psi(t)\|_{L^2} = \|\psi(0)\|_{L^2} = \|\psi_0\|_{L^2}.
  \]
  The above theorem also gives an $L^\infty$ bound
  (a so-called \emph{dispersive estimate})
  \[
    \|\psi(t)\|_{L^\infty}
    \le |2\pi t|^{-d / 2} \int_{\R^d} |\psi_0(y)|\, dy
    = |2\pi t|^{-d / 2} \|\psi_0\|_{L^1}.
  \]
\end{remark}

  \chapter{Jan.~15 --- Strichartz Estimates}

\section{Interpolation Results}
\begin{remark}[Interpolation]
  Consider a linear operator $T$ which maps
  $T : L^{p_1} \to L^{q_1}$ and
  $T : L^{p_2} \to L^{q_2}$, where $1 \le p_1 \le p_2 \le \infty$.
  Then $T$ also maps $T : L^p \to L^q$ for any $p, q$
  such that
  \[
    \frac{1}{p} = \frac{\theta}{p_1} + \frac{1 - \theta}{p_2}
    \quad \text{and} \quad
    \frac{1}{q} = \frac{\theta}{q_1} + \frac{1 - \theta}{q_2}
  \]
  for some $0 \le \theta \le 1$. More specifically, if
  $\|Tf\|_{L^{q_1}} \le C_1 \|f\|_{L^{p_1}}$ and
  $\|Tf\|_{L^{q_2}} \le C_2 \|f\|_{L^{p_2}}$, then
  \[
    \|Tf\|_{L^q} \le C_1^\theta C_2^{1 - \theta} \|f\|_{L^p}.
  \]
  This $L^p$ interpolation is a result from real and
  functional analysis. Note that by interpolation,
  we have
  \[
    \|\psi\|_{L^{p'}(\R^d)} \le C |t|^{-d(1 / p - 1 / 2)} \|\psi_0\|_{L^p(\R^d)}
  \]
  for $1 \le p \le 2$, where $p'$ is the
  \emph{H\"older conjugate} of $p$, i.e.
  $1 / p' + 1 / p = 1$.
\end{remark}

\section{Strichartz Estimates}

\begin{remark}
  We will now consider the inhomogeneous Schr\"odinger
  equation:
  \[
    \begin{cases}
      i \psi_t + \frac{1}{2}\Delta \psi = F, & F \in \mathcal{S}_{x, t} \\
      \psi(0) = \psi_0, & \psi_0 \in \mathcal{S},
    \end{cases}
  \]
  where $F \in \mathcal{S}_{x, t}$ means that $F$ is
  Schwartz in both $x$ and $t$. We can solve this
  via the \emph{Duhamel formula}:
  \[
    \psi(t) = e^{i t \Delta / 2} \psi_0 - i \int_0^t e^{i(t - s) \Delta / 2} F(s)\, ds,
  \]
  where $e^{it \Delta / 2}$ is the \emph{linear
  propagator} given by
  \[
    e^{it \Delta / 2} \psi_0
    = (e^{-it |\xi|^2 / 2} \widehat{\psi}_0)^\vee
    = \frac{1}{(2\pi)^d} \int_{\R^d} e^{ix \cdot \xi} e^{-it |\xi|^2 / 2} \widehat{\psi}_0(\xi)\, d\xi.
  \]
\end{remark}

\begin{theorem}[Strichartz estimates]\label{thm:strichartz}
  For $p' = 2 + 4 / d$, we have the estimate\footnote{Here $A \lesssim B$ means that $A \le CB$ for some prescribed constant $C$.}
  \[
    \|\psi\|_{L^{p'}_{t, x}(\R \times \R^d)}
    \lesssim \|\psi_0\|_{L^2_x(\R^d)}
    + \|F\|_{L^{p}_{t, x}(\R \times \R^d)}.
  \]
\end{theorem}

\begin{remark}
  If $F = 0$, this is the bound
  \[
    \|\psi\|_{L^{p'}_{t, x}}
    \lesssim \|\psi_0\|_{L^2}
  \]
  for $p' > 2$. Formally, this means that we gain
  integrability in $x$. Note that this gain in
  integrability is not pointwise in time, i.e.
  we do \emph{not} have $\|\psi(t)\|_{L^\infty_t L^{p'}_x} \lesssim \|\psi_0\|_{L^2_x}$. We must instead average
  over $t$.
\end{remark}

\begin{remark}
  Why $p'$ and why do we also pick $p'$ in the time
  integration? Actually, $p'$ is the only possible
  choice for the above result. This follows by a
  scaling argument: Set
  \[
    \psi_\lambda(t, x) = \psi(\lambda^2 t, \lambda x),
    \quad
    (\psi_\lambda)_0(x) = \psi_0(\lambda x),
    \quad
    F_\lambda(t, x) = \lambda^2 F(\lambda^2 t, \lambda x).
  \]
  Then $\psi_\lambda$ solves the equation
  \[
    \begin{cases}
    i \partial_t \psi_\lambda + \frac{1}{2} \Delta \psi_\lambda = F_\lambda, \\
    \psi_\lambda(0) = (\psi_{\lambda})_0.
    \end{cases}
  \]
  If the above theorem makes sense, then it must hold
  for both $\psi_\lambda$ and $\psi$. Now
  \[
    \|\psi_\lambda\|_{L^{p'}_{t, x}}
    = \lambda^{-d / p'} \lambda^{-2 / p'} \|\psi\|_{L^{p'}_{t, x}}
  \]
  by a change of variables, and
  \[
    \|(\psi_\lambda)_0\|_{L^2_x}
    = \lambda^{-d / 2} \|\psi_0\|_{L^2_x}.
  \]
  Now if $F = 0$, then we have the estimates
  \[
    \|\psi\|_{L^{p'}_{t, x}} \lesssim \|\psi_0\|_{L^2_x}
    \quad \text{and} \quad
    \|\psi_\lambda\|_{L^{p'}_{t, x}}
    \lesssim \|(\psi_\lambda)_0\|_{L^2_x}, \tag{$*$}
  \]
  Using the scaling computations in the second estimate
  in $(*)$ implies that
  \[
    \| \psi \|_{L^{p'}_{t, x}} \lambda^{-d / p'} \lambda^{-2 / p'}
    \lesssim \lambda^{-d / 2} \|\psi_0\|_{L^2_x}.
  \]
  This inequality should hold independent of $\lambda$,
  since otherwise taking $\lambda \to \infty$ or
  $\lambda \to 0$ yields a contradiction with the
  first inequality in $(*)$.
  Thus the powers in $\lambda$ should match:
  \[
    -\frac{d}{p'} - \frac{2}{p'} = -\frac{d}{2},
  \]
  so we find that $p'$ must be
  \[
    p' = \frac{d + 2}{d / 2} + \frac{2d + 4}{d} = 2 + \frac{4}{d}.
  \]
  This uniquely determines $p'$. Now consider $F \ne 0$.
  Using a similar computation as before, we have
  \[
    \|F_\lambda\|_{L^q_{t, x}} = \lambda^2 \lambda^{-d / q} \lambda^{-2 / q} \|F\|_{L^q_{t, x}}.
  \]
  Then the theorem says that
  $\|\psi_\lambda\|_{L^{p'}_{t, x}} \lesssim \|\psi_0\|_{L^2_x} + \|F\|_{L^q_{t, x}}$,
  so we have
  \[
    \|\psi\|_{L^{p'}_{t, x}} \lambda^{-d / p'} \lambda^{-2 / p'}
    \lesssim \lambda^{-d / 2} \| \psi_0 \|_{L^2_x} + \lambda^2 \lambda^{-d / q} \lambda^{-2 / q} \|F\|_{L^q_{t, x}}.
  \]
  Again the estimate should hold independent of
  $\lambda$, so the powers in $\lambda$ must match:
  \[
    -\frac{d}{p'} - \frac{2}{p'} = 2 - \frac{d}{q} - \frac{2}{q} = -\frac{d}{2},
  \]
  which then gives $p$ as
  \[
    p = \left(1 - \frac{1}{p'}\right)^{-1} = \left(1 - \frac{d}{2d + 4}\right)^{-1} = \frac{2d + 4}{d + 4}.
  \]
\end{remark}

\begin{lemma}
  Let $\psi(t) = e^{it \Delta / 2} \psi_0$.
  Then for $1 \le p \le 2$,
  \[
    \|\psi(t) \|_{L^{p'}_x(\R^d)}
    \lesssim |t|^{-d (1 / p - 1 / 2)} \|\psi_0\|_{L^p_x(\R^d)}.
  \]
\end{lemma}

\begin{proof}
  This is the interpolation result from the beginning
  of class.
\end{proof}

\begin{lemma}[Hardy-Littlewood-Sobolev inequality]
  Let $0 < \alpha < 1$ and $g \in \mathcal{S}(\R)$.
  Let
  \[
    (T_\alpha g)(t) = \int_{-\infty}^\infty |t - s|^{-\alpha} g(s)\, ds.
  \]
  Then we have
  $\|T_\alpha g\|_{L^q(\R)} \lesssim \|g\|_{L^p(\R)}$,
  where $1 < p < q < \infty$ such that
  $1 + 1 / q = \alpha + 1 / p$.
\end{lemma}

\begin{proof}
  One approach is via harmonic analysis and maximal
  functions. An alternative approach can be found in
  Theorem 4.3 of Analysis by Lieb and Loss.
\end{proof}

\begin{remark}
  Recall \emph{Young's inequality} that for
  \[
    h(t) = \int f(t - s) g(s)\, dx,
  \]
  we have $\|h\|_{L^r} \le \|f\|_{L^p} \|g\|_{L^q}$,
  where $1 / r + 1 = 1 / q + 1 / p$. The
  Hardy-Littlewood-Sobolev inequality can be
  seen as a generalized Young's inequality: If
  $f(s) = |s|^{-\alpha}$, then $f$ barely fails to be
  in $L^{1 / \alpha}$. Informally, we can think of
  ``$f \in L^{1 / \alpha}$,'' and the standard
  Young's inequality would imply
  Hardy-Littlewood-Sobolev.
\end{remark}

\begin{remark}
  We have $q > p$ in the Hardy-Littlewood-Sobolev
  inequality, so we gain some integrability via
  fractional integration for $p > 1$
  (the type of integral defining
  $T_\alpha g$ is known as \emph{fractional integration}).
\end{remark}

\begin{proof}[Proof of Theorem \ref{thm:strichartz}]
  This proof is left for next class.
\end{proof}

\end{document}
