\documentclass[12pt, letterpaper, oneside]{book}
\usepackage[margin={0.6in, 0.75in}]{geometry}
\usepackage{microtype}
% \usepackage{kpfonts}
\usepackage{amsmath, amssymb, amsthm}
\usepackage{parskip}
\usepackage[many]{tcolorbox}
\usepackage{footnote}
\usepackage{cancel}
\usepackage{titlesec}
\usepackage{pgffor}
\usepackage{mathtools}
\usepackage[shortlabels, inline]{enumitem}
\usepackage{hyperref}
\usepackage{tikz-cd}
\usepackage{bbm}

\usepackage[overload]{textcase}

\renewcommand{\chaptername}{Lecture}
\newtheorem{axiom}{Axiom}[chapter]
\newtheorem{theorem}{Theorem}[chapter]
\newtheorem{prop}{Proposition}[chapter]
\newtheorem{corollary}{Corollary}[theorem]
\newtheorem{lemma}{Lemma}[chapter]
\theoremstyle{definition}
\newtheorem{definition}{Definition}[chapter]
\newtheorem{exercise}{Exercise}[chapter]
\newtheorem{example}{Example}[definition]
\newtheorem*{remark}{Remark}

\tcbset{sharp corners, breakable, enhanced, parbox=false}
\newtcolorbox{mybox}[3][]
{
  colframe = #2!150,
  colback  = #2!5,
  coltitle = #2!0!white,  
  title    = {#3},
  #1,
}

\titleformat{\chapter}[display]
    {\normalfont\huge\bfseries}{\chaptertitlename\ \thechapter}{20pt}{\Huge}
\titlespacing*{\chapter}{0pt}{0pt}{40pt}

\newcommand{\R}{\mathbb{R}}
\newcommand{\N}{\mathbb{N}}
\newcommand{\Z}{\mathbb{Z}}
\newcommand{\C}{\mathbb{C}}
\newcommand{\Q}{\mathbb{Q}}
\newcommand{\F}{\mathbb{F}}
\newcommand{\sphere}{\mathbb{S}}
\newcommand{\ZF}{\mathsf{ZF}}
\newcommand{\ZFC}{\mathsf{ZFC}}
\newcommand{\AC}{\mathsf{AC}}

\newcommand{\T}{\mathcal{T}}
\newcommand{\B}{\mathcal{B}}

\newcommand{\HH}{\mathcal{H}}
\newcommand{\Hac}{\mathcal{H}_{\mathrm{ac}}}
\newcommand{\Hsc}{\mathcal{H}_{\mathrm{sc}}}
\newcommand{\Hpp}{\mathcal{H}_{\mathrm{pp}}}

\DeclareMathOperator{\Vol}{Vol}
\DeclareMathOperator{\Int}{int}
\DeclareMathOperator{\area}{area}
\DeclareMathOperator{\curl}{curl}
\DeclareMathOperator{\maps}{maps}
\DeclareMathOperator{\supp}{supp}
\DeclareMathOperator{\re}{Re}
\DeclareMathOperator{\im}{Im}
\DeclareMathOperator{\id}{id}
\DeclareMathOperator{\Ran}{Ran}
\DeclareMathOperator{\Spec}{Spec}
\DeclareMathOperator{\sign}{sign}

\newcommand{\seminorm}[1]{\left\lvert\hspace{-1 pt}\left\lvert\hspace{-1 pt}\left\lvert {#1}\right\lvert\hspace{-1 pt}\right\lvert\hspace{-1 pt}\right\lvert}

\title{MATH 8803: Nonlinear Dispersive Equations}
\author{Frank Qiang\\Instructor: Gong Chen}
\date{Georgia Institute of Technology\\Spring 2025}

\begin{document}
  \maketitle

  \begingroup
  \let\cleardoublepage\clearpage
  \tableofcontents
  \endgroup

  % \foreach \i in {00, 01, 02, 03, 04, ..., 50} {%
  %   \edef\FileName{lectures/lecture\i.tex}%     The % here are necessary to eliminate any
  %   \IfFileExists{\FileName}{%  spurious spaces that may get inserted
  %      \input{\FileName}%       at these points
  %   }
  % }
  \chapter{Jan.~6 --- Introduction to Dispersion}

\section{Introduction to Dispersion}
\begin{definition}
  An evolution equation is \emph{dispersive} if
  when no boundary conditions are imposed (e.g. on
  $\R^n$), its wave solutions spread out in space
  as they evolve in time.
\end{definition}

\begin{example}
  Two classic examples of dispersive equations are:
  \begin{itemize}
    \item The \emph{Schr\"odinger equation}: $i u_t + \Delta u = 0$.
    \item The \emph{Airy (linearized KdV) equation}:
      $u_t + u_{xxx} = 0$.
  \end{itemize}
\end{example}

\begin{remark}
  Consider the equation
  $u_t + p(\partial_x) u = 0$, where $p$ is a polynomial,
  and a plane-wave solution
  \[
    u(t, x) = e^{i(kx - \omega t)}
    = e^{ik(x - (\omega / k) t)}.
  \]
  Here $k$ is the \emph{wave number} or
  \emph{space frequency}, and $\omega$ is the
  \emph{(time) frequency}. Plugging the plane-wave
  solution into the equation, we obtain the
  relation $\omega(k) = -i p(ik)$, i.e.
  \[
    \frac{\omega(k)}{k} = \frac{1}{ik} p(ik).
  \]
  The above equation is known as the
  \emph{dispersive relation}. This gives the
  traveling speed of the plane-wave solution with
  wave number $k$,
  which is called the \emph{phase velocity}.
\end{remark}

\begin{example}
  The following are some examples of dispersive relations:
  \begin{itemize}
    \item For the \emph{linear advection equation}
      $u_t + c u_x = 0$ with $c \in \R$, one
      can compute that $\omega / k = c$.
    \item For the Schr\"odinger equation
      $i u_t + \frac{1}{2} \Delta u = 0$, we have
      $\omega / k = k / 2 \in \R$.
     
      In this case of the Schr\"odinger equation,
      plane waves with large wave number (large space
      frequency) travel faster than low-frequency waves.
  \end{itemize}
\end{example}

\begin{remark}
  In general, dispersion means that different
  frequency plane waves travel at different speeds.
\end{remark}

\begin{remark}
  Given initial data $u_0$,
  we can write using the Fourier transform that
  \[
    u_0 = \int \widehat{u}_0(k) e^{ikx} \, dk.
  \]
  Then we get the solution $u$ as
  \[
    u(t, x) = \int \widehat{u}_0(k) e^{i k (x - (\omega(k) / k) t)} \, dk.
  \]
\end{remark}

\begin{example}
  In the case of the linear advection equation, we
  obtain the solution as
  \[
    u(t, x) = \int \widehat{u}_0(k) e^{i k (x - ct)} \, dk
    = u_0(x - ct).
  \]
  For the Schr\"odinger equation, we instead have
  the solution
  \[
    u(t, x) = \int \widehat{u}_0(k) e^{i k (x - (k / 2) t)} \, dk.
  \]
  Since different $k$ travels at different speeds,
  the original profile quickly spreads out.
\end{example}

\begin{exercise}
  Calculate the dispersive relation $\omega / k$
  for the linearized KdV equation $u_t + u_{xxx} = 0$.
\end{exercise}

\begin{example}
  The \emph{KdV equation} is given by
  \[
    \partial_t u + \partial_{xxx} u + 6 u \partial_x u = 0.
  \]
  This equation is used to model shallow water
  surfaces, and is a nonlinear dispersive
  equation. Russell observed a great bump of water
  in a channel that traveled for a long time
  and kept its shape. This is due to the
  nonlinear effects in the KdV equation, and these
  effects are called \emph{solitons}.
\end{example}

\begin{definition}
  A \emph{soliton} is a self-reinforcing solitary
  wave (a wave packet or pulse) that maintains
  its shape while traveling at a constant speed.
\end{definition}

\section{Fourier Transform and the Free Schr\"odinger Equation}
Consider the following free Schr\"odinger equation:
\[
  \begin{cases}
    i \partial_t \psi + \frac{1}{2} \Delta \psi = 0, \\
    \psi|_{t = 0} = \psi_0.
  \end{cases}
\]
We will solve this equation using the \emph{Fourier transform}
\[
  \widehat{f}(\xi) = \int_{\R^d} e^{-ix \cdot \xi} f(x)\, dx.
\]
Note that one can recover $f$ from its Fourier
transform via the \emph{inversion formula}
\[
  f(x) = \frac{1}{(2\pi)^d} \int_{\R^d} e^{ix \cdot \xi} \widehat{f}(\xi)\, d\xi.
\]

\begin{exercise}
  Check that $(\partial_{x_j} f)^\wedge = i \xi_j \widehat{f}$.
\end{exercise}

Applying the Fourier transform to the free Schr\"odinger
equation, one has
\[
  i \partial_t \psi + \frac{1}{2} \Delta \psi = 0 \quad
  \xrightarrow{\text{F.T.}} \quad
  i \partial_t \widehat{\psi} - \frac{1}{2} |\xi|^2 \widehat{\psi} = 0
\]
and initial condition $\widehat{\psi}(0, \xi) = \widehat{\psi}_0(\xi)$.
So for fixed $\xi$, we have an ODE, so we
can solve the equation via
\[
  \widehat{\psi}(t, \xi) =
  e^{-i |\xi|^2 t / 2} \widehat{\psi}_0(\xi).
\]
Now by applying the inverse Fourier transform, we obtain
the solution
\[
  \psi(t, x) = \frac{1}{(2\pi)^d} \int e^{ix\xi} \widehat{\psi}(t, \xi)\, d\xi
  = \frac{1}{(2\pi)^d} \int e^{ix\xi} e^{-i |\xi|^2 t / 2} \widehat{\psi}_0(\xi)\, d\xi.
\]
Recalling Plancherel's theorem that
$\|f\|_{L^2} = C\|\widehat{f}\|_{L^2}$ (for a constant
$C$ independent of $f$), we obtain
\[
  \|\psi(t, x)\|_{L^2}
  = C\|\widehat{\psi}(t, \xi)\|_{L^2}
  = C\|\widehat{\psi}(0, \xi)\|_{L^2}
  = \|\psi(0, x)\|_{L^2}
  = \|\psi_0(x)\|_{L^2},
\]
where the second equality is noticing that
$e^{-i |\xi|^2 t / 2}$ is purely imaginary. This is
a rigorous justification that the linear Schr\"odinger
evolution preserves the $L^2$ norm of the solution.

\begin{exercise}
  Compute that
  \[
    \frac{d}{dt} \int_{\R^d} |\psi(t, x)|^2\, dx = 0.
  \]
  This is an alternative way to show that
  the $L^2$ norm of the solution is preserved.
\end{exercise}

\section{Sobolev Spaces}
\begin{definition}
  The \emph{Sobolev spaces} $H^{\gamma} = W^{\gamma, 2}$
  for $\gamma \in \R$
  are defined via the norm
  \[
    \|f\|_{H^{\gamma}}
    = \left(\int_{\R^d} (1 + |\xi|^2)^\gamma |\widehat{f}(\xi)|^2\, d\xi\right)^{1 / 2}.
  \]
  The \emph{homogeneous Sobolev spaces} $\dot{H}^\gamma$
  are defined by the norm
  \[
    \|f\|_{\dot{H}^\gamma}
    = \left(\int_{\R^d} |\xi|^{2\gamma} |\widehat{f}(\xi)|^2\, d\xi\right)^{1 / 2}.
  \]
\end{definition}

\begin{remark}
  If $\gamma \in \N$ and $d = 1$, then
  \[
    \|f\|_{H^{\gamma}}
    \sim \sum_{m = 0}^\gamma \| \partial_x^m f \|_{L^2}.
  \]
  In particular, this means that
  $f \in H^\gamma$ if and only if
  $\partial_x^m f \in L^2$ for all $m \le \gamma$.
\end{remark}

\begin{exercise}
  Check that if $f_\lambda(x) = f(\lambda x)$, then
  $\widehat{f_\lambda}(\xi) = \lambda^{-d} \widehat{f}(\xi / \lambda)$.
\end{exercise}

\begin{remark}
  In the Sobolev spaces, this means that (change
  variables $\eta = \xi / \lambda$ for the last equality)
  \[
    \|f_\lambda\|_{\dot{H}^{\gamma}}
    = \left(\int_{\R^d} |\xi|^{2\gamma} |\widehat{f_\lambda}(\xi)|^2\, d\xi\right)^{1 / 2}
    = \left(\int_{\R^d} |\xi|^{2\gamma} |\lambda^{-d} \widehat{f}(\xi / \lambda)|^2\, d\xi\right)^{1 / 2}
    = \lambda^{\gamma - d / 2} \|f\|_{\dot{H}^{\gamma}}.
  \]
\end{remark}

\begin{lemma}
  In the Schr\"odinger equation, $\|\psi(t)\|_{H^{\gamma}} = \|\psi_0\|_{H^{\gamma}}$
  and $\|\psi(t)\|_{\dot{H}^{\gamma}} = \|\psi_0\|_{\dot{H}^{\gamma}}$
  for all $t$ and $\gamma$.
\end{lemma}

\begin{proof}
  We can compute that
  \[
    \|\psi(t)\|_{\dot{H}^{\gamma}}
    = \int_{\R^d} |\xi|^{2\gamma} |\widehat{\psi}(t, \xi)|^2\, d\xi
    = \int_{\R^d} |\xi|^{2\gamma} |e^{-i |\xi|^2 / 2}\widehat{\psi_0}(\xi)|^2\, d\xi
    = \int_{\R^d} |\xi|^{2\gamma} |\widehat{\psi_0}(\xi)|^2\, d\xi
    = \|\psi_0\|_{\dot{H}^{\gamma}}.
  \]
  The same argument works for the $H^{\gamma}$ case
  after replacing $|\xi|^{2\gamma}$ with
  $(1 + |\xi|^2)^\gamma$.
\end{proof}

  \chapter{Jan.~8 --- Special Solutions}

\section{Special Solutions}
\begin{example}
  The following are special solutions to the
  Schr\"odinger equation:
  \begin{enumerate}
    \item Gaussian: $\psi_0 = e^{-|x|^2 / 2}$.
      One can compute the Fourier transform and get
      \[
        \widehat{\psi}_0(\xi)
        = \int_{\R^d} e^{-ix \cdot \xi} e^{-|x|^2 / 2} \, dx
        = \int_{\R^d} e^{-|x + i\xi|^2 / 2} e^{-|\xi|^2 / 2} \, dx
        = e^{-|\xi|^2 / 2}\int_{\R^d} e^{-|x + i\xi|^2 / 2} \, dx.
      \]
      The last integral is a contour integral in the
      complex plane along $\Im z = \xi$, and
      we can deform the contour via Cauchy's theorem
      to the real axis to obtain (the integrand is
      analytic on $0 \le \Im z \le \xi$)
      \[
        \widehat{\psi}_0(\xi)
        = e^{-|\xi|^2 / 2}\int_{\R^d} e^{-|x|^2 / 2} \, dx
        = (2\pi)^{d / 2} e^{-|\xi|^2 / 2}.
      \]
      Then taking inverse Fourier transforms, we obtain
      the solution
      \begin{align*}
        \psi(t, x)
        = (2\pi)^{-d} \int_{\R^d} e^{i(x \cdot \xi - |\xi|^2 t / 2)} \widehat{\psi}_0(\xi) \, d\xi
        &= (2\pi)^{-d / 2} \int_{\R^d} e^{i(x \cdot \xi - |\xi|^2 t / 2)} e^{-|\xi|^2 / 2} \, d\xi \\
        &= (2\pi)^{-d / 2} \int_{\R^d} e^{-\frac{1}{2}(1 + it)|\xi|^2} e^{ix \cdot \xi} \, d\xi.
      \end{align*}
      Now formally put $\eta = (1 + it)^{1 / 2} \xi$
      to get
      \[
        \psi(t, x) = (2\pi)^{-d / 2} (1 + it)^{-d / 2} \int_{\R^d} e^{-\frac{1}{2}|\eta|^2} e^{ix \eta / (1 + it)^{1 / 2}}\, d\eta.
      \]
      Fill in the details of the above change of
      variables as an exercise (e.g. one has to
      worry about choosing a branch cut when taking
      the square root). Computing the integral explicitly,
      one obtains
      \[
        \psi(t, x) = (1 + it)^{-d / 2} e^{-|x|^2 / (2(1 + it))}.
      \]
      One can from this that $\psi$ has decay in
      time. Furthermore, one can see that
      \[
        |\psi(t, x)|^2 = (1 + t^2)^{-d / 2} e^{-|x|^2 / (1 + t^2)}.
      \]
      From this we can observe an $L^\infty$ decay
      of $\psi$ like $t^{-d / 2}$, and that the influence
      region of the solution grows like order $t$.
      We can also see again from this explicit
      computation that $\|\psi(t)\|_{L^2} = C$.
    \item Modulated Gaussian: $\psi_0 = e^{-|x|^2 / 2} e^{ix \cdot v}$.
      The Fourier transform of this initial data is
      \[
        \widehat{\psi}_0(\xi)
        = (2\pi)^{d / 2} e^{-i|\xi - v|^2 / 2}.
      \]
      So the solution corresponding to this initial
      data is
      \begin{align*}
        \psi(t, x)
        = (2\pi)^{-d} \int_{\R^d} e^{i(x \cdot \xi - |\xi|^2 t / 2)} \widehat{\psi}_0(\xi) \, d\xi
        &= (2\pi)^{-d / 2} \int_{\R^d} e^{i(x \cdot \xi - |\xi|^2 t / 2)} e^{-|\xi - v|^2} \, d\xi \\
        &= e^{ix \cdot v} e^{-|v|^2 t / 2} (2\pi)^{-d / 2}
        \int_{\R^d} e^{i(x - vt) \cdot \xi} e^{-(1 + it) |\xi|^2 / 2}\, d\xi \\
        &= e^{ix \cdot v} e^{-|v|^2 t / 2}
        (1 + it)^{-d / 2}
        \exp\left(-\frac{|x - vt|^2}{2(1 + it)}\right).
      \end{align*}
      From this we can see that the influence region
      of the solution moves with velocity $v$.
    \item Fundamental solution: We want a
      \emph{fundamental solution} $K$ such that
      $K$ solves
      \[
        i \partial_t K + \frac{1}{2} \Delta K = 0
        \quad \text{and} \quad K|_{t = 0} = \delta_0.
      \]
      We will find $K$ by scaling arguments. Suppose
      such a $K$ exists. Then we must have
      \[
        \psi(t, x) = \int_{\R^d} K(t, x - y) \psi_0(y)\, dy \tag{1}
      \]
      since $K|_{t = 0} = \delta_0$.
      Now define the scaling
      $\psi_\lambda(t, x) = \psi(\lambda^2 t, \lambda x)$.
      Then $\psi_\lambda$ also solves
      \[
        i \partial_t \psi_\lambda + \frac{1}{2} \Delta \psi_\lambda = 0
      \]
      and we have the initial condition
      $\psi_\lambda(0, x) = \psi_0(\lambda x)$. Then
      \[
        \psi_\lambda(t, x)
        = \int_{\R^d} K(t, x - y)\psi_0(\lambda y)\, dy
        = \psi(\lambda^2 t, \lambda x).
      \]
      Setting $t' = \lambda^2 t$, $x' = \lambda x$, and
      $y' = \lambda y$, we get
      \[
        \psi(t', x')
        = \frac{1}{\lambda^d} \int_{\R^d} K\left(\frac{t'}{\lambda^2}, \frac{x' - y'}{\lambda}\right)
        \psi_0(y')\, dy'.
        \tag{2}
      \]
      Comparing (1) and (2), we see that we must have
      \[
        K(t, x - y) = \lambda^{-d} K\left(\frac{t}{\lambda^2}, \frac{x - y}{\lambda}\right).
      \]
      Setting $u = x - y$, we get
      \[
        K(t, u) = \lambda^{-d} K\left(\frac{t}{\lambda^2}, \frac{u}{\lambda}\right).
      \]
      Thus we expect
      $K(t, x) = t^{-d / 2} \Phi(|x|^2 / t)$
      for some $\Phi$. Now we use the fact that
      $i \partial_t K + \frac{1}{2} \Delta K = 0$.
      Setting $m = |x|^2 / t$, one can
      plug in the above guess for $K$ to obtain
      (note that $\Delta = \nabla \cdot \nabla$)
      \[
        -\frac{id}{2} t^{-d / 2 - 1} \Phi(m)
        - it^{-d / 2} \Phi'(m) \frac{m}{t}
        + \frac{1}{2} t^{-d / 2} \nabla \cdot \left(\frac{2x}{t} \Phi'(m)\right) = 0.
      \]
      Then we get
      \[
        -i \frac{d}{2} \Phi(m) - im \Phi'(m)
        + d \Phi'(m) + 2m \Phi''(m) = 0,
      \]
      which gives
      \[
        d\left(\Phi'(m) - \frac{i}{2} \Phi(m)\right)
        + 2m \frac{d}{dm} \left(\Phi'(m) - \frac{i}{2}\Phi(m)\right) = 0.
      \]
      Now observe that
      $\Phi(m) = e^{im / 2}$ solves the above equation.
      Since $\Phi(m)$ solves the
      equation, $c \Phi(m)$ also solves the equation for
      any $c \in \C$, and thus we have
      \[
        K(t, x) = c t^{-d / 2} \Phi(|x|^2 / t)
        = c t^{-d / 2} e^{i|x|^2 / 2t}.
      \]
      To determine $c$, we use $K|_{t = 0} = \delta_0$,
      from which one can obtain $c = (2\pi i)^{-d / 2}$. Thus
      \[
        K(t, x) = (2\pi it)^{-d / 2} e^{i|x|^2 / 2t}.
      \]
      The rough computation is that since
      $\widehat{K}(0, \xi) = 1$, we have
      \[
        K = (2\pi)^{-d} \int_{\R^d} e^{i(x \cdot \xi - |\xi|^2 t / 2)} \widehat{K}(0, \xi)\, d\xi
        = (2\pi)^{-d} \int_{\R^d} e^{i(x \cdot \xi - |\xi|^2 t / 2)}\, d\xi.
      \]
      This is not necessarily integrable a priori,
      but one can take limits and obtain
      \begin{align*}
        K
        = (2\pi)^{-d} \lim_{\epsilon \to 0^+}
        \int_{\R^d} e^{ix \cdot \xi} e^{-(\epsilon + it)|\xi|^2 / 2}\, d\xi
        &= \lim_{\epsilon \to 0^+} (\epsilon + it)^{-d / 2} (2\pi)^{-d / 2} e^{-|x|^2 / (2(\epsilon + it))} \\
        &= (2\pi it)^{-d / 2} e^{-|x|^2 / 2it}.
      \end{align*}
      Note that this computation matches the result
      of the previous scaling argument.
  \end{enumerate}
\end{example}

\begin{theorem}
  Let $\psi_0 \in \mathcal{S}(\R^d)$.\footnote{Here $\mathcal{S}(\R^d)$ is the space of \emph{Schwartz functions}.}
  Then there exists a solution to
  \[
    \begin{cases}
      i \partial_t \psi + \frac{1}{2} \Delta \psi = 0, \\
      \psi|_{t = 0} = \psi_0,
    \end{cases}
  \]
  which is unique and given by
  \[
    \psi(t, x) = \int_{\R^d} K(t, x - y) \psi_0(y)\, dy
    = (2\pi it)^{-d / 2} \int_{\R^d} e^{-|x - y|^2 / 2it} \psi_0(y)\, dy.
  \]
\end{theorem}

\begin{proof}
  This theorem is a summary of the results of
  the previous explicit computations.
\end{proof}

\begin{remark}
  Recall that the Schr\"odinger evolution
  preserves the $L^2$ norm of a solution, i.e.
  \[
    \|\psi(t)\|_{L^2} = \|\psi(0)\|_{L^2} = \|\psi_0\|_{L^2}.
  \]
  The above theorem also gives an $L^\infty$ bound
  (a so-called \emph{dispersive estimate})
  \[
    \|\psi(t)\|_{L^\infty}
    \le |2\pi t|^{-d / 2} \int_{\R^d} |\psi_0(y)|\, dy
    = |2\pi t|^{-d / 2} \|\psi_0\|_{L^1}.
  \]
\end{remark}

  \chapter{Jan.~15 --- Strichartz Estimates}

\section{Interpolation Results}
\begin{remark}[Interpolation]
  Consider a linear operator $T$ which maps
  $T : L^{p_1} \to L^{q_1}$ and
  $T : L^{p_2} \to L^{q_2}$, where $1 \le p_1 \le p_2 \le \infty$.
  Then $T$ also maps $T : L^p \to L^q$ for any $p, q$
  such that
  \[
    \frac{1}{p} = \frac{\theta}{p_1} + \frac{1 - \theta}{p_2}
    \quad \text{and} \quad
    \frac{1}{q} = \frac{\theta}{q_1} + \frac{1 - \theta}{q_2}
  \]
  for some $0 \le \theta \le 1$. More specifically, if
  $\|Tf\|_{L^{q_1}} \le C_1 \|f\|_{L^{p_1}}$ and
  $\|Tf\|_{L^{q_2}} \le C_2 \|f\|_{L^{p_2}}$, then
  \[
    \|Tf\|_{L^q} \le C_1^\theta C_2^{1 - \theta} \|f\|_{L^p}.
  \]
  This $L^p$ interpolation is a result from real and
  functional analysis. Note that by interpolation,
  we have
  \[
    \|\psi\|_{L^{p'}(\R^d)} \le C |t|^{-d(1 / p - 1 / 2)} \|\psi_0\|_{L^p(\R^d)}
  \]
  for $1 \le p \le 2$, where $p'$ is the
  \emph{H\"older conjugate} of $p$, i.e.
  $1 / p' + 1 / p = 1$.
\end{remark}

\section{Strichartz Estimates}

\begin{remark}
  We will now consider the inhomogeneous Schr\"odinger
  equation:
  \[
    \begin{cases}
      i \psi_t + \frac{1}{2}\Delta \psi = F, & F \in \mathcal{S}_{x, t} \\
      \psi(0) = \psi_0, & \psi_0 \in \mathcal{S},
    \end{cases}
  \]
  where $F \in \mathcal{S}_{x, t}$ means that $F$ is
  Schwartz in both $x$ and $t$. We can solve this
  via the \emph{Duhamel formula}:
  \[
    \psi(t) = e^{i t \Delta / 2} \psi_0 - i \int_0^t e^{i(t - s) \Delta / 2} F(s)\, ds,
  \]
  where $e^{it \Delta / 2}$ is the \emph{linear
  propagator} given by
  \[
    e^{it \Delta / 2} \psi_0
    = (e^{-it |\xi|^2 / 2} \widehat{\psi}_0)^\vee
    = \frac{1}{(2\pi)^d} \int_{\R^d} e^{ix \cdot \xi} e^{-it |\xi|^2 / 2} \widehat{\psi}_0(\xi)\, d\xi.
  \]
\end{remark}

\begin{theorem}[Strichartz estimates]\label{thm:strichartz}
  For $p' = 2 + 4 / d$, we have the estimate\footnote{Here $A \lesssim B$ means that $A \le CB$ for some prescribed constant $C$.}
  \[
    \|\psi\|_{L^{p'}_{t, x}(\R \times \R^d)}
    \lesssim \|\psi_0\|_{L^2_x(\R^d)}
    + \|F\|_{L^{p}_{t, x}(\R \times \R^d)}.
  \]
\end{theorem}

\begin{remark}
  If $F = 0$, this is the bound
  \[
    \|\psi\|_{L^{p'}_{t, x}}
    \lesssim \|\psi_0\|_{L^2}
  \]
  for $p' > 2$. Formally, this means that we gain
  integrability in $x$. Note that this gain in
  integrability is not pointwise in time, i.e.
  we do \emph{not} have $\|\psi(t)\|_{L^\infty_t L^{p'}_x} \lesssim \|\psi_0\|_{L^2_x}$. We must instead average
  over $t$.
\end{remark}

\begin{remark}
  Why $p'$ and why do we also pick $p'$ in the time
  integration? Actually, $p'$ is the only possible
  choice for the above result. This follows by a
  scaling argument: Set
  \[
    \psi_\lambda(t, x) = \psi(\lambda^2 t, \lambda x),
    \quad
    (\psi_\lambda)_0(x) = \psi_0(\lambda x),
    \quad
    F_\lambda(t, x) = \lambda^2 F(\lambda^2 t, \lambda x).
  \]
  Then $\psi_\lambda$ solves the equation
  \[
    \begin{cases}
    i \partial_t \psi_\lambda + \frac{1}{2} \Delta \psi_\lambda = F_\lambda, \\
    \psi_\lambda(0) = (\psi_{\lambda})_0.
    \end{cases}
  \]
  If the above theorem makes sense, then it must hold
  for both $\psi_\lambda$ and $\psi$. Now
  \[
    \|\psi_\lambda\|_{L^{p'}_{t, x}}
    = \lambda^{-d / p'} \lambda^{-2 / p'} \|\psi\|_{L^{p'}_{t, x}}
  \]
  by a change of variables, and
  \[
    \|(\psi_\lambda)_0\|_{L^2_x}
    = \lambda^{-d / 2} \|\psi_0\|_{L^2_x}.
  \]
  Now if $F = 0$, then we have the estimates
  \[
    \|\psi\|_{L^{p'}_{t, x}} \lesssim \|\psi_0\|_{L^2_x}
    \quad \text{and} \quad
    \|\psi_\lambda\|_{L^{p'}_{t, x}}
    \lesssim \|(\psi_\lambda)_0\|_{L^2_x}, \tag{$*$}
  \]
  Using the scaling computations in the second estimate
  in $(*)$ implies that
  \[
    \| \psi \|_{L^{p'}_{t, x}} \lambda^{-d / p'} \lambda^{-2 / p'}
    \lesssim \lambda^{-d / 2} \|\psi_0\|_{L^2_x}.
  \]
  This inequality should hold independent of $\lambda$,
  since otherwise taking $\lambda \to \infty$ or
  $\lambda \to 0$ yields a contradiction with the
  first inequality in $(*)$.
  Thus the powers in $\lambda$ should match:
  \[
    -\frac{d}{p'} - \frac{2}{p'} = -\frac{d}{2},
  \]
  so we find that $p'$ must be
  \[
    p' = \frac{d + 2}{d / 2} + \frac{2d + 4}{d} = 2 + \frac{4}{d}.
  \]
  This uniquely determines $p'$. Now consider $F \ne 0$.
  Using a similar computation as before, we have
  \[
    \|F_\lambda\|_{L^q_{t, x}} = \lambda^2 \lambda^{-d / q} \lambda^{-2 / q} \|F\|_{L^q_{t, x}}.
  \]
  Then the theorem says that
  $\|\psi_\lambda\|_{L^{p'}_{t, x}} \lesssim \|\psi_0\|_{L^2_x} + \|F\|_{L^q_{t, x}}$,
  so we have
  \[
    \|\psi\|_{L^{p'}_{t, x}} \lambda^{-d / p'} \lambda^{-2 / p'}
    \lesssim \lambda^{-d / 2} \| \psi_0 \|_{L^2_x} + \lambda^2 \lambda^{-d / q} \lambda^{-2 / q} \|F\|_{L^q_{t, x}}.
  \]
  Again the estimate should hold independent of
  $\lambda$, so the powers in $\lambda$ must match:
  \[
    -\frac{d}{p'} - \frac{2}{p'} = 2 - \frac{d}{q} - \frac{2}{q} = -\frac{d}{2},
  \]
  which then gives $p$ as
  \[
    p = \left(1 - \frac{1}{p'}\right)^{-1} = \left(1 - \frac{d}{2d + 4}\right)^{-1} = \frac{2d + 4}{d + 4}.
  \]
\end{remark}

\begin{lemma}
  Let $\psi(t) = e^{it \Delta / 2} \psi_0$.
  Then for $1 \le p \le 2$,
  \[
    \|\psi(t) \|_{L^{p'}_x(\R^d)}
    \lesssim |t|^{-d (1 / p - 1 / 2)} \|\psi_0\|_{L^p_x(\R^d)}.
  \]
\end{lemma}

\begin{proof}
  This is the interpolation result from the beginning
  of class.
\end{proof}

\begin{lemma}[Hardy-Littlewood-Sobolev inequality]
  Let $0 < \alpha < 1$ and $g \in \mathcal{S}(\R)$.
  Let
  \[
    (T_\alpha g)(t) = \int_{-\infty}^\infty |t - s|^{-\alpha} g(s)\, ds.
  \]
  Then we have
  $\|T_\alpha g\|_{L^q(\R)} \lesssim \|g\|_{L^p(\R)}$,
  where $1 < p < q < \infty$ such that
  $1 + 1 / q = \alpha + 1 / p$.
\end{lemma}

\begin{proof}
  One approach is via harmonic analysis and maximal
  functions. An alternative approach can be found in
  Theorem 4.3 of Analysis by Lieb and Loss.
\end{proof}

\begin{remark}
  Recall \emph{Young's inequality} that for
  \[
    h(t) = \int f(t - s) g(s)\, dx,
  \]
  we have $\|h\|_{L^r} \le \|f\|_{L^p} \|g\|_{L^q}$,
  where $1 / r + 1 = 1 / q + 1 / p$. The
  Hardy-Littlewood-Sobolev inequality can be
  seen as a generalized Young's inequality: If
  $f(s) = |s|^{-\alpha}$, then $f$ barely fails to be
  in $L^{1 / \alpha}$. Informally, we can think of
  ``$f \in L^{1 / \alpha}$,'' and the standard
  Young's inequality would imply
  Hardy-Littlewood-Sobolev.
\end{remark}

\begin{remark}
  We have $q > p$ in the Hardy-Littlewood-Sobolev
  inequality, so we gain some integrability via
  fractional integration for $p > 1$
  (the type of integral defining
  $T_\alpha g$ is known as \emph{fractional integration}).
\end{remark}

\begin{proof}[Proof of Theorem \ref{thm:strichartz}]
  This proof is left for next class.
\end{proof}

  \chapter{Jan.~22 --- Strichartz Estimates, Part 2}

\section{Proof of Strichartz Estimates}

\begin{proof}[Proof of Theorem \ref{thm:strichartz}]
  The first step is a $TT^*$ argument. Define the
  operator $T$ by $Tf = e^{it \Delta / 2} f$. We know
  that $T : L^2_x(\R^d) \to L^\infty_t(\R \to L^2_x(\R^d))$.
  The adjoint $T^* : L^1_t L^2_x \to L^2_x$ is defined
  via the relation
  \[
    \langle f, T^* G \rangle_{L^2_x}
    = \langle Tf, G \rangle_{L^2_{t, x}}
    = \iint (e^{it \Delta / 2} f)(x) \overline{G}(t, x)\, dt dx
    = \int f(x) \int \overline{(e^{-it \Delta / 2} G(t, \cdot))}(x)\, dt dx,
  \]
  and so we have the formula
  \[
    T^* G = \int (e^{-it \Delta / 2} G(t, \cdot))(x)\, dt.
  \]
  Then we can see that
  \[
    (TT^* G)(t, x)
    = \int (e^{i(t - s) \Delta / 2} G(s, \cdot))(x)\, ds
    = \left(e^{i t \Delta / 2} \int e^{-is \Delta / 2} G(s, \cdot)\, ds\right)(x).
  \]
  Note that there is a convolution structure in the
  time variable. Clearly $TT^* : L^1_t L^2_x \to L^\infty_x L^2_x$.
  Then the goal for now will be to show that
  \[
    \|TT^* G\|_{L^{p'}_{t, x}} \le C \|G\|_{L^p_{t, x}}.
  \]
  To do this, observe that by the above expression for
  $TT^* G$ and Minkowski's inequality, we have
  \[
    \|TT^* G\|_{L^{p'}_{x}}
    \le \int \|e^{i(t - s) \Delta / 2} G(s)\|_{L^{p'}_x}\, ds
    \le C \int |t - s|^{-d(1 / p - 1 / 2)} \|G(s)\|_{L^p_x}\, ds,
  \]
  where the second inequality follows by
  Lemma \ref{lem:decay}. Now by Lemma \ref{lem:hardy-littlewood-sobolev},
  \[
    \|TT^* G\|_{L^q_t L^{p'}_x}
    \le C \|G(s)\|_{L^p_{t, x}}
  \]
  if $1 / q + 1 = d(1 / p - 1 / 2) + 1 / p$
  (also check that $0 < \alpha = d(1 / p - 1 / 2) < 1$,
  where $p = (2d + 4) / (d + 4)$).
  From this relation, we find that we must have
  $q = 2 + 4 / d = p'$, so we have shown the goal.

  Thus we have proved that $TT^* : L^p_{t, x} \to L^{p'}_{t, x}$, where
  $p = (2d + 4) / (d + 4)$. Now
  \[
    \|T^* G\|^2_{L^2_x} = \langle T^* G, T^* G \rangle_{L^2_x}
    = \langle TT^* G, G \rangle_{L^2_{t, x}}
    \le \|TT^* G\|_{L^{p'}_{t, x}} \|G\|_{L^p_{t, x}}
    \le C \|G\|_{L^p_{t, x}}^2,
  \]
  where the first inequality follows by H\"older's inequality
  and the second follows from the goal we just proved.
  Thus we conclude that $T^* : L^p_{t, x} \to L^2_x$, and
  that $T : L^2_x \to L^{p'}_{t, x}$ by duality.
  Therefore,
  \[
    \|T \psi_0\|_{L^{p'}_{t, x}}
    = \|e^{it \Delta / 2} \psi_0\|_{L^{p'}_{t, x}}
    \le C\|\psi_0\|_{L^2_x},
  \]
  so we have proved the Strichartz estimates when
  $F = 0$.

  In the case when $F \ne 0$, by Duhamel's formula
  we have
  \[
    \psi(t) = e^{it \Delta / 2} \psi_0
    - i \int_0^t e^{i(t - s) \Delta / 2} F(s)\, ds,
  \]
  so by the triangle inequality we find that
  \[
    \|\psi\|_{L^{p'}_{t, x}}
    \le \|e^{it \Delta / 2} \psi_0\|_{L^{p'}_{t, x}}
    + \left\|
    \int_0^t e^{i(t - s) \Delta / 2} F(s)\, ds
    \right\|
    \le C \|\psi_0\|_{L^2_x}
    + \left\| \int_0^t \|e^{i(t - s)\Delta / 2} F(s)\|_{L^{p'}_x}\, ds \right\|_{L^{p'}_t},
  \]
  where the last inequality on the second term
  follows by Minkowski's inequality. Then using
  Lemma \ref{lem:decay} and Lemma \ref{lem:hardy-littlewood-sobolev} in the same fashion as before, we can bound
  the latter term by
  \[
    \left\| \int_0^t \|e^{i(t - s)\Delta / 2} F(s)\|_{L^{p'}_x}\, ds \right\|_{L^{p'}_t}
    \le
    C \left\| \int_{-\infty}^\infty |t - s|^{-d (1 / p - 1 / 2)} \|F(s)\|_{L^{p'}_x}\, ds \right\|_{L^{p'}_t}
    \le C \|F\|_{L^p_{t, x}}.
  \]
  Plugging this bound back in gives the desired
  inequality $\|\psi\|_{L^{p'}_{t, x}} \le C \|\psi_0\|_{L^2_x} + C \|F\|_{L^p_{t, x}}$.
\end{proof}

\begin{remark}
  Note that the term
  \[
    TT^* G = \int_{-\infty}^\infty e^{i(t - s) \Delta / 2} G(s)\, ds
  \]
  looks similar to term from the Duhamel formula
  \[
    \int_0^t e^{i(t - s) \Delta / 2} F(s)\, ds.
  \]
  However, it is possible that these two have
  different estimates, which is why we had to
  argue separately.
\end{remark}

\section{Strichartz Estimates and Harmonic Analysis}
\begin{remark}
  The original intention of Strichartz for these
  estimates was for use in harmonic analysis.
  The Strichartz estimates can actually be derived
  from the \emph{Stein-Tomas restriction theorem}.

  Let $S \subseteq \R^n$ with $n = d + 1$, where
  $S$ is a hypersurface.
  If $f \in L^1$, then one can show (e.g.
  using the Riemann-Lebesgue lemma) that
  $\widehat{f} \in L^\infty$. So we can conclude that
  $\widehat{f}$ has pointwise meaning, i.e.
  $\widehat{f}(\xi)$ makes sense pointwise. In
  particular, we can make sense of $\widehat{f}(\xi)$
  on the hypersurface $S$.

  On the other hand, if $f \in L^2$, then by Plancherel's theorem,
  $\widehat{f} \in L^2$ as well. But
  an $L^2$ function has no pointwise interpretation,
  i.e. we can modify it on a set of measure zero without
  changing the function.
  In particular, it is meaningless to restrict
  the function $\widehat{f}$ to $S$, since $S$ is a set
  of measure zero in $\R^n$.

  In general, what about $f \in L^p$ for
  $1 < p < 2$? This is the topic of the
  \emph{restriction theorems} in harmonic analysis.
  It turns out that the choice of which $p$ work
  depends on the ``curvature'' of $S$.
\end{remark}

\begin{theorem}[Stein-Tomas restriction theorem]
  Let $n = d + 1$ and $S \subseteq \R^n$ be a
  hypersurface with non-vanishing Gaussian curvature.
  Let $\sigma_S$ be the corresponding surface measure,
  and let $\phi$ be a compactly supported  on $S$.
  Then we have
  \[
    \|(\phi \sigma_S)^\vee\|_{L^r(\R^n)} \le
    C \|\phi\|_{L^2(\sigma_S)},
  \]
  where $r = (2n + 2) / (n - 1)$.
\end{theorem}

\begin{remark}
  Now recall the explicit formula for a solution $\psi$
  to the Schr\"odinger equation:
  \[
    \psi(t, s) = \int e^{i(x \cdot \xi - t |\xi|^2 / 2)} \widehat{\psi}_0(\xi)\, d\xi.
  \]
  Also define the hypersurface
  $S = \{(\xi, \tau) : \tau = -|\xi|^2 / 2, \xi \in \R^d\}$.\footnote{This particular hypersurface is called the \emph{characteristic surface} of the Schr\"odinger equation.}
  Then
  \[
    \psi(t, x) = (\phi \sigma_S)^\vee (t, x), \quad
    \phi(\xi, \tau) = \widehat{\psi}_0(\xi), \quad
    \sigma_S(d \xi, d \tau) = (2 \pi)^d d\xi.
  \]
  Indeed, we can see that
  \[
    \psi(t, x) = (2\pi)^{-d} \int e^{i(x \cdot \xi + t \tau)} \phi(\xi, \tau)\, \sigma_S(d \xi, d \tau)
    = \int e^{i(x \cdot \xi - t |\xi|^2 / 2)} \widehat{\psi}_0(\xi)\, d\xi.
  \]
  Then, the Stein-Tomas restriction theorem tells us
  that
  $\|(\phi \sigma_S)^\vee\|_{L^r(\R^n)} \le C \|\phi\|_{L^2(\sigma_S)}$,
  which implies
  \[
    \|\psi\|_{L^{2 + 4 / d}_{t, x}(\R \times \R^d)}
    \le C \|\widehat{\psi}_0\|_{L^2_\xi}
    \le C \|\psi_0\|_{L^2_x},
  \]
  where the last inequality follows by
  Plancherel's theorem. Note that the estimate
  above technically only holds for those $\psi$
  for which $\widehat{\psi}_0$ is compactly supported,
  but one can extend this by a density argument.
\end{remark}

  \chapter{Jan.~27 --- Kato Smoothing}

\section{Kato Smoothing}

\begin{theorem}[Kato $1 / 2$ smoothing estimate]
  Let $d \ge 2$ and $\chi \ge 0$ be a smooth
  cutoff function such that $\widehat{\chi}$ is
  compactly supported. Then we have the estimate
  \[
    \|\chi(x) (-\Delta)^{1 / 4} e^{i t \Delta / 2} f\|_{L^2_{x, t}} \le C\|f\|_{L^2}.
  \]
\end{theorem}

\begin{remark}
  The above theorem says that when we localize, we
  gain a smoothing effect ($1 / 2$ derivatives).
  Also note that this effect is not pointwise, but rather
  after integrating in time.
\end{remark}

\begin{remark}
  Let $f(t) = e^{it \Delta / 2} f_0$, where
  $f_0 = e^{-|x|^2 / 2} e^{-ix v}$. Think of this
  as a quantum particle at the origin with an initial
  velocity $v$, so
  that the particle will stay in
  $B(0, 1)$ for a period of order $O(1 / |v|)$. Then
  \[
    \left(\int_0^{1 / |v|} \int_{B(0, 1)} |f|^2\, dx dt\right)^{1 / 2}
    \sim \left(\frac{1}{|v|}\right)^{1 / 2}
    = \frac{1}{|v|^{1 / 2}}.
  \]
  Now $(-\Delta)^{1 / 2} f \sim |v|^{1 / 2}$, so these
  factors cancel each other out.
  This matches the above estimate.
\end{remark}

\begin{theorem}[1-D Kato smoothing estimate]
  For $d = 1$, we have
  \[
    \sup_x \|(-\Delta)^{1 / 4} e^{it \Delta / 2} f\|_{L^2_t}
    \le C \|f\|_{L^2}.
  \]
\end{theorem}

\begin{proof}
  For $d = 1$, by the Fourier transform we have
  \begin{align*}
    (-\Delta)^{1 / 4} e^{it \Delta / 2} f
    &= \frac{1}{2\pi} \int_\R e^{ix \xi} |\xi|^{1 / 2} e^{-it \xi^2 / 2} \widehat{f}(\xi)\, d\xi \\
    &= \frac{1}{2\pi} \int_{-\infty}^0 e^{ix \xi} |\xi|^{1 / 2} e^{-it \xi^2 / 2} \widehat{f}(\xi)\, d\xi
    + \frac{1}{2\pi} \int_{0}^\infty e^{ix \xi} |\xi|^{1 / 2} e^{-it \xi^2 / 2} \widehat{f}(\xi)\, d\xi.
  \end{align*}
  Since we want an $L^2_t$ bound, it indicates
  that we should apply Plancherel's theorem in time.
  We will prove the estimate for the latter integral, and
  the former integral is left as an exercise.
  Set $\eta = \xi^2$ with $d\eta = 2\xi\, d\xi = 2 \sqrt{\eta}\, d\xi$.
  Applying this change of variables, we obtain
  \[
    (-\Delta)^{1 / 4} e^{it \Delta / 2} f
    = \frac{1}{2\pi} \int_0^\infty e^{ix \sqrt{\eta}} |\eta|^{1 / 4} e^{-it \eta / 2} \widehat{f}(\sqrt{\eta}) \frac{1}{2\sqrt{\eta}} \, d\eta
    = \frac{1}{4\pi} \int_0^\infty e^{ix \sqrt{\eta}}
    |\eta|^{-1 / 4} e^{-it\eta / 2} \widehat{f}(\sqrt{\eta})\, d\eta.
  \]
  Fix $x$ and denote $h(\eta) = e^{ix \sqrt{\eta}} |\eta|^{-1 / 4} \widehat{f}(\sqrt{\eta})$. Then we have
  \[
    (-\Delta)^{1 / 4} e^{it \Delta / 2} f
    = \frac{1}{4\pi} \int_0^\infty e^{-it \eta / 2} h(\eta)\, d\eta = (\star).
  \]
  By Plancherel's theorem in time, we see that
  \[
    \| (\star) \|_{L^2_t}
    \le \|h(\eta)\|_{L^2_\eta}.
  \]
  Then we can estimate (setting $z = \sqrt{\eta}$ with
  $dz = d\eta / (2\sqrt{\eta})$)
  \[
    \int_0^\infty |h(\eta)|^2\, d\eta
    = \int_0^\infty |\eta|^{-1 / 2} |\widehat{f}(\sqrt{\eta})|^2\, d\eta
    = 2 \int_0^\infty |\widehat{f}(z)|^2\, dz
    = C \|f\|_{L^2}^2,
  \]
  where the last inequality follows by Plancherel's
  theorem in $z$. This yields
  $\|(\star)\|_{L^2_t} \le C \|f\|_{L^2}$, and since
  these estimates are all independent
  of $x$, we get that $\sup_x \|(\star)\|_{L^2_t} \le C\|f\|_{L^2}$.
  Putting this together with an identical estimate
  for the other integral, we obtain the desired
  bound.
\end{proof}

\begin{remark}
  The above 1-D version is stronger and implies the
  statement
  \[
    \| \chi (-\Delta)^{1 / 4} e^{it \Delta / 2} f\|_{L^2_{x, t}}
    \le C \|f\|_{L^2}
  \]
  for the 1-D case. Check this as an exercise.
\end{remark}

\section{Coarea Formula}

\begin{remark}
  In dimension $d$, suppose we have two nice functions
  $g$ and $u$ such that $u^{-1}(t)$ is a
  $(d - 1)$-dimension hypersurface. Then the
  \emph{coarea formula} says that
  \[
    \int_{\R^d} g(x) |\nabla u(x)|\, dx
    = \int_\R \int_{\{u(x) = t\}} g(x)\, d\sigma(x) dt,
  \]
  where $\sigma(x)$ is the surface measure on
  $\{u(x) = t\}$.

  Note that for $d = 1$, this says that
  \[
    \int g(x) |\partial_x u|\, dx
    = \int_\R \left(\int_{\{u(x) = t\}} g(x)\, dx\right) dt
    = \int_\R g(u^{-1}(t))\, dt.
  \]
  In particular, this is the change of
  variables formula where $\eta = u^{-1}(t)$ (so
  $u(\eta) = t$ and $dt = \partial_\eta u\, d \eta$).
\end{remark}

\begin{lemma}
  Let $F \in C_0^\infty$ and $\phi$ be smooth. Then
  one has
  \[
    \int_\R \int_{\R^d} e^{i \lambda \phi(x)} F(x)\, dx d\lambda
    = (2\pi)^d \int_{\{\phi = 0\}} \frac{F(x)}{|\nabla \phi(x)|}\, d\sigma(x).
  \]
\end{lemma}

\begin{proof}
  By the coarea formula (using $g(x) = e^{i \lambda \phi(x)} F(x) / |\nabla \phi(x)|$ and $u = \phi$), we have
  \[
    \int_\R \int_{\R^d} e^{i\lambda \phi(x)} F(x)\, dx d\lambda
    = \int_\R \int_\R e^{i \lambda y} \int_{\{\phi = y\}} \frac{F(x)}{|\nabla \phi(x)|}\, d\sigma(x) dy d\lambda.
  \]
  Denote by $h(y)$ the integral
  \[
    h(y) = \int_{\{\phi = y\}} \frac{F(x)}{|\nabla \phi(x)|}\, d\sigma(x).
  \]
  Then we can see that
  \[
    \int_\R \int_\R e^{i \lambda y} \int_{\{\phi = y\}} \frac{F(x)}{|\nabla \phi(x)|}\, d\sigma(x) dy d\lambda
    = \int_\R \int_\R e^{i \lambda y} h(y)\, dy d\lambda
    = \int_\R \widehat{h}(\lambda)\, d\lambda
    = (2\pi)^d h(0).
  \]
  This gives the desired equality after plugging
  in the definition of $h(0)$.
\end{proof}

  \chapter{Jan.~29 --- Kato Smoothing, Part 2}

\section{Proof of Kato Smoothing}

\begin{proof}[Proof of Theorem \ref{thm:kato-smoothing}]
  Note that we have
  (let $G = \chi^2$)
  \[
    (*) \int_\R \int_{\R^d}
    \chi^2 |(-\Delta)^{1 / 4} e^{it \Delta / 2} f|^2\, dx dt
    = \int_\R \langle (-\Delta)^{1 / 4} e^{i t \Delta / 2} f, G (-\Delta)^{1 / 4} e^{it \Delta / 2} f \rangle_{L^2_x} \, dt,
  \]
  so by Plancherel's theorem in $x \mapsto \xi$, we have
  \[
    (*)
    = \int_\R \langle |\xi|^{1 / 2} e^{-i |\xi|^2 / 2} \widehat{f}(\xi), H(\xi) \rangle_{L^2_\xi} \, dt,
  \]
  where $H$ is defined via (since the Fourier
  transform turns multiplication into convolution)
  \[
    H(\xi) = \int_\R \widehat{G}(\xi - \eta) |\eta|^{1 / 2} e^{-i t |\eta|^2 / 2} \widehat{f}(\eta)\, d\eta.
  \]
  Thus we have
  \[
    (*)
    = \int_{\R} \int_{\R^d} \int_{\R^d}
    |\xi|^{1 / 2} |\eta|^{1 / 2} \widehat{G}(\xi - \eta)
    \widehat{f}(\xi) \overline{\widehat{f}}(\eta)
    e^{-it(|\xi|^2 - |\eta|^2) / 2}\, d\xi d\eta dt
  .\]
  Applying Lemma \ref{lem:coarea-bound} with
  ($\lambda = t$)
  \[
    \phi = - \frac{|\xi|^2 + |\eta|^2}{2}, \quad
    F = |\xi|^{1 / 2} |\eta|^{1 / 2} \widehat{G}(\xi - \eta) \widehat{f}(\xi) \overline{\widehat{f}}(\eta),
  \]
  we have (since $\phi = 0$ implies
  $|\xi| = |\eta|$ and $|\nabla \phi| = \sqrt{|\xi|^2 + |\eta|^2}$)
  \[
    (*) = \int_{\{\phi = 0\}} \frac{F}{|\nabla \phi|}\, d\sigma
    = \int_{|\xi| = |\eta|} \frac{|\xi|^{1 / 2} |\eta|^{1 / 2}}{\sqrt{|\xi|^2 + |\eta|^2}} \widehat{G}(\xi - \eta) \widehat{f}(\xi) \overline{\widehat{f}}(\eta)\, d\sigma.
  \]
  Since $|\xi| = |\eta|$ on the region of integration,
  we have
  \begin{align*}
    (*) = \int_{|\xi| = |\eta|}
    \frac{|\xi|}{\sqrt{2}\xi} \widehat{G}(\xi - \eta) \widehat{f}(\xi) \overline{\widehat{f}}(\eta)\, d\sigma
    &\sim \int_{|\xi| = |\eta|}
    \widehat{G}(\xi - \eta) \widehat{f}(\xi) \overline{\widehat{f}}(\eta)\, d\sigma \\
    &\sim
    \int_{\R^d} |\widehat{G}(\xi - \eta)| (|\widehat{f}(\xi)|^2 + |\widehat{f}(\eta)|^2) \, d\sigma \\
    &\le \int_{\R^d} |\widehat{f}(\xi)|^2
    \int_{|\eta| = |\xi} |\widehat{G}(\xi - \eta)|\, d\sigma(\eta) \, d\xi
    \lesssim \int_{\R^d} |\widehat{f}(\xi)|^2 \, d\xi,
  \end{align*}
  where the last inequality is since (we claim the
  following)
  \[
    K(\xi) = \int_{|\eta| = |\xi|} |\widehat{G}(\xi - \eta)|\, d\sigma(\eta)
  \]
  satisfies $|K(\xi)| \le C$ for some constant $C$.
  For the claim, note that $\widehat{G}(z)$
  decays faster than $z^{-N}$ as $z \to \infty$ for
  every $N$. Then setting $z = \xi - \eta$, we have
  \[
    K(\xi) = \int_{|\eta| = |\xi|} |\widehat{G}(\xi - \eta)|\, d\sigma(\eta)
    = \int_{|\xi - z| = |\xi|} \widehat{G}(z)\, d\sigma(z).
  \]
  Note that this is an integral on a sphere. The
  integral will be bounded on a compact subset of the
  sphere and decays rapidly on the rest of the sphere,
  so we have $|K(\xi)| \le C$ for all $\xi \in \R^d$.
  This gives
  \[
    \|\chi^2 (-\Delta)^{1 / 4} e^{it \Delta / 2} f\|_{L^2_{x, t}}^2
    = 
    \int_\R \int_{\R^d} \chi^2 |(-\Delta)^{1 / 4} e^{it \Delta / 2} f|^2\, dx dt
    \lesssim \int_{\R^d} |\widehat{f}(\xi)|^2 \, d\xi
    \sim \|f\|_{L^2_x}^2,
  \]
  which is the desired estimate.
\end{proof}

\begin{remark}
  An alternative approach to prove the Kato smoothing
  estimate is the following: Take $\eta$ a
  Schwartz function such that
  $\supp \widehat{\eta} \in (-1, 1)$. Then it suffices
  to show estimates for
  \[
    (*) = \int_\R \int_{\R^d}
    |\eta(\epsilon t) \chi(x) (-\Delta)^{1 / 4} e^{it \Delta / 2} f|^2\, dx dt
  \]
  and then take $\epsilon \to 0$. By Plancherel's
  theorem in $x \mapsto \xi$, we have
  \[
    (*) = \int_\R \int_{\R^d}
    \left| \int \eta(\epsilon t) \widehat{\chi}(\xi - \xi') |\xi'|^{1 / 2} e^{it|\xi'|^2 / 2} \widehat{f}(\xi')\, d\xi' \right|^2 d\xi dt
  \]
  Next Plancherel's theorem in $t \mapsto \tau$
  implies that
  \[
    (*) = \int_\R \int_{\R^d}
    \left|
    \int_{\R^d} \widehat{\chi}(\xi - \xi')
    \frac{1}{\epsilon} \widehat{\eta}\left(\frac{\tau - |\xi'|^2 / 2}{\epsilon}\right) |\xi'|^{1 / 2} \widehat{f}(\xi')\, d\xi'
    \right|^2 d\xi d\tau.
  \]
  Cauchy-Schwarz in $\xi'$ implies that
  \[
    (*) \le
    \int_\R \int_{\R^d}
    \left[\left(\int \widehat{\chi}(\xi - \xi') \frac{1}{\epsilon} \widehat{\eta}\left(\frac{\tau - |\xi'|^2 / 2}{\epsilon}\right) d\xi'\right)
    \left(\int \widehat{\chi}(\xi - \xi') \frac{1}{\epsilon} \widehat{\eta} \left(\frac{\tau - |\xi'|^2}{\epsilon}\right) |\xi'| |\widehat{f}(\xi)|^2\, d\xi'\right)\right]
    d\xi d\tau
  .\]
  Now we count the area of the domain of integration:
  \[
    \begin{cases}
      -\epsilon < \tau - |\xi'|^2 / 2 < \epsilon & (1) \\
      |\xi - \xi'| \le 1 & (2)
    \end{cases}
  \]
  Then (2) implies that
  $||\xi| - |\xi'|| \le 1 \sim O(1)$, so
  \[
    ||\xi|^2 - |\xi'|^2|
    = ||\xi| - |\xi'|| \cdot ||\xi| + |\xi'||
    \sim O(1 + |\xi|),
  \]
  and thus $\tau \sim |\xi'|^2$. Fix $\xi$, then
  the size of the interval in which $\tau$ can vary
  is of order $O(1 + |\xi|)$. For $\xi'$,
  notice that $(1)$ implies
  \[
    |\sqrt{2\tau} - |\xi'|
    \sim O\left(\frac{\epsilon}{\sqrt{2\tau} + |\xi'|}\right)
    \sim O\left(\frac{\epsilon}{|\xi| + 1}\right).
  \]
  So the size of the interval in which
  $|\xi'|$ can vary is of order
  $O(\epsilon / (|\xi| + 1))$.  Then we have
  \[
    (*) \lesssim \iint \left(\frac{1}{\epsilon} \frac{\epsilon}{|\xi| + 1}\right)\left(\frac{1}{\epsilon} \frac{\epsilon}{|\xi| + 1}|\xi| |\widehat{f}(\xi')|^2\right) d\tau d\xi
    \lesssim \int \frac{1+ |\xi|}{|\xi| + 1} \frac{|\xi|}{|\xi| + 1} |\widehat{f}(\xi)|^2 \, d\xi
    \lesssim \int |\widehat{f}(\xi)|^2 \, d\xi.
  \]
\end{remark}

  \chapter{Feb.~3 --- Kato Smoothing, Part 3}

\section{Sharpness of Kato Smoothing}

\begin{lemma}[Kato smoothing is sharp]
  We have
  \[
    \|\chi(x) (-\Delta)^{1 / 4} e^{i t \Delta / 2} (e^{-|x|^2 / 2} e^{ixv})\|_{L^2_{x, t}} \sim 1
    \quad \text{and} \quad
    \sup_{v} \| \chi (-\Delta)^{1 / 4 + \delta} e^{it\Delta / 2} (e^{-|x|^2 / 2} e^{ixv}) \|_{L^2_{x, t}} = \infty.
  \]
  for any $\delta > 0$.
\end{lemma}

\begin{proof}
  Let $\supp \widehat{\chi} \subseteq B(0, 1)$ and
  $f_0 = e^{-|x|^2 / 2} e^{ixv}$. Choose
  $\eta$ smooth with $\widehat{\eta}$ compactly supported.
  Then
  \begin{align*}
    \int_{\R} \int_{\R^d}
    |\chi(x) \eta(\epsilon t) (-\Delta)^{1 / 4} e^{it \Delta / 2} f_0|^2\, dxdt
    &= \iint\left|\int \widehat{\chi}(\xi - \xi') \eta(\epsilon t) |\epsilon|^{1 / 2} e^{it|\xi|^2 / 2} \widehat{f}_0(\xi)\, d\xi\right|^2\, d\xi dt \\
    &= \iint \left| \int \widehat{\chi}(\xi - \xi') \frac{1}{\epsilon} \widehat{\eta}\left(\frac{\tau - |\xi|^2 / 2}{\epsilon}\right)|\xi|^{1 / 2} e^{-|\xi - v|^2 / 2}\, d\xi\right|^2\, d\xi' d\tau
  \end{align*}
  by Plancherel's theorem first in $x \mapsto \xi$
  and then in $t \mapsto \tau$, since
  $\widehat{\eta}$ and $\widehat{\chi}$ are compactly
  supported. Now, $-\epsilon < \tau - |\xi|^2 / 2 < \epsilon$
  implies
  \[
    ||\xi| - \sqrt{2\tau}| < \frac{2\epsilon}{|\xi| + \sqrt{2\tau}}.
  \]
  So the region of integration  for $|\xi|$ is an
  annulus-shaped area between $\sqrt{2\tau}$ and
  $2\epsilon / (|\xi| + \sqrt{2\tau})$ (plot the
  above inequality to see the region).
  Thus for fixed $\tau$, as $\epsilon \to 0$,
  we have
  \begin{align*}
    &\iint \left| \int \widehat{\chi}(\xi - \xi') \frac{1}{\epsilon} \widehat{\eta}\left(\frac{\tau - |\xi|^2 / 2}{\epsilon}\right)|\xi|^{1 / 2} e^{-|\xi - v|^2 / 2}\, d\xi\right|^2\, d\xi' d\tau \\
    &\quad \quad \sim \int_{\R} \int_{\R^d}
    \left| \int_{\sqrt{2\tau} \mathbb{S}^{d - 1}} \widehat{\chi}(\xi - \xi') e^{-|\xi - v|^2 / 2} |\xi|^{1 / 2} \frac{2\epsilon}{|\xi| + \sqrt{2\tau}} \frac{1}{\epsilon}\, d\sigma(\xi)\right|^2\, d\xi' d\tau \\
    & \quad \quad = \int_{\R} \int_{\R^d} \left| \int_{\sqrt{2\tau} \mathbb{S}^{d - 1}} \widehat{\chi}(\xi - \xi') e^{-|\xi - v|^2 / 2} |\xi|^{1 / 2} \frac{2}{|\xi| + \sqrt{2\tau}} \, d\sigma(\xi)\right|^2\, d\xi' d\tau.
  \end{align*}
  Since $\widehat{\chi}$ is supported in $B(0, 1)$, for
  fixed $\xi$ the range of values for $\xi'$ is $O(1)$
  (as $|\xi - \xi'| \le 1$). Since
  \[
    -\epsilon < \tau - \frac{1}{2} |\xi|^2 < \epsilon
    \implies -\epsilon + \frac{1}{2} |\xi|^2 < \tau < \epsilon + \frac{1}{2} |\xi|^2
  \]
  and $|\xi - v| \le O(1)$, we get
  \[
    -\epsilon + \frac{1}{2} ||v| - 1|^2
    < \tau < \epsilon + \frac{1}{2} ||v| + 1|^2.
  \]
  Expanding $||v| - 1|^2$ and $||v| + 1|^2$, we see
  that as $\epsilon \to 0$,
  we have $|2\tau - |v|^2| \lesssim |v| + 1$. Then
  \begin{align*}
    &\int_{\R} \int_{\R^d} \left| \int_{\sqrt{2\tau} \mathbb{S}^{d - 1}} \widehat{\chi}(\xi - \xi') e^{-|\xi - v|^2 / 2} |\xi|^{1 / 2} \frac{1}{|\xi| + \sqrt{2\tau}} \, d\sigma(\xi)\right|^2\, d\xi' d\tau \\
    & \quad \quad \lesssim \int_{\R} \left|\int_{\sqrt{2\tau} \mathbb{S}^{d - 1}} e^{-|\xi - v|^2 / 2} |\xi|^{1 / 2} \frac{1}{|\xi| + \sqrt{2\tau}}\, d\sigma(\xi)\right|^2\, d\tau \\
    & \quad \quad \lesssim \int_{\R} \left|\frac{|\sqrt{\tau}|^{1 / 2}}{\sqrt{2\tau}}\right|^2\, d\tau
    \approx \int_{|2\tau - |v|^2| \lesssim |v| + 1}
    \left|\frac{|\sqrt{\tau}|^{1 / 2}}{\sqrt{2\tau}}\right|^2\, d\tau
    \approx \frac{1}{v} \int_{|2\tau - |v|^2| \lesssim |v| + 1}\, d\tau
    \approx \frac{1}{v} \cdot v \approx O(1)
  \end{align*}
  uniformly in $v$. We can also see that if
  $|\xi|^{1 / 2}$ is replaced by $|\xi|^{1 / 2 + \delta}$,
  then $1 / |v|$ will become $1 / |v|^{1 - 4 \delta}$,
  so the above becomes $O(|v|^{4\delta})$, which
  goes to $\infty$ as $v \to \infty$.
\end{proof}

\begin{corollary}
  When $d \ge 2$, for $\epsilon > 0$, we have
  \[
    \|(1 + |x|)^{-1 / 2 -\epsilon} (-\Delta)^{1 / 4} e^{it\Delta / 2} f\|_{L^2_{x, t}} \le C_\epsilon \|f\|_{L^2_x}.
  \]
  This is the sharp Kato smoothing estimate for
  $d \ge 2$.
\end{corollary}

\section{Schr\"odinger Equation with Potential}
\begin{remark}
  Now we will consider the Schr\"odinger equation with
  a potential:
  \[
  \begin{cases}
    i\psi_t + \frac{1}{2} \Delta \psi + V(t, x) \psi = 0, \\
    \psi|_{t = 0} = \psi_0 \in H^\gamma(\R^d).
  \end{cases}
  \]
\end{remark}

\begin{lemma}
  If $\psi_0 \in H^\gamma(\R^d)$, then
  $e^{it\Delta / 2} \psi_0 \in C^0(\R, H^\gamma(\R^d)) \cap C^1(\R, H^{\gamma - 2}(\R^d))$.
\end{lemma}

\begin{proof}
  For each $t, s \in \R$, we can write (since
  $e^{is\Delta / 2}$ preserves each $H^\gamma$ norm)
  \[
    \|e^{it\Delta / 2} \psi_0 - e^{is\Delta / 2} \psi_0\|_{H^\gamma}^2
    = \|e^{i(t - s)\Delta} \psi_0 - \psi_0\|_{H^\gamma}^2
    = \int (1 + |\xi|^2)^{\gamma} |e^{i(t - s)|\xi|^2 / 2} - 1| |\widehat{\psi}_0(\xi)|^2\, d\xi.
  \]
  Since $\psi_0 \in H^\gamma(\R^d)$,
  as $t \to s$, the above quantity goes to $0$
  by the dominated convergence theorem. This shows that
  $e^{it\Delta / 2 \psi_0} \in C^0(\R, H^\gamma(\R^d)$.
  To check that $e^{it\Delta / 2} \psi_0 \in C^1(\R, H^{\gamma - 2}(\R^d))$, note
  \[
    \frac{d}{dt} e^{it\Delta / 2} \psi_0
    = \frac{i}{2} \Delta e^{it\Delta / 2} \psi_0.
  \]
  Thus it suffices to see that $\Delta e^{it\Delta / 2} \psi_0$ is continuous, i.e.
  \[
    \|\Delta e^{it / 2} \psi_0 - \Delta e^{is / 2} \psi_0\|_{H^{\gamma - 2}}^2 \to 0
  \]
  as $t \to s$, which follows by the dominated
  convergence theorem.
\end{proof}

\begin{theorem}
  Assume $V$ is real and that
  \[
    \sup_{t} \|\nabla_x^\alpha V(t, x) \|_{L^2_x} \le C_\alpha
    \quad \text{and} \quad
    \|\nabla_x^\alpha V(t, x) - \nabla_x^\alpha V(s, x)\|_{L^\infty_x} \to 0
  \]
  as $t \to s$ for every multi-index $\alpha$.
  Then for all $k \ge 0$, if $\psi_0 \in H^k(\R^d)$,
  there exists a unique solution
  $\psi \in C^0(\R, H^k(\R^d)) \cap C^1(\R, H^{k - 2}(\R^d))$
  to the Schr\"odinger equation with potential $V$,
  given by
  \[
    \psi(t) = e^{it\Delta / 2} \psi_0 - i \int_0^t e^{i(t - s)\Delta / 2} V(s) \psi(s)\, ds.
  \]
  Moreover, $\|\psi\|_{H^k(\R^d)} \le C_{k, V} (1 + |t|)^k$.
\end{theorem}

\begin{proof}
  Define the operator $A$ by the formula
  \[
    A(\phi) = e^{it \Delta / 2} \psi_0 - i\int_0^t e^{i(t - s)\Delta / 2} V(s) \phi(s)\, ds.
  \]
  Let $X = (C^0[0, T], H^k)$, and we want to find
  $T$ such that $A$ is a contraction in $X$ (do this
  as an exercise). Repeating this
  on $[T, 2T]$, $[2T, 3T]$, \dots, gives
  existence. Uniqueness follows by Gronwall's inequality.

  Now we prove the upper bound for the growth.
  Gronwall's inequality gives us
  \[
    \|\psi(t)\|_{H^k} \le c_1 e^{c_2|t|} \|\psi_0\|_{H^k}.
  \]
  Since $V$ is real (the argument before this step
  works even when $V$ is not real),
  \begin{align*}
    \frac{d}{dt} \|\psi(t)\|_{L^2}^2
    &= \frac{d}{dt} \langle \psi(t), \psi(t) \rangle
    = \frac{d}{dt}
    = \langle \frac{d}{dt} \psi, \psi \rangle
    + \langle \psi, \frac{d}{dt} \psi \rangle
    = 2 \re \langle \frac{d}{dt} \psi, \psi \rangle \\
    &= 2 \re \langle \frac{i}{2} \Delta \psi + iV(t, x) \psi, \psi \rangle
    = 2\re i \int \left(-\frac{1}{2} |\nabla \phi|^2 + V(t, x)|\psi|^2\right)\, dx = 0
  \end{align*}
  since the integral is real, so that $i$ times the
  integral is purely imaginary. So
  $\|\psi\|_{L^2}$ is conserved.
  Now let $\phi = \partial_{x_j} \psi(t, x)$, so that
  \[
    i\phi_t + \frac{1}{2} \Delta \phi + V \phi = -\partial_{x_j} V\psi. \tag{$*$}
  \]
  From the existence and uniqueness of the solution,
  we can use $V(t, 0) \psi_0$ to denote the solution
  to
  \[
    \begin{cases}
      i\partial_t + \frac{1}{2} \Delta \psi + V \psi = 0, \\
      \psi|_{t = 0} = \psi_0,
    \end{cases}
  \]
  and use $V(t, s) \psi_0$ denote the solution
  for initial condition $\psi(s) = \psi_0$. Then
  $(*)$ can be solved as
  \[
    \phi(t) = V(t, 0) \partial_{x_j} \psi_0
    + \int_0^t V(t, s) \partial_{x_j} V(s) \psi(s)\, ds.
  \]
  This gives
  \[
    \|\phi(t)\|_{L^2} \le
    \|V(t, 0) \partial_{x_j} \psi_0\|_{L^2}
    + \int_0^t \|V(t, s) \partial_{x_j} V(s) \psi(s)\|_{L^2}\, ds
  \]
  by Minkowski's inequality.  By the
  conservation of the $L^2$ norm,
  $\|V(t, 0) \partial_{x_j} \psi_0\|_{L^2} = \|\partial_{x_j} \psi_0 \|_{L^2}$, so
  \[
    \|V(t, s) \partial_{x_j} V(s) \psi(s)\|_{L^2}
    = \|\partial_{x_j} V(s) \psi(s)\|_{L^2}
    \lesssim \|\psi(s)\|_{L^2} = \|\psi(0)\|_{L^2}.
  \]
  Then we obtain
  \[
    \|\phi\|_{L^2} \lesssim
    \|\partial_{x_j} \psi_0\|_{L^2}
    + \int_0^t \|\psi(s)\|_{L^2}\, ds
    \le C_{v, 1} (1 + |t|) \|\psi(0)\|_{H^1},
  \]
  and inducting on $k$ gives the desired result.
\end{proof}

\begin{remark}
  Note that in the above theorem, we need a suitable
  integral (e.g. the Bochner integral) to be able to
  interpret the Duhamel formula in a Banach space.
\end{remark}

  \chapter{Feb.~5 --- Schr\"odinger Equation with Potential}

\section{Schr\"odinger Equation with Potential, Continued}

\begin{theorem}[Bourgain]
  Assume $d \ge 3$ and that $V$ is real. Suppose that
  \[
    \sup_{t} \| \nabla_x^\alpha V(t, x) \|_{L^\infty_x} \le C_\alpha
  \]
  for every multi-index $\alpha$ and that
  $\sup_{|t - t_0| \le 1} |V(t, x)|$ is compactly
  supported in $x$ with the diameter of the support
  independent of $t_0$ (e.g. $V(t, x) = V(x - ct)$).
  Then
  \[
    \|\psi(t)\|_{H^k} \le C_\epsilon (1 + |t|)^{\epsilon} \|\psi_0\|_{H^k}
  \]
  for every $\epsilon > 0$.
\end{theorem}

\begin{proof}
  Define (note that $\seminorm{f} = \|f\|_{L^\infty + L^2}$, the dual space of $\|g\|_{L^1 \cap  L^2}$)
  \[
    \seminorm{f} = \inf_{f = f_1 + f_2} (\|f_1\|_{L^2} + \|f_2\|_{L^\infty}).
  \]
  Then we can write
  \[
    \psi(t) = e^{it\Delta / 2} \psi_0 + i \int_0^t e^{i(t - s)\Delta / 2} V(s) \psi(s)\, ds.
  \]
  Taking the derivative and splitting the integral, we obtain
  \[
    \nabla^\alpha \psi(t) = e^{it \Delta / 2} \nabla^\alpha \psi_0
    + i \int_0^{t - A} e^{i(t - s)\Delta / 2} \nabla^\alpha (V\psi) \, ds
    + i \int_{t - A}^t e^{i(t - s)\Delta / 2} \nabla^\alpha (V\psi) \, ds,
  \]
  where $A > 0$ is a constant to be chosen later.
  We will measure the first integral in $L^\infty$ and
  the second integral in $L^2$. Taking norms, we
  find that
  \[
    \seminorm{\nabla^\alpha \psi(t)}
    \le \seminorm{e^{it\Delta / 2} \nabla^\alpha \psi_0}
    + \left\|\int_0^{t - A} e^{i(t - s)\Delta / 2} \nabla^\alpha (V\psi) \, ds\right\|_{L^\infty}
    + \left\|\int_{t - A}^t e^{i(t - s)\Delta / 2} \nabla^\alpha (V\psi) \, ds\right\|_{L^2}.
  \]
  We can estimate the first integral by
  \begin{align*}
    \left\|\int_0^{t - A} e^{i(t - s) \Delta / 2} \nabla^\alpha(V \psi)\, ds \right\|_{L^\infty}
    &\le \int_0^{t - A} \left\|e^{i(t - s)\Delta / 2} \nabla^\alpha(V \psi)\right\|_{L^\infty}\, ds \\
    &\lesssim \int_0^{t - A} |t - s|^{-d / 2} \|\nabla^\alpha(V \psi)\|_{L^1}\, ds
    \lesssim A^{-d / 2 + 1} \sup_{0 \le s \le t} \|\nabla^\alpha(V \psi)\|_{L^1},
  \end{align*}
  where the first inequality is by Minkowski's
  inequality, the second inequality is by the dispersive
  estimate, and the third is by explicit integration.
  Note that we needed the cutoff at $t - A$ here since
  $|t - s|^{-d / 2}$ is not integrable at $t = s$.
  Then by the product rule and Cauchy-Schwarz, we have
  \[
    \left\|\int_0^{t - A} e^{i(t - s) \Delta / 2} \nabla^\alpha(V \psi)\, ds \right\|_{L^\infty}
    \lesssim A^{-d / 2 + 1} \sup_{\substack{0 \le \beta \le \alpha \\ 0 \le s \le t}} \|\nabla^{\beta} V \nabla^{\alpha - \beta} \psi\|_{L^1}
    \lesssim A^{-d / 2 + 1} \sup_{\substack{0 \le \beta \le \alpha \\ 0 \le s \le t}} \|\nabla^{\beta} \psi\|_{L^2}.
  \]
  Note here that $d \ge 3$ implies $-d / 2 + 1 < 0$,
  so if $A$ is large, then $A^{-d / 2 + 1}$ is small.
  The above implies
  \[
    \left\|\int_0^{t - A} e^{i(t - s) \Delta / 2} \nabla^\alpha(V \psi)\, ds \right\|_{L^\infty}
    \lesssim A^{-d / 2 + 1} \sup_{\substack{0 \le \beta \le \alpha \\ 0 \le s \le t}} \seminorm{\nabla^{\beta} \psi}.
  \]
  For the second integral, we can write
  \begin{align*}
    \left\| \int_{t - A}^t e^{i(t - s) \Delta / 2} \nabla^\alpha (V \psi)\, ds \right\|_{L^2}
    &= \sup_{\|\phi\|_{L^2} = 1} \int_{t - A}^t \langle \phi, e^{i(t - s) \Delta / 2} \nabla^\alpha (V \psi) \rangle\, ds \\
    &= \sup_{\|\phi\|_{L^2 = 1}} \int_{t - A}^t
    \langle \phi e^{i(t - s) \Delta / 2} \chi \nabla^\alpha (V \psi) \rangle\, ds,
  \end{align*}
  where $\chi$ is a smooth cutoff function such that
  $\chi \equiv 1$ in the support of $V$. Then
  \[
    \left\| \int_{t - A}^t e^{i(t - s) \Delta / 2} \nabla^\alpha (V \psi)\, ds \right\|_{L^2}
    = \sup_{\|\phi\|_{L^2 = 1}}
    \int_{t - A}^t \langle (1 - \Delta)^{1 / 4} \chi e^{-i(t - s)\Delta / 2} \phi, (1 - \Delta)^{-1 / 4} \nabla^{\alpha} (V \psi) \rangle\, ds.
  \]
  By Cauchy-Schwarz in $x$ and $t$ and Kato
  $1 / 2$ smoothing, we obtain
  \begin{align*}
    &\left\| \int_{t - A}^t e^{i(t - s) \Delta / 2} \nabla^\alpha (V \psi)\, ds \right\|_{L^2} \\
    & \quad \quad \le \sup_{\|\phi\|_{L^2 = 1}}
    \left(\int_{t - A}^t \|(1 - \Delta)^{1 / 4} \chi e^{-i(t - s) \Delta / 2} \phi\|_{L^2_x}^2 \, ds\right)^{1 / 2} 
    \left(\int_{t - A}^t \|(1 - \Delta)^{-1 / 4} \nabla^\alpha(V\psi)\|_{L^2_x}^2 \, ds\right)^{1 / 2} \\
    & \quad \quad \lesssim \|\phi\|_{L^2}
    + \sqrt{A} \sup_{t - A \le s \le t} \|(1 - \Delta)^{-1 / 4} \nabla^\alpha(V\psi)\|_{L^2_x}.
  \end{align*}
  To deal with the last term, note that by Sobolev's
  inequality, we have
  \[
    \|f\|_{H^{k - 1 / 2}} \le C_k \|f\|_{H^k}^{1 - \delta} \|f\|_{L^2}^{\delta}
  \]
  for some $\delta = \delta(k)$. From this we get
  \[
    \sup_{|\alpha| = k}
    \|(1 - \Delta)^{-1 / 4} \nabla^\alpha(V\psi)\|_{L^2}
    \le C_k \|V\psi\|_{L^2}^\delta \sup_{|\beta| \le k} \|\nabla^\beta (V\psi)\|_{L^2}^{1 - \delta}
    \le C_k \|\psi_0\|_{L^2}^\delta \sup_{|\beta| \le k} \seminorm{\nabla^\beta (V\psi)}^{1 - \delta},
  \]
  where the second inequality is since the $L^2$
  norm is conserved. This means that
  \[
    \seminorm{\nabla^\alpha \psi}
    \lesssim \|\nabla^\alpha \psi_0\|_{L^2}
    + A^{1 - d / 2} \sup_{\substack{0 \le |\beta| \le |\alpha| \\ 0 \le s \le t}} \seminorm{\nabla^\beta \psi}
    + \sqrt{A} \cdot \|\psi_0\|_{L^2}^\delta \sup_{\substack{0 \le |\beta| \le |\alpha| \\ 0 \le s \le t}} \seminorm{\nabla^\beta \psi}^{1 - \delta}.
  \]
  For the last term, observe that
  \[
    x^{1 - \delta} y^\delta \le \frac{1}{1 - \delta} x + \frac{1}{\delta} y
  \]
  by Young's inequality, and we can also use that
  for $\eta > 0$,
  \[
  \left(\eta^{1 / (1 - \delta)} x\right)^{1 - \delta}
  \left(n^{-1 / \delta} y\right)^\delta
  \le \frac{\eta^{1 / (1 - \delta)}}{1 - \delta} x + 
  \frac{n^{-1 / \delta}}{\delta} y.
  \]
  Applying this to the last term, we obtain
  \[
    \|\psi_0\|_{L^2}^\delta \sup_{\substack{0 \le |\beta| \le |\alpha| \\ 0 \le s \le t}} \seminorm{\nabla^\beta \psi}^{1 - \delta}
    \le \frac{\eta^{-1 / \delta}}{\delta} \|\psi_0\|_{L^2}
    + \frac{\eta^{1 / (1 - \delta)}}{1 - \delta}
    \sup_{\substack{0 \le |\beta| \le |\alpha| \\ 0 \le s \le t}} \seminorm{\nabla^\beta \psi}.
  \]
  Putting everything together, we obtain
  \[
    \sup_{\substack{|\alpha| = k \\ 0 \le s \le t}}
    \seminorm{\nabla^\alpha \psi}
    \le \|\psi_0\|_{H^k} + A^{1 - d / 2} \sup_{\substack{0 \le s \le t \\ |\alpha| \le k}} \seminorm{\psi}
    + \frac{\sqrt{A} \cdot \eta^{-1 / \delta}}{\delta} \|\psi_0\|_{L^2}
    + \frac{\sqrt{A} \cdot \eta^{1 / (1 - \delta)}}{1 - \delta} \sup_{\substack{0 \le s \le t \\ |\alpha| \le k}} \seminorm{\psi}.
  \]
  Now take $A$ large, so that $A^{1 - d / 2}$ is very
  small (note that since $A \le t$, we also take
  $t$ to be large enough; for $t$ small,
  apply Gronwall). Take $\eta$ small, so that $\sqrt{A} \cdot \eta^{1 / (1 - \delta)}$
  is very small. Then we have
  \[
    \sup_{\substack{|\alpha| = k \\ 0 \le s \le t}}
    \seminorm{\nabla^\alpha \psi}
    \le C_{\alpha, \eta} \|\psi_0\|_{H^k}.
  \]
  Now recall that (since the Schr\"odinger evolution
  conserves the $H^k$ norms)
  \[
    \|\psi(t)\|_{H^k}
    \le \|\psi_0\|_{H^k} + \int_0^t \|V(s) \psi(s)\|_{H^k}\, ds.
  \]
  The above calculations show that
  \[
    \|V\psi\|_{H^k} \lesssim
    \sup_{|\alpha| \le k} \|\nabla^\alpha (V\psi)\|_{L^2}
    \lesssim \sup_{|\alpha| \le k} \seminorm{\nabla^\alpha \psi}
    \lesssim \|\psi_0\|_{H^k},
  \]
  and so we have
  \[
    \|\psi(t)\|_{H^k}
    \lesssim \|\psi_0\|_{H^k} + \int_0^t \|\psi_0 \|_{H^k}\, ds
    \lesssim (1 + |t|) \|\psi_0\|_{H^k}.
  \]
  Note that the growth of the norm is independent of
  $k$.
  To get $(1 + |t|)^\epsilon$, note that the
  $L^2$ norm is conserved. So for any $k$, take
  $k_1$ to be large enough such that $k / k_1 < \epsilon$.
  Then
  \begin{align*}
    \|\psi\|_{H^k}
    \lesssim \|\psi\|^{k / k_1}_{H^{k_1}} \|\psi\|_{L^2}^{1 - k / k_1}
    &\lesssim (1 + |t|)^{k / k_1} \|\psi_0\|_{H^{k_1}}^{k / k_1} \|\psi_0\|_{L^2}^{1 - k / k_1} \\
    &\le C_\epsilon (1 + |t|)^{k / k_1} \|\psi_0\|_{H^k}
    \le C_\epsilon (1 + |t|)^\epsilon \|\psi_0\|_{H^k},
  \end{align*}
  which is the desired result.
\end{proof}

\begin{remark}
  The compact support assumption could be replaced
  by rapid decay in the above theorem.
\end{remark}

  \chapter{Feb.~10 --- Spectral Theory}

\section{Review of Functional Analysis}

\begin{remark}
  Our goal now is to study the decay of a solution $\psi$ to
  \[
    i\psi_t - \frac{1}{2} \Delta \psi + V(x) \psi = 0, \quad \text{$V$ real}.
  \]
  We will study $e^{iHt} \psi_0$, where
  $H = -\Delta / 2 + V$, in particular the
  spectral theory for $H$ in $L^2(\R^d)$.
\end{remark}

\begin{definition}
  Let $\mathcal{H}$ be a Hilbert space, with inner product
  $\langle \cdot, \cdot \rangle$. We will
  use $\mathcal{H} = L^2(\R^d)$ with
  \[
    \langle f, g \rangle
    = \int_{\R^d} f \overline{g}\, dx.
  \]
  Let $T$ be a linear operator which is densely defined
  on $\mathcal{H}$.
  Let $D(T)$ denote the domain of $T$. We say that
  $T$ is \emph{symmetric} if
  $\langle Tx, y \rangle = \langle x, Ty \rangle$.
  The \emph{adjoint} $T^*$ of $T$ is defined as
  \[
    D(T^*) =
    \{
      f \in \mathcal{H} : \text{there exists $h \in \mathcal{H}$ such that } \langle T \phi, f \rangle = \langle \phi, h \rangle \text{ for all $\phi \in D(T)$}
    \},
  \]
  and we set $T^* f = h$. We say
  $T$ is \emph{closed} provided that the graph
  of $T$ is a closed subset of $\mathcal{H} \times \mathcal{H}$,
  i.e. if $f_n \to f$ in $\mathcal{H}$ with $f_n \in D(T)$
  and $Tf_n \to g$ in $\mathcal{H}$, then
  $f \in D(T)$ and $Tf = g$.
\end{definition}

\begin{definition}
  We say that a linear operator $T$ is \emph{self-adjoint} if
  $(T, D(T)) = (T^*, D(T^*))$.
\end{definition}

\begin{example}
  Let $T = i \partial_x$. For $\mathcal{H} = L^2(\R)$ and
  $D(T) = C^1_c(\R)$, one can check that $T$ is self-adjoint.
  But if we take $\mathcal{H} = L^2([0, \infty))$ and
  $D(T) = C^1_c([0, \infty))$, then $T$ is not self-adjoint
  (but $T$ is symmetric).
\end{example}

\begin{example}
  Let $T = - \partial_x^2$
  on $L^2([0, 1])$. Then for
  \[
    D(T) = \{f \in H^2([0, 1]) : f(0) = f(1) = f'(0) = f'(1) = 0\},
  \]
  $T$ is symmetric but not self-adjoint. One can check that
  $D(T^*) = H^2([0, 1])$.
\end{example}

\begin{lemma}
  Let $T$ be a densely defined linear operator on
  $\mathcal{H}$. Then:
  \begin{enumerate}
    \item If $T$ is symmetric, then
      $D(T^*) \supseteq D(T)$ (so $T^*$ is 
      densely defined) and all eigenvalues of $T$ are
      real.
    \item For every $T$, its adjoint $T^*$ is closed.
    \item We have $(\ker T)^\perp = \overline{\Ran(T^*)}$
      and $(\ker T^*)^\perp = \overline{\Ran(T)}$.
    \item If $T$ is self-adjoint, i.e. $(T, D(T)) = (T^*, D(T^*))$, then \[\Spec T = \C \setminus \{z \in \C : \text{$(T - z)^{-1}$ exists as a bounded linear operator}\} \subseteq \R\]
      and $\|(T - z)^{-1}\|_{\mathcal{L}(\mathcal{H}, \mathcal{H})} \lesssim |{\im z}|^{-1}$.
  \end{enumerate}
\end{lemma}

\begin{proof}
  (1) One can check this by definition.

  (2) Take a sequence $f_n \in D(T^*)$ with
  $f_n \to f$, and suppose that $T^* f_n \to g$. By
  definition,
  \[
    \langle T \phi, f_n \rangle
    = \langle \phi, T^* f_n \rangle
    \quad \text{for every $\phi \in D(T)$}.
  \]
  Then $f_n \to f$ implies that
  $\langle T \phi, f_n \rangle \to \langle T \phi, f \rangle$,
  and $T^* f_n \to g$ implies that $\langle \phi, T^* f_n \rangle \to \langle \phi, g \rangle$.
  Then we have
  $\langle T \phi, f \rangle = \langle \phi, g \rangle$
  for every $\phi \in D(T)$, so
  $f \in D(T^*)$ and $T^* f = g$ by definition.

  (3) Note that in general, if $A \subseteq \mathcal{H}$, then
  $A^\perp$ is closed. To see this, let $f_n \in A^\perp$
  and $f_n \to f$ in $\mathcal{H}$. Then
  $\langle f_n, a \rangle = 0$ for every $a \in A$
  since $f_n \in A^\perp$. So letting
  $n \to \infty$, we have $\langle f, a \rangle = 0$,
  i.e. $f \in A^\perp$.

  We first show that $\overline{\Ran(T)} \subseteq (\ker T^*)^\perp$.
  For any $h \in \Ran(T)$, there exists $f$ such that
  $h = Tf$. Then for any $g \in \ker T^* \subseteq D(T^*)$,
  we have
  \[
    \langle Tf, g \rangle
    = \langle f, T^* g \rangle = 0,
  \]
  so $f \in (\ker T^*)^\perp$.
  Thus $\Ran(T) \subseteq (\ker T^*)^\perp$, and taking
  closures,
  $\overline{\Ran(T)} \subseteq \overline{(\ker T^*)^\perp} = (\ker T^*)^\perp$.

  Now we show that $(\ker T^*)^\perp \subseteq \overline{\Ran(T)}$.
  Let $f \in (\overline{\Ran(T)})^\perp$, so
  $f \in (\Ran(T))^\perp$. So
  \[
    \langle T^* f, g \rangle = \langle f, Tg \rangle = 0, \quad \text{for all $g \in D(T)$},
  \]
  so $T^* f = 0$. This gives $f \in \ker(T^*)$.
  So $(\overline{\Ran(T)})^\perp \subseteq \ker(T^*)$.
  Then taking orthogonal complements,
  \[
    (\ker T^*)^\perp \subseteq ((\overline{\Ran(T)})^\perp)^\perp
    = \overline{\Ran(T)}
  \]
  since $\overline{\Ran(T)}$ is closed. This
  gives $(\ker T^*)^\perp = \overline{\Ran(T)}$, and
  the other equality follows by duality.

  (4) Let $T = T^*$ and $z = x + iy$. Then we can write
  \[
    |\langle (T - z) f, f \rangle|
    = |\langle (T - (x + iy)) f, f \rangle|
    = |\langle (T - x) f, f \rangle
    - iy \langle f, f \rangle|
    \ge |y| \|f\|_{\mathcal{H}}^2,
  \]
  since $\langle (T - x) f, f \rangle$ is real and
  $iy \langle f, f \rangle$ is purely imaginary. On the
  other hand,
  \[
    \langle (T - z) f, f \rangle
    \le \|f\|_{\mathcal{H}} \|(T - z) f\|_{\mathcal{H}},
  \]
  by Cauchy-Schwarz, so $\|(T - z) f\|_{\mathcal{H}} \ge |y| \|f\|_{\mathcal{H}}$
  where $y = \im Z$. So $\ker(T - z) = 0$ if $|y| \ne 0$.
  Now
  \[
    \overline{\Ran(T - z)} = (\ker (T^* - \overline{z}))^\perp
    = (\ker (T - \overline{z}))^\perp
    = \mathcal{H}
  \]
  if $|y| \ne 0$ by $(3)$, so we just need to show that
  $\Ran(T - z)$ is closed. Assume $(T - z) f_n \to g$.
  Note that
  \[
    \|(T - z) f_n - (T - z) f_m\|_{\mathcal{H}}
    \ge |y| \|f_n - f_m \|_{\mathcal{H}},
  \]
  so $(T - z) f_n$  being Cauchy implies that
  $\{f_n\}$ is also Cauchy. So $f_n \to f$ for some
  $f \in \mathcal{H}$, i.e. $T = T^*$ is closed.
  This then implies that $T - z$ is closed, from which it
  follows that $f \in D(T - z)$ and $(T - z)f = g$, so
  $\Ran(T - z)$ is closed if $|y| \ne 0$. So
  $\ker(T - z) = \{0\}$ and $\Ran(T - z) = \mathcal{H}$
  if $|y| \ne 0$. So $T - z$ is invertible, hence
  $\Spec T \subseteq \R$ and
  $\|(T - z)^{-1}\|_{\mathcal{L}(\mathcal{H}, \mathcal{H})} \lesssim |{\im z}|^{-1}$.
\end{proof}

\section{Spectral Theory}

\begin{definition}
  The \emph{resolvent} of $T$, denoted $\rho(T) \subseteq \C$
  is defined as follows: We
  have $\lambda \in \rho(T)$ if we can find a bounded
  linear operator $R(\lambda)$ from $\mathcal{H}$ to
  $\mathcal{H}$ such that
  \[
    (T - \lambda) R(\lambda) = R(\lambda)(T - \lambda) = I.
  \]
  The \emph{spectrum} of $T$ is $\sigma(T) = \C \setminus \rho(T)$.
\end{definition}

\begin{definition}
  Let $\mathcal{B}$ be the Borel sets in $\R$.
  A \emph{spectral measure} $E$ on $\R$ is an orthogonal
  projection valued measure on $\mathcal{B}$, i.e.
  a map $E$ such that $E(A)$ is an orthogonal projection
  for every $A \in \mathcal{B}$, and
  \begin{enumerate}
    \item $E(\varnothing) = 0$ and $E(\R) = I$;
    \item for $A, B \in \mathcal{B}$, we have
      $E(A) E(B) = E(A \cap B)$;
    \item for disjoint $A_j \in \mathcal{B}$, we have
      $E(\bigcup_{j = 1}^\infty A_j) = \sum_{j = 1}^\infty E(A_j)$.
  \end{enumerate}
\end{definition}

\begin{remark}
  Given a spectral measure $E$, for $x, y \in \mathcal{H}$,
  we can define
  \[
    E_{x, y}(A) = \langle E(A) x, y \rangle \quad
    \text{for $A \in \mathcal{B}$}.
  \]
  For fixed $x, y$,
  $E_{x, y}$ is a complex-valued Borel measure. In particular,
  $E_{x, x}$ is a positive Borel measure.
\end{remark}

\begin{theorem}[Spectral theorem]
  Let $(T, D(T))$ be self-adjoint on $\mathcal{H}$. Then
  there is a unique spectral measure $E$ such that
  \[
    \langle Tx, y \rangle
    = \int \lambda \, dE_{x, y}(\lambda)
    \quad \text{for all $x \in D(T)$, $y \in \mathcal{H}$}.
  \]
\end{theorem}

\begin{remark}
  We make the following remarks:
  \begin{enumerate}
    \item We have $\supp E_{x, y} \subseteq \sigma(T)$.
    \item If $f$ is a Borel measurable function, then
      we can interpret
      \[
        \langle f(T) x, y \rangle
        = \int f(\lambda) \, dE_{x, y}(\lambda).
      \]
      In particular, we have a map
      $f \mapsto f(T)$.
  \end{enumerate}
\end{remark}

  \chapter{Feb.~12 --- Spectral Theory, Part 2}

\section{Examples for Spectral Theory}

\begin{example}
  Let $\mathcal{H} = \R^n$ with the standard inner product, and
  let $T$ be a Hermitian matrix. Then $T$ is self-adjoint, and
  the corresponding spectral measure $E$ is given by
  \[
    E(I) = \sum_{j : \lambda_j \in I} P_{L_j},
  \]
  where $\lambda_1 \le \lambda_2 \le \dots \le \lambda_k$ are 
  the eigenvalues of $T$ without multiplicity and
  $L_1, \dots, L_k$ are eigenspaces.
\end{example}

\begin{example}
  Let $T$ be a self-adjoint operator. Then
  \[
    e^{itA} f = \left(\int e^{it\lambda)}\, E(d\lambda)\right) f.
  \]
  From this we have
  \[
    \|e^{itA} f\|_{L^2}^2
    = \langle \int e^{it\lambda}\, E(d\lambda) f, \int e^{it\lambda}\, E(d\lambda) f\rangle
    = \iint e^{it \lambda} e^{-it\mu}\, \langle E(d\lambda) E(d\mu) f, f \rangle.
  \]
  Fixing $\lambda$ and integrating over $\mu$, the
  only point that survives is $\mu = \lambda$, so
  \[
    \|e^{itA} f\|_{L^2}^2
    = \int \langle Ed(\lambda) f, f \rangle
    = \langle E(\R) f, f \rangle
    = \|f\|_{L^2}^2.
  \]
  since $E(\R) = I$. Thus if $T$ is self-adjoint, then
  $\|e^{itT} f\|_{L^2} = \|f\|_{L^2}$, i.e. $e^{itT}$ is
  unitary. Also,
  \[
    e^{itT} e^{isT} f = e^{i(t+s)T} f \quad \text{and} \quad
    e^{-itT} e^{itT} = I = e^{itT} e^{-itT}.
  \]
  The above are group properties for $e^{itT}$ as
  $t$ varies.
\end{example}

\begin{example}
  Let $\mathcal{H} = L^2(X, \mu)$, where $\mu$ is a positive
  probability measure. Define $T = M_\phi$ to be the
  multiplication operator by $\phi$, i.e.
  $Tf = \phi f$ for $\phi : X \to \R$. We have
  \[
    D(T) = \{f \in L^2 : \phi f \in L^2\},
  \]
  which is dense in $\mathcal{H}$ since
  $\mu(|\phi| > M) \to 0$ as $M \to \infty$. So $T$ is
  densely defined. Then
  \[
    \langle Tf, g \rangle
    = \int \phi f \overline{g}\, \mu(dx)
    = \int f \overline{\phi g}\, \mu(dx)
    = \langle f, Tg \rangle
  \]
  since $\phi$ is real. So $T$ is symmetric, which implies
  $D(T) \subseteq D(T^*)$. To get the
  reverse inclusion, notice that for $g \in D(T^*)$, we have
  $\langle Tf, g \rangle = \langle f, h \rangle$ for
  some $h \in \mathcal{H}$, for all $f \in D(T)$. Then
  \[
    \int f \overline{\phi g}\, \mu(dx)
    \int \phi f \overline{g}\, \mu(dx)
    = \langle Tf, g \rangle
    = \langle f, h \rangle
    = \int f\overline{h} \mu(dx).
  \]
  Since $D(T)$ is dense, we get $\phi g = h \in \mathcal{H} = L^2(X, \mu)$, so
  $g \in D(T)$.
  This gives $D(T^*) \subseteq D(T)$, so we see that
  $D(T^*) = D(T)$ and $T$ is self-adjoint.
  The spectral theorem then gives the spectral resolution
  $E$:
  \[
    E(I) f = \chi_{\{\phi \in I\}} f,
  \]
  where $\chi$ denotes an indicator function.
\end{example}

\begin{example}
  Let $T = -\Delta$ on $L^2(\R^d)$. Clearly
  $D(T) = H^2(\R^d)$ is dense in $L^2(\R^d)$. Integration
  by parts shows that $-\Delta$ is symmetric.
  Now suppose $g \in D(T^*)$. Then
  \[
    \langle -\Delta f, g \rangle
    = \langle f, h \rangle \quad \text{for all $f \in D(T)$ and some $h \in L^2$}.
  \]
  Since
  $\langle -\Delta f, g \rangle = \langle f, -\Delta g \rangle = L f$
  (note that $-\Delta g \in \mathcal{S}'$, i.e. in the
  sense of distributions), we have
  \[
    |Lf| \le \|f \|_{L^2} \| h \|_{L^2}
    = C \|f \|_{L^2},
  \]
  where $C = \| h \|_{L^2}$. So by the Hahn-Banach
  theorem, $L$ can be extended to a continuous functional
  on $L^2$ such that $\|L\|_{B(L^2, L^2)} \le \|h \|_{L^2}$.
  We can identify  $Lf = \langle f, -\Delta g \rangle$, and thus
  \[
    \|\Delta g\|_{L^2} \le \|h \|_{L^2}.
  \]
  This implies $\Delta g \in L^2$, so by the elliptic theory,
  $g \in H^2 = D(T)$. This gives
  $D(T^*) \subseteq D(T)$ and thus $D(T) = D(T^*)$ (since
  $D(T) \subseteq D(T^*)$ because $T$ is symmetric), so
  $T$ is self-adjoint. To understand the spectral
  resolution, we can use the Fourier transform:
  $\widehat{Tf} = \xi^2 \widehat{f}$. Writing
  $\mathcal{F}(f) = \widehat{f}$, we have
  \[
    \mathcal{F} T \mathcal{F}^{-1} = M_{\xi^2}.
  \]
  Note that $\mathcal{F}$ is unitary on $L^2_x \to L^2_\xi$ by
  Plancherel's theorem. This says that $\mathcal{F}$
  diagonalizes $\Delta$. Letting $E_T$ be the
  spectral resolution for $T$ and $E_{\xi^2}$ be the
  spectral resolutoin for $M_{\xi^2}$, we have
  \[
    \mathcal{F} E_T \mathcal{F}^{-1} = E_{\xi^2}.
  \]
  Since $E_{\xi}^2(I) = \chi_{\{\xi^2 \in I\}}$, we can write
  \[
    E_T(I) f = (\chi_{\{\xi^2 \in I\}} \widehat{f})^\vee.
  \]
\end{example}

\begin{example}
  Let $\mathcal{H} = L^2(\R^d)$ and
  $T = - \Delta / 2 + V$, where $V \in L^\infty$ is
  real. We claim that $T$ is self-adjoint. It is easy to check
  that $T$ is symmetric, so $D(T) \subseteq D(T^*)$,
  where $D(T) = H^2(\R^d)$. For the reverse inclusion,
  take $g \in D(T^*)$, so $\langle Tf, g \rangle = \langle f, h \rangle$
  for some $h \in \mathcal{H}$. Then
  \[
    \langle -\Delta f / 2, g \rangle
    = \langle f, h \rangle - \langle Vf, g \rangle,
  \]
  so we have
  \[
    | \langle - \Delta f / 2, g \rangle | 
    \le (\| h \|_{L^2} + \| V \|_{L^\infty}\| g \|_{L^2}) \| f \|_{L^2}.
  \]
  As before, the Hahn-Banach theorem implies that
  $-\Delta g \in L^2$, and elliptic theory yields
  $g \in H^2 = D(T)$. Thus $D(T^*) \subseteq D(T)$, so
  we see that $T$ is self-adjoint.
\end{example}

  \chapter{Feb.~17 --- Spectral Theory, Part 3}

\section{More Spectral Theory}

\begin{definition}
  Define the following parts of $\mathcal{H}$ (with
  respect to Lebesgue measure):
  \begin{align*}
    \mathcal{H}_{\mathrm{ac}}
    &= \{x \in \mathcal{H} : E_{x, x} \text{ is absolutely continuous}\}, \\
    \mathcal{H}_{\mathrm{sc}}
    &= \{x \in \mathcal{H} : E_{x, x} \text{ is singular continuous}\}, \\
    \mathcal{H}_{\mathrm{pp}}
    &= \{x \in \mathcal{H} : E_{x, x} \text{ is pure point}\}.
  \end{align*}
\end{definition}

\begin{lemma}
  Let $T$ be self-adjoint and $E$ its spectral
  resolution. Then
  $\mathcal{H} = \mathcal{H}_{\mathrm{ac}} \oplus \mathcal{H}_{\mathrm{sc}} \oplus \mathcal{H}_{\mathrm{pp}}$.
\end{lemma}

\begin{proof}
  We first show that $(1)$ $\mathcal{H}_{\mathrm{ac}}, \mathcal{H}_{\mathrm{sc}}, \mathcal{H}_{\mathrm{pp}}$ are closed
  subspaces, then that $(2)$ they are orthogonal to each
  other, and finally $(3)$ the decomposition for
  an arbitrary $f \in \HH$.

  $(1)$ Take $f, g \in \mathcal{H}_{\mathrm{ac}}$,
  we first show that $f + g \in \mathcal{H}_{\mathrm{ac}}$.
  If $S$ is a Borel set such that $|S| = 0$ (Lebesgue
  measure), then we want to show that
  $E(S)(f + g) = 0$. Since $f, g \in \Hac$, we note that
  \[
    E(S)f = E(S)g = 0.
  \]
  Indeed, for $f \in \Hac$, we have
  $\langle E(S) f, f \rangle = E_{f, f}(S) = 0$. Then
  \[
    0 = \langle E(S) f, f \rangle = \langle E(S) E(S) f, f \rangle
    = \langle E(S) f , E(S) f \rangle
  \]
  since $E(S)$ is a projection and also
  self-adjoint. This shows that $E(S)f = 0$ by the
  definiteness of the inner product, and
  similarly, $E(S) g = 0$.
  To check $E_{f + g, f + g}(S) = 0$, we note that
  \[
    E_{f + g, f + g}(S) = \langle E(S) (f + g), f + g \rangle
    = \langle E(S) f, f \rangle + \langle E(S) g, f \rangle
    + \langle E(S) f, g \rangle + \langle E(S) g, g \rangle
    = 0
  \]
  since $E(S) f = E(S) g = 0$. This implies
  that $f + g \in \Hac$, which shows that
  $\Hac$ is a subspace. To see that $\Hac$ is
  closed,
  let $f_n \in \Hac$ with $f_n \to f$ in $\HH$.
  Take any $S$ with $|S| = 0$, so
  \[
    \langle E(S)f, f \rangle
    =
    \lim_{n \to \infty} \langle E(S) f_n, f_n \rangle =
    \lim_{n \to \infty} 0 = 0.
  \]
  Thus $E(S) f = 0$, so $f \in \Hac$. Thus
  $\Hac$ is a closed subspace.

  Now let $f, g \in \Hsc$. Let $S_f, S_g$ be the support
  of $E_{f, f}, E_{g, g}$, respectively. Since
  $f, g \in \Hsc$, we must have $|S_f| = |S_g| = 0$.
  We also know $E_{f, f}(S_f^c) = E_{g, g}(S_g^c) = 0$,
  since $S_f^c, S_g^c$ are outside of the support
  of $E_{f, f}, E_{g, g}$. Let
  $S_{f + g} = S_f \cup S_g$, so
  $|S_{f + g}| = 0$ by the above. Furthermore,
  \[
    E(S_{f + g}^c)(f + g)
    = E(S_f^c \cap S_g^c)(f + g)
    = E(S_g^c) E(S_f^c) f + E(S_f^c) E(S_g^c) g
    = 0
  \]
  This implies that $\Hsc$ is a subspace.
  Now take $f_n \in \Hsc$ with $f_n \to f$ in $\HH$.
  Similarly, let $S_n$ be the support of the $E_{f_n, f_n}$.
  Let $S_f = \bigcup_{n = 1}^\infty S_{f_n}$, so
  $|S_f| = 0$ by countable subadditivity. Then
  \[
    E(S_f^c) f_n
    = \left(\prod_j \ne n E(S^c_{f_j})\right) \circ E(S_{f_n}^c) f_n
    = 0,
  \]
  and letting $n \to \infty$ implies that
  $E(S_f^c) f = 0$. Since the support of $E_{f, f}$ is
  of measure $0$ in the Lebesgue sense,
  $f \in \Hsc$.
  Thus we see that $\Hsc$ is a closed subspace.

  One can verify that $\Hpp$ is a closed subspace
  in a similar manner, which completes the proof for
  $(1)$.

  $(2)$ We claim that
  $\Hac$ is orthogonal to $\Hsc$ and $\Hpp$.
  Take $f \in \Hac$ and $g \in \Hsc \cup \Hpp$.
  Let $S_g$ be the support of $E_{g, g}$, so
  $|S_g| = 0$. Note that $g = E(S_g) g$. Then we can
  write
  \[
    \langle f, g \rangle
    = \langle f, E(S_g) g \rangle
    = \langle E(S_g) f, g \rangle = 0
  \]
  since $f$ is absolutely continuous with respect
  to Lebesgue measure. Thus $f \perp g$.

  A similar argument shows that
  $\Hsc \perp \Hpp$, so which proves $(2)$.

  $(3)$ Given $f \in \HH$, the measure $E_{f, f}$ can be
  decomposed (with respect to Lebesgue measure)
  into an absolutely continuous
  part $E_1$, a singular continuous part $E_2$, and
  a pure point part $E_3$. Denote the support of
  $E_1, E_2, E_3$ by $S_1, S_2, S_3$.
  Thus we can write
  \[
    f = E(S_1) f + E(S_2) f + E(S_3) f.
  \]
  We have $E(S_1) f \in \Hac$, $E(S_2) f \in \Hsc$,
  and $E(S_3) f \in \Hpp$, so we get
  $\HH = \Hac \oplus \Hsc \oplus \Hpp$.
\end{proof}

\begin{definition}
  Define $E_{\mathrm{ac}} = E \circ P_{\mathrm{ac}}$, where
  $P_{\mathrm{ac}}$ is the projection onto
  $\mathcal{H}_{\mathrm{ac}}$.
\end{definition}

\begin{lemma}
  Let $T$ be a self-adjoint operator and $E$ be its
  spectral resolution. Then for a.e. $\lambda \in \R$,
  \[
    \langle E_{\mathrm{ac}}(d\lambda) f, g \rangle
    = \langle \frac{1}{2\pi i} (R(\lambda + i0) - R(\lambda - i0)) f, g \rangle\, d \lambda,
  \]
  where $R(\lambda \pm i0) = \lim_{\epsilon \to 0^+} R(\lambda \pm i\epsilon)$
  and $R(\lambda \pm i \epsilon) = (T - (\lambda \pm i \epsilon))^{-1}$.
\end{lemma}

\begin{proof}
  By spectral theory, we can write
  \begin{align*}
    &\langle \frac{1}{2\pi i} ((T - (x + i\epsilon))^{-1} - (T - (x - i\epsilon))^{-1}) f, g \rangle \\
    & \quad \quad = \frac{1}{2\pi i} \left(\frac{1}{\mu - (\lambda + i\epsilon)} - \frac{1}{\mu - (\lambda - i\epsilon)}\right)
    \langle E(d\mu) f, g \rangle
    = \int \frac{1}{\pi} \frac{\epsilon}{(\mu - \lambda)^2 + \epsilon^2}\,  \langle E(d\mu) f, g \rangle.
  \end{align*}
  Note that this is the Poisson kernel in the
  upper half-plane. Letting $\epsilon \to 0^+$, we have
  \[
    \int \frac{1}{\pi} \frac{\epsilon}{(\mu - \lambda)^2 + \epsilon^2}\,  \langle E(d\mu) f, g \rangle
    \to \frac{\langle E_{\mathrm{ac}}(d\lambda) f, g \rangle}{d\lambda}
  \]
  for a.e. $\lambda$ (in terms of Lebesgue measure)
  since the Poisson kernel is an approximate identity.
\end{proof}

\begin{remark}
  What can we do with resolvents? We are interested
  in $- \Delta + V$. Take $V = 0$, then
  \[
    R(\lambda + i\epsilon)
    = (-\Delta - (\lambda + i\epsilon))^{-1}.
  \]
  Let $d = 3$, then we have
  \[
    ((-\Delta - (\lambda + i\epsilon))^{-1} f)^\wedge
    = \frac{\widehat{f}(\xi)}{\xi^2 - (\lambda + i\epsilon)}.
  \]
  Applying Fourier inversion, we get
  \[
    ((-\Delta - (\lambda + i\epsilon))^{-1} f)(x)
    = \frac{1}{(2\pi)^3} \int \frac{e^{ix\xi} \widehat{f}(\xi)}{\xi^2 - (\lambda + i\epsilon)}\, d\xi.
  \]
  Note that we can write
  \[
    ((-\Delta - (\lambda + i\epsilon))^{-1} f)(x)
    = \int K(x - y) f(x)\, dy,
  \]
  where the kernel $K$ satisfies
  \[
    \widehat{K}(\xi) =
    \frac{1}{\xi^2 - (\lambda + i\epsilon)}.
  \]
  By the inversion formula, this means that
  \[
    K = \frac{1}{(2\pi)^3} \int \frac{e^{ix\xi}}{\xi^2 - (\lambda + i\epsilon)}\, d\xi.
  \]

  Using polar coordinates, set $\xi = r \omega$, and
  we obtain
  \[
    K
    = \int_0^\infty \int_{\mathbb{S}^2} e^{i r x \cdot \omega}\, d \sigma_{\mathbb{S}^2}(\omega)\,
    \frac{r^2}{r^2 - (\lambda + i\epsilon)}\, dr. \tag{$*$}
  \]
  We can first compute that
  \[
    \int_{\mathbb{S}^2} e^{ia \omega \cdot \vec{e}_3}\, d \sigma_{\mathbb{S}^2}(\omega)
    = 2\pi \int_0^\pi e^{ia \cos \theta}\, \sin \theta\, d\theta
    = 2\pi \int_{-1}^1 e^{ia u}\, du
    = 4\pi \frac{\sin(au)}{a}.
  \]
  Then we can calculate the inner integral in
  $(*)$ by symmetry:
  \[
    \int_{\mathbb{S}^2} e^{i r x \cdot \omega}\, d \sigma_{\mathbb{S}^2}(\omega)
    = 4\pi \frac{\sin(r |x|)}{r |x|}.
  \]
  Substituting this into $(*)$, we get
  \begin{align*}
    K
    &= \frac{1}{2\pi^2} \int_0^\infty \frac{\sin(r |x|)}{r |x|} \frac{r^2}{r^2 - (\lambda + i\epsilon)}\, dr
    = \frac{1}{2\pi^2 |x|} \int_0^\infty \frac{\sin(r |x|)}{r^2 - (\lambda + i \epsilon)}\, r\, dr \\
    &= \frac{1}{4\pi^2 |x|} \int_{-\infty}^\infty \frac{\sin(r |x|)}{r^2 - (\lambda + i \epsilon)}\, r\, dr
    = \frac{1}{16 \pi^2 |x|} \int_{-\infty}^\infty
    (e^{i|x|r} - e^{-ir|x|}) \left(\frac{1}{r - \sqrt{\lambda + i\epsilon}} - \frac{1}{r + \sqrt{\lambda + i\epsilon}}\right) dr.
  \end{align*}
  By some complex analysis (residue computations),
  we get
  $K = e^{i|x| \sqrt{\lambda + i\epsilon}} / 4\pi |x|$,
  so that
  \[
    R(\lambda + i 0) = \lim_{\epsilon \to 0^+} R(\lambda + i\epsilon)
    = \frac{1}{4\pi |x|} e^{i|x| \sqrt{\lambda}},
    \quad -\infty < \lambda < \infty,
  \]
  where we take the principal branch of the square root.
\end{remark}

  \chapter{Feb.~19 --- Dispersive Decay with Potential}

\section{More on Computing Resolvents}

\begin{example}
  We will compute the resolvent for $-\partial_x^2$.
  For $\epsilon > 0$, we would like to study
  \[
    R(\lambda \pm i \epsilon) = (-\partial_x^2 - (\lambda \pm i \epsilon))^{-1}.
  \]
  It is reduced to compute the Green's function in
  $1$ dimension: $G_{\lambda + i\epsilon}(x, y)$ such
  that
  \[
    (-\partial_x^2 - (\lambda + i\epsilon)) G_{\lambda + i\epsilon}(x, y) = \delta(x - y)
  \]
  and is normalized by
  $G_{\lambda + i\epsilon}(-\infty, y) = G_{\lambda + i\epsilon}(\infty, y) = 0$. Note that
  the function
  \[
    g_1 = e^{i\sqrt{\lambda + i\epsilon} x}
  \]
  solves the ODE $(-\partial_x^2 - (\lambda + i\epsilon)) g_1 = 0$
  and satisfies $g_1(x) \to 0$ as $x \to \infty$.
  The function
  \[
    g_2 = e^{-i\sqrt{\lambda + i\epsilon} x}
  \]
  also solves $(-\partial_x^2 - (\lambda + i\epsilon)) g_2 = 0$
  but satisfies $g_2(x) \to 0$ as $x \to -\infty$.
  If $x < y$, then by the uniqueness of the ODE solution
  we should expect that $G_{\lambda + i\epsilon}$ is a multiple of $g_2$.
  Similarly, if $x > y$ we should expect that
  $G_{\lambda + i\epsilon}$ is a multiple of $g_1$.
  Stitching the solutions together, we find that
  \[
    G_{\lambda + i\epsilon}(x, y)
    = \frac{1}{W[g_1, g_2]}
    \begin{cases}
      g_2(x) g_1(y) & -\infty < x < y, \\
      g_1(x) g_2(y) & y < x < \infty,
    \end{cases}
  \]
  where $W[g_1, g_2] = g_1 g_2' - g_1' g_2 = -2i \sqrt{\lambda + i\epsilon}$ is the Wronskian.
  Passing $\epsilon \to 0$,
  \[
    R(\lambda + i 0) = \lim_{\epsilon \to 0^+} G_{\lambda + i\epsilon}(x, y)
    = -\frac{e^{i \sqrt{\lambda}|x - y|}}{2i \sqrt{\lambda}}.
  \]
  We can also note that
  \[
    R(\lambda - i 0) = \overline{R(\lambda + i 0)}
    = \frac{e^{-i \sqrt{\lambda}|x - y|}}{2i \sqrt{\lambda}}.
  \]
\end{example}

\begin{exercise}
  By spectral theory, we know that we can write
  \[
    e^{t \partial_x^2 t} f = \frac{1}{2\pi i} \int_0^\infty e^{i\lambda t} [R(\lambda + i0) - R(\lambda - i0)] f\, d\lambda.
  \]
  We have an explicit formula for
  $R(\lambda \pm i 0)$ from the above computation, e.g.
  \[
    R(\lambda + i 0) f = -\int_{\R} \frac{e^{i \sqrt{\lambda}|x - y|}}{2i \sqrt{\lambda}} f(y)\, dy.
  \]
  Check that this is the same formula for
  $e^{i\partial_x^2 t} f$ that
  we obtained via Fourier transforms.
\end{exercise}

\begin{remark}
  The Fourier transform only works well for
  standard differential operators and not for
  perturbed versions such as $-\partial_x^2 + V$.
  Spectral theory will allow us to
  better understand such operators.
\end{remark}

\section{Dispersive Decay with Potential}

\begin{remark}
  Our next goal is to show the dispersive decay
  for $e^{iH t} f$, where $H = -\Delta + V$, in
  $d = 1, 3$. In particular, for $V$ sufficiently
  nice and $P_{\mathrm{c}} = P_{\mathrm{ac}}$, we will show that
  \[
    \| e^{iHt} P_{\mathrm{c}} f \|_{L^\infty} \lesssim t^{-d / 2} \|f\|_{L^1}, \quad d = 1, 3.
  \]
  If $V$ is nice, then $\Hsc = \varnothing$, so
  $\HH = \Hac \oplus \Hpp$ and we may simply write
  $P_{\mathrm{c}} = P_{\mathrm{ac}}$.
\end{remark}

\begin{theorem}\label{thm:dispersive-decay-1d}
  Let $d = 1$ and $H = -\partial_x^2 + V$, where
  $V$ is sufficiently nice (e.g. enough decay).
  Then
  \[
    \|e^{iHt} P_{\mathrm{c}} f\|_{L^\infty} \lesssim t^{-1 / 2} \|f\|_{L^1}.
  \]
\end{theorem}

\begin{remark}
  Why do we need $P_{\mathrm{c}}$? This is because
  if $f \in \Ran(P_{\mathrm{pp}})$, then $Hf = \lambda f$
  for some $\lambda \in \R$ (since $H$ is self-adjoint).
  Then $e^{iHt} f = e^{i\lambda t} f$, which is
  oscillatory in $t$ and does not decay.
\end{remark}

\begin{lemma}[High energy part]
  Let
  $\chi \in C^\infty(\R)$ such that
  $\chi(\lambda) \equiv 1$ for $\lambda \ge 2 \lambda_0$
  and $\chi(\lambda) \equiv 0$ for $\lambda \le \lambda_0$
  for some large but fixed $\lambda_0$. Then we have
  \[
    \| e^{iHt} \chi(H) f \|_{L^\infty} \lesssim t^{-1 / 2} \|f\|_{L^1}.
  \]
\end{lemma}

\begin{proof}
  This proof is due to Goldberg--Schlag (2004), we
  will give an alternative (more direct but less
  general) proof later on. Note that
  $\sigma(-\partial_x^2) = [0, \infty)$, so
  $\langle -\partial_x^2 f, f \rangle \ge 0$.
  So one can always invert $-\partial_x^2 - \lambda$
  for $\lambda < 0$.
  Then the absolutely continuous spectrum of
  $-\partial_x^2 + V$ will also be $[0, \infty)$
  by Weyl's theorem provided that $V$ decays
  sufficiently fast. Then by spectral theory,
  one can write
  \[
    \langle e^{iHt} \chi(H) P_{\mathrm{c}} f, g \rangle
    = \int_0^\infty  e^{i\lambda t} \chi(\lambda) \langle (R(\lambda + i0) - R(\lambda - i0)) f, g \rangle\, d\lambda.
  \]
  The resolvent in this case is
  \[
    R(\lambda \pm i 0)
    = \lim_{\epsilon \to 0^+}
    (-\partial_x^2 + V - (\lambda \pm i \epsilon))^{-1}
    = \lim_{\epsilon \to 0^+}
    (H_0 + V - (\lambda \pm i \epsilon))^{-1}
  \]
  where $H_0 = -\partial_x^2$. Then one can write
  \[
    R(\lambda \pm i\epsilon)
    = (H_0 + V - (\lambda \pm i \epsilon))^{-1}
    = R_0(\lambda \pm i\epsilon) - R_0(\lambda \pm i\epsilon) V R(\lambda \pm i\epsilon).
  \]
  The above is the \emph{second resolvent identity},
  which follows from
  \[
    (H - (\lambda \pm i \epsilon)) R(\lambda \pm i \epsilon) = I
  \]
  and
  \[
    -(H - (\lambda \pm i \epsilon)) R_0(\lambda \pm i \epsilon)
    = (H_0 + V - (\lambda \pm i \epsilon)) R_0(\lambda \pm i \epsilon)
    = V R_0(\lambda \pm i \epsilon) + I,
  \]
  which gives
  \[
    -(H - (\lambda \pm i \epsilon)) (R(\lambda \pm i \epsilon) V R_0(\lambda \pm i \epsilon))
    = V R_0(\lambda \pm i \epsilon).
  \]
  Adding the above equations gives the desired
  identity. Thus we can formally write
  \[
    R(\lambda \pm i \epsilon)
    = R_0(\lambda \pm i \epsilon)
    - R_0 (\lambda \pm i \epsilon) V R(\lambda \pm i \epsilon)
    + \dots
    = \sum_{n = 0}^\infty R_0(\lambda \pm i \epsilon) (-V R_0(\lambda \pm i \epsilon))^n,
  \]
  which is known as the \emph{Born series}.\footnote{Formally, one can see this as the \emph{Neumann series} for $(H - (\lambda \pm i \epsilon))^{-1} = (H_0 - (\lambda \pm i \epsilon))^{-1} (1 + V / (H_0 - (\lambda + i\epsilon)))^{-1}$ and apply the geometric series formula.}
  Now we claim that for $\lambda > \lambda_0$ with
  $\lambda_0$ large enough,
  \[
    \langle R(\lambda + i 0)f, g \rangle
    = \sum_{n = 0}^\infty \langle R_0(\lambda + i 0) (-V R_0(\lambda + i 0))^n f, g \rangle
  \]
  for Schwartz functions $f, g \in \mathcal{S}$. To
  see this, we can write
  \[
    \langle R_0(\lambda + i\epsilon) f, g \rangle
    = -\iint \frac{1}{2i \sqrt{\lambda + i\epsilon}} e^{i\sqrt{\lambda + i\epsilon} |x - y|} f(y)\, dy\, \overline{g(x)}\, dx.
  \]
  Thus we have the bound
  \[
    |\langle R_0(\lambda + i\epsilon) f, g \rangle|
    \le \frac{1}{2 \sqrt{\lambda}} \|f\|_{L^1} \|g\|_{L^1}.
  \]
  In general, we can estimate
  \begin{align*}
    &|\langle R_0(\lambda + i\epsilon)(VR_0(\lambda + i\epsilon))^n f, g \rangle| \\
    &\quad \quad \le \int \dots \int \frac{1}{|2\sqrt{\lambda + \epsilon}|^{n + 1}} |V(x_1) \dots V(x_n)| |f(x_0)| |g(x_{n + 1})|\, dx_0 \dots dx_{n + 1} \\
    &\quad \quad \le \frac{1}{(2\sqrt{\lambda + i\epsilon})^{n + 1}} \|V \|_{L^1}^n \|f\|_{L^1} \|g\|_{L^1}.
  \end{align*}
  Now take $\lambda_0$ such that $2\sqrt{\lambda_0} > \|V\|_{L^1}$, so that the series
  \[
    \sum_{n = 0}^\infty \langle R_0(\lambda + i 0) (-V R_0(\lambda + i 0))^n f, g \rangle
  \]
  converges uniformly in $\lambda$. We will finish the
  proof next time.
\end{proof}

  \chapter{Feb.~24 --- Dispersive Decay with Potential, Part 2}

\section{High Energy Estimates, Continued}
\begin{prop}
  We have the estimate
  \[
    |\langle e^{iH t} \chi(H) P_{\mathrm{ac}} f, g \rangle|
    \lesssim t^{-1 / 2} \|f\|_{L^1} \|g\|_{L^1}.
  \]
\end{prop}

\begin{proof}
  Using the Born series, we have
  \begin{align*}
    &|\langle e^{iH t} \chi(H) P_{\mathrm{ac}} f, g \rangle| \\
    &\le \sum_{n = 0}^\infty \left|
    \frac{(-1)^n}{2\pi i} \int_0^\infty e^{it\lambda} \chi(\lambda) \left(
      \langle R_0(\lambda + i0) (V R_0(\lambda + i0))^n f, g \rangle
      - \langle R_0(\lambda - i0) (V R_0(\lambda - i0))^n f, g \rangle
    \right) d\lambda
    \right|.
  \end{align*}
  The general term in the above series is
  (the minus term is similar)
  \[
    \int_0^\infty e^{it\lambda} \chi(\lambda)
    \langle R_0(\lambda + i0) (V R_0(\lambda + i0))^n f, g \rangle\, d\lambda.
  \]
  Changing variables using $\lambda = \eta^2$, we have
  \[
    \int_0^\infty e^{it \eta^2} \chi(\eta^2) \cdot 2\eta
    \langle R_0(\eta^2 + i0) (V R_0(\eta^2 + i0))^n f, g \rangle\, d\eta.
  \]
  Using the calculation that
  $R_0(\lambda \pm i 0) = \pm i e^{\pm i |x| \sqrt{\lambda}} / 2\sqrt{\lambda}$,
  we have $R_0(\eta^2 + i 0) = i e^{i|x| \eta} / 2\eta$,
  and we get
  \[
    \iint f(x_0) g(x_{n + 1})
    \int \dots \int V(x_1) \dots V(x_n)
    \int_0^\infty e^{it\eta^2} 2\eta \cdot \chi(\eta^2)
    \frac{e^{i \eta \sum_{j = 1}^{n + 1} |x_j - x_{j - 1}|}}{(-2\eta i)^{n + 1}}\, d\eta\, dx_1 \dots dx_n\, dx_0 dx_{n + 1}.
  \]
  Note the above integral by $(*)$, and by Minkowski's
  inequality we have
  \begin{align*}
    &|(*)| \\
    &\le
    \int \dots \int |f(x_0)| |g(x_{n + 1})| V(x_1) \dots V(x_n)
    \sup_{x_0, \dots, x_{n + 1}} \left| \int_0^\infty e^{it\eta^2} 2\eta \cdot \chi(\eta^2) \frac{e^{i \eta \sum_{j = 1}^{n + 1} |x_j - x_{j - 1}|}}{(-2\eta i)^{n + 1}}\, d\eta \right|
    dx_0 \dots dx_{n + 1}.
  \end{align*}
  Setting $a = \sum_{j = 1}^{n + 1} |x_j - x_{j - 1}|$,
  the inner integral satisfies
  \[
    \sup_{x_0, \dots, x_{n + 1}} \left| \int_0^\infty e^{it\eta^2} 2\eta \cdot \chi(\eta^2) \frac{e^{i \eta \sum_{j = 1}^{n + 1} |x_j - x_{j - 1}|}}{(-2\eta i)^{n + 1}}\, d\eta \right|
    = \sup_a \left| \int_0^\infty e^{it\eta^2} 2\eta \cdot \chi(\eta^2) \frac{e^{i \eta a}}{(-2\eta i)^{n + 1}}\, d\eta \right|.
  \]
  By a suitable change of variables, it suffices to study
  \[
    \sup_a \left| \int_{-\infty}^\infty e^{it\eta^2} 2\eta \cdot \chi(\eta^2) \frac{e^{i \eta a}}{(-2\eta i)^{n + 1}}\, d\eta \right|
    = \sup_a \left| \int_{-\infty}^\infty e^{it\eta^2 + i\eta a} \widehat{\mu}_n(\eta)\, d\eta \right|,
  \]
  where we have
  \[
    \widehat{\mu}_n(\eta) = \frac{2\eta \cdot \chi(\eta^2)}{(-2\eta i)^{n + 1}}
    = \|e^{i\partial_x^2 t} \mu_n \|_{L^\infty}
    \le t^{-1 / 2}\|\mu_n \|_{L^1}.
  \]
  The key idea is to reduce
  the computation to the $1$-D free Schr\"odinger
  equation. Writing
  \[
    \mu_n(x) = \int_{-\infty}^\infty \frac{\chi(\lambda^2)}{\lambda^n} e^{-i\lambda x}\, d\lambda,
  \]
  we want to estimate $\|\mu_n\|_{L^1}$. For $n = 0$,
  we have
  \[
    \mu_0(x) = \int_{-\infty}^\infty \chi(\lambda^2) e^{-i\lambda x}\, d\lambda
    = \int_{-\infty}^\infty (1 + (1 - \chi(\lambda^2))) e^{-i\lambda x}\, d\lambda
    = \delta_0 + \int_{-\infty}^\infty (1 - \chi(\lambda))^2 e^{-i\lambda x}\, d\lambda.
  \]
  The latter integral is a Schwartz
  function since $1 - \chi(\lambda)$ is compactly
  supported. So $\|\mu_0\|_{L^1} < \infty$.
  For general $n$, we would like to estimate
  $\|x^2 \mu_n(x)\|_{L^\infty}$:
  \[
    \|x^2 \mu_n(x)\|_{L^\infty}
    = \|(\widehat{\mu}_n''(x))^\vee\|_{L^\infty}
    \le \|\widehat{\mu}_n''\|_{L^1}
    = \left\| \left(\frac{\chi(\lambda^2)}{\lambda^n}\right)'' \right\|
    \lesssim \lambda_0^{-n / 2},
  \]
  where the second inequality is by Young's inequality
  and the last inequality is because
  \[
    \left(\frac{\chi(\lambda^2)}{\lambda^n}\right)''
    \lesssim \left|\frac{\chi'(\lambda^2)}{\lambda^{n + 2}}\right|
    + \left|\frac{\lambda \chi'(\lambda^2)}{\lambda^{n + 1}}\right|
    + \left|\frac{\lambda^2 \chi''(\lambda^2)}{\lambda^{n}}\right|
  \]
  and the observation that $\chi', \chi''$ are
  compactly supported functions. Thus we have
  \[
    |\mu_n(x)| \lesssim (\lambda_0)^{-n / 2} |x|^{-2}.
  \]
  This gives us decay as $x \to \infty$, but we also
  need to estimate $\|\mu_n\|_{L^\infty}$: For $n \ge 2$,
  \[
    \| \mu_n \|_{L^\infty}
    \lesssim \|\widehat{\mu}_n\|_{L^1}
    = \left\| \frac{\chi(\lambda^2)}{\lambda^n} \right\|
    \lesssim (\lambda_0)^{-n / 2}
  \]
  since $\widehat{\mu}_n$ is integrable for $n \ge 2$.
  When $n = 1$, we have
  \[
    \|\mu_1(x)\|_{L^\infty}
    = \left\| \left(\frac{\chi(\lambda^2)}{\lambda}\right)^\vee \right\|_{L^\infty}
    = \| (\chi(\lambda^2))^\vee * (1 / \lambda)^\vee \|_{L^\infty}
    = \| \chi(\lambda^2) \|_{L^1} \|(1 / \lambda)^\vee\|_{L^\infty} < \infty
  \]
  since $\|\chi(\lambda^2)\|_{L^1}$ is finite and
  $(1 / \lambda)^\vee = -i \sign(x)$, so
  $\|(1 / \lambda)^\vee\|_{L^\infty} \le 1$. Thus
  for $n \ge 1$,
  \[
    \|\mu_n\|_{L^\infty} \lesssim (\lambda_0)^{-n / 2}
    \quad \text{and} \quad
    \|\mu_n(x)\| \lesssim (\lambda_0)^{-n / 2} |x|^{-2},
  \]
  which gives $\|\mu_n \|_{L^1} \lesssim (\lambda_0)^{-n / 2}$.
  When $n = 0$, we already know
  $\|\mu_0\|_{L^1} < \infty$. This gives
  \[
    \sup_a \left| \int_0^\infty e^{it\eta^2} 2\eta \cdot \chi(\eta^2) \frac{e^{i \eta a}}{(-2\eta i)^{n + 1}}\, d\eta \right|
    \lesssim |t|^{-1 / 2} (\lambda_0)^{-n / 2},
  \]
  and so we have
  \[
    (*) \lesssim \|f\|_{L^1} \|g\|_{L^1} \|V\|_{L^1}^n (\lambda_0)^{-n / 2} |t|^{-1 / 2}.
  \]
  As long as $\lambda_0^{-1 / 2} \| V \|_{L^1} < 1$,
  we can sum the Born series and get
  \[
    \langle e^{iHt} \chi(H) P_{\mathrm{ac}} f, g \rangle
    \le |t|^{-1 / 2} \|f\|_{L^1} \|g\|_{L^1} \sum_{n = 0}^\infty (\lambda_0^{-1 / 2} \|V\|_{L^1})^n
    \lesssim |t|^{-1 / 2} \|f\|_{L^1} \|g\|_{L^1},
  \]
  which is the desired estimate.
\end{proof}

\begin{remark}
  The takeaway from this proof is that for the
  high frequency part, we use a Born series, the
  free resolvent, and explicit computations.
\end{remark}

\section{Low Energy Estimates}

\begin{remark}
  In this setting, the influence of the potential
  becomes more serious.
\end{remark}

\begin{lemma}[Low energy]
  Let $V$ be ``nice'' and $\chi$ a cutoff function
  near $0$. Then
  \[
    \| e^{iHt} \chi(H) P_{\mathrm{ac}} f \|_{L^\infty}
    \le |t|^{-1 / 2} \|f\|_{L^1}.
  \]
\end{lemma}

\begin{proof}
  We use spectral theory. Write
  \[
    e^{iHt} \chi(H) P_{\mathrm{ac}} f
    = \frac{1}{2\pi i} \int_0^\infty \chi(\lambda) e^{-it\lambda} [R(\lambda + i0) - R(\lambda - i0)] f\, d\lambda.
  \]
  We will use the convention (note that
  $(\eta + i\epsilon)^2 = \eta^2 - \epsilon^2 + 2i\eta \epsilon$)
  \[
    R(\eta^2 + i 0) = \lim_{\epsilon \to 0^+} R((\eta + i\epsilon)^2)
    = \lim_{\epsilon \to 0^+} R(\eta^2 + i \sign (\eta) \epsilon).
  \]
  Then we can write
  \[
    e^{iHt} \chi(H) P_{\mathrm{ac}} f
    = \frac{1}{\pi i} \int_{-\infty}^\infty \eta \cdot \chi(\eta^2) e^{it\eta^2} R(\eta^2 + i0) f\, d\eta.
  \]
  Using a Green's function, we can write
  \[
    R(\lambda^2 + i 0)(x, y)
    = \frac{f_+(\lambda, y)f_-(\lambda, x)}{W[f_+(\lambda, \cdot), f_-(\lambda, \cdot)]} \mathbbm\{x < y\}
    + \frac{f_+(\lambda, x)f_-(\lambda, y)}{W[f_+(\lambda, \cdot), f_-(\lambda, \cdot)]} \mathbbm\{x > y\},
  \]
  where $f_{\pm}$ are the Jost functions solving
  \[
    \begin{cases}
      -f_{\pm}'' + V f_{\pm} = \lambda^2 f_{\pm}, \\
      |f_+(\lambda, x) - e^{i\lambda x}| \xrightarrow[x \to \infty]{} 0, \\
      |f_-(\lambda, x) - e^{-i\lambda x}| \xrightarrow[x \to \infty]{} 0.
    \end{cases}
  \]
  The conclusion from this is that
  \begin{align*}
    \langle e^{iHt} \chi(H) P_{\mathrm{ac}} f, g \rangle
    &=
    \frac{1}{\pi i} \int_{-\infty}^\infty
    \int e^{it \lambda^2} \lambda \cdot  \chi(\lambda^2)
    \frac{f_+(\lambda, y) f_-(\lambda, x)}{W(\lambda)}\, d\lambda\, f(x) \overline{g(y)} \mathbbm{1}_{\{x < y\}}\, dx\, dy \\
    & \quad \quad +
    \frac{1}{\pi i} \int_{-\infty}^\infty
    \int e^{it \lambda^2} \lambda \cdot  \chi(\lambda^2)
    \frac{f_+(\lambda, x) f_-(\lambda, y)}{W(\lambda)}\, d\lambda\, f(x) \overline{g(y)} \mathbbm{1}_{\{y < x\}}\, dx\, dy.
  \end{align*}
  We will finish the proof next time.
\end{proof}

  \chapter{Feb.~26 --- Dispersive Decay with Potential, Part 3}

\section{Low Energy Estimates, Continued}
\begin{proof}[Proof of Lemma \ref{lem:low-energy}, continued]
  Our goal is to understand $f_\pm$. Define
  $m_+(\lambda, x) = e^{i\lambda x} f_+(\lambda)$,
  so that
  \[
    m_+(\lambda, x) \to 1 \quad \text{as $x \to \infty$}
  \]
  ``Volterra iteration'' gives more precise
  estimates, but we will go with a rougher approach:
  $m_+$ satisfies
  \[
    \partial_x^2 m_+(\lambda, x) + 2i \lambda \partial_x m_+(\lambda, x) + V(x) m_+(\lambda, x) = 0.
  \]
  Then we have the \emph{Volterra equation}
  \[
    m_+(\lambda, x) = 1 + \int_x^\infty D_\lambda(x - y) V(y) m_+(\lambda, y)\, dy, \quad
    D_\lambda = \int_0^x e^{2i\lambda z}\, dz = \frac{e^{2i\lambda x} - 1}{2 i \lambda}.
  \]
  Denote $\dot{m}_+(\lambda, x) = \partial_\lambda m_+(\lambda, x)$
  and $\ddot{m}_+(\lambda, x) = \partial_\lambda^2 m_+(\lambda, x)$,
  which satisfy
  \[
    \dot{m}_+(\lambda, x) = \int_x^\infty D_\lambda(x - y) V(y) \dot{m}_+(\lambda, y)\, dy
    + \int_x^\infty \dot{D}_\lambda(x - y) V(y) m_+(\lambda, y)\, dy
  \]
  and a similar integral equation for $\ddot{m}_+$. We
  want to
  show the existence of $m_+$, and we are interested
  in the behavior of $m_+$ as $x \to \infty$.
  We will solve the equation on $[x_0, \infty)$ for
  $x_0$ large enough. Set
  \[
    z_+(\lambda, x)
    = \langle \lambda \rangle \frac{m_+(k, x) - 1}{W_+'(x)}, \quad
    W_+'(x) = \int_x^\infty \langle y \rangle |V(y)|\, dy.
  \]
  where $\langle \lambda \rangle = \sqrt{1 + \lambda^2}$
  is the \emph{Japanese bracket}.
  Note that $W_+'(x) \to 0$ with the same rate, and if
  \[
    \int \langle y \rangle^n |V(y)|\, dy < \infty,
  \]
  then $|W_+'(x)| \le \langle x \rangle^{1 - n}$.
  From the equation of $m_+$, we can see that
  \[
    (I - L) z_+(k, x) = z_+(k, x) - L z_+
    = \frac{\langle \lambda \rangle}{W_+'(x)} \int_x^\infty D_\lambda(y - x) V(y)\, dy, \quad
  \]
  where $Lz_+$ is given by
  \[
    L z_+ = \frac{1}{W_+'(x)} \int_x^\infty D_\lambda(y - x) V(y) W_+'(y) z_+(y) \, dy.
  \]
  Our goal is to show that $(I - L)$ is invertible
  on $L^\infty([x_0, \infty))$.
  Note that we have
  $|D_\lambda(z)| \lesssim |z| / |\lambda|$
  since $|e^{ix} - 1| \le |x|$, and also
  $|D_\lambda(z)| \le |z|$. Then we can estimate
  \[
    \left\|\langle \lambda \rangle W_+' \int_x^\infty D_\lambda(y - x) V(y)\, dy \right\|_{L^\infty([x_0, \infty))}
    \lesssim
    \left\|\langle \lambda \rangle W_+' \int_x^\infty \frac{|y - x|}{|\lambda|} V(y)\, dy \right\|_{L^\infty([x_0, \infty))}
    < \infty
  \]
  So if $(I - L) z_+ = G$, then from the
  above computation $\|G\|_{L^\infty([x_0, \infty))} < \infty$.
  We also have
  \begin{align*}
    \|Lz_+\|_{L^\infty([x_0, \infty))}
    &\lesssim \left\|\frac{1}{W_+'(x)} \int_x^\infty |y - x| V(y) W_+'(y) \|z_+\|_{L^\infty}\, dy \right\|_{L^\infty([x_0, \infty))} \\
    &\lesssim W_+'(x_0) \|z_+\|_{L^\infty([x_0, \infty))}\left\|\frac{1}{W_+'(x)} \int_x^\infty |y - x| |V(y)|\, dy \right\|_{L^\infty([x_0, \infty))} \\
    &\lesssim W_+'(x_0) \|z_+\|_{L^\infty([x_0, \infty))}.
  \end{align*}
  As $|x_0| \to \infty$, we have
  $\|Lz_+\|_{L^\infty([x_0, \infty))} \ll \|z_+\|_{L^\infty([x_0, \infty))}$,
  so $(I - L)$ is invertible on
  $L^\infty([x_0, \infty))$. Thus $z_+$ exists
  in $L^\infty([x_0, \infty))$, hence
  $m_+$ also exists in $L^\infty([x_0, \infty))$.

  We still need to work on $(-\infty, x_0)$.
  For $-1 \le x \le x_0$, we have
  \[
    |z_+(\lambda, x)|
    \lesssim 1 + \int_x^{x_0} \langle y \rangle |V(y)| |z_+(\lambda, y)|\, dy,
  \]
  so by Gronwall's inequality,
  $|z_+(\lambda, x)| \lesssim 1$. For $x \le 0$,
  we can notice that
  \[
    \frac{|z_+(\lambda, x)|}{\langle x \rangle}
    \lesssim 1 + \frac{1}{\langle x \rangle} \int_x^0 \langle x - y\rangle |V(y)| \frac{|z_+(\lambda, y)|}{\langle y \rangle}\, dy
    \lesssim 1 + \int_x^0 \langle y\rangle |V(y)| \frac{|z_+(\lambda, y)|}{\langle y \rangle}\, dy
  \]
  Applying Gronwall's inequality again, we obtain
  $|z_+(\lambda, x) / \langle x \rangle| \lesssim 1$.\footnote{Note that one can also apply Gronwall's inequality for $x > 0$, but the resulting estimate $|z_+(\lambda, x)| \lesssim \langle x \rangle$ does not give precise asymptotic information for $z_+$ as $x \to \infty$.}
  So we know the existence of $f_+$ and $m_+$, and
  a similar argument shows the existence of
  $f_-$ and $m_-$. We can also obtain estimates for
  $\dot{m}_\pm$ and $\ddot{m}_\pm$ in a similar fashion.
  We continue the proof from here next time.
\end{proof}

\begin{remark}
  Inserting the formula for $m_+$ again in the
  Volterra equation yields
  \begin{align*}
    m_+(\lambda, x)
    &= 1 + \int_x^\infty D_\lambda(x - y) V(y) m_+(\lambda, y)\, dy \\
    &= 1 + \int_x^\infty D_\lambda(x - y) V(y) \, dy
    + \int_x^\infty D_\lambda(x - y) V(y) \int_y^\infty D_\lambda(y - z) V(z) m_+(\lambda, z)\, dz\, dy \\
    &= \dots.
  \end{align*}
  Iterating this process (Volterra iteration) and
  bounding each term not
  involving $m_+$ gives more precise estimates (argue
  that one ends up with a summable series and get
  estimates from there).
\end{remark}

  \chapter{Mar.~3 --- Dispersive Decay with Potential, Part 4}

\section{Low Energy Estimates, Continued}
\begin{lemma}
  Suppose $W(0) \ne 0$. Then we have
  \[
    \sup_{x < y} \left|\int_{-\infty}^\infty e^{it\lambda^2} \frac{\lambda \chi(\lambda^2)}{W(\lambda)} f_+(\lambda, y) f_-(\lambda, x)\, d\lambda\right|
    \le |t|^{-1 / 2}.
  \]
\end{lemma}

\begin{proof}
  We first consider the case $x < 0 < y$. Then
  $f_+(\lambda, y) \to e^{i y \lambda}$ as
  $y \to \infty$ and
  $f_-(\lambda, x) \to e^{-i x \lambda}$ as
  $x \to -\infty$. So we can write
  \begin{align*}
    \left|\int_{-\infty}^\infty e^{it\lambda^2} e^{i\lambda(y - x)} \frac{\lambda \chi(x^2)}{W(\lambda)} m_+(\lambda, y)m_-(\lambda, x)\, d\lambda\right|
    &= \left|e^{i\partial_x^2 t} \left( \frac{\lambda \chi(\lambda^2)}{W(\lambda)} m_+(\lambda, y) m_-(\lambda, x)\right)^\vee\right| \\
    &\lesssim |t|^{-1 / 2}
    \left\| \left(\frac{\lambda \chi(\lambda^2)}{W(\lambda)} m_+(\lambda, y) m_-(\lambda, x)\right)^\vee \right\|_{L^1}.
  \end{align*}
  Let $\tau$ be the dual variable of $\lambda$, and
  write the inner function as $G(x, y; \tau)$, where
  $x, y$ are fixed. Then we would like to bound
  $\|G(x, y; \tau)\|_{L^1_\tau}$, for which
  it suffices to show $|G(x, y; \tau)| \lesssim \langle \tau \rangle^{-2}$.
  By Fourier duality, it suffices to bound
  the derivatives of $G$ in $\lambda$:
  \[
    \partial_\lambda^2 \left(\frac{\lambda \chi(\lambda^2)}{W(\lambda)} m_+(\lambda, y) m_-(\lambda, x)\right), \quad
    \partial_\lambda \left(\frac{\lambda \chi(\lambda^2)}{W(\lambda)} m_+(\lambda, y) m_-(\lambda, x)\right), \quad
    \left(\frac{\lambda \chi(\lambda^2)}{W(\lambda)} m_+(\lambda, y) m_-(\lambda, x)\right).
  \]
  Recalling that $\chi$ is compactly supported, it
  is sufficent to bound the $L^\infty$ norms of the
  above quantities. So we just need to show that
  $\partial_{\lambda}^j W(\lambda)$ and
  $\partial_{\lambda}^j m_\pm$
  are bounded for $j = 0, 1, 2$. Recall that
  \[
    W(\lambda)
    = W[f_+(\lambda, x), f_-(\lambda, x)]
    = [e^{i\lambda x} m_+(\lambda, x), e^{-i\lambda x} m_-(\lambda, x)]
    = [m_+(\lambda, 0), m_-(\lambda, 0)],
  \]
  where we assumed $W(0) \ne 0$.
  Note that the Wronskian is independent of the point
  at which we evaluate, so that $W(\lambda)$
  does not depend on $x$. We previously
  estimated
  $\partial_\lambda^j m_\pm$ for $x \ge 0$ and
  $j = 0, 1, 2$, so we have $L^\infty$ bounds
  for $\partial_\lambda^j W(\lambda)$ and
  $\partial_\lambda^j m_\pm$ for $j = 0, 1, 2$.
  This proves pointwise decay in this case.

  Now we consider the case $0 \le x < y$. We need
  to analyze $f_+(\lambda, y) f_-(\lambda, x)$ (note
  that $f_-(\lambda, x)$ behaves badly as
  $x \to \infty$, we only know that $f_-(\lambda, x)$
  might grow linearly as $x \to \infty$). To deal
  with this, we use
  some $1$-D scattering theory. Note that
  $f_\pm(\lambda, x)$ solve the ODE
  $-\partial_x^2 f_\pm + V f_\pm = \lambda^2 f$,
  so we can write
  \[
    f_\pm(\lambda, x) = \alpha_{\pm}(\lambda) f_{\mp} f(\lambda, x)
    + \beta_{\pm}(\lambda) f_{\mp}(-\lambda, x)
  \]
  for $\lambda \ne 0$. Since
  $f_-(\lambda, x)$ and $f_-(-\lambda, x)$ are
  linearly independent ($W[f_-(\lambda, x), f_-(-\lambda, x)] = 2i\lambda$ and $\lambda \ne 0$).
  Since $0 \le x < y$, we can use this to write
  \begin{align*}
    f_+(\lambda, y) f_-(\lambda, x)
    &= f_+(\lambda, y)(\alpha_-(\lambda) f_+(\lambda, x) + \beta_-(\lambda) f_+(-\lambda, x)) \\
    &= \alpha_-(\lambda) f_+(\lambda, y) f_+(\lambda, x)
    + \beta_-(\lambda) f_+(\lambda, y) f_+(-\lambda, x),
  \end{align*}
  and we can now make the same argument as in the
  previous case to get the decay estimate.

  In the final case $x < y \le 0$, the idea is the same
  as when $x_0 \le x < y$:
  We can write $f_+(\lambda, y)$ as a linear combination of
  $f_-(-\lambda, y)$ and $f_-(\lambda, y)$ and
  proceed as before.
\end{proof}

\begin{remark}
  This completes the proof of the low energy part.
\end{remark}

\begin{remark}
  Consider the two \emph{scattering relations} from above
  \begin{align*}
    f_+(\lambda, x)
    &= \alpha_+(\lambda) f_-(\lambda, x) + \beta_+(\lambda) f_-(-\lambda, x) \\
    f_-(\lambda, x)
    &= \alpha_-(\lambda) f_+(\lambda, x) + \beta_-(\lambda) f_+(-\lambda, x).
  \end{align*}
  We can rewrite the first equation as
  \begin{align*}
    \frac{1}{\beta_+(\lambda)} f_+(\lambda, x)
    = \frac{\alpha_+(\lambda)}{\beta_+(\lambda)} f_-(\lambda, x) + f_-(-\lambda, x),
  \end{align*}
  and we find that
  \begin{align*}
    \frac{1}{\beta_+(\lambda)} f_+(\lambda, x)
    &\xrightarrow[x \to +\infty]{} \frac{1}{\beta_+(\lambda)} e^{i\lambda x} \\
    \frac{\alpha_+(\lambda)}{\beta_+(\lambda)} f_-(\lambda, x)
    &\xrightarrow[x \to -\infty]{} \frac{\alpha_+(\lambda)}{\beta_+(\lambda)} e^{-i\lambda x} \\
    f_-(-\lambda, x)
    &\xrightarrow[x \to -\infty]{} e^{i\lambda x}.
  \end{align*}
  We call $T(\lambda) = 1 / \beta_+(\lambda)$ the
  \emph{transmission coefficient} and
  $R_+(\lambda) = \alpha_+(\lambda) / \beta_+(\lambda)$
  the \emph{reflection coefficient}.
  One also has the relation
  $|T(\lambda)|^2 + |R_+(\lambda)|^2 = 1$.
\end{remark}

\section{Distorted Fourier Transforms}

\begin{remark}
  We give an alternate approach to proving
  the low energy estimates via ``distorted Fourier
  transforms.'' Recall that the usual Fourier transform
  diagonalizes differentiation. The ``distorted
  Fourier transform'' $\widetilde{\mathcal{F}}$
  will diagonalize $-\partial_x^2 + V$, i.e. it will
  satisfy
  \[
    \widetilde{\mathcal{F}}((-\partial_x^2 + V) g)(\xi)
    = \xi^2 \widetilde{\mathcal{F}}(g).
  \]
  In Fourier space, the scattering relations from
  before become:
  \begin{align*}
    T(\xi) f_+(x, \xi)
    &= f_-(x, -\xi) + R_-(\xi) f_-(x, \xi) \\
    T(\xi) f_-(x, \xi)
    &= f_+(x, -\xi) + R_+(\xi) f_+(x, \xi)
  \end{align*}
  Recall that for $H = -\partial_x^2 + V$, we have
  the spectral resolution
  \[
    (H - (\xi^2 + i 0))^{-1}(x, y)
    = R(\xi^2 + i 0)(x, y)
    = \frac{f_+(x, \xi) f_-(y, \xi) \mathbbm{1}_{\{x \ge y\}} + f_-(x, \xi) f_+(y, \xi) \mathbbm{1}_{\{x < y\}}}{W[f_+(\cdot, \xi), f_-(\cdot, \xi)]}
  \]
  for $x, y \in \R$ and $\xi \in \R$.
\end{remark}

\begin{lemma}
  The density of the spectral resolution
  $E(d\xi^2)$ on $[0, \infty)$ can be written as
  \[
    \frac{E(d\xi^2)}{d\xi}(x, y)
    = \frac{|T(\xi)|^2}{2\pi} [f_+(x, \xi) f_+(y, -\xi) + f_-(x, \xi) f_-(y, -\xi)].
  \]
  Alternatively, it can also be written as
  \[
    \frac{E(d\xi^2)}{d\xi}(x, y)
    = \frac{1}{\pi}
    \begin{cases}
      \re [T(\xi) f_+(x, \xi) f_-(y, \xi)], & \text{if $x \ge y$}, \\
      \re [T(\xi) f_+(y, \xi) f_-(x, \xi)], & \text{if $x \ge y$}.
    \end{cases}
  \]
\end{lemma}

\begin{proof}
  Recall that (make a change of variables $\lambda = \xi^2$
  in the usual formula for $E(d\lambda)$)
  \[
    \frac{E(d\xi^2)}{d\xi}(x, y)
    = \frac{\xi}{\pi i} [R(\xi^2 + i0) - R(\xi^2 - i0)](x, y).
  \]
  Then we can write
  \[
    \frac{E(d\xi^2)}{d\xi}(x, y)
    = \frac{\xi}{2\pi} [W(\xi)^{-1} f_+(x, \xi) f_-(y, \xi) - W(-\xi)^{-1} f_+(x, -\xi) f_-(y, -\xi)].
  \]
  Using the scattering relations, we have
  $T(\xi) W(\xi) = -2 \xi i$, so we have
  \[
    \frac{E(d\xi^2)}{d\xi}(x, y)
    = \frac{1}{2\pi}(T(\xi) f_+(x, \xi) f_-(y, \xi) + T(-\xi) f_+(x, -\xi) f_-(y, -\xi)).
  \]
  Write $f_-(y, \xi)$ in the first term
  using $f_+(y, \xi)$ and $f_-(y, -\xi)$, and
  write $f_+(x, -\xi)$ in the second term
  as $f_-(x, \xi)$ and $f_+(x, \xi)$ to obtain
  (note that $T(-\xi) = T(\xi)$ and
  $R_{\pm}(-\xi) = \overline{R_{\pm}(\xi)}$)
  \begin{align*}
    \frac{E(d\xi^2)}{d\xi}(x, y)
    &= \frac{1}{2\pi} [T(\xi) f_+(x, \xi)(T(-\xi) f_+(y, -\xi) - R_-(-\xi) f_-(y, \xi)) \\
    & \quad \quad + T(-\xi)(T(\xi)f_-(x, \xi) - R_+(\xi) f_+(x, \xi)) f_-(y, -\xi)] \\
    &= \frac{|T(\xi)|^2}{2\pi} [f_+(x, \xi) f_+(y, -\xi) + f_-(x, \xi) f_-(y, -\xi)].
  \end{align*}
  This proves the case $x > y$, and $x < y$ follows
  similarly.
\end{proof}

\begin{definition}
  Define the \emph{distorted Fourier basis}
  \[
    e(x, \xi) = \frac{1}{\sqrt{2\pi}}
    \begin{cases}
      T(\xi) f_+(x, \xi) & \text{if $\xi \ge 0$}, \\
      T(-\xi) f_-(x, -\xi) & \text{if $\xi < 0$}.
    \end{cases}
  \]
\end{definition}

  \chapter{Mar.~5 --- Scattering Relations}

\section{Distorted Fourier Transform, Continued}

\begin{remark}
  Let $H = -\partial_x^2 + V$ and $P_{\mathrm{c}}$
  be the projection onto the continuous spectrum of
  $\mathcal{H}$. By spectral theory, we can write
  \[
    H P_{\mathrm{c}} f
    = \frac{1}{2\pi i} \int_0^\infty \lambda E(d\lambda) f,
  \]
  which gives
  \[
    P_c f
    = \frac{1}{2\pi i} \int_0^\infty E(d\lambda) f
    = \frac{1}{2\pi i} [R(\lambda + i 0) - R(\lambda - i 0)] f\, d\lambda
  .\]
  Setting $\lambda = \xi^2$, we find
  \begin{align*}
    P_{\mathrm{c}} f
    &= \frac{1}{2\pi i} \int_0^\infty 2\xi [R(\xi^2 + i 0) - R(\xi^2 - i 0)] f\, d\xi \\
    &= \int_{\R} \int_0^\infty
    \frac{|T(\xi)|^2}{2\pi} [f_+(x, \xi)f_+(y, -\xi) + f_-(x, \xi) f_-(y, -\xi)] f(y)\,
    d\xi dy \\
    &= 
    \int_{\R} \int_0^\infty
    \frac{|T(\xi)|^2}{2\pi} (f_+(x, \xi)f_+(y, -\xi)) f(y)\,
    d\xi dy
    +
    \int_{\R} \int_0^\infty
    \frac{|T(\xi)|^2}{2\pi} (f_-(x, \xi)f_-(y, -\xi)) f(y)\,
    d\xi dy.
  \end{align*}
  We keep the first piece and
  change variables $\xi \mapsto -\xi$ in the second
  piece to get
  \begin{align*}
    P_{\mathrm{c}} f
    &= \int_{\R} \int_0^\infty
    \frac{|T(\xi)|^2}{2\pi} (f_+(x, \xi)f_+(y, -\xi)) f(y)\,
    d\xi dy
    +
    \int_{\R} \int_{-\infty}^0
    \frac{|T(\xi)|^2}{2\pi} (f_-(x, -\xi)f_-(y, \xi)) f(y)\,
    d\xi dy \\
    &= \int_\R e(x, \xi) \int_\R \overline{e(y, \xi)} f(y)\, dy d\xi,
  \end{align*}
  where $e(x, \xi)$ is the distorted Fourier basis
  (note that
  $\overline{f_+}(y, \xi) = f_+(y, -\xi)$
  and $\overline{f_-}(y, -\xi) = f_-(y, \xi)$).
  Note when $V = 0$, we have
  $f_\pm = e^{\pm i x \xi}$ and
  $T(\xi) = 1$, so we recover the standard Fourier
  transform.
\end{remark}

\begin{corollary}
  Let $H = -\partial_x^2 + V$. Define the
  \emph{distorted Fourier transform} by
  \[
    \widetilde{F}(f) = \int_{\R} \overline{e(x, \xi)} f(x)\, dx \quad \text{and} \quad
    \widetilde{F}^{-1}(h) = \int_{\R} e(x, \xi) h(x)\, dx.
  \]
  Then for any Borel function $G$, we have
  the formula
  \[
    G(H) P_{\mathrm{c}} f
    = \widetilde{F}^{-1} [G(\xi) \widetilde{F}(f)(\xi)].
  \]
\end{corollary}

\begin{remark}
  Using the distorted Fourier transform, we can write
  \[
    e^{iHt} P_{\mathrm{c}} f
    = \int_\R e(x, \xi) e^{it \xi^2} \widetilde{\mathcal{F}}(f)(\xi)\, d\xi. \tag{$*$}
  \]
  Note that for $\xi \ge 0$, we have
  $e(x, \xi) = T(\xi) f_+(x, \xi) \sim e^{ix \xi}$
  as $x \to \infty$. So at least morally,
  \[
    \int_\R e(x, \xi) e^{it \xi^2} \widetilde{\mathcal{F}}(f)(\xi)\, d\xi
    = e^{-i\partial_x^2 t} [\mathcal{F}^{-1} [\widetilde{\mathcal{F}}(f)]].
  \]
  On the other hand, we do not know the precise
  behavior of $T(\xi) f_+(x, \xi)$ as $x \to -\infty$.
  But we can use the scattering relation and write
  $f_+$ in terms of $f_-$:
  \[
    T(\xi) f_+(x, \xi) = f_-(x, -\xi) + R_-(\xi) f_-(x, \xi),
  \]
  so we can write
  \[
    \int_{\R} e^{ix\xi} e^{it \xi^2} \widetilde{F}(f)(\xi)\, d\xi
    + \int_{\R} R_-(\xi) e^{-ix\xi} e^{it \xi^2} \widetilde{F}(f)(\xi)\, d\xi,
  \]
  which we can estimate as a $1$-D oscillatory
  integral. Then using $(*)$, we get
  $\|e^{iHt} P_{\mathrm{c}} f\|_{L^\infty} \lesssim |t|^{-1 / 2}$.
\end{remark}

\section{Scattering Relations}

\begin{remark}
  Recall that we used the scattering relations
  \begin{align*}
    T(\xi) f_+(x, \xi) = f_-(x, -\xi) + R_-(\xi) f_-(x, \xi), \\
    T(\xi) f_-(x, \xi) = f_+(x, -\xi) + R_+(\xi) f_+(x, \xi).
  \end{align*}
  Also recall that
  \[
    f_+(x, \xi) = \alpha_+(\xi) f_-(x, \xi) + \beta_+(\xi) f_-(x, -\xi) \tag{$\star$}
  \]
  since $f_+(x, \xi)$ and $f_-(x, -\xi)$ are
  linearly independent for $\xi \ne 0$. Similarly,
  we can write
  \[
    f_-(x, \xi)
    = \alpha_-(\xi) f_+(x, \xi)
    + \beta_-(\xi) f_+(x, -\xi). \tag{$\star\star$}
  \]
  The scattering relations then come from dividing
  these equations by $\alpha_{\pm}(\xi)$. We will now
  try to understand these coefficients in more detail.
  Recall that
  \begin{align*}
    W(\lambda)
    = W[f_+(x, \lambda), f_-(x, \lambda)]
    &= W[\alpha_+(\lambda) f_-(x, \lambda) + \beta_+(\lambda) f_-(x, -\lambda), f_-(x, \lambda)] \\
    &= \beta_+(\lambda) W[f_-(x, -\lambda), f_-(x, \lambda)].
  \end{align*}
  Now passing $x \to -\infty$ (since the Wronskian
  is independent of the point at which we evaluate),
  we have
  \[
    W(\lambda) = \beta_+(\lambda)
    \begin{vmatrix}
      e^{i\lambda x} & i\lambda e^{i\lambda x} \\
      e^{-i\lambda x} & -i\lambda e^{-i\lambda x} \\
    \end{vmatrix}
    = -2i\lambda \beta_+(\lambda).
  \]
  Using the same idea, we can compute that
  \[
    W[f_+(x, \lambda), f_-(x, -\lambda)]
    = \alpha_+(\lambda) W[f_-(x, \lambda), f_-(x, -\lambda)]
    = 2i\lambda \alpha_+(\lambda).
  \]
  Continuing in this manner, we also get
  $W[f_-(x, \lambda), f_+(x, -\lambda)] = -2i\lambda \alpha_-(\lambda)$
  and
  \[
    \beta_-(\lambda) = \beta_+(\lambda)
    = \frac{W(\lambda)}{-2i\lambda}.
  \]
  From this, we can also conclude
  \[
    \alpha_+ = \frac{W[f_+(\lambda), f_-(-\lambda)]}{2i\lambda} \quad \text{and} \quad
    \alpha_- = \frac{W[f_-(\lambda), f_+(-\lambda)]}{-2i\lambda}.
  \]
  Substituting $(\star\star)$ into $(\star)$, we get
  \begin{align*}
    f_+(x, \xi)
    &= \alpha_+(\xi) (\alpha_-(\xi) f_+(x, \xi) + \beta_-(\xi) f_+(x, -\xi))
    + \beta_+(\xi) (\alpha_-(-\xi) f_+(x, -\xi) + \beta_-(-\xi) f_+(x, \xi)) \\
    &= [\alpha_+(\xi) \beta_-(\xi) + \beta_+(\xi) \alpha_-(-\xi)] f_+(x, -\xi)
    + [\alpha_+(\xi) \alpha_-(\xi) + \beta_+(\xi) \beta_-(-\xi)] f_+(x, \xi).
  \end{align*}
  This then tells us that
  \[
    \alpha_+(\xi) \alpha_-(\xi) + \beta_+(\xi) \beta_-(-\xi) = 1
    \quad \text{and} \quad
    \alpha_+(\xi) \beta_-(\xi) + \beta_+(\xi) \alpha_-(-\xi) = 0.
  \]
  Using our formulas for $\alpha_{\pm}$ and $\beta_{\pm}$ in the second equation,
  we get
  \[
    2i \xi \beta(-\xi) = W(-\xi) = \overline{W(\xi)}
    = 2 i\xi \beta(\xi) \quad \text{and} \quad
    -2i \xi \alpha_+(\xi) = 2i\xi \overline{\alpha_-(\xi)},
  \]
  so $\beta(-\xi) = \beta(\xi)$, and
  $\alpha_+(\xi) = -\overline{\alpha_-(\xi)}$
  or $\alpha_-(\xi) = -\overline{\alpha_+(\xi)}$.
  By the first equation, this implies
  \[
    -|\alpha_+(\xi)|^2 + |\beta(\xi)|^2 = 1.
  \]
  Dividing by $|\beta(\xi)|^2$ and using some algebra,
  we find that
  \[
    \frac{1}{|\beta(\xi)|^2}
    + \frac{|\alpha_+(\xi)|^2}{|\beta(\xi)|^2} = 1.
  \]
  Setting $T(\xi) = 1 / \beta(\xi)$ and
  $R_{\pm}(\xi) = \alpha_{\mp}(\xi) / \beta(\xi)$,
  we get the relationss
  \[
    |T(\xi)|^2 + |R_{\pm}(\xi)|^2 = 1.
  \]
  We can also define
  \[
    S(\xi) =
    \begin{bmatrix}
      T(\xi) & R_-(\xi) \\
      R_+(\xi) & T(\xi)
    \end{bmatrix},
  \]
  which is called the \emph{scattering matrix}.
\end{remark}

  \chapter{Mar.~10 --- Compact Operators}

\section{Conclusion on Dispersive Decay}
\begin{remark}
  Recall that we have previously shown if
  $V$ is sufficiently nice, then
  \[
    \|e^{iHt} P_{\mathrm{c}} f\|_{L^\infty}
    \lesssim |t|^{-\frac{1}{2}}.
  \]
  In particular, if we consider the following equation:
  \[
    \begin{cases}
      i \partial_t \psi - \partial_x^2 \psi + V \psi = F, \\
      \psi(0) = f,
    \end{cases}
  \]
  then we have the Strichartz estimate (which
  follows by a $TT^*$ argument)
  \[
    \|P_{\mathrm{c}} \psi\|_{L^6_{t, x}} \lesssim \|f\|_{L^2_x} + \|F\|_{L^{6 / 5}_{t, x}}.
  \]
\end{remark}

\section{Compact Operators}

\begin{definition}
  A linear operator $A : \HH \to \HH$ is \emph{compact}
  if it satisfies one of the following (equivalent)
  conditions:
  \begin{enumerate}
    \item $A$ takes bounded sets to relatively
      compact sets;
    \item $A$ takes bounded sets to sets that
      have $\epsilon$-nets for all $\epsilon > 0$ (see below);
    \item if $\{u_n\}$ is a bounded sequence in
      $\HH$, then $\{A u_n\}$ has a strongly
      convergent subsequence;
    \item $A$ takes weakly convergent to strongly
      convergent sequences.
  \end{enumerate}
\end{definition}

\begin{definition}
  Let $K$ be a metric space. We say that $E \subseteq K$
  is an $\epsilon$-net of $K$ if for every $x \in K$,
  we can find $y \in E$ such that $d(x, y) < \epsilon$.
\end{definition}

\begin{theorem}[Spectral theorem for compact self-adjoint operators]
  Let $A : \HH \to \HH$ be compact and self-adjoint.
  Then we can find an orthonormal basis
  $\{\phi_n\}_{n = 1}^\infty$ of $\HH$ such that
  \[
    A \phi_n = \lambda_n \phi_n,
    \quad \lambda_n \in \R,
  \]
  and the $\lambda_n$ can only accumulate at zero.
\end{theorem}

\begin{definition}
  The \emph{essential spectrum} of $A$ is the spectrum
  of $A$ minus the isolated eigenvalues of finite
  multiplicity.
\end{definition}

\begin{theorem}[Weyl's criterion]
  Let $A, B : \HH \to \HH$ be self-adjoint. Assume
  \[
    (A + M)^{-1} - (B + M)^{-1} = K
  \]
  for some compact operator $K$ and
  constant $M$. Then the essential spectrum of
  $A$ and $B$ are the same.
\end{theorem}

\begin{proof}
  We only sketch a proof here. We first claim that if
  $\lambda$ is in the essential spectrum of $A$, then
  for every $\epsilon > 0$, there exists
  an infinite dimensional subspace $L$ of $\HH$
  such that
  \[
    \|(A + M)^{-1} f - (\lambda + M)^{-1} f\|_{\HH}
    \le \epsilon \|f\|_{\HH}
  \]
  for all $f \in L$.
  This is by spectral theory, as we can write
  \[
    (A + M)^{-1} f - (\lambda + M)^{-1} f
    = \int \left(\frac{1}{t + M} - \frac{1}{\lambda + M}\right) E(dt) f.
  \]
  Then since $\lambda$ is fixed, for any $\epsilon > 0$
  we can find $\delta > 0$ (by continuity) such that
  \[
    \int_{\lambda - \delta}^{\lambda + \delta}
    \left|\frac{1}{t + M} - \frac{1}{\lambda + M}\right|^2 \langle E(d\lambda) f, f \rangle \le \epsilon^2.
  \]
  We can take $L = \Ran E([\lambda - \delta, \lambda + \delta])$.
  By spectral theory again, for any $f \in L$, we have
  \[
    \|(A + M)^{-1} f - (\lambda + M)^{-1} f\|_{\HH}^2
    = \int_{\lambda - \delta}^{\lambda + \delta}
    \left|\frac{1}{t + M} - \frac{1}{\lambda + M}\right|^2 \langle E(d\lambda) f, f \rangle
    \le \epsilon^2 \|f\|_{\HH}^2,
  \]
  since $E(d\lambda) f = f$ as
  $f \in L$. So it only remains to check that $L$ is
  infinite dimensional. To see this, note that if
  $L$ was finite dimensional, then
  $\langle E([\lambda - \delta, \lambda + \delta]) f, f \rangle$
  is a purely atomic measure, a contradiction with
  $\lambda$ being in the essential spectrum of $A$.\footnote{This claim says that the essential spectrum is related to the infinite dimensional subspaces of $\HH$.}

  Now the converse of the claim is also true (exercise).
  We will use this to show that $\lambda$ is
  essential spectrum of $B$. By the converse of
  the claim, it suffices to find $\widetilde{L}$
  such that $\dim \widetilde{L} = \infty$ and
  for every $g \in \widetilde{L}$,
  \[
    \|(B + M)^{-1} g - (\lambda + M)^{-1} g\|_{\HH}
    \le 2 \epsilon \|g\|_{\HH}.
  \]
  We now claim that there $\widetilde{L}$ with
  $\dim \widetilde{L} = \infty$ and $\widetilde{L} \subseteq L$ such that
  \[
    \|PK^2 P g\|_{\HH} \le 2 \epsilon \|g\|_{\HH}
  \]
  for all $g \in \widetilde{L}$, where
  $P = E([\lambda - \delta, \lambda + \delta])$.
  To see this, note that $T = PK^2 P$ is a self-adjoint
  compact operator, so by the spectral theorem
  we can find $\{\eta_n\}_{n = 1}^\infty$ and
  $\{\phi_n\}_{n = 1}^\infty$ with
  $T\phi_n = \eta_n \phi_n$ and $\eta_n \to 0$ (Since
  $L$ is an infinite dimensional subspace of $\HH$,
  we can find a basis $\{b_m\}_{m = 1}^\infty$ for $L$.
  Then $b_n \to 0$ weakly. A compact operator takes
  weakly convergent sequences to strongly convergent
  sequences, so $T b_n \to 0$ strongly. In particular,
  it is impossible that $\{\eta_n\}_{n = 1}^\infty$ is
  a finite subset of $\R$. So it has an
  accumulation point by the Bolzano-Weierstrass
  theorem, which must be $0$.) Take
  \[
    \widetilde{L} = \mathrm{span}\{\phi_n : |\eta_n| \le \epsilon^2\},
  \]
  which is infinite dimensional since
  $\eta_n \to 0$. Then for any $g \in \widetilde{L}$,
  \[
    \|Kg\|^2_{\HH} = \langle PK^2 P g, g \rangle
    \le \epsilon^2 \|g\|^2_{\HH}.
  \]
  We can write
  \begin{align*}
    (B + M)^{-1} g - (\lambda + M)^{-1} g
    &= (B + M)^{-1} g - (A + M)^{-1} g
    + (A + M)^{-1} g - (\lambda + M)^{-1} g \\
    &= K g + [(A + M)^{-1} g - (\lambda + M)^{-1} g],
  \end{align*}
  and thus we have
  \[
    \|(B + M)^{-1} g - (\lambda + M)^{-1} g\|_{\HH}
    \le \|Kg\|_{\HH} + \|(A + M)^{-1} g - (\lambda + M)^{-1} g\|_{\HH}
    \le 2 \epsilon \|g\|_{\HH}.
  \]
  This implies that $\lambda$ lies in the essential
  spectrum of $B$. We conclude the theorem by symmetry.
\end{proof}

\begin{example}
  Let $A = -\Delta$ (with essential spectrum
  $[0, \infty)$)
  and $B = -\Delta + V$, where $V$ decays to
  $0$ as $|x| \to \infty$. Then for $M > 0$,
  we can see that (using
  the resolvent identities)
  \begin{align*}
    (A + M)^{-1} - (B + M)^{-1}
    &= (-\Delta + M)^{-1} - (-\Delta + V + M)^{-1} \\
    &= \sum_{n = 1}^\infty (-\Delta + M)^{-1}
    (-V(-\Delta + M)^{-1})^n.
  \end{align*}
  Notice that $\|(-\Delta + M)^{-1}\|_{L^2 \to L^2} \le 1 / M$,
  since
  $((-\Delta + M)^{-1})^\wedge = 1 / (\xi^2 + M) \le 1 / M$
  for $M > 0$. Now
  \[
    \|V(-\Delta + M)^{-1}\|_{L^2 \to L^2} \le \frac{\|V\|_{L^\infty}}{M},
  \]
  so if $M > \|V\|_{L^\infty}$, then
  we have a convergent series. So in order to show
  that $(A + M)^{-1} - (B + M)^{-1}$ is compact, it
  suffices to show $K = V(-\Delta + M)^{-1}$ is compact
  (since the sum and composition of compact operators
  is again compact). Note that a loss of compactness
  can come from either the physical or the frequency
  side. But we have
  \[
    ((-\Delta + M)^{-1})^\wedge = \frac{1}{\xi^2 + M},
  \]
  which localizes a function on the frequency side,
  and the decay of $V$ localizes a function on the
  physical side. This gives intuition for why
  $K$ should be compact.

  A more rigorous argument is the following: Let
  $B \subseteq L^2(\R^d)$ be the unit ball, and set
  \[
    W = \{w = V(-\Delta + M)^{-1} v : \|v\|_{L^2} \le 1\}.
  \]
  We would like to show that $W$ is relatively compact,
  which we will do by appealing to the Arzela-Ascoli
  theorem. For this, we must check boundedness and
  equicontinuity. Define
  \[
    W_\epsilon = \{w_\epsilon = \chi_\epsilon * w : w \in W\},
  \]
  where $\chi_\epsilon$ is the standard mollifier.
  For boundedness, we have
  \[
    |w_\epsilon(x)| = |\chi_\epsilon * w|
    \lesssim \epsilon^{-n / 2} \|w\|_{L^2}
    \lesssim \epsilon^{-n / 2}
  \]
  and $\|w_\epsilon(x) - w(x)\|_{L^2} \lesssim \epsilon$,
  which proves boundedness. For
  equicontinuity, we note that
  \[
    |w_\epsilon(x) - w_\epsilon(y)|
    \le \left|\int (\chi_\epsilon(x - z) - \chi_\epsilon(y - z))\right| w(z)\, dz
    \lesssim \epsilon^{-n / 2} \epsilon^{-1} |x - y|,
  \]
  which proves equicontinuity. Thus we can apply
  the Arzela-Ascoli theorem to get that $W_\epsilon$ is
  relatively compact. Since this holds for all
  $\epsilon > 0$, we can also conclude that
  $W$ is relatively compact.

  Finally, by Weyl's criterion, we conclude that
  the essential spectrum of $B$ is also $[0, \infty)$.
\end{example}

  \chapter{Mar.~12 --- Spectral Theory and Negative Eigenvalues}

\section{Spectral Theory for \texorpdfstring{$-\Delta + V$}{-Delta + V}}
\begin{lemma}
  Suppose $V \to 0$ as $|x| \to \infty$.
  Then the following are equivalent:
  \begin{enumerate}
    \item There exists an eigenvalue $\lambda < 0$
      of $H = -\Delta + V$.
    \item There exists $f \in H^2$
      such that $\langle H f, f \rangle < 0$.
  \end{enumerate}
\end{lemma}

\begin{proof}
  $(1 \Rightarrow 2)$ Let $f$ be the eigenfunction,
  so that $H f = \lambda f$. Then
  $\langle Hf, f \rangle  = \lambda \langle f, f \rangle < 0$.

  $(2 \Rightarrow 1)$
  If $\langle Hf , f \rangle < 0$, then by the
  spectral theorem, the spectrum of $H$ must extend
  below zero. But the essential spectrum of
  $H$ is $[0, \infty)$, so any part of the spectrum below
  must be an eigenvalue.
\end{proof}

\begin{theorem}[Birman-Schwinger]
  In $\R^3$, let $V^{-}$ denote the negative part
  of $V$, and suppose that
  \[
    \int_{\R^3} \int_{\R^3} \frac{|V^-(x)| |V^-(y)|}{|x - y|} \,dx dy < (4\pi)^2.
  \]
  Then $H = -\Delta + V$ has no negative eigenvalues.
\end{theorem}

\begin{remark}
  If $V \ge 0$, then integration by parts yields
  \[
    \langle (-\Delta + V) f, f \rangle
    = \langle \nabla f, \nabla f \rangle + \langle V f, f \rangle
    = \|\nabla f\|^2 + \langle V f, f \rangle \ge 0.
  \]
  So the previous lemma implies that $-\Delta + V$
  has no negative eigenvalues.
\end{remark}

\begin{proof}
  We argue by contradiction. Assume the exists an
  eigenvalue $\lambda < 0$. Then by the previous lemma,
  we can find $f \in H^2$ such that
  $\langle Hf, f \rangle < 0$. Write
  $V = V^+ + V^-$,
  where $V^+, V^-$ are the positive and negative parts
  of $V$, respectively. Since $V^+ \ge 0$, we have
  \begin{align*}
    \langle Hf, f \rangle
    = \langle (-\Delta + V) f, f \rangle
    &= \langle (-\Delta + V^+ + V^-) f, f \rangle \\
    &= \langle (-\Delta + V^-) f, f \rangle + \langle V^+ f, f \rangle
    \ge \langle (-\Delta + V^-) f, f \rangle.
  \end{align*}
  Let $\widetilde{H} = -\Delta + V^-$, so
  $\langle \widetilde{H} f, f \rangle < 0$.
  By the previous lemma, $\widetilde{H}$ has a
  negative eigenvalue $\widetilde{\lambda}$, and
  we can find $\phi$ such that
  $(-\Delta + V^-) \phi
    = \widetilde{H} \phi = \widetilde{\lambda} \phi$.
    Then $(-\Delta - \widetilde{\lambda}) \phi = -V^- \phi \in L^2$, and we can write
    \[
      \phi = -(-\Delta - \widetilde{\lambda})^{-1} V^- \phi,
    \]
    where $(-\Delta - \widetilde{\lambda}$ is invertible
    since $-\widetilde{\lambda} > 0$. Note that
    $V^- \le 0$, so
    \[
      \underbrace{\sqrt{|V^-|}}_{\psi} = \underbrace{\sqrt{|V^-|} (-\Delta - \widetilde{\lambda})^{-1}\sqrt{|V^-|}}_T \underbrace{(\sqrt{|V^-|}\phi)}_{\psi}.
    \]
    The above says $\psi = T \psi$, so
    $\|T\|_{L^2 \to L^2} \ge 1$ (recall that
    $\|T\|_{L^2 \to L^2} = \sup_{f \in L^2} (\|Tf\|_{L^2} / \|f\|_{L^2}))$.
    But
    \[
      Tf(x)
      = \sqrt{|V^{-}|(x)} \int_\R (-\Delta - \widetilde{\lambda})^{-1}(x, y) \sqrt{|V^{-}|(y)} f(y) \,dy.
    \]
    Such a $T$ is called a \emph{Hilbert-Schmidt operator}.
    These operators have kernels, so we have
    \begin{align*}
      \|T\|_{L^2 \to L^2}
      \le \|T\|_{\mathrm{H.S.}}
      &= \left(\int_{\R^3} \int_{\R^3} |\mathrm{kernel}(T)|^2(x, y)\, dxdy\right)^{1 / 2} \\
      &= \left(\int_{\R^3} \int_{\R^3} \sqrt{|V^{-}|}(x) ((-\Delta - \widetilde{\lambda})^{-1}(x, y) \sqrt{|V^-|}(y))^2\, dxdy\right)^{1 / 2} \\
      &= \left(\int_{\R^3} \int_{\R^3} |V^{-}|(x) \frac{e^{-i\sqrt{\widetilde{\lambda}}|x - y|}}{(4\pi|x - y|)^2} |V^-|(y) \, dxdy\right)^{1 / 2} \\
      &= \left(\int_{\R^3} \int_{\R^3} \frac{|V^-|(x) |V^-(y)|}{(4\pi |x - y|)^2} \, dxdy\right)^{1 / 2}
      < 1
    \end{align*}
    by assumption, which contradicts
    $\|T\|_{L^2 \to L^2} \ge 1$.
\end{proof}

\begin{remark}
  The full version of Birman-Schwinger says that
  \[
    \text{\# of negative eigenvalues of $H$}
    \le \frac{1}{(4\pi)^2} \int_{\R^3} \int_{\R^3} \frac{|V^-(x)| |V^-(y)|}{|x - y|} \,dx dy.
  \]
  A proof can be seen in Reed-Simon IV.
\end{remark}

\begin{lemma}
  In one dimension, if $V \in L^1(\R)$ and
  \[
    \int_{-\infty}^\infty V(x)\, dx < 0,
  \]
  then $H = -\partial_x^2 + V$ has a negative
  eigenvalue.
\end{lemma}

\begin{proof}
  By the first lemma from the day, it suffices
  to find $f \in H^2$ such that $\langle Hf, f \rangle < 0$.
  Let $\phi$ be a nice function, and
  set $f(x) = \phi(\epsilon x)$ for
  $\epsilon \ll 1$ to be determined. We have
  \begin{align*}
    \langle Hf, f \rangle
    = \langle f', f' \rangle + \langle Vf, f \rangle
    &= \epsilon^2 \int |\phi'(\epsilon x)|^2\, dx
    + \int V \phi^2(\epsilon x)\, dx \\
    &= \epsilon \int |\phi'|^2\, dx
    + \int V \phi^2(\epsilon x) \, dx
    \xrightarrow[\epsilon \to 0]{} 0 + |\phi^2(0)|^2 \int V\, dx < 0
  \end{align*}
  by the dominated convergence theorem.
  Thus taking $\epsilon$ small enough, we can get
  $\langle Hf, f \rangle < 0$.
\end{proof}

\begin{remark}
  Consider
  \[
    (*) = \iint \frac{|V(x)| |V(y)|}{|x - y|^2}\, dxdy
  \]
  in $\R^3$. We know that if $(*)$ is small enough,
  then $H$ behaves similarly to $-\Delta$.
  The \emph{Kato norm} of $V$ is
  \[
    \|V\|_{K}
    = \sup_x \int \frac{|V(y)|}{|x - y|}\, dy.
  \]
  This norm is scaling invariant. In 2004,
  Rodnianski-Schlag showed that
  if $\|V\|_K < 4\pi$, then
  \[
    \|e^{itH} P_{\mathrm{ac}} f\|_{L^\infty}
    \lesssim |t|^{-3 / 2} \|f\|_{L^1}.
  \]
  Another result by Beceanu-Goldberg in 2010 showed
  that if $\|V\|_{K} < \infty$ and $V$ is ``nice''
  such that $H$ has
  ``no zero resonance'' nor ``zero eigenfunctions,''
  then
  \[
    \|e^{itH} P_{\mathrm{ac}} f\|_{L^\infty}
    \lesssim |t|^{-3 / 2} \|f\|_{L^1}.
  \]
\end{remark}

\section{Agmon Bound}
\begin{remark}
  We consider the equation $(-\Delta + V) \psi = E\psi$
  with $E < 0$ and $\psi \in H^2$.
\end{remark}

\begin{theorem}[Agmon bound]
  Let $V \in C(\R^d)$ and $V \to 0$ as $|x| \to \infty$.
  Then
  \[
    \int_{\R^d} e^{\alpha |x|} |\psi(x)|^2\, dx < \infty
  \]
  for some $\alpha > 0$. If $V$ is smooth and
  $\|D^\gamma V\|_{L^\infty} < \infty$ for
  $|\gamma| \ge (n + 1) / 2$, then
  $|\psi(x)| \lesssim e^{-\alpha |x| / 2}$.
\end{theorem}

\begin{remark}
  Some intuition is the following:
  Consider $-\psi'' + V(x) \psi = E\psi$ in $1$-D,
  and suppose that
  \[
    \psi(x) \sim \exp\left(-\int_0^x\sqrt{(V - E)^+}\, dy\right).
  \]
  Then $\psi'(x) \sim -\sqrt{(V - E)^+} \psi(x)$ and
  $\psi''(x) \sim (V - E)^+ \psi(x) - (V'(x) / \sqrt{(V - E)^+}) \psi(x)$.
  In the limit $x \to \infty$, we have
  $V \to 0$ and $E < 0$, so
  $(V - E)^+ = V - E$. We also have $V' \to 0$, so
  $V'(x) / \sqrt{(V - E)^+}$ is small and so
  $\psi$ approximately solves
  $-\psi'' + V(x) \psi = E\psi$.
\end{remark}

\begin{proof}
  Define the \emph{Agmon metric} by
  \[
    P_E(x) = \inf_{\text{$\gamma$ a path $0 \to x$}}
    \int_0^1 \sqrt{(V(\gamma(t)) - E)^+} |\gamma'(t)|\, dt.
  \]
  Then one can check directly that $|\nabla P_E(x)| \le \sqrt{(V(x) - E)^+}$
  and $P_E(x) \le (\|V\|_{L^\infty} + |E|)^{1 / 2} |x|$.
  The crucial estimate is the following: Let
  $w(x) = \exp(\min\{2(1 - \epsilon P_E(x), N)\})$
  for some $0 < \epsilon \ll 1$ and $N \gg 1$, and let
  $\phi \in C^\infty$ with $\phi = 1$ for
  $|x| \gg 1$ and $\supp \phi \subseteq \{x : V(x) - E > 0 \}$.
  Then
  \[
    \int w(x) |\psi(x)|^2 \phi^2(x)\, dx \lesssim \|\psi\|_{L^2}^2.
  \]
  We will finish the proof next class.
\end{proof}

\begin{remark}
  In classic mechanics, the \emph{total energy} of
  a particle is given by
  \[
    E = V(x) + \frac{1}{2} mv^2,
  \]
  where $V$ is the \emph{potential energy} and $mv^2 / 2$
  is the \emph{kinetic energy}. Note that this means
  $V \le E$, so we cannot find a classical particle
  with $V > E$. For a quantum particle, however,
  we can find them in the region
  $R = \{x \in \R^n : V > E\}$. This region $R$ is
  called the \emph{classically-forbidden region}.
\end{remark}

  \chapter{Mar.~24 --- Agmon Bound}

\section{Proof of the Agmon Bound}

\begin{proof}[Proof of Theorem \ref{thm:agmon}]
  Define the \emph{Agmon metric} by
  \[
    P_E(x) = \inf_{\text{$\gamma$ a path $0 \to x$}}
    \int_0^1 \sqrt{(V(\gamma(t)) - E)^+} |\gamma'(t)|\, dt.
  \]
  Then one can check directly that $|\nabla P_E(x)| \le \sqrt{(V(x) - E)^+}$
  and $P_E(x) \le (\|V\|_{L^\infty} + |E|)^{1 / 2} |x|$.
  The crucial estimate is the following: Let
  $w(x) = \exp(\min\{2(1 - \epsilon P_E(x), N)\})$
  for some $0 < \epsilon \ll 1$ and $N \gg 1$, and let
  $\phi \in C^\infty$ with $\phi = 1$ for
  $|x| \gg 1$ and $\supp \phi \subseteq \{x : V(x) - E > 0 \}$.
  Then
  \[
    \int w(x) |\psi(x)|^2 \phi^2(x)\, dx \lesssim \|\psi\|_{L^2}^2.
  \]
  Note that this is finite since $\psi$ is an
  eigenfunction, so it lies in $L^2$. Take
  $\delta > 0$ to be determined later. Take
  $\phi$ as above but with $V - E > \delta$ in the
  support of $\phi$. Set $w_N(x) = \exp(\min\{2(1 - \epsilon P_E(x)), N\})$, and
  \[
    I_N = \int w_N(x) |\psi(x)|^2 \phi^2(x)\, dx.
  \]
  We claim that $I_n \le C$ independent of
  $N$. To see this, note that
  \begin{align*}
    \delta I_N
    \le \int (V - E) w_N(x) |\psi(x)|^2 \phi^2(x)\, dx
    &= \int (V - E) \psi(x) \overline{\psi(x)} w_N(x) \phi^2(x) \, dx \\
    &= \int (\Delta \psi) \overline{\psi(x)} w_N(x) \phi^2(x)\, dx
  \end{align*}
  since $-\Delta \psi + V \psi = E\psi$. Then
  integrate by parts to get
  \begin{align*}
    \delta I_N
    &\le -\int |\nabla \psi|^2 w_N(x) \phi^2(x)\, dx
    + 2\int |\nabla \psi| |\overline{\psi}| |w_N(x)| |\phi| |\nabla \phi|\, dx \\
    & \quad \quad + 2(1 - \epsilon) \int |\nabla \psi| |\overline{\psi}| |w_N| |\phi^2| \sqrt{V - E}\, dx, \tag{$*$}
  \end{align*}
  where we used $|\nabla P_E| \le \sqrt{(V - E)^+}$.
  For the second term, we can apply Cauchy-Schwarz to
  get
  \[
    2\int |\nabla \psi| |\overline{\psi}| |w_N(x)| |\phi| |\nabla \phi|\, dx
    \le \epsilon \int |\nabla \psi|^2 w_N(x) \phi^2\, dx
    + \frac{1}{\epsilon} \int |\psi|^2 w_N(x) |\nabla \phi|^2\, dx.
  \]
  Note that the $|\nabla \phi|^2$ term is compactly
  supported (since $\phi(x) = 1$ for all $|x|$ large
  enough). The first term above of the same form
  as the first term in $(*)$, so we can combine these
  later (with a coefficient of $\epsilon - 1$
  so far). For the third term in $(*)$, we can write
  \begin{align*}
    2(1 - \epsilon) \int |\nabla \psi| |\overline{\psi}| |w_N| |\phi^2| \sqrt{V - E}\, dx
    &\le (1 - \epsilon) \int |\nabla \psi|^2 w_N \phi^2\, dx \\
    &\quad \quad + (1 - \epsilon) \int |\psi|^2 (V - E) w_N(x) \phi^2\, dx
  \end{align*}
  The first term here also matches with the
  first term in $(*)$, and the three terms cancel
  out (we previously had a coefficient of $\epsilon - 1$).
  Now the remaining term is actually the same
  as the original bound on $\delta I_N$, but with a
  coefficient of $1 - \epsilon$ instead of $1$. Putting
  all of this together, we get
  \[
    \epsilon \int |\psi|^2 (V - E) w_N(x) \phi^2\, dx
    \le \frac{1}{\epsilon} \int |\psi|^2 w_N(x) |\nabla \psi|^2\, dx.
  \]
  Thus we have the bound
  \[
    |\delta I_N| \epsilon \le \frac{1}{\epsilon} \int |\psi|^2 w_N(x) |\nabla \psi|^2\, dx.
  \]
  The term $|\nabla \psi|^2$ is compactly supported
  and $w_N(x)$ is bounded (depending on $N$).
  Taking $N \to \infty$,
  \[
    |\delta I_N| \epsilon
    \le \frac{1}{\epsilon} \int_{|x| \le R_{\delta, V, E}} e^{2C_{V, E} |x|} |\psi(x)|^2\, dx
    \lesssim \frac{1}{\epsilon} e^{2C_{V, E} R_{\delta, V, E}} \int |\psi|^2\, dx,
  \]
  where $\delta, \epsilon$ are fixed and small.
  Thus $I_N < \infty$ independent of $N$.
  So if we pass $N \to \infty$, we have
  \[
    \int e^{2(1 - \epsilon) P_E(x)} |\psi|^2 \phi^2(x)\, dx < \infty.
  \]
  Note that
  the complement of $A = \{x : \phi(x) = 1\}$ is
  compact, so the integral on $A^c$ is finite, which
  gives
  \[
    \int e^{2(1 - \epsilon) P_E(x)} |\psi|^2\, dx < \infty.
  \]
  To convert $2(1 - \epsilon) P_E(x)$ to $\alpha |x|$,
  notice that in the support of $\phi$,
  we have $V - E > \delta$, so
  \[
    P_E(x) = \inf_{\text{$\gamma$ a path $0 \to x$}}
    \int_0^1 \sqrt{(V(\gamma(t)) - E)^+} |\gamma'(t)|\, dt
    \le (\|V\|_{L^\infty} + |E|) |x|.
  \]
  Also, $(V - E)^{1 / 2} |x| \lesssim P_E(x)$
  in the support of $\phi$, so
  $|P_E(x)| \sim (V - E)^{1 / 2} |x| \ge |E|^{1 / 2} |x|$ in this region.
  Thus if we take $\alpha = 2(1 - \epsilon) |E|^{1 / 2}$,
  we get the desired weighted $L^2$ bound.

  Now we show the pointwise bound. Since
  $(-\Delta + V) \psi = E \psi$, by elliptic
  theory we get $\psi \in H^2$. So
  \[
    \int (|\nabla \psi|^2 + V|\psi|^2)
    = \int E |\psi|^2 < 0
    \implies \int |\nabla \psi|^2 \le \|V\|_{L^\infty}
    \int |\psi|^2 < \infty.
  \]
  Taking $\partial_\alpha$, we get
  $-\Delta (\partial_\alpha \psi) + \partial_\alpha(V \psi) = E(\partial_\alpha \psi)(\partial_{\alpha} \overline{\psi})$, so
  integration by parts gives
  \begin{align*}
    \int |\nabla \partial_\alpha \psi|^2
    &\le \int |\partial_\alpha (V\psi) |\partial_\alpha \psi|\, dx
    \le \int |\partial_\alpha V\psi| |\partial_\alpha \psi|\, dx
    + \int V|\partial_\alpha \psi|^2\, dx \\
    &\le \|\partial_\alpha V\|_{L^\infty}
    \int |\psi| |\partial_\alpha \psi|^2\, dx
    + \int V |\partial_\alpha \psi|^2\, dx
    \lesssim \|\partial_\alpha V\|_{L^\infty} \left(\int |\psi|^2 + |\partial_\alpha \psi|^2\right) \\
    &\lesssim (\|\partial_\alpha V\|_{L^\infty} + 1) (\|V\|_{L^\infty} + 1) \int |\psi|^2.
  \end{align*}
  By induction, one can show
  \[
    \int |\nabla \partial_\gamma \psi|^2
    \le C \|V\|_{C^\gamma} \|\psi\|_{L^2}^2.
  \]
  Note that if we consider a compact region, then
  the pointwise bound is immediate:
  \[
    \int_{B_R(x_0)} e^{\alpha |x|} |\psi(x)|^2\, dx
    \sim e^{\alpha |x_0|} \int_{B_R(x_0)} |\psi|^2\, dx,
  \]
  and we can use Sobolev embedding to convert this
  to an $L^\infty$ bound.
  In general, write
  \[
    \int (-\Delta \psi + E\psi) \overline{\psi(x)} \phi^2(x - y)\, dx
    = E \int |\psi|^2 \phi^2(x - y)\, dx,
  \]
  where $\phi$ is a compactly supported function for
  every $y \in \R^n$. We can integrate by parts
  on the left:
  \begin{align*}
    E \int |\psi|^2 \phi^2(x - y)\, dx
    &= \int |\nabla \psi|^2 \phi^2(x - y)\, dx
    + \int V(x) |\psi|^2 \phi^2(x - y)\, dx \\
    &\quad \quad + 2 \int (\nabla \psi) \overline{\psi} \phi(x - y) \nabla \phi(x - y)\, dx.
  \end{align*}
  Thus we can estimate
  \begin{align*}
    \int |\nabla \psi|^2 \phi^2(x - y)\, dx
    &\le E \int |\psi|^2 \phi^2(x - y)\, dx
    + \|V\|_{L^\infty} \int |\psi|^2 \phi^2(x - y)\, dx \\
    &\quad \quad + \left|2 \int (\nabla \psi) \overline{\psi} \phi(x - y) \nabla \phi(x - y)\, dx\right| \\
    &\le E \int |\psi|^2 \phi^2(x - y)\, dx
    + \|V\|_{L^\infty} \int |\psi|^2 \phi^2(x - y)\, dx \\
    &\quad \quad + \frac{1}{2} \int |\nabla \psi|^2 \phi^2(x - y)\, dx
    + 4 \int |\psi|^2 |\nabla \phi(x - y)|^2\, dx,
  \end{align*}
  and we get a bound
  \[
    \frac{1}{2} \int |\nabla \psi|^2 \phi^2(x - y)\, dx
    \le 4 \int |\psi|^2 |\nabla \phi(x - y)|^2\, dx
    + (\|V\|_{L^\infty} + E) \int |\psi|^2 \phi^2(x - y)\, dx.
  \]
  Pick $\phi$ smooth such that $\phi(z) = 1$ when
  $|z| \le 1$ and $\phi(z) = 0$ when $|z| > 2$. Then
  \[
    \int_{|x - y| \le 1} |\nabla \psi|^2
    \le (\|V\|_{L^\infty} + |E| + 1) \int_{|x - y| \le 2} |\psi|^2\, dx.
  \]
  Then by induction,
  \[
    \int_{|x - y| \le 1} |\nabla \partial^\gamma \psi|^2 \lesssim C(|E| + \|V\|_{C^\gamma} + 1)
    \int_{|x - y| \le 2} |\psi|^2\, dx.
  \]
  Using Sobolev embedding, we get (for large enough $K$)
  \[
    \sup_{|x - y| \le 1} |\psi(x)| \le
    C \sum_{|\gamma| \le K} \int_{|x - y| \le 1} |\partial^\gamma \psi|^2\, dx
    \le C \int_{|x - y| \le 2} |\psi|^2\, dx.
  \]
  By the Agmon weighted estimate, we have
  \[
    \sup_{|x - y| \le 1} |\psi(x)| \lesssim
    e^{-\alpha|y|} \int e^{\alpha |x|} |\psi|^2\, dx
    \lesssim e^{-\alpha |y|},
  \]
  which gives us the pointwise bound.
\end{proof}

\begin{remark}
  Localizing is important since
  the Sobolev embedding $W^{\alpha, p} \hookrightarrow L^q$
  in $\R^n$
  requires
  \[
    \frac{1}{p} - \frac{1}{q} = \frac{|\alpha|}{n}.
  \]
  We want $q = \infty$, $p = 2$, so $|\alpha| = n / 2$.
  However, Sobolev embedding
  into $L^\infty$ requires $|\alpha| > n / 2$.
\end{remark}

  \chapter{Mar.~26 --- Scattering Theory}

\section{Sharp Agmon Estimate}
\begin{theorem}[Sharp Agmon estimate in $\R^3$]
  In $\R^3$, suppose $(-\Delta + V) \psi = E\psi$ with
  $E < 0$, and
  let $V$ have ``nice'' decay (the precise
  condition is $V \in L^{3 / 2, 1})$ Then
  \[
    |\psi(x)| \lesssim \frac{1}{\langle x \rangle} e^{-(\sqrt{-E}) |x|}.
  \]
\end{theorem}

\begin{remark}
  Recall that we have previously shown
  $|\psi| \le e^{-(1 - \epsilon) P_E(x)}$, where
  $P_E(x) \sim (\sqrt{-E}) |x|$. Note that this
  bound is not sharp, we have an $\epsilon$ loss
  in the exponent.
\end{remark}

\section{Short-Range Scattering Theory}

\begin{remark}
  Let $H_0 = -\Delta$ and $H = -\Delta + V$.
  We would like to compare $e^{i H_0 t}$ and $e^{i H t}$.
  More precisely, for $f \in L^2$,
  we would like to determine if we can  find
  $g$ such that
  \[
    \|e^{-itH_0} f - e^{-itH} g\|_{L^2} \xrightarrow[t \to \infty]{} 0. \tag{$*$}
  \]
  The goal is the following: For a prescribed free flow,
  can we find a perturbed flow such that they match as
  $t \to \infty$? This is known as the
  \emph{wave operator}
  $\Omega_{\mp} = \lim_{t \to \pm \infty} e^{itH} e^{-itH_0}$
  (in the strong $L^2$ sense).

  Note that the equation $(*)$ is equivalent to (since
  $e^{itH}$ is an isometry on $L^2$)
  \[
    \|e^{itH} e^{-i t H_0} f - g\|_{L^2} \xrightarrow[t \to \infty]{} 0,
  \]
  so if the wave operator $\Omega_-$ exists,
  then $\|\Omega_- f - g\|_{L^2} = 0$, so
  we can take $g = \Omega_- f$.
\end{remark}

\begin{theorem}\label{thm:wave-op-properties}
  Suppose that $\sup_x |V(x)| (1 + |x|)^{1 + \epsilon} < \infty$. Then
  \begin{enumerate}
    \item $\Omega_{\mp}$ exist;
    \item $\Omega_{\mp}$ are isometries;
    \item (intertwining properties)
      $\Omega_- e^{-itH_0} = e^{-itH} \Omega_-$, so
      $e^{-itH_0} = \Omega_-^{-1} e^{-itH} \Omega_-$
      and $e^{-itH} = \Omega_- e^{-itH_0} \Omega_-^{-1}$;
    \item $\Omega_- H_0 f = H \Omega_- f$
      for $f \in H^2$;
    \item $H \circ P_{\Ran(\Omega_-)} = \Omega_- H_0 \Omega_-^*$
      in $H^2$, so if
      $f \in \Ran(\Omega_-)$ then $\Omega_- H_0 \Omega_-^* f = Hf$;
    \item $\Ran(\Omega_-) \subseteq L^2_{\mathrm{ac}}(H) = \HH_{\mathrm{ac}}$.
  \end{enumerate}
\end{theorem}

\begin{proof}
  $(1)$ The following proof is known as \emph{Cook's method}.
  For a dense subspace of $L^2$, we will show that
  $e^{itH} e^{-itH_0} f$ is a Cauchy sequence
  in time. Fix $s \ge t$, then
  \[
    e^{isH} e^{-isH_0} f - e^{itH} e^{-itH_0} f
    = \int_{t}^s \frac{d}{d\tau} (e^{i\tau H} e^{-i\tau H_0} f) \, d\tau.
  \]
  We would like to show that
  $\|e^{isH} e^{-isH_0} f - e^{itH} e^{-itH_0} f\|_{L^2} \to 0$
  as $s, t \to \infty$, for which is suffices to prove
  \[
    \left|\int_{t}^s \frac{d}{d\tau} (e^{i\tau H} e^{-i\tau H_0} f) \, d\tau\right|
    \xrightarrow[t, s \to \infty]{} 0.
  \]
  We can compute that (recall that $H = H_0 + V$)
  \[
    \frac{d}{d\tau} (e^{i\tau H} e^{-i\tau H_0} f)
    = e^{i\tau H} H e^{-i\tau H_0} f
    - e^{i\tau H} H_0 e^{-i\tau H_0} f
    = e^{i\tau H} V e^{-i\tau H_0} f.
  \]
  Then we can bound (since $e^{i\tau H}$ is an isometry)
  \[
    \left\|\int_{t}^s \frac{d}{d\tau} (e^{i\tau H} e^{-i\tau H_0} f) \, d\tau\right\|_{L^2}
    = \left\|\int_{t}^s e^{i\tau H} V e^{-i \tau H_0} \, d\tau\right\|_{L^2}
    \le \int_{t}^s \left\|V e^{-i \tau H_0}\right\|_{L^2} \, d\tau.
  \]
  We want to show that $\|V e^{-i \tau H_0} f\|_{L^2}$
  is integrable in $\tau$, so the above goes to $0$ as $s, t \to \infty$.\footnote{Note that if $V$ is nice, then $\|Ve^{-i\tau H_0} f\|_{L^2} \lesssim \|e^{-i\tau H_0} f\|_{L^\infty} \lesssim |\tau|^{-d / 2}$, which is integrable for $d \ge 3$. But for $d = 1, 2$, we need a more careful argument.}
  Define
  \[
    \mathcal{C} = \{f \in \mathcal{S}(\R^d)
      : \widehat{f}(\xi) = 0 \text{ for  $|\xi| < a$ and $|\xi| > b$, for some $a, b \in \R^+$}
    \},
  \]
  and note that $\mathcal{C}$ is dense in $L^2$.
  Take $f \in \mathcal{C}$, then for some $a, b$
  we have $\widehat{f}(\xi) = 0$ whenever
  $|\xi| < a$ or $|\xi| > b$. We can write
  \[
    e^{-i t H_0} f
    = \int e^{ix\xi} e^{-i |\xi|^2 t}
    \widehat{f}(\xi)\, d\xi.
  \]
  Write $e^{ix \xi} e^{-i |\xi|^2 t} = e^{it (-|\xi|^2 + \xi(x / t))} = e^{it \phi(\xi)}$.
  We use a stationary phase argument to estimate
  this integral: The
  critical point $\nabla \phi(\xi) = 0$ is the point
  where the oscillations stop. When $\nabla \phi \ne 0$,
  we can write
  \[
    \int e^{it\phi(\xi)} \widehat{f}(\xi)\, d\xi
    = \int \left(\frac{1}{it \nabla \phi(\xi)} \frac{d}{d\xi} e^{it \phi(\xi)}\right) \widehat{f}(\xi)\, d\xi
    = -\frac{1}{t} \int e^{it \phi(\xi)} \frac{d}{d\xi} \left(\widehat{f}(\xi) \frac{1}{\nabla \phi(\xi)}\right)\, d\xi
  \]
  by integration by parts. So it suffices to
  consider the critical point, which happens when
  $x / t - 2 \xi = 0$, or $\xi = -x / 2t$.
  Note that $\widehat{f}(\xi)$ is supported in
  $a < |\xi| < b$, so if
  $|x / 2t| < a$ or $|x / 2t| < b$, then
  \[
    \int e^{ix \xi} e^{-it |\xi|^2} \widehat{f}(\xi)\, d\xi
  \]
  has no critical points. So we get that
  \[
    \left|\int e^{ix \xi} e^{-it |\xi|^2} \widehat{f}(\xi)\, d\xi\right|
    \lesssim |t|^{-n}, \quad
    \text{for any $n \in \N$}.
  \]
  So if $|x| < 2at$ or $|x| > 2tb$,
  then $\|e^{iH_0 f}\|_{L^\infty} \lesssim t^{-n}$
  for any $n \in \N$.
  So we focus on $2at < |x| < 2tb$.
  But in this region, by the pointwise decay of
  $e^{i H_0 t} f$, we have
  $\|e^{i H_0 t} f\|_{L^\infty} \lesssim t^{-d / 2}$.
  Then
  \begin{align*}
    \|V e^{i H_0 t f}\|_{L^2(2at < |x| < 2bt)}
    &\lesssim \frac{1}{(1 + |t|)^{1 + \epsilon}} \|e^{iH_0 t} f\|_{L^2(2at < |x| < 2bt)}
    \lesssim \frac{1}{(1 + |t|)^{1 + \epsilon}} t^{d / 2}
    \|e^{iH_0 t} f\|_{L^\infty} \\
    &\lesssim \frac{1}{(1 + |t|)^{1 + \epsilon}} t^{d / 2}
    t^{-d / 2}
    \lesssim \frac{1}{(1 + |t|)^{1 + \epsilon}}
  \end{align*}
  where we used the hypothesis that
  $|V| \lesssim 1 / (1 + |x|)^{1 + \epsilon}$.
  So
  \[
    \|Ve^{i\tau H_0} f\|_{L^2}
    = \|Ve^{i\tau H_0} f\|_{L^2(|x| < 2a\tau)}
    + \|Ve^{i\tau H_0} f\|_{L^2(|x| > 2b\tau)}
    + \|Ve^{i\tau H_0} f\|_{L^2(2a\tau < |x| < 2b\tau)}.
  \]
  We have already shown that the last term is
  integrable. For the
  first term, we have
  \[
    \|Ve^{i\tau H_0} f\|_{L^2(|x| < 2a\tau)}
    \lesssim \|V\|_{L^2(|x| < 2a\tau)} \| e^{-i\tau H_0} f \|_{L^\infty(|x| < 2a \tau)}
    \lesssim 2a|\tau|^{d / 2} \cdot \tau^{-n} \quad \text{for any $n \in \N$},
  \]
  so if we take $n > d / 2 + 1$, then this
  term is integrable.
  Finally, for the second term,
  \[
    \|Ve^{i\tau H_0} f\|_{L^2(|x| > 2b\tau)}
    \lesssim \|V\|_{L^\infty(|x| > 2b\tau)} \| e^{-i\tau H_0} f \|_{L^2}
    \lesssim \frac{1}{(1 + |\tau|)^{1 + \epsilon}},
  \]
  which is integrable (note that $\|e^{-i \tau H_0} f\|_{L^2} = \|f\|_{L^2}$).
  Thus we obtain
  \[
    \|Ve^{-i \tau H_0}\|_{L^2} \lesssim \tau^{-1 - \epsilon},
  \]
  which is integrable in $\tau$. The conclusion from
  this is that for $s \ge t$,
  \[
    \|e^{isH} e^{-isH_0} f - e^{itH} e^{-itH_0} f\|_{L^2}
    \xrightarrow[s, t \to \infty]{} 0,
  \]
  so $\lim_{t \to \infty} e^{itH} e^{-itH_0} f$ exists
  strongly in $L^2$.
  A similar argument shows that $\Omega_+$ exists.

  $(2)$ This follows since
  $e^{-itH_0}, e^{itH}$ are both isometries,
  so that $e^{itH} e^{-itH_0}$ is an isometry.
  A strong limit of isometries is still an isometry,
  so we get that $\Omega_{\mp}$ are also isometries.

  $(3)$ We can see that (note the group properties
  of $e^{itH}$ and $e^{itH_0}$)
  \begin{align*}
    \Omega_- e^{-it H_0} f
    &= \left(\lim_{s \to \infty} e^{isH -isH_0}\right)
    e^{-itH_0} f
    = e^{-iHt} \left(\lim_{s \to \infty} e^{itH} e^{isH} e^{-isH} e^{-itH_0}\right) f \\
    &= e^{-iHt} \lim_{s \to \infty} \left(e^{i(t + s) H} e^{-i(s + t) H_0}\right) f
    = e^{-iHt} \Omega_- f,
  \end{align*}
  which proves the desired result.

  We finish the proof of $(4)$-$(6)$ next time.
\end{proof}

  \chapter{Mar.~31 --- Scattering Theory, Part 2}

\section{Short-Range Scattering Theory, Continued}
\begin{proof}[Proof of Theorem \ref{thm:wave-op-properties}, continued]
  $(4)$ The idea is to use the Laplace transform.
  By $(3)$, we have
  \[
    \Omega_- \int_0^\infty e^{-itH_0} e^{-t\epsilon}\, dt
    = \int_0^\infty \Omega_- e^{-itH_0} e^{-t\epsilon}\, dt
    = \int_0^\infty e^{-itH} \Omega_- e^{-t\epsilon}\, dt
    = \left(\int_0^\infty e^{-itH} e^{-t\epsilon}\, dt\right) \Omega_-
  \]
  since $\Omega_-$ is independent of $t$.
  Integrating both sides, we get
  \[
    \Omega(iH_0 + \epsilon)^{-1}
    = (iH + \epsilon)^{-1} \Omega_-.
  \]
  Note that these are bounded $L^2 \to H^2$
  operators. Applying $(iH + \epsilon)$ on the left
  on both sides,
  \[
    (i H + \epsilon) \Omega_- (i H_0 + \epsilon)^{-1}
    = \Omega_-
  \]
  in $L^2$. For any $f \in L^2$, we have
  $(i H_0 + \epsilon)^{-1} f \in H^2$, and
  $\Omega_- (i H_0 + \epsilon)^{-1} f \in H^2$
  by the isometry property, so
  $(i H + \epsilon) \Omega_- (i H_0 + \epsilon)^{-1} f \in L^2$.
  Now $(i H + \epsilon) \Omega_- (i H_0 + \epsilon)^{-1} g = \Omega_- g$
  for any $g \in L^2$, and we can find $f \in H^2$
  such that $g = (i H_0 + \epsilon)^{-1} f$, so
  we get
  $(iH + \epsilon) \Omega_- f = \Omega_- (iH_0 + \epsilon) f$.
  Finally, passing $\epsilon \to 0$, we
  obtain $iH \Omega_- f = \Omega_- iH_0 f$,
  so we have $H \Omega_- f = \Omega_- H_0 f$.

  $(5)$ Define $Q = \Omega_-^* \Omega_-$, where
  $\Omega_-^*$ is the adjoint of $\Omega_-$.
  We claim that $Q = \id$ and
  $\Omega_- \Omega_-^* = P_{\Ran(\Omega_-)}$.
  We clearly have $Q^* = Q \ge 0$,
  since $\langle Qf, f \rangle = \|\Omega_- f\|_{L^2}^2 = \|f\|_{L^2}^2 \ge 0$ (recall that
  $\Omega_-$ is an isometry). Then
  \[
    \langle Qf, f \rangle = \langle f, f \rangle
    \implies \langle (Q - \id) f, f \rangle = 0
  \]
  for all $f$. But $Q - \id$ is self-adjoint, so
  spectral theory implies that
  $Q - \id = 0$, i.e. $Q = \id$.
  On the other hand, $P = \Omega_- \Omega_-^*$
  is a projection since we have
  \[
    P^2 = \Omega_- \Omega_-^* \Omega_- \Omega_-^*
    = \Omega_- \id \Omega_-^* = \Omega_- \Omega_-^* = P.
  \]
  One can also see that $P^* = P \ge 0$, so
  $P$ is an orthogonal projection. So it only
  remains to check the range. Since $P$ is a
  projection, $\ker P = (\Ran P)^\perp$, and we
  can identify that
  \[
    \ker P
    = \{f : \langle Pf, f \rangle = 0\}
    = \{f : \|\Omega_-^* f \|_{L^2}^2 = 0\}
    = \ker \Omega_-^*
  \]
  Since $(\ker T)^\perp = \overline{\Ran T^*}$,
  we find that (note that $\Ran P$ is closed since
  $P$ is a projection)
  \[
    \Ran P = (\ker P)^\perp
    = (\ker \Omega_-^*)^\perp
    = \overline{\Ran \Omega_-}.
  \]
  One can check that $\Omega_-$ is a closed
  operator, so $\overline{\Ran \Omega_-} = \Ran \Omega_-$
  and $\Ran P = \Ran \Omega_-$. So $P = \Omega_- \Omega_-^*$
  is an orthogonal projection onto
  $\Ran(\Omega_-)$. From $(4)$, taking
  $f = \Omega_-^* g$, we get
  $\Omega_- H_0 \Omega_-^* = H \Omega_- \Omega_-^* = H P$.

  $(6)$ Let $E(d\lambda)$ be the
  spectral resolution for $H$. Then by $(5)$,
  $E(d\lambda) P_{\Ran \Omega_-}
    = \Omega_- E_0(d\lambda) \Omega_-^*$, so
    \[
      P_{\Ran \Omega_-} E(d\lambda) P_{\Ran \Omega_-}
      = \Omega_- \Omega_-^* \Omega_- E_0(d\lambda) \Omega_-^*
      = \Omega_- E_0(d\lambda) \Omega_-^*,
    \]
  where $\Omega_-^* \Omega_- = \id$ by $(4)$.
  Thus we see that
  \[
    \langle E(d\lambda) P_{\Ran \Omega_-} f, P_{\Ran \Omega_-} f \rangle
    = \langle E_0(d\lambda) \Omega_-^* f, \Omega_-^* f \rangle.
  \]
  Since $E_0$ is the free case, it is
  always absolutely continuous with respect
  to the Lebesgue measure, so we have
  $P_{\Ran \Omega_-} f \in \Hac$, i.e.
  $\Ran \Omega_- \subseteq \Hac$.
\end{proof}

\begin{corollary}
  If $T$ is a Borel measurable function,
  then
  $T(H) \Omega_- f = \Omega_- T(H_0) f$.
\end{corollary}

\begin{proof}
  This follows from the proof of $(4)$ in the
  previous theorem.
\end{proof}

\begin{remark}
  When do we have the equality
  $\Ran \Omega_- = \Hac$? We have the following
  result:
\end{remark}

\begin{prop}
  Assume $d \ge 3$ and $|V(x)| \lesssim (1 + |x|)^{1 - \epsilon}$ (this is the \emph{short-range condition})
  with $\|V\|_{L^2(\R^d)} < \infty$. Assume that
  \[
    \|e^{-itH}P_{\mathrm{ac}} f \|_{L^\infty}
    \lesssim |t|^{-d / 2} \|f\|_{L^1}.
  \]
  Then $\Ran \Omega_- = \Hac$.
\end{prop}

\begin{proof}
  Define $\widetilde{\Omega} = \lim_{t \to \infty} e^{itH} e^{-itH} P_{\mathrm{ac}}(H)$ in $L^2$.
  To show that $\widetilde{\Omega}$ exists, we
  again use Cook's method:
  \begin{align*}
    \int_1^\infty \left\| \frac{d}{dt} e^{itH_0} e^{-itH} P_{\mathrm{ac}} f \right\|_{L^2} dt
    &\le \int_1^\infty \| e^{itH_0} V e^{-itH} P_{\mathrm{ac}} f \|_{L^2}\, dt
    \le \int_1^\infty \| V e^{-itH} P_{\mathrm{ac}} f \|_{L^2}\, dt \\
    &\le
    \int_1^\infty \| V \|_{L^2} \| e^{-itH} P_{\mathrm{ac}} f \|_{L^\infty}\, dt
    \lesssim \int_1^\infty |t|^{-d / 2} \|f\|_{L^1}\, dt < \infty
  \end{align*}
  for $d \ge 3$ (technically we should use
  $f \in L^2$, but it suffices to consider the
  dense subspace $L^1 \cap L^2 \subseteq L^2$ for
  Cook's method).
  So $\widetilde{\Omega}$ exists. Then
  $\Omega_- \widetilde{\Omega} f = \lim_{t \to \infty} (e^{itH} e^{-itH_0}) \widetilde{\Omega} f$, so
  for all $\eta > 0$, we can find $T_\eta$ such
  that for all $t \ge T_\eta$,
  \[
    \Omega_- \widetilde{\Omega} f
    = e^{itH} e^{-itH_0} \widetilde{\Omega} f
    + o_{L^2}(\eta).
  \]
  Since $\widetilde{\Omega}$ also exists,
  there is $\widetilde{T}_\eta$ such that
  for all $t \ge \widetilde{T}_\eta$,
  \[
    \widetilde{\Omega} f
    = e^{itH_0} e^{-itH}
    P_{\mathrm{ac}} f + o_{L^2}(\eta).  
  \]
  So for $t \ge \max\{T_\eta, \widetilde{T}_\eta\}$,
  we can write
  \[
    \Omega_- \widetilde{\Omega} f
    = e^{itH} e^{-itH_0} (e^{itH_0} e^{-itH} P_{\mathrm{ac}}) f + o_{L^2}(\eta)
    = P_{\mathrm{ac}} f + o_{L^2}(\eta).
  \]
  This holds for arbitrary $\eta$, so
  $\Omega_- \widetilde{\Omega} f = P_{\mathrm{ac}} f$
  for every $f \in L^2$. So
  $\Ran \Omega_- = \Ran P_{\mathrm{ac}}(H) = \Hac$.
\end{proof}

\begin{remark}
  Recall that when $d = 3$,
  \[
    \sup_x \int \frac{|V(y)|}{|x - y|}\, dy
    < \infty. \tag{$*$}
  \]
  Assuming that $H$ has no zero eigenvalues or
  resonance, we have
  \[
    \|e^{-itH} P_{\mathrm{ac}}(H) f\|_{L^\infty}
    \lesssim |t|^{-3 / 2} \|f\|_{L^1}.
  \]
  In particular, if $(*) \le 4\pi$,
  then the dispersive estimate holds and
  the proposition applies.
\end{remark}

  \chapter{Apr.~2 --- Classical Scattering}

\section{Consequences of Wave Operators}

\begin{theorem}[Enss]
  Suppose $|V(x)| \le 1 / (1 + |x|)^{1 + \epsilon}$
  for $\epsilon > 0$. Then
  \[
    \Ran \Omega_{\mp} = L^2_{\mathrm{ac}} = \Hac
    \quad \text{and} \quad
    L^2_{\mathrm{sc}} = \varnothing.
  \]
\end{theorem}

\begin{corollary}
  If $|V(x)| \le 1 / (1 + |x|)^{1 + \epsilon}$ for
  $\epsilon > 0$, then for all $f \in L^2(\R^d)$,
  there exist $\{f_j\}_{j = 0}^N \in L^2$
  such that $Hf_j = \lambda_j f_j$ for some
  $\lambda_j \le 0$ and
  \[
    e^{-iHt} f
    = \sum_{j = 1}^N e^{-i \lambda_j t} f_j + e^{iH_0 t} f_0 + R(t)
  \]
  such that $\|R(t)\|_{L^2} \to 0$ as $t \to \infty$.
\end{corollary}

\begin{proof}
  By Enss's theorem, we have
  $L^2 = L^2_{\mathrm{ac}} + L^2_{\mathrm{pp}}$, so
  we can write
  \[
    f = g + \sum_{j = 1}^N f_j,
  \]
  where $g \in L^2_{\mathrm{ac}}$ and
  $\{f_j\}$ spans $L^2_{\mathrm{pp}}$ (so
  $Hf_j = \lambda_j f_j$). Then
  \[
    e^{-iHt} f = e^{-iHt} g + \sum_{j = 1}^N e^{-i \lambda_j t} f_j. \tag{$*$}
  \]
  Since $g \in L^2_{\mathrm{ac}} = \Ran \Omega_-$,
  we can find $f_0 \in L^2$ such that
  $\Omega_- f_0 = g$. By the definition of
  $\Omega_-$,
  \[
    \| e^{-iHt g} - e^{-iH_0 t} f_0 \|_{L^2} \xrightarrow[t \to \infty]{} 0,
  \]
  so $R(t) = e^{-iHt} g - e^{-iH_0 t} f_0 \to 0$
  in $L^2$ as $t \to \infty$. This proves the
  claim after substituting in $(*)$.
\end{proof}

\section{Classical Scattering}

\begin{remark}
  Consider the following equation for $x, \xi \in \R^d$:
  \[
    \begin{cases}
      \dot{x}(t) = \xi(t), \\
      \dot{\xi}(t) = -\nabla V(x(t)). \\
    \end{cases}
    \tag{$+$}
  \]
  Let $x(t; y, \eta)$, $\xi(t; y, \eta)$
  denote the solution to
  $(+)$ with $x(0) = y$ and $\dot{x}(0) = \xi(0) = \eta)$.
  Define
  \[
    \Sigma_{\pm} =
    \left\{
      (y, \eta) \in \R^{2d} :
      V(y) + \frac{1}{2} |\eta|^2 > 0 \text{ and }
      \limsup_{t \to \pm \infty} |x(t; y, \eta)| = \infty
    \right\},
  \]
  and set $\Sigma_{\mathrm{scattering}} = \Sigma_+ \cap \Sigma_-$.
  Denote $E(t) = V(x(t)) + |\xi(t)|^2 / 2$
  with $(x(t), \xi(t))$ solving $(+)$, so
  \[
    \frac{d}{dt} E(t)
    = \nabla V(x(t)) \cdot \dot{x}(t)
    + \dot{\xi}(t) \cdot \xi(t)
    = \nabla V(x(t)) \cdot \xi(t) - \nabla V(x(t)) \cdot \xi(t)
    = 0.
  \]
  This shows that we have conservation of the
  energy $E$.
\end{remark}

\begin{lemma}
  Assume that $|\nabla V(x)| \le C_0 (1 + |x|)^{-1 - \epsilon}$
  and $|V(x)| \to 0$ as $x \to \infty$. Then
  $(y, \eta) \in \Sigma_+$ if and only if
  $\lim_{t \to \infty} \xi(t; y, \eta)$ exists
  and $\lim_{t \to \infty} \xi(t; y, \eta) \ne 0$.
\end{lemma}

\begin{proof}
  $(\Leftarrow)$ Suppose
  $\lim_{t \to \infty} \xi(t; y, \eta) \ne 0$.
  So for some constant $c > 0$, we will have
  $|\dot{x}(t)| \ge c$ for all $t$ large enough.
  Clearly this implies $|x(t)| \to \infty$ as
  $t \to \infty$.
  We can also see that
  \[
    E(t) = V(x(t)) + \frac{1}{2} |\xi(t)|^2
    \xrightarrow[t \to \infty]{} \frac{1}{2} |\xi(t)|^2 > 0
  \]
  since $|V| \to 0$ as $|x| \to \infty$, so
  by conservation of energy,
  $E(0) = V(y) + |\eta|^2 / 2 > 0$.
  So $(y, \eta) \in \Sigma_+$.

  $(\Rightarrow)$ Assume $(y, \eta) \in \Sigma_+$,
  we want to show $|x(t; y, \eta)| \ge at$
  for some $a > 0$. If this can be done, then
  by the fundamental theorem of calculus, we
  can estimate
  \begin{align*}
    |\xi(t; y, \eta) - \xi(s; y, \eta)|
    &\le \left|\int_s^t \dot{\xi}(\tau; y, \eta)\, d\tau\right|
    = \int_s^t |\nabla V(x(\tau; y, \eta))|\, d\tau
    \le C_0 \int_s^t (1 + x(\tau; y, \eta))^{-1 - \epsilon}\, d\tau \\
    &\le C_0 \int_s^t (a|\tau|)^{-1 - \epsilon}\, d\tau
    = \frac{C_0}{a^{1 + \epsilon}} \int_s^\infty \tau^{-1 - \epsilon}\, d\tau
    \xrightarrow[s \to \infty]{} 0
  \end{align*}
  since $\tau^{-1 - \epsilon}$ is integrable.
  So $\xi(t; y, \eta)$ is a Cauchy sequence in
  $t$, so $\lim_{t \to \infty} \xi(t; y, \eta)$ exists.
  We also know that
  $\lim_{t \to \infty} \xi(t; y, \eta) \ne 0$
  since $|x(t; y, \eta)| \ge a|t|$, so the
  result would be proved.

  So it suffices to show $|x(t; y, \eta)| \ge at$
  for some $a > 0$.
  % We claim that
  % \[
  %   |x(t)| \ge \max\left\{\frac{R}{2}, C_1(t - t_0)\right\}
  % \]
  % for some $t_0$ large and $C_1$ independent
  % of $R$, which we will prove using a
  % bootstrap argument. Assume the claim holds
  % on $[t_0, T]$, then we show that it can be
  % extended. We can estimate
  % \begin{align*}
  %   |\xi(t) - \xi(t_0)|
  %   &\le \int_{t_0}^t |\nabla V(x(\tau))|\, d\tau
  %   = \int_{t_0}^{t_0 + R / 2C_1} |\nabla V(x(\tau))|\, d\tau
  %   + \int_{t_0 + R / 2C_1}^t |\nabla V(x(\tau))|\, d\tau \\
  %   &\le \int_{t_0}^{t_0 + R / 2 C_1} \frac{C_0}{|x(\tau)|^{1 + \epsilon}}\, d\tau
  %   + \int_{t_0 + R / 2C_1}^t \frac{C_0}{|x(\tau)|^{1 + \epsilon}}\, d\tau \\
  %   &\le \int_{t_0}^{t_0 + R / 2 C_1} C_0 \left(\frac{R}{2}\right)^{-1 - \epsilon}\, d\tau
  %   + \int_{t_0 + R / 2C_1}^t C_0 (C_1(\tau - t_0))^{-1 - \epsilon}\, d\tau \\
  %   &\le C_0 \frac{2\epsilon R^{-\epsilon}}{C_1}
  %   + \frac{C_0 C_1^{-1 - \epsilon}}{\epsilon}
  %   \left(\frac{R}{2C_1}\right)^{-\epsilon}
  %   = \frac{C_0}{C_1} 2^\epsilon \left(1 + \frac{1}{\epsilon}\right) R^{-\epsilon}.
  % \end{align*}
  % By the fundamental theorem of calculus, we have
  % \begin{align*}
  %   x(t) = x(t_0) + \int_{t_0}^t \xi(\tau)\, d\tau
  %   &= x(t_0) + (t - t_0) \xi(t_0) + \int_{t_0}^t (\xi(\tau) - \xi(t_0))\, d\tau \\
  %   &\ge \sqrt{R^2 + |t - t_0|^2 |\xi(t_0)|^2}
  %   - \int_{t_0}^t \frac{C_0}{C_1} 2^\epsilon \left(1 + \frac{1}{\epsilon}\right) R^{-\epsilon}\, d\tau \\
  %   &\ge \frac{R}{2} + \frac{(t - t_0) |\xi(t_0)|}{\sqrt{2}} - (t - t_0) \frac{C_0}{C_1} 2^\epsilon \left(1 + \frac{1}{\epsilon}\right) R^{-\epsilon} \\
  %   &\ge \frac{R}{2} + (t - t_0) \left(\frac{|\xi(t_0)|}{\sqrt{2}} - \frac{C_0}{C_1} 2^\epsilon \left(1 + \frac{1}{\epsilon}\right) R^{-\epsilon}\right).
  % \end{align*}
  % Note that $t_0$ is given, so $|\xi(t_0)|$ is
  % fixed. The second part of the inner sum
  % can be made small by taking $R$ large enough,
  % so we can choose $C_1, R$ such that
  % \[
  %   \frac{|\xi(t_0)|}{\sqrt{2}}
  %   - \frac{C_0}{C_1} 2^\epsilon \left(1 + \frac{1}{\epsilon}\right) R^{-\epsilon}
  %   < C_1.
  % \]
  % Then we can applying a bootstrap to
  % extend the estimate beyond $T$ and get
  % $x(t) \ge C_1(t - t_0)$, which then implies
  % that $|x(t)| \ge at$ for some $a > 0$ and
  % $t$ large enough.
  Set $A(t) = |x(t)|^2 = \langle x(t), x(t) \rangle$, this is a
  \emph{Lyapunov function} from ODE theory.
  Differentiating in time, we get
  \[
    \frac{d}{dt} A(t)
    = 2 \langle x(t), \dot{x}(t) \rangle
    = 2 \langle x(t), \xi(t) \rangle
  \]
  and similarly
  \begin{align*}
    \frac{d^2}{dt^2} A(t)
    &= 2 \langle \dot{x}(t), \xi(t) \rangle
    + 2 \langle x(t), \dot{\xi}(t) \rangle
    = 2 |\xi(t)|^2 - 2 x(t) \cdot \nabla V(x(t)) \\
    &= 4 \left(\frac{1}{2} |\xi(t)|^2 + V(x(t))\right)
    - 4V(x(t)) - 2x(t) \cdot \nabla V(x(t)).
  \end{align*}
  Note that $|\xi(t)|^2 / 2 + V(x(t)) = E(t) = E_0 > 0$
  and as $|x(t)| \to \infty$, we have
  $|V(x(t))| \to 0$. We also get that
  $|x(t) \cdot \nabla V(x(t))| \to 0$
  as $|x(t)| \to \infty$
  since $|\nabla V(x(t))| \le C_0 (1 + |x(t)|)^{-1 - \epsilon}$
  by assumption. So we can find $R_0$
  large such that when $|x(t)| > R_0$,
  we have $\ddot{A}(t) > E_0 / 2$. We claim that we
  can find $t_0$ with
  \[
    |x(t_0)| > R_0 \quad \text{and} \quad
    \frac{d}{dt} A(t_0) > 0.
  \]
  To see this, note that $A(t) = |x(t)|^2$, so
  $A(t) \to \infty$ as $t \to \infty$, so
  such a $t_0$ must exist. Then at $t_0$,
  \[
    A(t_0) > R_0^2, \quad \dot{A}(t_0) > 0, \quad
    \ddot{A}(t_0) > \frac{E_0}{2},
  \]
  so for any $t_0 \le t \le t_0 + \delta$, we
  have
  $\dot{A}(t) > \dot{A}(t_0) > 0$,
  $\ddot{A}(t) > E_0 / 2 > 0$, and $A(t) > R_0^2$. Thus
  we have
  \begin{align*}
    \dot{A}(t)
    &= \int_{t_0}^t \ddot{A}(s)\, ds + \dot{A}(t_0)
    \ge (t - t_0) \frac{E_0}{2} + \dot{A}(t_0), \\
    A(t)
    & = A(t_0) + \int_{t_0}^t \dot{A}(s)\, ds
    \ge \frac{1}{2} (t - t_0)^2 \frac{E_0}{2}
    + (t - t_0) \dot{A}(t_0) + A(t_0),
  \end{align*}
  so $x(t) = \sqrt{A(t)} \ge a|t - t_0|$
  for some $a > 0$. This proves the
  claim after a bootstrap argument.
\end{proof}

\begin{remark}
  Physically, the above lemma tells us that
  if $(y, \eta) \in \Sigma_+$, then a classical
  particle will escape from the potential $V$
  and move away with ``super-linear''
  trajectory.
\end{remark}

\begin{prop}\label{prop:classical-scattering}
  Assume $|\nabla V(x)| \le C_0 (1 + |x|)^{-2 - \epsilon}$
  and $|\nabla^2 V(x)| \le C_1 (1 + |x|)^{-2 - \epsilon}$.
  Then for every $(y, \eta) \in \Sigma_+$,
  there exists $(a, p) \in \R^{2d}$ with
  $p \ne 0$, such that
  \[
    |x(t; y, \eta) - (tp + a)| \xrightarrow[t \to \infty]{} 0 \quad \text{and} \quad
    |p - \xi(t; y, \eta)| \xrightarrow[t \to \infty]{} 0.
  \]
  Furthermore, the converse also holds, i.e.
  for every $(a, p) \in \R^{2d}$ with $p \ne 0$,
  we can find $(y, \eta) \in \Sigma_+$ such that
  the above conditions hold.
\end{prop}

\begin{proof}
  $(\Rightarrow)$ Let $(y, \eta) \in \Sigma_+$.
  From the lemma, we know
  $\lim_{t \to \infty} \xi(t; y, \eta)$ exists,
  so set
  \[
    p = \lim_{t \to \infty} \xi(t; y, \eta) \ne 0.
  \]
  We can write
  \[
    x(t; y, \eta)
    = y + \int_0^t \xi(s; y, \eta) \, ds
    = y + p t + \int_0^t (\xi(s) - p)\, ds
    = y + pt + \int_0^\infty \int_s^\infty \nabla V(x(\tau))\, d\tau ds
  \]
  where we think of $p$ as $\xi(\infty)$. Then 
  we have
  \[
    x(t; y, \eta) =
    y + pt + \int_0^\infty \int_s^\infty \nabla V(x(\tau))\, d\tau ds
    - \int_t^\infty \int_s^\infty \nabla V(x(\tau))\, d\tau ds,
  \]
  where the latter integral goes to $0$ as
  $t \to \infty$. The former integral is
  a constant, so we can define $a$ such that
  $a - y$ equals that integral. This proves
  the claim.

  $(\Leftarrow)$ Given $(a, p) \in \R^{2d}$ with
  $p \ne 0$, we want to find $x(t)$, $\xi(t)$
  solving $(+)$ of the form
  \[
    x(t) = a + pt + R(t),
  \]
  where $R(t) \to 0$ as $t \to \infty$. Because
  of the equation $(+)$, we can formally write
  \[
    x(t) = a + pt + R(t)
    = a + pt - \int_t^\infty \int_s^\infty \nabla V(x(\tau))\, d\tau ds.
  \]
  This is the same as the situation from above.
  We can write
  \[
    a + pt + R(t) = a + pt
    - \int_t^\infty \int_s^\infty \nabla V(a + \tau p + R(\tau)\, d\tau ds,
  \]
  which gives the equation
  \[
    R(t) = -\int_t^\infty \int_s^\infty \nabla V(a + \tau p + R(\tau))\, d\tau ds.
  \]
  We can solve this uniquely via
  contraction mapping, which we will do
  next time.
\end{proof}

  \chapter{Apr.~7 --- Classical Scattering, Part 2}

\section{Classical Scattering, Continued}

\begin{proof}[Proof of Proposition \ref{prop:classical-scattering}, continued]
  We continue by solving the equation
  \[
    R(t) = -\int_t^\infty \int_s^\infty \nabla V(a + \tau p + R(\tau))\, d\tau ds.
  \]
  We do this by contraction mapping in the
  unit ball in $C([T, \infty); \R^d)$: Set
  \[
    A(z(t))
    = -\int_t^\infty \int_s^\infty \nabla V(a + \tau p + z(\tau))\, d\tau ds.
  \]
  Note that $|a + \tau p + z(\tau)| \ge |a + \tau p| - 1 \ge \tau |p| / 2$
  if $\tau$ is large enough. Then
  \[
    |A(z(t))|
    \le \int_t^\infty \int_s^\infty \left(\frac{1}{2} \tau p\right)^{-2 - \epsilon}\, d\tau ds
    \le |p|^{-2 - \epsilon} T^{-\epsilon} \le 1
  \]
  if we pick $T$ sufficiently large. We can
  also see that
  \begin{align*}
    |A(z(t)) - A(\widetilde{z}(t))|
    &\le \int_t^\infty \int_s^\infty
    |\nabla V(a + \tau p + z(\tau)) - \nabla V(a + \tau p + \widetilde{z}(\tau))|\, d\tau ds \\
    &\le \int_t^\infty \int_s^\infty
    \left(\frac{1}{2} \tau |p|\right)^{-2 - \epsilon}
    |z(\tau) - \widetilde{z}(\tau)|\, d\tau ds
    \le p^{-2 - \epsilon} T^{-\epsilon} \|z - \widetilde{z}\|_{C([T, \infty); \R^d)},
  \end{align*}
  which is a contraction if $T$ is large.
  So we can find a fixed point $R(t)$.
\end{proof}

\begin{remark}
  The map $\Omega_-(a, p) = (y, \eta)$
  can be seen as a classical analogue of the
  wave operator.
\end{remark}

\section{More on Scattering for the Schr\"odinger Equation}
\begin{lemma}
  Let $H = -\Delta + V$ and $f \in L^2_{\mathrm{ac}}$.
  Then we have
  \[
    \|\chi_{\{|x| \le R\}} e^{itH} f\|_{L^2} \xrightarrow[t \to \infty]{} 0
  \]
\end{lemma}

\begin{remark}
  We have $\|e^{itH} f\|_{L^2} = \|f\|_{L^2}$,
  but if we localize the solution, then
  \[
    \|\chi_{\{|x| \le R\}} e^{itH} f\|_{L^2} \xrightarrow[t \to \infty]{} 0
  \]
  if $f \in L^2_{\mathrm{ac}}$. But if
  $f$ is an eigenfunction of $H$, i.e.
  $f \in L^2$ with $e^{iH t} f = e^{i\lambda t} f$,
  this does not hold.
  In particular, if a quantum particle is
  not trapped by the potential $V$, then
  the probability of finding it in any
  compact region will go to $0$ as
  $t \to \infty$.
\end{remark}

\end{document}
