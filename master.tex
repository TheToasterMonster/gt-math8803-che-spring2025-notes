\documentclass[12pt, letterpaper, oneside]{book}
\usepackage[margin={0.6in, 0.75in}]{geometry}
\usepackage{microtype}
% \usepackage{kpfonts}
\usepackage{amsmath, amssymb, amsthm}
\usepackage{parskip}
\usepackage[many]{tcolorbox}
\usepackage{footnote}
\usepackage{cancel}
\usepackage{titlesec}
\usepackage{pgffor}
\usepackage[shortlabels, inline]{enumitem}
\usepackage{hyperref}
\usepackage{tikz-cd}

\usepackage[overload]{textcase}

\renewcommand{\chaptername}{Lecture}
\newtheorem{axiom}{Axiom}[chapter]
\newtheorem{theorem}{Theorem}[chapter]
\newtheorem{prop}{Proposition}[chapter]
\newtheorem{corollary}{Corollary}[theorem]
\newtheorem{lemma}{Lemma}[chapter]
\theoremstyle{definition}
\newtheorem{definition}{Definition}[chapter]
\newtheorem{exercise}{Exercise}[chapter]
\newtheorem{example}{Example}[definition]
\newtheorem*{remark}{Remark}

\tcbset{sharp corners, breakable, enhanced, parbox=false}
\newtcolorbox{mybox}[3][]
{
  colframe = #2!150,
  colback  = #2!5,
  coltitle = #2!0!white,  
  title    = {#3},
  #1,
}

\titleformat{\chapter}[display]
    {\normalfont\huge\bfseries}{\chaptertitlename\ \thechapter}{20pt}{\Huge}
\titlespacing*{\chapter}{0pt}{0pt}{40pt}

\newcommand{\R}{\mathbb{R}}
\newcommand{\N}{\mathbb{N}}
\newcommand{\Z}{\mathbb{Z}}
\newcommand{\C}{\mathbb{C}}
\newcommand{\Q}{\mathbb{Q}}
\newcommand{\F}{\mathbb{F}}
\newcommand{\sphere}{\mathbb{S}}
\newcommand{\ZF}{\mathsf{ZF}}
\newcommand{\ZFC}{\mathsf{ZFC}}
\newcommand{\AC}{\mathsf{AC}}

\newcommand{\T}{\mathcal{T}}
\newcommand{\B}{\mathcal{B}}

\DeclareMathOperator{\Vol}{Vol}
\DeclareMathOperator{\Int}{int}
\DeclareMathOperator{\area}{area}
\DeclareMathOperator{\curl}{curl}
\DeclareMathOperator{\maps}{maps}
\DeclareMathOperator{\id}{id}
\DeclareMathOperator{\con}{\#}

\title{MATH 8803: Nonlinear Dispersive Equations}
\author{Frank Qiang\\Instructor: Gong Chen}
\date{Georgia Institute of Technology\\Spring 2025}

\begin{document}
  \maketitle

  \begingroup
  \let\cleardoublepage\clearpage
  \tableofcontents
  \endgroup

  % \foreach \i in {00, 01, 02, 03, 04, ..., 50} {%
  %   \edef\FileName{lectures/lecture\i.tex}%     The % here are necessary to eliminate any
  %   \IfFileExists{\FileName}{%  spurious spaces that may get inserted
  %      \input{\FileName}%       at these points
  %   }
  % }
  \chapter{Jan.~6 --- Introduction to Dispersion}

\section{Introduction to Dispersion}
\begin{definition}
  An evolution equation is \emph{dispersive} if
  when no boundary conditions are imposed (e.g. on
  $\R^n$), its wave solutions spread out in space
  as they evolve in time.
\end{definition}

\begin{example}
  Two classic examples of dispersive equations are:
  \begin{itemize}
    \item The \emph{Schr\"odinger equation}: $i u_t + \Delta u = 0$.
    \item The \emph{Airy (linearized KdV) equation}:
      $u_t + u_{xxx} = 0$.
  \end{itemize}
\end{example}

\begin{remark}
  Consider the equation
  $u_t + p(\partial_x) u = 0$, where $p$ is a polynomial,
  and a plane-wave solution
  \[
    u(t, x) = e^{i(kx - \omega t)}
    = e^{ik(x - (\omega / k) t)}.
  \]
  Here $k$ is the \emph{wave number} or
  \emph{space frequency}, and $\omega$ is the
  \emph{(time) frequency}. Plugging the plane-wave
  solution into the equation, we obtain the
  relation $\omega(k) = -i p(ik)$, i.e.
  \[
    \frac{\omega(k)}{k} = \frac{1}{ik} p(ik).
  \]
  The above equation is known as the
  \emph{dispersive relation}. This gives the
  traveling speed of the plane-wave solution with
  wave number $k$,
  which is called the \emph{phase velocity}.
\end{remark}

\begin{example}
  The following are some examples of dispersive relations:
  \begin{itemize}
    \item For the \emph{linear advection equation}
      $u_t + c u_x = 0$ with $c \in \R$, one
      can compute that $\omega / k = c$.
    \item For the Schr\"odinger equation
      $i u_t + \frac{1}{2} \Delta u = 0$, we have
      $\omega / k = k / 2 \in \R$.
     
      In this case of the Schr\"odinger equation,
      plane waves with large wave number (large space
      frequency) travel faster than low-frequency waves.
  \end{itemize}
\end{example}

\begin{remark}
  In general, dispersion means that different
  frequency plane waves travel at different speeds.
\end{remark}

\begin{remark}
  Given initial data $u_0$,
  we can write using the Fourier transform that
  \[
    u_0 = \int \widehat{u}_0(k) e^{ikx} \, dk.
  \]
  Then we get the solution $u$ as
  \[
    u(t, x) = \int \widehat{u}_0(k) e^{i k (x - (\omega(k) / k) t)} \, dk.
  \]
\end{remark}

\begin{example}
  In the case of the linear advection equation, we
  obtain the solution as
  \[
    u(t, x) = \int \widehat{u}_0(k) e^{i k (x - ct)} \, dk
    = u_0(x - ct).
  \]
  For the Schr\"odinger equation, we instead have
  the solution
  \[
    u(t, x) = \int \widehat{u}_0(k) e^{i k (x - (k / 2) t)} \, dk.
  \]
  Since different $k$ travels at different speeds,
  the original profile quickly spreads out.
\end{example}

\begin{exercise}
  Calculate the dispersive relation $\omega / k$
  for the linearized KdV equation $u_t + u_{xxx} = 0$.
\end{exercise}

\begin{example}
  The \emph{KdV equation} is given by
  \[
    \partial_t u + \partial_{xxx} u + 6 u \partial_x u = 0.
  \]
  This equation is used to model shallow water
  surfaces, and is a nonlinear dispersive
  equation. Russell observed a great bump of water
  in a channel that traveled for a long time
  and kept its shape. This is due to the
  nonlinear effects in the KdV equation, and these
  effects are called \emph{solitons}.
\end{example}

\begin{definition}
  A \emph{soliton} is a self-reinforcing solitary
  wave (a wave packet or pulse) that maintains
  its shape while traveling at a constant speed.
\end{definition}

\section{Fourier Transform and the Free Schr\"odinger Equation}
Consider the following free Schr\"odinger equation:
\[
  \begin{cases}
    i \partial_t \psi + \frac{1}{2} \Delta \psi = 0, \\
    \psi|_{t = 0} = \psi_0.
  \end{cases}
\]
We will solve this equation using the \emph{Fourier transform}
\[
  \widehat{f}(\xi) = \int_{\R^d} e^{-ix \cdot \xi} f(x)\, dx.
\]
Note that one can recover $f$ from its Fourier
transform via the \emph{inversion formula}
\[
  f(x) = \frac{1}{(2\pi)^d} \int_{\R^d} e^{ix \cdot \xi} \widehat{f}(\xi)\, d\xi.
\]

\begin{exercise}
  Check that $(\partial_{x_j} f)^\wedge = i \xi_j \widehat{f}$.
\end{exercise}

Applying the Fourier transform to the free Schr\"odinger
equation, one has
\[
  i \partial_t \psi + \frac{1}{2} \Delta \psi = 0 \quad
  \xrightarrow{\text{F.T.}} \quad
  i \partial_t \widehat{\psi} - \frac{1}{2} |\xi|^2 \widehat{\psi} = 0
\]
and initial condition $\widehat{\psi}(0, \xi) = \widehat{\psi}_0(\xi)$.
So for fixed $\xi$, we have an ODE, so we
can solve the equation via
\[
  \widehat{\psi}(t, \xi) =
  e^{-i |\xi|^2 t / 2} \widehat{\psi}_0(\xi).
\]
Now by applying the inverse Fourier transform, we obtain
the solution
\[
  \psi(t, x) = \frac{1}{(2\pi)^d} \int e^{ix\xi} \widehat{\psi}(t, \xi)\, d\xi
  = \frac{1}{(2\pi)^d} \int e^{ix\xi} e^{-i |\xi|^2 t / 2} \widehat{\psi}_0(\xi)\, d\xi.
\]
Recalling Plancherel's theorem that
$\|f\|_{L^2} = C\|\widehat{f}\|_{L^2}$ (for a constant
$C$ independent of $f$), we obtain
\[
  \|\psi(t, x)\|_{L^2}
  = C\|\widehat{\psi}(t, \xi)\|_{L^2}
  = C\|\widehat{\psi}(0, \xi)\|_{L^2}
  = \|\psi(0, x)\|_{L^2}
  = \|\psi_0(x)\|_{L^2},
\]
where the second equality is noticing that
$e^{-i |\xi|^2 t / 2}$ is purely imaginary. This is
a rigorous justification that the linear Schr\"odinger
evolution preserves the $L^2$ norm of the solution.

\begin{exercise}
  Compute that
  \[
    \frac{d}{dt} \int_{\R^d} |\psi(t, x)|^2\, dx = 0.
  \]
  This is an alternative way to show that
  the $L^2$ norm of the solution is preserved.
\end{exercise}

\section{Sobolev Spaces}
\begin{definition}
  The \emph{Sobolev spaces} $H^{\gamma} = W^{\gamma, 2}$
  for $\gamma \in \R$
  are defined via the norm
  \[
    \|f\|_{H^{\gamma}}
    = \left(\int_{\R^d} (1 + |\xi|^2)^\gamma |\widehat{f}(\xi)|^2\, d\xi\right)^{1 / 2}.
  \]
  The \emph{homogeneous Sobolev spaces} $\dot{H}^\gamma$
  are defined by the norm
  \[
    \|f\|_{\dot{H}^\gamma}
    = \left(\int_{\R^d} |\xi|^{2\gamma} |\widehat{f}(\xi)|^2\, d\xi\right)^{1 / 2}.
  \]
\end{definition}

\begin{remark}
  If $\gamma \in \N$ and $d = 1$, then
  \[
    \|f\|_{H^{\gamma}}
    \sim \sum_{m = 0}^\gamma \| \partial_x^m f \|_{L^2}.
  \]
  In particular, this means that
  $f \in H^\gamma$ if and only if
  $\partial_x^m f \in L^2$ for all $m \le \gamma$.
\end{remark}

\begin{exercise}
  Check that if $f_\lambda(x) = f(\lambda x)$, then
  $\widehat{f_\lambda}(\xi) = \lambda^{-d} \widehat{f}(\xi / \lambda)$.
\end{exercise}

\begin{remark}
  In the Sobolev spaces, this means that (change
  variables $\eta = \xi / \lambda$ for the last equality)
  \[
    \|f_\lambda\|_{\dot{H}^{\gamma}}
    = \left(\int_{\R^d} |\xi|^{2\gamma} |\widehat{f_\lambda}(\xi)|^2\, d\xi\right)^{1 / 2}
    = \left(\int_{\R^d} |\xi|^{2\gamma} |\lambda^{-d} \widehat{f}(\xi / \lambda)|^2\, d\xi\right)^{1 / 2}
    = \lambda^{\gamma - d / 2} \|f\|_{\dot{H}^{\gamma}}.
  \]
\end{remark}

\begin{lemma}
  In the Schr\"odinger equation, $\|\psi(t)\|_{H^{\gamma}} = \|\psi_0\|_{H^{\gamma}}$
  and $\|\psi(t)\|_{\dot{H}^{\gamma}} = \|\psi_0\|_{\dot{H}^{\gamma}}$
  for all $t$ and $\gamma$.
\end{lemma}

\begin{proof}
  We can compute that
  \[
    \|\psi(t)\|_{\dot{H}^{\gamma}}
    = \int_{\R^d} |\xi|^{2\gamma} |\widehat{\psi}(t, \xi)|^2\, d\xi
    = \int_{\R^d} |\xi|^{2\gamma} |e^{-i |\xi|^2 / 2}\widehat{\psi_0}(\xi)|^2\, d\xi
    = \int_{\R^d} |\xi|^{2\gamma} |\widehat{\psi_0}(\xi)|^2\, d\xi
    = \|\psi_0\|_{\dot{H}^{\gamma}}.
  \]
  The same argument works for the $H^{\gamma}$ case
  after replacing $|\xi|^{2\gamma}$ with
  $(1 + |\xi|^2)^\gamma$.
\end{proof}

\end{document}
