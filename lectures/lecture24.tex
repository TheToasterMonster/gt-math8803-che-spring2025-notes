\chapter{Apr.~14 --- Scattering Theory, Part 3}

\section{More Scattering Theory}

\begin{proof}[Proof of Lemma \ref{lem:local}]
  By density, it suffices to check
  the statement for $f = (H + i)^{-1} g$. Note
  that the operator
  $(H + i)^{-1}$ is well-defined since
  the spectrum of $H$ lies on the real line. If
  $f = L^2_{\mathrm{ac}}$, then
  $P_{\mathrm{ac}} g = g$ by spectral theory.
  In terms of $g$, we need to show that
  \[
    \|\chi_{\{|x| \le R\}} (H + i)^{-1} e^{itH} g\|_{L^2} \xrightarrow[t \to \infty]{} 0.
  \]
  The operator $C = \chi_{\{|x| \le R\}} (H + i)^{-1}$
  is compact, so it can be approximated
  by finite-rank operators. So it suffices
  to prove $\|M e^{itH} g\|_{L^2} \to 0$
  as $t \to \infty$, where $M$ is a finite-rank
  operator. We can further assume that
  the rank of $M$ is $1$, i.e. we can find
  $\psi_1, \psi_2 \in L^2$ such that
  $Mg = \langle g, \psi_2 \rangle \psi_1$. So
  \[
    \|M e^{itH} g \|_{L^2}
    = \| \psi_1 \langle e^{itH} g, \psi_2 \rangle \|_{L^2}
    = \| \psi_1 \|_{L^2} |\langle e^{itH} g, \psi_2 \rangle|.
  \]
  By spectral theory, write
  \[
    \|M e^{itH} g \|_{L^2}
    = \left|
    \int_{-\infty}^\infty e^{-it \lambda}
    \langle E(d\lambda) g, \psi_2
    \right|.
  \]
  Write $\langle E(d\lambda) g, \psi_2 \rangle = G(\lambda)\, d\lambda$ for some continuous
  function $G$ (since
  $g \in L^2_{\mathrm{ac}}$, so we can
  differentiate with respect to Lebesgue measure).
  Then we have
  \[
    \int_{-\infty}^\infty e^{-it\lambda}
    \langle E(d\lambda) g, \psi_2 \rangle
    = \int_{-\infty}^\infty e^{-it\lambda} G(\lambda)\, d\lambda
    \xrightarrow[t \to \infty]{} 0
  \]
  by the Riemann-Lebesgue lemma since
  $G$ is continuous, which proves the result.
\end{proof}

\begin{theorem}[RAGE: Ruelle-Amrein-Georgescu-Enss]
  Let $f \in L^2_{\mathrm{c}} = L^2_{\mathrm{ac}} \oplus L^2_{\mathrm{sc}}$.
  Then
  \[
    \frac{1}{T} \int_{-T}^T \| \chi_{\{|x| \le R\}} e^{-itH} f \|_{L^2}^2\, dt
    \xrightarrow[T \to \infty]{} 0.
  \]
\end{theorem}

\begin{proof}
  By density,
  we consider $f = (H + i)^{-1} g$. Using
  the same reduction as before, it suffices
  to show
  \[
    \frac{1}{T} \int_{-T}^T \|\langle e^{itH} g, \psi_2 \rangle \psi_1\|_{L^2}^2\, dt \xrightarrow[T \to \infty]{} 0
    \iff
    \frac{1}{T} \int_{-T}^T |\langle e^{itH} g, \psi_2\rangle|^2\, dt \xrightarrow[T \to \infty]{} 0.
  \]
  As $T \to \infty$, by spectral theory,
  \[
    \frac{1}{T} \int_{-T}^T \left| \int_{-\infty}^\infty e^{i\lambda t} \langle E(d\lambda) g, \psi_2 \rangle\, d\lambda \right|^2 dt
    = \frac{1}{T} \int_{-T}^T \left| \int_{-\infty}^\infty e^{i t \lambda} \mu_{g, \psi_2}(d\lambda) \right|^2 dt
    = \frac{1}{T} \int_{-T}^T |\widehat{\mu}_{g, \psi_2}(t)|^2\, dt
  \]
  for some measure $\mu_{g, \psi_2}$.
  Wiener's theorem (Theorem \ref{thm:wiener})
  then implies that
  \[
    \frac{1}{T} \int_{-T}^T |\widehat{\mu}_{g, \psi_2}(t)|^2\, dt
    \xrightarrow[T \to \infty]{}
    2 \sum_{\tau \in \R} |\mu_{g, \psi_2}(\{\tau\})|^2 = 0
  \]
  since $\mu_{g, \psi_2}$ is a continuous
  measure, so
  $\mu_{g, \psi_2}(\{\tau\}) = 0$ for all $\tau \in \R$.
\end{proof}

\begin{theorem}[Wiener]\label{thm:wiener}
  Let $\mu \in \mathcal{M}(\R)$, i.e. $\mu$ is
  a measure on $\R$. Then
  \[
    \frac{1}{2T} \int_{-T}^T |\widehat{\mu}(t)|^2\, dt
    \xrightarrow[T \to \infty]{} \sum_{\tau \in \R} |\mu(\{\tau\})|^2.
  \]
\end{theorem}

\begin{proof}
  For $\mu, \nu \in \mathcal{M}(\R)$ and
  a measurable set $E$, denote
  \[
    \mu * \nu(E)
    = \int_\R \mu(E - t)\, \nu(dt).
  \]
  Set $\mu^\#(E) = \overline{\mu(-E)}$, so that
  $\widehat{\mu^\#}(\xi) = \overline{\widehat{\mu}(\xi)}$.
  Then we have
  \begin{align*}
    \mu * \mu^\#(\{0\})
    = \int_\R \mu(\{-t\})\, \mu^\#(dt)
    &= \sum_{\text{$-t$ atom of $\mu$}}
    \mu(\{-t\}) \mu^\#(\{t\})
    = \sum_{\text{$t$ atom of $\mu$}}
    \mu(\{t\}) \mu^\#(\{-t\}) \\
    &= \sum_{\text{$t$ atom of $\mu$}}
    \mu(\{t\}) \overline{\mu(\{t\})}
    = \sum_{\text{$t$ atom of $\mu$}}
    |\mu(\{t\})|^2.
  \end{align*}
  We can also compute that
  \[
    \frac{1}{2T} \int_{-T}^T e^{it\tau} \widehat{\mu}(\tau)\, dt
    = \frac{1}{2T} \int_{-T}^T e^{it\tau} \int_{-\infty}^\infty e^{-is\tau} \mu(ds) dt
    = \frac{1}{2T} \int_{-T}^T \int_{-\infty}^\infty e^{i(t - s)\tau} \mu(ds) dt.
  \]
  By Fubini's theorem, we can write
  \begin{align*}
    \frac{1}{2T} \int_{-T}^T e^{it\tau} \widehat{\mu}(\tau)\, dt
    &= \int_{-\infty}^\infty \left(\frac{1}{2T} \int_{-T}^T e^{i(t - s)\tau} dt \right) \mu(ds)
    = \int_{-\infty}^\infty \frac{1}{2T} D_T(s - t)\, \mu(ds) \\
    &= \mu(t) + \int_{-\infty}^\infty \frac{1}{2T} D_T(s - t) \left(\mu(ds) - \mu(t) \delta_{\{t\}}(ds)\right),
  \end{align*}
  where $D_T$ is the Dirichlet kernel. Note
  that $D_T$ satisfies
  \[
    \int_{-\infty}^\infty \frac{1}{2T} D_T(s - t) \delta_{\{t\}}(ds) = 1,
  \]
  and $D_T(\tau) / 2T$ is an approximation
  of the identity, so we have
  \[
    \frac{1}{2T} \int_{-T}^T e^{it\tau} \widehat{\mu}(\tau)\, dt
    = \mu(t) + \int_{-\infty}^\infty \frac{1}{2T} D_T(s - t) \left(\mu(ds) - \mu(t) \delta_{\{t\}}(ds)\right)
    \xrightarrow[T \to \infty]{} \mu(t).
  \]
  Applying this to the measure
  $\mu * \mu^\#(\{0\})$, we obtain
  \[
    \frac{1}{2T} \int_{-T}^T |\widehat{\mu}|^2\, d\tau
    =
    \frac{1}{2T} \int_{-T}^T \widehat{\mu} \cdot \widehat{\mu^\#}\, d\tau
    = \frac{1}{2T} \int_{-T}^T (\mu * \mu^\#)^{\wedge}\, dt
    \xrightarrow[T \to \infty]{} \mu * \mu^\#(\{0\})
    = \sum_{\text{$t$ atom of $\mu$}}
    |\mu(\{t\})|^2,
  \]
  which is the desired result.
\end{proof}
