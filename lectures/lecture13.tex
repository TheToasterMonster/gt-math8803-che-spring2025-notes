\chapter{Feb.~24 --- Dispersive Decay with Potential, Part 2}

\section{High Energy Estimates, Continued}
\begin{prop}
  We have the estimate
  \[
    |\langle e^{iH t} \chi(H) P_{\mathrm{ac}} f, g \rangle|
    \lesssim t^{-1 / 2} \|f\|_{L^1} \|g\|_{L^1}.
  \]
\end{prop}

\begin{proof}
  Using the Born series, we have
  \begin{align*}
    &|\langle e^{iH t} \chi(H) P_{\mathrm{ac}} f, g \rangle| \\
    &\le \sum_{n = 0}^\infty \left|
    \frac{(-1)^n}{2\pi i} \int_0^\infty e^{it\lambda} \chi(\lambda) \left(
      \langle R_0(\lambda + i0) (V R_0(\lambda + i0))^n f, g \rangle
      - \langle R_0(\lambda - i0) (V R_0(\lambda - i0))^n f, g \rangle
    \right) d\lambda
    \right|.
  \end{align*}
  The general term in the above series is
  (the minus term is similar)
  \[
    \int_0^\infty e^{it\lambda} \chi(\lambda)
    \langle R_0(\lambda + i0) (V R_0(\lambda + i0))^n f, g \rangle\, d\lambda.
  \]
  Changing variables using $\lambda = \eta^2$, we have
  \[
    \int_0^\infty e^{it \eta^2} \chi(\eta^2) \cdot 2\eta
    \langle R_0(\eta^2 + i0) (V R_0(\eta^2 + i0))^n f, g \rangle\, d\eta.
  \]
  Using the calculation that
  $R_0(\lambda \pm i 0) = \pm i e^{\pm i |x| \sqrt{\lambda}} / 2\sqrt{\lambda}$,
  we have $R_0(\eta^2 + i 0) = i e^{i|x| \eta} / 2\eta$,
  and we get
  \[
    \iint f(x_0) g(x_{n + 1})
    \int \dots \int V(x_1) \dots V(x_n)
    \int_0^\infty e^{it\eta^2} 2\eta \cdot \chi(\eta^2)
    \frac{e^{i \eta \sum_{j = 1}^{n + 1} |x_j - x_{j - 1}|}}{(-2\eta i)^{n + 1}}\, d\eta\, dx_1 \dots dx_n\, dx_0 dx_{n + 1}.
  \]
  Note the above integral by $(*)$, and by Minkowski's
  inequality we have
  \begin{align*}
    &|(*)| \\
    &\le
    \int \dots \int |f(x_0)| |g(x_{n + 1})| V(x_1) \dots V(x_n)
    \sup_{x_0, \dots, x_{n + 1}} \left| \int_0^\infty e^{it\eta^2} 2\eta \cdot \chi(\eta^2) \frac{e^{i \eta \sum_{j = 1}^{n + 1} |x_j - x_{j - 1}|}}{(-2\eta i)^{n + 1}}\, d\eta \right|
    dx_0 \dots dx_{n + 1}.
  \end{align*}
  Setting $a = \sum_{j = 1}^{n + 1} |x_j - x_{j - 1}|$,
  the inner integral satisfies
  \[
    \sup_{x_0, \dots, x_{n + 1}} \left| \int_0^\infty e^{it\eta^2} 2\eta \cdot \chi(\eta^2) \frac{e^{i \eta \sum_{j = 1}^{n + 1} |x_j - x_{j - 1}|}}{(-2\eta i)^{n + 1}}\, d\eta \right|
    = \sup_a \left| \int_0^\infty e^{it\eta^2} 2\eta \cdot \chi(\eta^2) \frac{e^{i \eta a}}{(-2\eta i)^{n + 1}}\, d\eta \right|.
  \]
  By a suitable change of variables, it suffices to study
  \[
    \sup_a \left| \int_{-\infty}^\infty e^{it\eta^2} 2\eta \cdot \chi(\eta^2) \frac{e^{i \eta a}}{(-2\eta i)^{n + 1}}\, d\eta \right|
    = \sup_a \left| \int_{-\infty}^\infty e^{it\eta^2 + i\eta a} \widehat{\mu}_n(\eta)\, d\eta \right|,
  \]
  where we have
  \[
    \widehat{\mu}_n(\eta) = \frac{2\eta \cdot \chi(\eta^2)}{(-2\eta i)^{n + 1}}
    = \|e^{i\partial_x^2 t} \mu_n \|_{L^\infty}
    \le t^{-1 / 2}\|\mu_n \|_{L^1}.
  \]
  The key idea is to reduce
  the computation to the $1$-D free Schr\"odinger
  equation. Writing
  \[
    \mu_n(x) = \int_{-\infty}^\infty \frac{\chi(\lambda^2)}{\lambda^n} e^{-i\lambda x}\, d\lambda,
  \]
  we want to estimate $\|\mu_n\|_{L^1}$. For $n = 0$,
  we have
  \[
    \mu_0(x) = \int_{-\infty}^\infty \chi(\lambda^2) e^{-i\lambda x}\, d\lambda
    = \int_{-\infty}^\infty (1 + (1 - \chi(\lambda^2))) e^{-i\lambda x}\, d\lambda
    = \delta_0 + \int_{-\infty}^\infty (1 - \chi(\lambda))^2 e^{-i\lambda x}\, d\lambda.
  \]
  The latter integral is a Schwartz
  function since $1 - \chi(\lambda)$ is compactly
  supported. So $\|\mu_0\|_{L^1} < \infty$.
  For general $n$, we would like to estimate
  $\|x^2 \mu_n(x)\|_{L^\infty}$:
  \[
    \|x^2 \mu_n(x)\|_{L^\infty}
    = \|(\widehat{\mu}_n''(x))^\vee\|_{L^\infty}
    \le \|\widehat{\mu}_n''\|_{L^1}
    = \left\| \left(\frac{\chi(\lambda^2)}{\lambda^n}\right)'' \right\|
    \lesssim \lambda_0^{-n / 2},
  \]
  where the second inequality is by Young's inequality
  and the last inequality is because
  \[
    \left(\frac{\chi(\lambda^2)}{\lambda^n}\right)''
    \lesssim \left|\frac{\chi'(\lambda^2)}{\lambda^{n + 2}}\right|
    + \left|\frac{\lambda \chi'(\lambda^2)}{\lambda^{n + 1}}\right|
    + \left|\frac{\lambda^2 \chi''(\lambda^2)}{\lambda^{n}}\right|
  \]
  and the observation that $\chi', \chi''$ are
  compactly supported functions. Thus we have
  \[
    |\mu_n(x)| \lesssim (\lambda_0)^{-n / 2} |x|^{-2}.
  \]
  This gives us decay as $x \to \infty$, but we also
  need to estimate $\|\mu_n\|_{L^\infty}$: For $n \ge 2$,
  \[
    \| \mu_n \|_{L^\infty}
    \lesssim \|\widehat{\mu}_n\|_{L^1}
    = \left\| \frac{\chi(\lambda^2)}{\lambda^n} \right\|
    \lesssim (\lambda_0)^{-n / 2}
  \]
  since $\widehat{\mu}_n$ is integrable for $n \ge 2$.
  When $n = 1$, we have
  \[
    \|\mu_1(x)\|_{L^\infty}
    = \left\| \left(\frac{\chi(\lambda^2)}{\lambda}\right)^\vee \right\|_{L^\infty}
    = \| (\chi(\lambda^2))^\vee * (1 / \lambda)^\vee \|_{L^\infty}
    = \| \chi(\lambda^2) \|_{L^1} \|(1 / \lambda)^\vee\|_{L^\infty} < \infty
  \]
  since $\|\chi(\lambda^2)\|_{L^1}$ is finite and
  $(1 / \lambda)^\vee = -i \sign(x)$, so
  $\|(1 / \lambda)^\vee\|_{L^\infty} \le 1$. Thus
  for $n \ge 1$,
  \[
    \|\mu_n\|_{L^\infty} \lesssim (\lambda_0)^{-n / 2}
    \quad \text{and} \quad
    \|\mu_n(x)\| \lesssim (\lambda_0)^{-n / 2} |x|^{-2},
  \]
  which gives $\|\mu_n \|_{L^1} \lesssim (\lambda_0)^{-n / 2}$.
  When $n = 0$, we already know
  $\|\mu_0\|_{L^1} < \infty$. This gives
  \[
    \sup_a \left| \int_0^\infty e^{it\eta^2} 2\eta \cdot \chi(\eta^2) \frac{e^{i \eta a}}{(-2\eta i)^{n + 1}}\, d\eta \right|
    \lesssim |t|^{-1 / 2} (\lambda_0)^{-n / 2},
  \]
  and so we have
  \[
    (*) \lesssim \|f\|_{L^1} \|g\|_{L^1} \|V\|_{L^1}^n (\lambda_0)^{-n / 2} |t|^{-1 / 2}.
  \]
  As long as $\lambda_0^{-1 / 2} \| V \|_{L^1} < 1$,
  we can sum the Born series and get
  \[
    \langle e^{iHt} \chi(H) P_{\mathrm{ac}} f, g \rangle
    \le |t|^{-1 / 2} \|f\|_{L^1} \|g\|_{L^1} \sum_{n = 0}^\infty (\lambda_0^{-1 / 2} \|V\|_{L^1})^n
    \lesssim |t|^{-1 / 2} \|f\|_{L^1} \|g\|_{L^1},
  \]
  which is the desired estimate.
\end{proof}

\begin{remark}
  The takeaway from this proof is that for the
  high frequency part, we use a Born series, the
  free resolvent, and explicit computations.
\end{remark}

\section{Low Energy Estimates}

\begin{remark}
  In this setting, the influence of the potential
  becomes more serious.
\end{remark}

\begin{lemma}[Low energy]\label{lem:low-energy}
  Let $V$ be ``nice'' and $\chi$ a cutoff function
  near $0$. Then
  \[
    \| e^{iHt} \chi(H) P_{\mathrm{ac}} f \|_{L^\infty}
    \le |t|^{-1 / 2} \|f\|_{L^1}.
  \]
\end{lemma}

\begin{proof}
  We use spectral theory. Write
  \[
    e^{iHt} \chi(H) P_{\mathrm{ac}} f
    = \frac{1}{2\pi i} \int_0^\infty \chi(\lambda) e^{-it\lambda} [R(\lambda + i0) - R(\lambda - i0)] f\, d\lambda.
  \]
  We will use the convention (note that
  $(\eta + i\epsilon)^2 = \eta^2 - \epsilon^2 + 2i\eta \epsilon$)
  \[
    R(\eta^2 + i 0) = \lim_{\epsilon \to 0^+} R((\eta + i\epsilon)^2)
    = \lim_{\epsilon \to 0^+} R(\eta^2 + i \sign (\eta) \epsilon).
  \]
  Then we can write
  \[
    e^{iHt} \chi(H) P_{\mathrm{ac}} f
    = \frac{1}{\pi i} \int_{-\infty}^\infty \eta \cdot \chi(\eta^2) e^{it\eta^2} R(\eta^2 + i0) f\, d\eta.
  \]
  Using a Green's function, we can write
  \[
    R(\lambda^2 + i 0)(x, y)
    = \frac{f_+(\lambda, y)f_-(\lambda, x)}{W[f_+(\lambda, \cdot), f_-(\lambda, \cdot)]} \mathbbm\{x < y\}
    + \frac{f_+(\lambda, x)f_-(\lambda, y)}{W[f_+(\lambda, \cdot), f_-(\lambda, \cdot)]} \mathbbm\{x > y\},
  \]
  where $f_{\pm}$ are the Jost functions solving
  \[
    \begin{cases}
      -f_{\pm}'' + V f_{\pm} = \lambda^2 f_{\pm}, \\
      |f_+(\lambda, x) - e^{i\lambda x}| \xrightarrow[x \to \infty]{} 0, \\
      |f_-(\lambda, x) - e^{-i\lambda x}| \xrightarrow[x \to \infty]{} 0.
    \end{cases}
  \]
  The conclusion from this is that
  \begin{align*}
    \langle e^{iHt} \chi(H) P_{\mathrm{ac}} f, g \rangle
    &=
    \frac{1}{\pi i} \int_{-\infty}^\infty
    \int e^{it \lambda^2} \lambda \cdot  \chi(\lambda^2)
    \frac{f_+(\lambda, y) f_-(\lambda, x)}{W(\lambda)}\, d\lambda\, f(x) \overline{g(y)} \mathbbm{1}_{\{x < y\}}\, dx\, dy \\
    & \quad \quad +
    \frac{1}{\pi i} \int_{-\infty}^\infty
    \int e^{it \lambda^2} \lambda \cdot  \chi(\lambda^2)
    \frac{f_+(\lambda, x) f_-(\lambda, y)}{W(\lambda)}\, d\lambda\, f(x) \overline{g(y)} \mathbbm{1}_{\{y < x\}}\, dx\, dy.
  \end{align*}
  We will finish the proof next time.
\end{proof}
