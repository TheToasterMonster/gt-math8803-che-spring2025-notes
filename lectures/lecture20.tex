\chapter{Mar.~26 --- Scattering Theory}

\section{Sharp Agmon Estimate}
\begin{theorem}[Sharp Agmon estimate in $\R^3$]
  In $\R^3$, suppose $(-\Delta + V) \psi = E\psi$ with
  $E < 0$, and
  let $V$ have ``nice'' decay (the precise
  condition is $V \in L^{3 / 2, 1})$ Then
  \[
    |\psi(x)| \lesssim \frac{1}{\langle x \rangle} e^{-(\sqrt{-E}) |x|}.
  \]
\end{theorem}

\begin{remark}
  Recall that we have previously shown
  $|\psi| \le e^{-(1 - \epsilon) P_E(x)}$, where
  $P_E(x) \sim (\sqrt{-E}) |x|$. Note that this
  bound is not sharp, we have an $\epsilon$ loss
  in the exponent.
\end{remark}

\section{Short-Range Scattering Theory}

\begin{remark}
  Let $H_0 = -\Delta$ and $H = -\Delta + V$.
  We would like to compare $e^{i H_0 t}$ and $e^{i H t}$.
  More precisely, for $f \in L^2$,
  we would like to determine if we can  find
  $g$ such that
  \[
    \|e^{-itH_0} f - e^{-itH} g\|_{L^2} \xrightarrow[t \to \infty]{} 0. \tag{$*$}
  \]
  The goal is the following: For a prescribed free flow,
  can we find a perturbed flow such that they match as
  $t \to \infty$? This is known as the
  \emph{wave operator}
  $\Omega_{\mp} = \lim_{t \to \pm \infty} e^{itH} e^{-itH_0}$
  (in the strong $L^2$ sense).

  Note that the equation $(*)$ is equivalent to (since
  $e^{itH}$ is an isometry on $L^2$)
  \[
    \|e^{itH} e^{-i t H_0} f - g\|_{L^2} \xrightarrow[t \to \infty]{} 0,
  \]
  so if the wave operator $\Omega_-$ exists,
  then $\|\Omega_- f - g\|_{L^2} = 0$, so
  we can take $g = \Omega_- f$.
\end{remark}

\begin{theorem}
  Suppose that $\sup_x |V(x)| (1 + |x|)^{1 + \epsilon} < \infty$. Then
  \begin{enumerate}
    \item $\Omega_{\mp}$ exist;
    \item $\Omega_{\mp}$ are isometries;
    \item (intertwining properties)
      $\Omega_- e^{-itH_0} = e^{-itH} \Omega_-$, so
      $e^{-itH_0} = \Omega_-^{-1} e^{-itH} \Omega_-$
      and $e^{-itH} = \Omega_- e^{-itH_0} \Omega_-^{-1}$;
    \item $\Omega_- H_0 f = H \Omega_- f$
      for $f \in H^2$;
    \item $H \circ P_{\Ran(\Omega_-)} = \Omega_- H_0 \Omega_-^*$
      in $H^2$;
    \item $\Ran(\Omega_-) \subseteq L^2_{\mathrm{ac}}(H) = \HH_{\mathrm{ac}}$.
  \end{enumerate}
\end{theorem}

\begin{proof}
  $(1)$ The following proof is known as \emph{Cook's method}.
  For a dense subspace of $L^2$, we will show that
  $e^{itH} e^{-itH_0} f$ is a Cauchy sequence
  in time. Fix $s \ge t$, then
  \[
    e^{isH} e^{-isH_0} f - e^{itH} e^{-itH_0} f
    = \int_{t}^s \frac{d}{d\tau} (e^{i\tau H} e^{-i\tau H_0} f) \, d\tau.
  \]
  We would like to show that
  $\|e^{isH} e^{-isH_0} f - e^{itH} e^{-itH_0} f\|_{L^2} \to 0$
  as $s, t \to \infty$, for which is suffices to prove
  \[
    \left|\int_{t}^s \frac{d}{d\tau} (e^{i\tau H} e^{-i\tau H_0} f) \, d\tau\right|
    \xrightarrow[t, s \to \infty]{} 0.
  \]
  We can compute that (recall that $H = H_0 + V$)
  \[
    \frac{d}{d\tau} (e^{i\tau H} e^{-i\tau H_0} f)
    = e^{i\tau H} H e^{-i\tau H_0} f
    - e^{i\tau H} H_0 e^{-i\tau H_0} f
    = e^{i\tau H} V e^{-i\tau H_0} f.
  \]
  Then we can bound (since $e^{i\tau H}$ is an isometry)
  \[
    \left\|\int_{t}^s \frac{d}{d\tau} (e^{i\tau H} e^{-i\tau H_0} f) \, d\tau\right\|_{L^2}
    = \left\|\int_{t}^s e^{i\tau H} V e^{-i \tau H_0} \, d\tau\right\|_{L^2}
    \le \int_{t}^s \left\|V e^{-i \tau H_0}\right\|_{L^2} \, d\tau.
  \]
  We want to show that $\|V e^{-i \tau H_0} f\|_{L^2}$
  is integrable in $\tau$, so the above goes to $0$ as $s, t \to \infty$.\footnote{Note that if $V$ is nice, then $\|Ve^{-i\tau H_0} f\|_{L^2} \lesssim \|e^{-i\tau H_0} f\|_{L^\infty} \lesssim |\tau|^{-d / 2}$, which is integrable for $d \ge 3$. But for $d = 1, 2$, we need a more careful argument.}
  Define
  \[
    \mathcal{C} = \{f \in \mathcal{S}(\R^d)
      : \widehat{f}(\xi) = 0 \text{ for  $|\xi| < a$ and $|\xi| > b$, for some $a, b \in \R^+$}
    \},
  \]
  and note that $\mathcal{C}$ is dense in $L^2$.
  Take $f \in \mathcal{C}$, then for some $a, b$
  we have $\widehat{f}(\xi) = 0$ whenever
  $|\xi| < a$ or $|\xi| > b$. We can write
  \[
    e^{-i t H_0} f
    = \int e^{ix\xi} e^{-i |\xi|^2 t}
    \widehat{f}(\xi)\, d\xi.
  \]
  Write $e^{ix \xi} e^{-i |\xi|^2 t} = e^{it (-|\xi|^2 + \xi(x / t))} = e^{it \phi(\xi)}$.
  We use a stationary phase argument to estimate
  this integral: The
  critical point $\nabla \phi(\xi) = 0$ is the point
  where the oscillations stop. When $\nabla \phi \ne 0$,
  we can write
  \[
    \int e^{it\phi(\xi)} \widehat{f}(\xi)\, d\xi
    = \int \left(\frac{1}{it \nabla \phi(\xi)} \frac{d}{d\xi} e^{it \phi(\xi)}\right) \widehat{f}(\xi)\, d\xi
    = -\frac{1}{t} \int e^{it \phi(\xi)} \frac{d}{d\xi} \left(\widehat{f}(\xi) \frac{1}{\nabla \phi(\xi)}\right)\, d\xi
  \]
  by integration by parts. So it suffices to
  consider the critical point, which happens when
  $x / t - 2 \xi = 0$, or $\xi = -x / 2t$.
  Note that $\widehat{f}(\xi)$ is supported in
  $a < |\xi| < b$, so if
  $|x / 2t| < a$ or $|x / 2t| < b$, then
  \[
    \int e^{ix \xi} e^{-it |\xi|^2} \widehat{f}(\xi)\, d\xi
  \]
  has no critical points. So we get that
  \[
    \left|\int e^{ix \xi} e^{-it |\xi|^2} \widehat{f}(\xi)\, d\xi\right|
    \lesssim |t|^{-n}, \quad
    \text{for any $n \in \N$}.
  \]
  So if $|x| < 2at$ or $|x| > 2tb$,
  then $\|e^{iH_0 f}\|_{L^\infty} \lesssim t^{-n}$
  for any $n \in \N$.
  So we focus on $2at < |x| < 2tb$.
  But in this region, by the pointwise decay of
  $e^{i H_0 t} f$, we have
  $\|e^{i H_0 t} f\|_{L^\infty} \lesssim t^{-d / 2}$.
  Then
  \begin{align*}
    \|V e^{i H_0 t f}\|_{L^2(2at < |x| < 2bt)}
    &\lesssim \frac{1}{(1 + |t|)^{1 + \epsilon}} \|e^{iH_0 t} f\|_{L^2(2at < |x| < 2bt)}
    \lesssim \frac{1}{(1 + |t|)^{1 + \epsilon}} t^{d / 2}
    \|e^{iH_0 t} f\|_{L^\infty} \\
    &\lesssim \frac{1}{(1 + |t|)^{1 + \epsilon}} t^{d / 2}
    t^{-d / 2}
    \lesssim \frac{1}{(1 + |t|)^{1 + \epsilon}}
  \end{align*}
  where we used the hypothesis that
  $|V| \lesssim 1 / (1 + |x|)^{1 + \epsilon}$.
  So
  \[
    \|Ve^{i\tau H_0} f\|_{L^2}
    = \|Ve^{i\tau H_0} f\|_{L^2(|x| < 2a\tau)}
    + \|Ve^{i\tau H_0} f\|_{L^2(|x| > 2b\tau)}
    + \|Ve^{i\tau H_0} f\|_{L^2(2a\tau < |x| < 2b\tau)}.
  \]
  We have already shown that the last term is
  integrable. For the
  first term, we have
  \[
    \|Ve^{i\tau H_0} f\|_{L^2(|x| < 2a\tau)}
    \lesssim \|V\|_{L^2(|x| < 2a\tau)} \| e^{-i\tau H_0} f \|_{L^\infty(|x| < 2a \tau)}
    \lesssim 2a|\tau|^{d / 2} \cdot \tau^{-n} \quad \text{for any $n \in \N$},
  \]
  so if we take $n > d / 2 + 1$, then this
  term is integrable.
  Finally, for the second term,
  \[
    \|Ve^{i\tau H_0} f\|_{L^2(|x| > 2b\tau)}
    \lesssim \|V\|_{L^\infty(|x| > 2b\tau)} \| e^{-i\tau H_0} f \|_{L^2}
    \lesssim \frac{1}{(1 + |\tau|)^{1 + \epsilon}},
  \]
  which is integrable (note that $\|e^{-i \tau H_0} f\|_{L^2} = \|f\|_{L^2}$).
  Thus we obtain
  \[
    \|Ve^{-i \tau H_0}\|_{L^2} \lesssim \tau^{-1 - \epsilon},
  \]
  which is integrable in $\tau$. The conclusion from
  this is that for $s \ge t$,
  \[
    \|e^{isH} e^{-isH_0} f - e^{itH} e^{-itH_0} f\|_{L^2}
    \xrightarrow[s, t \to \infty]{} 0,
  \]
  so $\lim_{t \to \infty} e^{itH} e^{-itH_0} f$ exists
  strongly in $L^2$.
  A similar argument shows that $\Omega_+$ exists.

  $(2)$ This follows since
  $e^{-itH_0}, e^{itH}$ are both isometries,
  so that $e^{itH} e^{-itH_0}$ is an isometry.
  A strong limit of isometries is still an isometry,
  so we get that $\Omega_{\mp}$ are also isometries.

  $(3)$ We can see that (note the group properties
  of $e^{itH}$ and $e^{itH_0}$)
  \begin{align*}
    \Omega_- e^{-it H_0} f
    &= \left(\lim_{s \to \infty} e^{isH -isH_0}\right)
    e^{-itH_0} f
    = e^{iHt} \left(\lim_{s \to \infty} e^{itH} e^{isH} e^{-isH} e^{-itH_0}\right) f \\
    &= e^{-iHt} \lim_{s \to \infty} \left(e^{i(t + s) H} e^{-i(s + t) H_0}\right) f
    = e^{-iHt} \Omega_- f,
  \end{align*}
  which proves the desired result.
\end{proof}
