\chapter{Jan.~22 --- Strichartz Estimates, Part 2}

\section{Proof of Strichartz Estimates}

\begin{proof}[Proof of Theorem \ref{thm:strichartz}]
  The first step is a $TT^*$ argument. Define the
  operator $T$ by $Tf = e^{it \Delta / 2} f$. We know
  that $T : L^2_x(\R^d) \to L^\infty_t(\R \to L^2_x(\R^d))$.
  The adjoint $T^* : L^1_t L^2_x \to L^2_x$ is defined
  via the relation
  \[
    \langle f, T^* G \rangle_{L^2_x}
    = \langle Tf, G \rangle_{L^2_{t, x}}
    = \iint (e^{it \Delta / 2} f)(x) \overline{G}(t, x)\, dt dx
    = \int f(x) \int \overline{(e^{-it \Delta / 2} G(t, \cdot))}(x)\, dt dx,
  \]
  and so we have the formula
  \[
    T^* G = \int (e^{-it \Delta / 2} G(t, \cdot))(x)\, dt.
  \]
  Then we can see that
  \[
    (TT^* G)(t, x)
    = \int (e^{i(t - s) \Delta / 2} G(s, \cdot))(x)\, ds
    = \left(e^{i t \Delta / 2} \int e^{-is \Delta / 2} G(s, \cdot)\, ds\right)(x).
  \]
  Note that there is a convolution structure in the
  time variable. Clearly $TT^* : L^1_t L^2_x \to L^\infty_x L^2_x$.
  Then the goal for now will be to show that
  \[
    \|TT^* G\|_{L^{p'}_{t, x}} \le C \|G\|_{L^p_{t, x}}.
  \]
  To do this, observe that by the above expression for
  $TT^* G$ and Minkowski's inequality, we have
  \[
    \|TT^* G\|_{L^{p'}_{x}}
    \le \int \|e^{i(t - s) \Delta / 2} G(s)\|_{L^{p'}_x}\, ds
    \le C \int |t - s|^{-d(1 / p - 1 / 2)} \|G(s)\|_{L^p_x}\, ds,
  \]
  where the second inequality follows by
  Lemma \ref{lem:decay}. Now by Lemma \ref{lem:hardy-littlewood-sobolev},
  \[
    \|TT^* G\|_{L^q_t L^{p'}_x}
    \le C \|G(s)\|_{L^p_{t, x}}
  \]
  if $1 / q + 1 = d(1 / p - 1 / 2) + 1 / p$
  (also check that $0 < \alpha = d(1 / p - 1 / 2) < 1$,
  where $p = (2d + 4) / (d + 4)$).
  From this relation, we find that we must have
  $q = 2 + 4 / d = p'$, so we have shown the goal.

  Thus we have proved that $TT^* : L^p_{t, x} \to L^{p'}_{t, x}$, where
  $p = (2d + 4) / (d + 4)$. Now
  \[
    \|T^* G\|^2_{L^2_x} = \langle T^* G, T^* G \rangle_{L^2_x}
    = \langle TT^* G, G \rangle_{L^2_{t, x}}
    \le \|TT^* G\|_{L^{p'}_{t, x}} \|G\|_{L^p_{t, x}}
    \le C \|G\|_{L^p_{t, x}}^2,
  \]
  where the first inequality follows by H\"older's inequality
  and the second follows from the goal we just proved.
  Thus we conclude that $T^* : L^p_{t, x} \to L^2_x$, and
  that $T : L^2_x \to L^{p'}_{t, x}$ by duality.
  Therefore,
  \[
    \|T \psi_0\|_{L^{p'}_{t, x}}
    = \|e^{it \Delta / 2} \psi_0\|_{L^{p'}_{t, x}}
    \le C\|\psi_0\|_{L^2_x},
  \]
  so we have proved the Strichartz estimates when
  $F = 0$.

  In the case when $F \ne 0$, by Duhamel's formula
  we have
  \[
    \psi(t) = e^{it \Delta / 2} \psi_0
    - i \int_0^t e^{i(t - s) \Delta / 2} F(s)\, ds,
  \]
  so by the triangle inequality we find that
  \[
    \|\psi\|_{L^{p'}_{t, x}}
    \le \|e^{it \Delta / 2} \psi_0\|_{L^{p'}_{t, x}}
    + \left\|
    \int_0^t e^{i(t - s) \Delta / 2} F(s)\, ds
    \right\|
    \le C \|\psi_0\|_{L^2_x}
    + \left\| \int_0^t \|e^{i(t - s)\Delta / 2} F(s)\|_{L^{p'}_x}\, ds \right\|_{L^{p'}_t},
  \]
  where the last inequality on the second term
  follows by Minkowski's inequality. Then using
  Lemma \ref{lem:decay} and Lemma \ref{lem:hardy-littlewood-sobolev} in the same fashion as before, we can bound
  the latter term by
  \[
    \left\| \int_0^t \|e^{i(t - s)\Delta / 2} F(s)\|_{L^{p'}_x}\, ds \right\|_{L^{p'}_t}
    \le
    C \left\| \int_{-\infty}^\infty |t - s|^{-d (1 / p - 1 / 2)} \|F(s)\|_{L^{p'}_x}\, ds \right\|_{L^{p'}_t}
    \le C \|F\|_{L^p_{t, x}}.
  \]
  Plugging this bound back in gives the desired
  inequality $\|\psi\|_{L^{p'}_{t, x}} \le C \|\psi_0\|_{L^2_x} + C \|F\|_{L^p_{t, x}}$.
\end{proof}

\begin{remark}
  Note that the term
  \[
    TT^* G = \int_{-\infty}^\infty e^{i(t - s) \Delta / 2} G(s)\, ds
  \]
  looks similar to the term from the Duhamel formula
  \[
    \int_0^t e^{i(t - s) \Delta / 2} F(s)\, ds.
  \]
  However, it is possible that these two have
  different estimates, which is why we had to
  argue separately.
\end{remark}

\section{Strichartz Estimates and Harmonic Analysis}
\begin{remark}
  The original intention of Strichartz for these
  estimates was for use in harmonic analysis.
  The Strichartz estimates can actually be derived
  from the \emph{Stein-Tomas restriction theorem}.

  Let $S \subseteq \R^n$ with $n = d + 1$, where
  $S$ is a hypersurface.
  If $f \in L^1$, then one can show (e.g.
  using the Riemann-Lebesgue lemma) that
  $\widehat{f} \in L^\infty$. So we can conclude that
  $\widehat{f}$ has pointwise meaning, i.e.
  $\widehat{f}(\xi)$ makes sense pointwise. In
  particular, we can make sense of $\widehat{f}(\xi)$
  on the hypersurface $S$.

  On the other hand, if $f \in L^2$, then by Plancherel's theorem,
  $\widehat{f} \in L^2$ as well. But
  an $L^2$ function has no pointwise interpretation,
  i.e. we can modify it on a set of measure zero without
  changing the function.
  In particular, it is meaningless to restrict
  the function $\widehat{f}$ to $S$, since $S$ is a set
  of measure zero in $\R^n$.

  In general, what about $f \in L^p$ for
  $1 < p < 2$? This is the topic of the
  \emph{restriction theorems} in harmonic analysis.
  It turns out that the choice of which $p$ work
  depends on the ``curvature'' of $S$.
\end{remark}

\begin{theorem}[Stein-Tomas restriction theorem]
  Let $n = d + 1$ and $S \subseteq \R^n$ be a
  hypersurface with non-vanishing Gaussian curvature.
  Let $\sigma_S$ be the corresponding surface measure,
  and let $\phi$ be a compactly supported  on $S$.
  Then we have
  \[
    \|(\phi \sigma_S)^\vee\|_{L^r(\R^n)} \le
    C \|\phi\|_{L^2(\sigma_S)},
  \]
  where $r = (2n + 2) / (n - 1)$.
\end{theorem}

\begin{remark}
  Now recall the explicit formula for a solution $\psi$
  to the Schr\"odinger equation:
  \[
    \psi(t, s) = \int e^{i(x \cdot \xi - t |\xi|^2 / 2)} \widehat{\psi}_0(\xi)\, d\xi.
  \]
  Also define the hypersurface
  $S = \{(\xi, \tau) : \tau = -|\xi|^2 / 2, \xi \in \R^d\}$.\footnote{This particular hypersurface is called the \emph{characteristic surface} of the Schr\"odinger equation.}
  Then
  \[
    \psi(t, x) = (\phi \sigma_S)^\vee (t, x), \quad
    \phi(\xi, \tau) = \widehat{\psi}_0(\xi), \quad
    \sigma_S(d \xi, d \tau) = (2 \pi)^d d\xi.
  \]
  Indeed, we can see that
  \[
    \psi(t, x) = (2\pi)^{-d} \int e^{i(x \cdot \xi + t \tau)} \phi(\xi, \tau)\, \sigma_S(d \xi, d \tau)
    = \int e^{i(x \cdot \xi - t |\xi|^2 / 2)} \widehat{\psi}_0(\xi)\, d\xi.
  \]
  Then, the Stein-Tomas restriction theorem tells us
  that
  $\|(\phi \sigma_S)^\vee\|_{L^r(\R^n)} \le C \|\phi\|_{L^2(\sigma_S)}$,
  which implies
  \[
    \|\psi\|_{L^{2 + 4 / d}_{t, x}(\R \times \R^d)}
    \le C \|\widehat{\psi}_0\|_{L^2_\xi}
    \le C \|\psi_0\|_{L^2_x},
  \]
  where the last inequality follows by
  Plancherel's theorem. The estimate
  above technically only holds for those $\psi$
  where $\widehat{\psi}_0$ is compactly supported,
  but one can extend this by a density argument to
  $\psi_0 \in L^2_x$.

  See Chapter 11 of Muscalu-Schlag
  (Volume I) for more details.
\end{remark}
