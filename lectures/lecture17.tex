\chapter{Mar.~10 --- Compact Operators}

\section{Conclusion on Dispersive Decay}
\begin{remark}
  Recall that we have previously shown if
  $V$ is sufficiently nice, then
  \[
    \|e^{iHt} P_{\mathrm{c}} f\|_{L^\infty}
    \lesssim t^{-1 / 2} \|f\|_{L^1}.
  \]
  In particular, if we consider the following equation:
  \[
    \begin{cases}
      i \partial_t \psi - \partial_x^2 \psi + V \psi = F, \\
      \psi(0) = f,
    \end{cases}
  \]
  then we have the Strichartz estimate (which
  follows by a $TT^*$ argument)
  \[
    \|P_{\mathrm{c}} \psi\|_{L^6_{t, x}} \lesssim \|f\|_{L^2_x} + \|F\|_{L^{6 / 5}_{t, x}}.
  \]
\end{remark}

\section{Compact Operators}

\begin{definition}
  A linear operator $A : \HH \to \HH$ is \emph{compact}
  if it satisfies one of the following (equivalent)
  conditions:
  \begin{enumerate}
    \item $A$ takes bounded sets to relatively
      compact sets;
    \item $A$ takes bounded sets to sets that
      have $\epsilon$-nets for all $\epsilon > 0$ (see below);
    \item if $\{u_n\}$ is a bounded sequence in
      $\HH$, then $\{A u_n\}$ has a strongly
      convergent subsequence;
    \item $A$ takes weakly convergent to strongly
      convergent sequences.
  \end{enumerate}
\end{definition}

\begin{definition}
  Let $K$ be a metric space. We say that $E \subseteq K$
  is an $\epsilon$-net of $K$ if for every $x \in K$,
  we can find $y \in E$ such that $d(x, y) < \epsilon$.
\end{definition}

\begin{theorem}[Spectral theorem for compact self-adjoint operators]
  Let $A : \HH \to \HH$ be compact and self-adjoint.
  Then we can find an orthonormal basis
  $\{\phi_n\}_{n = 1}^\infty$ of $\HH$ such that
  \[
    A \phi_n = \lambda_n \phi_n,
    \quad \lambda_n \in \R,
  \]
  and the $\lambda_n$ can only accumulate at zero.
\end{theorem}

\begin{definition}
  The \emph{essential spectrum} of $A$ is the spectrum
  of $A$ minus the isolated eigenvalues of finite
  multiplicity.
\end{definition}

\begin{theorem}[Weyl's criterion]
  Let $A, B : \HH \to \HH$ be self-adjoint. Assume
  \[
    (A + M)^{-1} - (B + M)^{-1} = K
  \]
  for some compact operator $K$ and
  constant $M$. Then the essential spectrum of
  $A$ and $B$ are the same.
\end{theorem}

\begin{proof}
  We only sketch a proof here. We first claim that if
  $\lambda$ is in the essential spectrum of $A$, then
  for every $\epsilon > 0$, there exists
  an infinite dimensional subspace $L$ of $\HH$
  such that
  \[
    \|(A + M)^{-1} f - (\lambda + M)^{-1} f\|_{\HH}
    \le \epsilon \|f\|_{\HH}
  \]
  for all $f \in L$.
  This is by spectral theory, as we can write
  \[
    (A + M)^{-1} f - (\lambda + M)^{-1} f
    = \int \left(\frac{1}{t + M} - \frac{1}{\lambda + M}\right) E(dt) f.
  \]
  Then since $\lambda$ is fixed, for any $\epsilon > 0$
  we can find $\delta > 0$ (by continuity) such that
  \[
    \int_{\lambda - \delta}^{\lambda + \delta}
    \left|\frac{1}{t + M} - \frac{1}{\lambda + M}\right|^2 \langle E(d\lambda) f, f \rangle \le \epsilon^2.
  \]
  We can take $L = \Ran E([\lambda - \delta, \lambda + \delta])$.
  By spectral theory again, for any $f \in L$, we have
  \[
    \|(A + M)^{-1} f - (\lambda + M)^{-1} f\|_{\HH}^2
    = \int_{\lambda - \delta}^{\lambda + \delta}
    \left|\frac{1}{t + M} - \frac{1}{\lambda + M}\right|^2 \langle E(d\lambda) f, f \rangle
    \le \epsilon^2 \|f\|_{\HH}^2,
  \]
  since $E(d\lambda) f = f$ as
  $f \in L$. So it only remains to check that $L$ is
  infinite dimensional. To see this, note that if
  $L$ was finite dimensional, then
  $\langle E([\lambda - \delta, \lambda + \delta]) f, f \rangle$
  is a purely atomic measure, a contradiction with
  $\lambda$ being in the essential spectrum of $A$.\footnote{This claim says that the essential spectrum is related to the infinite dimensional subspaces of $\HH$.}

  Now the converse of the claim is also true (exercise).
  We will use this to show that $\lambda$ is
  essential spectrum of $B$. By the converse of
  the claim, it suffices to find $\widetilde{L}$
  such that $\dim \widetilde{L} = \infty$ and
  for every $g \in \widetilde{L}$,
  \[
    \|(B + M)^{-1} g - (\lambda + M)^{-1} g\|_{\HH}
    \le 2 \epsilon \|g\|_{\HH}.
  \]
  We now claim that there $\widetilde{L}$ with
  $\dim \widetilde{L} = \infty$ and $\widetilde{L} \subseteq L$ such that
  \[
    \|PK^2 P g\|_{\HH} \le 2 \epsilon \|g\|_{\HH}
  \]
  for all $g \in \widetilde{L}$, where
  $P = E([\lambda - \delta, \lambda + \delta])$.
  To see this, note that $T = PK^2 P$ is a self-adjoint
  compact operator, so by the spectral theorem
  we can find $\{\eta_n\}_{n = 1}^\infty$ and
  $\{\phi_n\}_{n = 1}^\infty$ with
  $T\phi_n = \eta_n \phi_n$ and $\eta_n \to 0$ (Since
  $L$ is an infinite dimensional subspace of $\HH$,
  we can find a basis $\{b_m\}_{m = 1}^\infty$ for $L$.
  Then $b_n \to 0$ weakly. A compact operator takes
  weakly convergent sequences to strongly convergent
  sequences, so $T b_n \to 0$ strongly. In particular,
  it is impossible that $\{\eta_n\}_{n = 1}^\infty$ is
  a finite subset of $\R$. So it has an
  accumulation point by the Bolzano-Weierstrass
  theorem, which must be $0$.) Take
  \[
    \widetilde{L} = \mathrm{span}\{\phi_n : |\eta_n| \le \epsilon^2\},
  \]
  which is infinite dimensional since
  $\eta_n \to 0$. Then for any $g \in \widetilde{L}$,
  \[
    \|Kg\|^2_{\HH} = \langle PK^2 P g, g \rangle
    \le \epsilon^2 \|g\|^2_{\HH}.
  \]
  We can write
  \begin{align*}
    (B + M)^{-1} g - (\lambda + M)^{-1} g
    &= (B + M)^{-1} g - (A + M)^{-1} g
    + (A + M)^{-1} g - (\lambda + M)^{-1} g \\
    &= K g + [(A + M)^{-1} g - (\lambda + M)^{-1} g],
  \end{align*}
  and thus we have
  \[
    \|(B + M)^{-1} g - (\lambda + M)^{-1} g\|_{\HH}
    \le \|Kg\|_{\HH} + \|(A + M)^{-1} g - (\lambda + M)^{-1} g\|_{\HH}
    \le 2 \epsilon \|g\|_{\HH}.
  \]
  This implies that $\lambda$ lies in the essential
  spectrum of $B$. We conclude the theorem by symmetry.
\end{proof}

\begin{example}
  Let $A = -\Delta$ (with essential spectrum
  $[0, \infty)$)
  and $B = -\Delta + V$, where $V$ decays to
  $0$ as $|x| \to \infty$. Then for $M > 0$,
  we can see that (using
  the resolvent identities)
  \begin{align*}
    (A + M)^{-1} - (B + M)^{-1}
    &= (-\Delta + M)^{-1} - (-\Delta + V + M)^{-1} \\
    &= \sum_{n = 1}^\infty (-\Delta + M)^{-1}
    (-V(-\Delta + M)^{-1})^n.
  \end{align*}
  Notice that $\|(-\Delta + M)^{-1}\|_{L^2 \to L^2} \le 1 / M$,
  since
  $((-\Delta + M)^{-1})^\wedge = 1 / (\xi^2 + M) \le 1 / M$
  for $M > 0$. Now
  \[
    \|V(-\Delta + M)^{-1}\|_{L^2 \to L^2} \le \frac{\|V\|_{L^\infty}}{M},
  \]
  so if $M > \|V\|_{L^\infty}$, then
  we have a convergent series. So in order to show
  that $(A + M)^{-1} - (B + M)^{-1}$ is compact, it
  suffices to show $K = V(-\Delta + M)^{-1}$ is compact
  (since the sum and composition of compact operators
  is again compact). Note that a loss of compactness
  can come from either the physical or the frequency
  side. But we have
  \[
    ((-\Delta + M)^{-1})^\wedge = \frac{1}{\xi^2 + M},
  \]
  which localizes a function on the frequency side,
  and the decay of $V$ localizes a function on the
  physical side. This gives intuition for why
  $K$ should be compact.

  A more rigorous argument is the following: Let
  $B \subseteq L^2(\R^d)$ be the unit ball, and set
  \[
    W = \{w = V(-\Delta + M)^{-1} v : \|v\|_{L^2} \le 1\}.
  \]
  We would like to show that $W$ is relatively compact,
  which we will do by appealing to the Arzela-Ascoli
  theorem. For this, we must check boundedness and
  equicontinuity. Define
  \[
    W_\epsilon = \{w_\epsilon = \chi_\epsilon * w : w \in W\},
  \]
  where $\chi_\epsilon$ is the standard mollifier.
  For boundedness, we have
  \[
    |w_\epsilon(x)| = |\chi_\epsilon * w|
    \lesssim \epsilon^{-n / 2} \|w\|_{L^2}
    \lesssim \epsilon^{-n / 2}
  \]
  and $\|w_\epsilon(x) - w(x)\|_{L^2} \lesssim \epsilon$,
  which proves boundedness. For
  equicontinuity, we note that
  \[
    |w_\epsilon(x) - w_\epsilon(y)|
    \le \left|\int (\chi_\epsilon(x - z) - \chi_\epsilon(y - z)) w(z)\, dz \right|
    \lesssim \epsilon^{-n / 2} \epsilon^{-1} |x - y|,
  \]
  which proves equicontinuity. Thus we can apply
  the Arzela-Ascoli theorem to get that $W_\epsilon$ is
  relatively compact. Since this holds for all
  $\epsilon > 0$, we can also conclude that
  $W$ is relatively compact.

  Finally, by Weyl's criterion, we conclude that
  the essential spectrum of $B$ is also $[0, \infty)$.
\end{example}
