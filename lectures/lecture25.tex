\chapter{Apr.~16 --- Nonlinear Equations}

\section{Strichartz Estimates, Revisited}
\begin{definition}
  A pair $(p, q)$ is called \emph{(Strichartz) admissible}
  if
  \[\frac{2}{p} + \frac{d}{q} = \frac{d}{2}\]
  and $2 \le p \le \infty$ with
  $(p, q) \ne (2, \infty)$.
\end{definition}

\begin{theorem}[General Strichartz]\label{thm:strichartz-full}
  Suppose $i \partial_t \psi - \Delta \psi = h$
  in $\R \times \R^d$ and $\psi(0) = f$. Then
  \[
    \|\psi\|_{L^p_t L^q_x}
    \lesssim \|f\|_{L^2} + \|h\|_{L^{p'}_t L^{q'}_x}
  \]
  for $(p, q)$ Strichartz admissible and
  $1 / p + 1 / p' = 1 / q + 1 / q' = 1$.
\end{theorem}

\begin{remark}
  Recall that the $(p, q)$ Strichartz admissible
  restriction comes from scaling. If we define
  \[
    \psi_\lambda(t, x) = \psi(\lambda^2 t, \lambda x),
  \]
  then $i \partial_t \psi_\lambda - \Delta \psi_\lambda = h(\lambda^2 t, \lambda x) / \lambda^2$
  and $\psi_\lambda(0) = f(\lambda x)$. The
  bound on $\psi_\lambda$ forces
  $2 / p + d / q = d / 2$.
\end{remark}

\begin{remark}
  If $p < \infty$, then $q > 2$. So in the
  time average sense, solutions
  gain integrability. This is related to the
  smoothing phenomenon of the Schr\"odinger
  equation.
\end{remark}

\begin{proof}[Proof of Theorem \ref{thm:strichartz-full}]
  We use a $TT^*$ argument. Let
  $Tf = e^{-i\Delta t} f$, then $T^*$ acts
  on $g(t, x)$ by
  \[
    T^* g
    = \int_{-\infty}^\infty e^{is\Delta} g(s)\, ds.
  \]
  The goal is to show the following inequality:
  \[
    \|Tf\|_{L^p_t L^q_x}
    \lesssim \|f\|_{L^2}
    \iff \|T^* g\|_{L^2} \lesssim \|g\|_{L^{p'}_t L^{q'}_x}
    \iff \|T T^* g\|_{L^p_t L^q_x}
    \lesssim \|g\|_{L^{p'}_t L^{q'}_x}.
  \]
  Recall that $\|Tf\|_{L^q_x} \lesssim t^{-d(1 / q' - 1 / q) / 2} \|f\|_{L^{q'}_x}$
  for $1 \le q' \le 2$. We have
  \begin{align*}
    \|TT^* g\|_{L^q_x}
    =
    \left\|
    \int_{-\infty}^\infty e^{i(t - s)\Delta} g(s)\, ds
    \right\|_{L^q_x}
    &\le \int_{-\infty}^\infty
    \|e^{i(t - s)\Delta} g(s)\|_{L^q_x}\, ds \\
    &\le \int_{-\infty}^\infty
    |t - s|^{-d (1 / q' - 1 / q) / 2}
    \|g(s)\|_{L^{q'}_x}\, ds.
  \end{align*}
  Fractional integration then implies that
  for $1 + 1 / p = 1 / p' + d(1 / q' - 1 / q) / 2$ (provided that we have
  $0 < d (1 / q' - 1 / q) / 2 < 1$, which happens
  if and only if $p > 2$),
  \[
    \|TT^* g\|_{L^p_t L^q_x}
    \lesssim \left\|\int_{-\infty}^\infty |t - s|^{-d(1 / q' - 1 / q) / 2}
    \|g\|_{L^{q'}_x}\, ds\right\|_{L^p_t}
    \lesssim \|g\|_{L^{p'}_t L^{q'}_x}.
  \]
  This proves that $\|Tf\|_{L^p_t L^q_x} = \|e^{i\Delta t} f\|_{L^p_t L^q_x} \lesssim \|f\|_{L^2}$.
  To move to the inhomogeneous case, we can
  simply use Duhamel's formula. If
  $\psi$ satisfies the equation
  \[
    \psi(t) = \int_0^t e^{i(t - s)\Delta} h(s)\, ds
    = \int_{-\infty}^\infty \chi_{\{0 < s < t\}}
    e^{i(t - s)\Delta} h(s)\, ds,
  \]
  then we can run the same argument as above
  to get
  \[
    \|\psi\|_{L^q_x}
    \lesssim \int_{-\infty}^\infty \chi_{\{0 < s < t\}} \|h(s)\|_{L^{q}_x}\, ds
    \lesssim \int_{-\infty}^\infty \|e^{i(t - s)\Delta} h(s)\|_{L^q_x}\, ds.
  \]
  This is now the same expression as before,
  so the same argument as before applies, and
  we get
  \[
    \|\psi\|_{L^p_t L^q_x} \lesssim \|h\|_{L^{p'}_t L^{q'}_x}.
  \]
  This proves the inhomogeneous case.
\end{proof}

\begin{remark}
  In the above estimates, we require
  $p > 2$. For $d = 2$ and the pair
  $(p, q) = (2, \infty)$,
  the Strichartz estimate fails in general.
  To get $p = 2$ for $d \ge 3$, one needs to
  use a more complex argument (Keel-Tao), and
  these are called \emph{endpoint} Strichartz
  estimates.
\end{remark}

\section{Existence and Uniqueness for Nonlinear Equations}

\begin{remark}
  Now we want to consider the equation
  \[
    \begin{cases}
      i \partial_t \psi + \Delta \psi = \mu |\psi|^{4 / d} \psi, & (t, x) \in \R \times \R^d, \\
      \psi(0) = \psi_0.
    \end{cases}
    \tag{$*$}
  \]
  The reason for
  $|\psi|^{4 / d} \psi$ is that it is
  \emph{$L^2$-critical} (the $L^2$ norm
  is important for Schr\"odinger equations).
  If $\psi$ is a solution, then
  $\psi_{\lambda} = \lambda^{d / 2} \psi(\lambda^2 t, \lambda x)$
  satisfies $\|\psi_{\lambda}\|_{L^2} = \|\psi\|_{L^2}$.
  Moreover, for the nonlinearity $|\psi|^{4 / d} \psi$, the scaled solution
  $\psi_\lambda$ solves
  \[
    i \partial_t \psi_\lambda + \Delta \psi_\lambda
    = \mu |\psi_\lambda|^{4 / d} \psi_\lambda,
  \]
  which is the same equation that $\psi$ solves.
  So the equation with $|\psi|^{4 / d} \psi$ is
  \emph{scaling invariant}.
\end{remark}

\begin{definition}\label{def:solution}
  Given $\psi_0 \in L^2(\R^d)$,
  $\psi(t)$ is a \emph{solution} if
  \[
    \psi(t) = e^{it\Delta} \psi_0
    + i \mu \int_0^\infty e^{i(t - s)\Delta}
    |\psi(s)|^{4 / d} \psi(s)\, ds
  \]
  and $\psi(t)$ lies in
  $X = (C_t \cap L^\infty_t)([0, \infty); L^2_x(\R^d)) \cap L^{p_0}_t([0, \infty); L^{2p_0}_x(\R^d))$
  with $p_0 = 1 + 4 / d = (d + 4) / d$.
\end{definition}

\begin{remark}
  Note that $(p_0, 2p_0)$ is Strichartz
  admissible since
  \[
    \frac{2}{p_0} + \frac{d}{2p_0}
    = \frac{4 + d}{2p_0}
    = \left(\frac{4 + d}{2}\right) \frac{1}{p_0}
    = \frac{d}{2}.
  \]
\end{remark}

\begin{remark}
  The motivation for choosing
  $p_0 = 1 + 4 / d$ is
  the following: We have
  \[
    \|\psi\|_{L^2}
    \le \|\psi_0\|_{L^2}
    + \left\|\int_0^\infty e^{i(t - s)\Delta} |\psi(s)|^{4 / d} \psi(s)\, ds\right\|_{L^2}
    \le \|\psi_0\|_{L^2}
    + \int_0^\infty \||\psi(s)|^{4 / d} \psi(s)\|_{L^2}\, ds.
  \]
  One can check that
  $\||\psi|^{4 / d} \psi\|_{L^2_x} \le \| |\psi|^{4 / d + 1} \|_{L^2_x} = \|\psi\|_{L^{2p_0}_x}^{p_0}$, so
  \[
    \int_0^\infty \||\psi(s)|^{4 / d} \psi(s)\|_{L^2}\, ds
    \lesssim
    \int_0^\infty \|\psi\|_{L^{2p_0}_x}^{p_0}\, ds
    \lesssim \|\psi\|_{L^{p_0}_t L^{2p_0}_x}^{p_0}.
  \]
\end{remark}

\begin{corollary}
  The equation $(*)$ has a unique
  solution in the sense of Definition \ref{def:solution},
  provided that $\|\psi_0\|_{L^2} \ll 1$
  is small enough depending on $\mu$ and $d$.
\end{corollary}

\begin{proof}
  For simplicity, take $d = 2$, so
  $\mu |\psi|^{4 / d} \psi = \mu |\psi|^2 \psi$
  and $p_0 = 4 / d + 1 = 3$. Define
  \[
    A \psi
    = e^{it\Delta} \psi_0 + i \mu \int_0^\infty
    e^{i(t - s)\Delta} |\psi|^2 \psi\, ds,
  \]
  and we will show that $A$ is a contraction
  in $X$ if
  $\|\psi_0\|_{L^2}$ is small enough.
  By the Strichartz estimate,
  \[
    \|A\psi\|_{X}
    \le C(\|\psi_0\|_{L^2} + |\mu| \|\psi\|_{X}^3).
  \]
  To show that $A$ is a contraction in a
  ball in $X$ with radius $R$, take
  $R = C_1 \epsilon$. We want
  \[
    C(\|\psi_0\|_{L^2} + |\mu| \|\psi\|_{X}^3)
    \le C(\epsilon + |\mu| (C_1 \epsilon)^3)
    < C_1 \epsilon,
  \]
  so we can take $C_1 = 2C$ and $\epsilon$
  small enough such that
  $C(1 + |\mu| C_1^3 \epsilon^2) < C_1$,
  which will make $A$ map $\psi$ back into
  the ball of radius $R$ in $X$. For the
  difference estimate, we have
  \[
    A(\psi) - A(\varphi)
    = i \mu \int_0^\infty e^{i(t - s)\Delta}
    (|\psi|^2 \psi - |\varphi|^2 \varphi)\, ds,
  \]
  so we can estimate (note that
  $|\psi|^2 \psi - |\varphi|^2 \varphi = |\psi|^2 \psi - |\psi|^2 \varphi + |\psi|^2 \varphi - |\varphi|^2 \varphi$)
  \[
    \|A(\psi) - A(\varphi)\|_{X}
    \le |\mu| \| |\psi|^2 \psi - |\varphi|^2 \varphi\|_{L^1_t L^2_x}
    \le C |\mu| (\|\psi\|_X + \|\varphi\|_X)^2 \|\psi - \varphi\|_X,
  \]
  so if we take $\epsilon$ small enough so that
  $C |\mu|(2R)^2 = C|\mu| (C_1 \epsilon)^2 < 1$, then we
  get a contraction. Thus $A$ has a unique
  fixed point in the ball of radius $R$ in
  $X$, which is the desired solution.
\end{proof}

\begin{remark}
  Suppose $H = -\Delta + V$ has no eigenvalues
  and $H$ is ``nice'' at zero energy (e.g.
  if $V$ is very small), then recall that
  we have
  \[
    \|e^{iHt} f\|_{L^\infty}
    \lesssim t^{-d / 2} \|f\|_{L^1}.
  \]
  In this setting, we can show
  the general Strichartz estimates hold, and
  one can show global existence for
  \[
    i \partial_t \psi - \Delta \psi + V \psi
    = |\psi|^{4 / d} \psi
  \]
  with small initial data.
\end{remark}
