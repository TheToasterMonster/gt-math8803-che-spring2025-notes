\chapter{Feb.~3 --- Kato Smoothing, Part 3}

\section{Sharpness of Kato Smoothing}

\begin{lemma}[Kato smoothing is sharp]
  We have
  \[
    \|\chi(x) (-\Delta)^{1 / 4} e^{i t \Delta / 2} (e^{-|x|^2 / 2} e^{ixv})\|_{L^2_{x, t}} \sim 1
    \quad \text{and} \quad
    \sup_{v} \| \chi (-\Delta)^{1 / 4 + \delta} e^{it\Delta / 2} (e^{-|x|^2 / 2} e^{ixv}) \|_{L^2_{x, t}} = \infty.
  \]
  for any $\delta > 0$.
\end{lemma}

\begin{proof}
  Let $\supp \widehat{\chi} \subseteq B(0, 1)$ and
  $f_0 = e^{-|x|^2 / 2} e^{ixv}$. Choose
  $\eta$ smooth with $\widehat{\eta}$ compactly supported.
  Then
  \begin{align*}
    \int_{\R} \int_{\R^d}
    |\chi(x) \eta(\epsilon t) (-\Delta)^{1 / 4} e^{it \Delta / 2} f_0|^2\, dxdt
    &= \iint\left|\int \widehat{\chi}(\xi - \xi') \eta(\epsilon t) |\epsilon|^{1 / 2} e^{it|\xi|^2 / 2} \widehat{f}_0(\xi)\, d\xi\right|^2\, d\xi dt \\
    &= \iint \left| \int \widehat{\chi}(\xi - \xi') \frac{1}{\epsilon} \widehat{\eta}\left(\frac{\tau - |\xi|^2 / 2}{\epsilon}\right)|\xi|^{1 / 2} e^{-|\xi - v|^2 / 2}\, d\xi\right|^2\, d\xi' d\tau
  \end{align*}
  by Plancherel's theorem first in $x \mapsto \xi$
  and then in $t \mapsto \tau$, since
  $\widehat{\eta}$ and $\widehat{\chi}$ are compactly
  supported. Now, $-\epsilon < \tau - |\xi|^2 / 2 < \epsilon$
  implies
  \[
    ||\xi| - \sqrt{2\tau}| < \frac{2\epsilon}{|\xi| + \sqrt{2\tau}}.
  \]
  So the region of integration  for $|\xi|$ is an
  annulus-shaped area between $\sqrt{2\tau}$ and
  $2\epsilon / (|\xi| + \sqrt{2\tau})$ (plot the
  above inequality to see the region).
  Thus for fixed $\tau$, as $\epsilon \to 0$,
  we have
  \begin{align*}
    &\iint \left| \int \widehat{\chi}(\xi - \xi') \frac{1}{\epsilon} \widehat{\eta}\left(\frac{\tau - |\xi|^2 / 2}{\epsilon}\right)|\xi|^{1 / 2} e^{-|\xi - v|^2 / 2}\, d\xi\right|^2\, d\xi' d\tau \\
    &\quad \quad \sim \int_{\R} \int_{\R^d}
    \left| \int_{\sqrt{2\tau} \mathbb{S}^{d - 1}} \widehat{\chi}(\xi - \xi') e^{-|\xi - v|^2 / 2} |\xi|^{1 / 2} \frac{2\epsilon}{|\xi| + \sqrt{2\tau}} \frac{1}{\epsilon}\, d\sigma(\xi)\right|^2\, d\xi' d\tau \\
    & \quad \quad = \int_{\R} \int_{\R^d} \left| \int_{\sqrt{2\tau} \mathbb{S}^{d - 1}} \widehat{\chi}(\xi - \xi') e^{-|\xi - v|^2 / 2} |\xi|^{1 / 2} \frac{2}{|\xi| + \sqrt{2\tau}} \, d\sigma(\xi)\right|^2\, d\xi' d\tau.
  \end{align*}
  Since $\widehat{\chi}$ is supported in $B(0, 1)$, for
  fixed $\xi$ the range of values for $\xi'$ is $O(1)$
  (as $|\xi - \xi'| \le 1$). Since
  \[
    -\epsilon < \tau - \frac{1}{2} |\xi|^2 < \epsilon
    \implies -\epsilon + \frac{1}{2} |\xi|^2 < \tau < \epsilon + \frac{1}{2} |\xi|^2
  \]
  and $|\xi - v| \le O(1)$, we get
  \[
    -\epsilon + \frac{1}{2} ||v| - 1|^2
    < \tau < \epsilon + \frac{1}{2} ||v| + 1|^2.
  \]
  Expanding $||v| - 1|^2$ and $||v| + 1|^2$, we see
  that as $\epsilon \to 0$,
  we have $|2\tau - |v|^2| \lesssim |v| + 1$. Then
  \begin{align*}
    &\int_{\R} \int_{\R^d} \left| \int_{\sqrt{2\tau} \mathbb{S}^{d - 1}} \widehat{\chi}(\xi - \xi') e^{-|\xi - v|^2 / 2} |\xi|^{1 / 2} \frac{1}{|\xi| + \sqrt{2\tau}} \, d\sigma(\xi)\right|^2\, d\xi' d\tau \\
    & \quad \quad \lesssim \int_{\R} \left|\int_{\sqrt{2\tau} \mathbb{S}^{d - 1}} e^{-|\xi - v|^2 / 2} |\xi|^{1 / 2} \frac{1}{|\xi| + \sqrt{2\tau}}\, d\sigma(\xi)\right|^2\, d\tau \\
    & \quad \quad \lesssim \int_{\R} \left|\frac{|\sqrt{\tau}|^{1 / 2}}{\sqrt{2\tau}}\right|^2\, d\tau
    \approx \int_{|2\tau - |v|^2| \lesssim |v| + 1}
    \left|\frac{|\sqrt{\tau}|^{1 / 2}}{\sqrt{2\tau}}\right|^2\, d\tau
    \approx \frac{1}{v} \int_{|2\tau - |v|^2| \lesssim |v| + 1}\, d\tau
    \approx \frac{1}{v} \cdot v \approx O(1)
  \end{align*}
  uniformly in $v$. We can also see that if
  $|\xi|^{1 / 2}$ is replaced by $|\xi|^{1 / 2 + \delta}$,
  then $1 / |v|$ will become $1 / |v|^{1 - 4 \delta}$,
  so the above becomes $O(|v|^{4\delta})$, which
  goes to $\infty$ as $v \to \infty$.
\end{proof}

\begin{corollary}
  When $d \ge 2$, for $\epsilon > 0$, we have
  \[
    \|(1 + |x|)^{-1 / 2 -\epsilon} (-\Delta)^{1 / 4} e^{it\Delta / 2} f\|_{L^2_{x, t}} \le C_\epsilon \|f\|_{L^2_x}.
  \]
  This is the sharp Kato smoothing estimate for
  $d \ge 2$.
\end{corollary}

\section{Schr\"odinger Equation with Potential}
\begin{remark}
  Now we will consider the Schr\"odinger equation with
  a potential:
  \[
  \begin{cases}
    i\psi_t + \frac{1}{2} \Delta \psi + V(t, x) \psi = 0, \\
    \psi|_{t = 0} = \psi_0 \in H^\gamma(\R^d).
  \end{cases}
  \]
\end{remark}

\begin{lemma}
  If $\psi_0 \in H^\gamma(\R^d)$, then
  $e^{it\Delta / 2} \psi_0 \in C^0(\R, H^\gamma(\R^d)) \cap C^1(\R, H^{\gamma - 2}(\R^d))$.
\end{lemma}

\begin{proof}
  For each $t, s \in \R$, we can write (since
  $e^{is\Delta / 2}$ preserves each $H^\gamma$ norm)
  \[
    \|e^{it\Delta / 2} \psi_0 - e^{is\Delta / 2} \psi_0\|_{H^\gamma}^2
    = \|e^{i(t - s)\Delta} \psi_0 - \psi_0\|_{H^\gamma}^2
    = \int (1 + |\xi|^2)^{\gamma} |e^{i(t - s)|\xi|^2 / 2} - 1| |\widehat{\psi}_0(\xi)|^2\, d\xi.
  \]
  Since $\psi_0 \in H^\gamma(\R^d)$,
  as $t \to s$, the above quantity goes to $0$
  by the dominated convergence theorem. This shows that
  $e^{it\Delta / 2 \psi_0} \in C^0(\R; H^\gamma(\R^d)$.
  To check that $e^{it\Delta / 2} \psi_0 \in C^1(\R; H^{\gamma - 2}(\R^d))$, note
  \[
    \frac{d}{dt} e^{it\Delta / 2} \psi_0
    = \frac{i}{2} \Delta e^{it\Delta / 2} \psi_0.
  \]
  Thus it suffices to see that $\Delta e^{it\Delta / 2} \psi_0$ is continuous, i.e.
  \[
    \|\Delta e^{it / 2} \psi_0 - \Delta e^{is / 2} \psi_0\|_{H^{\gamma - 2}}^2 \to 0
  \]
  as $t \to s$, which follows by the dominated
  convergence theorem.
\end{proof}

\begin{theorem}
  Assume that $V$ is real with
  \[
    \sup_{t} \|\nabla_x^\alpha V(t, x) \|_{L^2_x} \le C_\alpha
    \quad \text{and} \quad
    \|\nabla_x^\alpha V(t, x) - \nabla_x^\alpha V(s, x)\|_{L^\infty_x} \to 0
  \]
  as $t \to s$ for every multi-index $\alpha$.
  Then for all $k \ge 0$, if $\psi_0 \in H^k(\R^d)$,
  there exists a unique solution
  $\psi \in C^0(\R; H^k(\R^d)) \cap C^1(\R, H^{k - 2}(\R^d))$
  to the Schr\"odinger equation with potential $V$,
  given by
  \[
    \psi(t) = e^{it\Delta / 2} \psi_0 + i \int_0^t e^{i(t - s)\Delta / 2} V(s) \psi(s)\, ds.
  \]
  Moreover, $\|\psi\|_{H^k(\R^d)} \le C_{k, V} (1 + |t|)^k$.
\end{theorem}

\begin{proof}
  Define the operator $A$ by the formula
  \[
    A(\phi) = e^{it \Delta / 2} \psi_0 - i\int_0^t e^{i(t - s)\Delta / 2} V(s) \phi(s)\, ds.
  \]
  Let $X = (C^0[0, T], H^k)$, and we want to find
  $T$ such that $A$ is a contraction in $X$ (do this
  as an exercise). Repeating this
  on $[T, 2T]$, $[2T, 3T]$, \dots, gives
  existence. Uniqueness follows by Gronwall's inequality.

  Now we prove the upper bound for the growth.
  Gronwall's inequality gives us
  \[
    \|\psi(t)\|_{H^k} \le c_1 e^{c_2|t|} \|\psi_0\|_{H^k}.
  \]
  Since $V$ is real (the argument before this step
  works even when $V$ is not real),
  \begin{align*}
    \frac{d}{dt} \|\psi(t)\|_{L^2}^2
    &= \frac{d}{dt} \langle \psi(t), \psi(t) \rangle
    = \frac{d}{dt}
    = \langle \frac{d}{dt} \psi, \psi \rangle
    + \langle \psi, \frac{d}{dt} \psi \rangle
    = 2 \re \langle \frac{d}{dt} \psi, \psi \rangle \\
    &= 2 \re \langle \frac{i}{2} \Delta \psi + iV(t, x) \psi, \psi \rangle
    = 2\re i \int \left(-\frac{1}{2} |\nabla \phi|^2 + V(t, x)|\psi|^2\right)\, dx = 0
  \end{align*}
  since the integral is real, so that $i$ times the
  integral is purely imaginary. So
  $\|\psi\|_{L^2}$ is conserved.
  Now let $\phi = \partial_{x_j} \psi(t, x)$, so that
  \[
    i\phi_t + \frac{1}{2} \Delta \phi + V \phi = -\partial_{x_j} V\psi. \tag{$*$}
  \]
  From the existence and uniqueness of the solution,
  we can use $V(t, 0) \psi_0$ to denote the solution
  to
  \[
    \begin{cases}
      i\partial_t + \frac{1}{2} \Delta \psi + V \psi = 0, \\
      \psi|_{t = 0} = \psi_0,
    \end{cases}
  \]
  and use $V(t, s) \psi_0$ denote the solution
  for initial condition $\psi(s) = \psi_0$. Then
  $(*)$ can be solved as
  \[
    \phi(t) = V(t, 0) \partial_{x_j} \psi_0
    + \int_0^t V(t, s) \partial_{x_j} V(s) \psi(s)\, ds.
  \]
  This gives
  \[
    \|\phi(t)\|_{L^2} \le
    \|V(t, 0) \partial_{x_j} \psi_0\|_{L^2}
    + \int_0^t \|V(t, s) \partial_{x_j} V(s) \psi(s)\|_{L^2}\, ds
  \]
  by Minkowski's inequality.  By the
  conservation of the $L^2$ norm,
  $\|V(t, 0) \partial_{x_j} \psi_0\|_{L^2} = \|\partial_{x_j} \psi_0 \|_{L^2}$, so
  \[
    \|V(t, s) \partial_{x_j} V(s) \psi(s)\|_{L^2}
    = \|\partial_{x_j} V(s) \psi(s)\|_{L^2}
    \lesssim \|\psi(s)\|_{L^2} = \|\psi(0)\|_{L^2}.
  \]
  Then we obtain
  \[
    \|\phi\|_{L^2} \lesssim
    \|\partial_{x_j} \psi_0\|_{L^2}
    + \int_0^t \|\psi(s)\|_{L^2}\, ds
    \le C_{1, V} (1 + |t|) \|\psi(0)\|_{H^1},
  \]
  and inducting on $k$ gives the result
  $\|\psi(t)\|_{H^k} \le C_{k, V} (1 + |t|)^k \|\psi(0)\|_{H^k}$.
\end{proof}

\begin{remark}
  Note that in the above theorem, we need a suitable
  integral (e.g. the Bochner integral) to be able to
  interpret the Duhamel formula in a Banach space.
\end{remark}
