\chapter{Apr.~7 --- Classical Scattering, Part 2}

\section{Classical Scattering, Continued}

\begin{proof}[Proof of Proposition \ref{prop:classical-scattering}, continued]
  We continue by solving the equation
  \[
    R(t) = -\int_t^\infty \int_s^\infty \nabla V(a + \tau p + R(\tau))\, d\tau ds.
  \]
  We do this by contraction mapping in the
  unit ball in $C([T, \infty); \R^d)$: Set
  \[
    A(z(t))
    = -\int_t^\infty \int_s^\infty \nabla V(a + \tau p + z(\tau))\, d\tau ds.
  \]
  Note that $|a + \tau p + z(\tau)| \ge |a + \tau p| - 1 \ge \tau |p| / 2$
  if $\tau$ is large enough. Then
  \[
    |A(z(t))|
    \le \int_t^\infty \int_s^\infty \left(\frac{1}{2} \tau p\right)^{-2 - \epsilon}\, d\tau ds
    \le |p|^{-2 - \epsilon} T^{-\epsilon} \le 1
  \]
  if we pick $T$ sufficiently large. We can
  also see that
  \begin{align*}
    |A(z(t)) - A(\widetilde{z}(t))|
    &\le \int_t^\infty \int_s^\infty
    |\nabla V(a + \tau p + z(\tau)) - \nabla V(a + \tau p + \widetilde{z}(\tau))|\, d\tau ds \\
    &\le \int_t^\infty \int_s^\infty
    \left(\frac{1}{2} \tau |p|\right)^{-2 - \epsilon}
    |z(\tau) - \widetilde{z}(\tau)|\, d\tau ds
    \le p^{-2 - \epsilon} T^{-\epsilon} \|z - \widetilde{z}\|_{C([T, \infty); \R^d)},
  \end{align*}
  which is a contraction if $T$ is large.
  So we can find a fixed point $R(t)$.
\end{proof}

\begin{remark}
  The map $\Omega_-(a, p) = (y, \eta)$
  can be seen as a classical analogue of the
  wave operator.
\end{remark}

\section{More on Scattering for the Schr\"odinger Equation}
\begin{lemma}\label{lem:local}
  Let $H = -\Delta + V$ and $f \in L^2_{\mathrm{ac}}$.
  Then we have
  \[
    \|\chi_{\{|x| \le R\}} e^{itH} f\|_{L^2} \xrightarrow[t \to \infty]{} 0
  \]
\end{lemma}

\begin{remark}
  We have $\|e^{itH} f\|_{L^2} = \|f\|_{L^2}$,
  but if we localize the solution, then
  \[
    \|\chi_{\{|x| \le R\}} e^{itH} f\|_{L^2} \xrightarrow[t \to \infty]{} 0
  \]
  if $f \in L^2_{\mathrm{ac}}$. But if
  $f$ is an eigenfunction of $H$, i.e.
  $f \in L^2$ with $e^{iH t} f = e^{i\lambda t} f$,
  this does not hold.
  In particular, if a quantum particle is
  not trapped by the potential $V$, then
  the probability of finding it in any
  compact region will go to $0$ as
  $t \to \infty$.
\end{remark}
