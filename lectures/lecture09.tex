\chapter{Feb.~10 --- Spectral Theory}

\section{Review of Functional Analysis}

\begin{remark}
  Our goal now is to study the decay of a solution $\psi$ to
  \[
    i\psi_t - \frac{1}{2} \Delta \psi + V(x) \psi = 0, \quad \text{$V$ real}.
  \]
  We will study $e^{iHt} \psi_0$, where
  $H = -\Delta / 2 + V$, in particular the
  spectral theory for $H$ in $L^2(\R^d)$.
\end{remark}

\begin{definition}
  Let $\mathcal{H}$ be a Hilbert space, with inner product
  $\langle \cdot, \cdot \rangle$. We will
  use $\mathcal{H} = L^2(\R^d)$ with
  \[
    \langle f, g \rangle
    = \int_{\R^d} f \overline{g}\, dx.
  \]
  Let $T$ be a linear operator which is densely defined
  on $\mathcal{H}$.
  Let $D(T)$ denote the domain of $T$. We say that
  $T$ is \emph{symmetric} if
  $\langle Tx, y \rangle = \langle x, Ty \rangle$.
  The \emph{adjoint} $T^*$ of $T$ is defined as
  \[
    D(T^*) =
    \{
      f \in \mathcal{H} : \text{there exists $h \in \mathcal{H}$ such that } \langle T \phi, f \rangle = \langle \phi, h \rangle \text{ for all $\phi \in D(T)$}
    \},
  \]
  and we set $T^* f = h$. We say
  $T$ is \emph{closed} provided that the graph
  of $T$ is a closed subset of $\mathcal{H} \times \mathcal{H}$,
  i.e. if $f_n \to f$ in $\mathcal{H}$ with $f_n \in D(T)$
  and $Tf_n \to g$ in $\mathcal{H}$, then
  $f \in D(T)$ and $Tf = g$.
\end{definition}

\begin{definition}
  We say that a linear operator $T$ is \emph{self-adjoint} if
  $(T, D(T)) = (T^*, D(T^*))$.
\end{definition}

\begin{example}
  Let $T = i \partial_x$. For $\mathcal{H} = L^2(\R)$ and
  $D(T) = C^1_c(\R)$, one can check that $T$ is self-adjoint.
  But if we take $\mathcal{H} = L^2([0, \infty))$ and
  $D(T) = C^1_c([0, \infty))$, then $T$ is not self-adjoint
  (but $T$ is symmetric).
\end{example}

\begin{example}
  Let $T = - \partial_x^2$
  on $L^2([0, 1])$. Then for
  \[
    D(T) = \{f \in H^2([0, 1]) : f(0) = f(1) = f'(0) = f'(1) = 0\},
  \]
  $T$ is symmetric but not self-adjoint. One can check that
  $D(T^*) = H^2([0, 1])$.
\end{example}

\begin{lemma}
  Let $T$ be a densely defined linear operator on
  $\mathcal{H}$. Then:
  \begin{enumerate}
    \item If $T$ is symmetric, then
      $D(T^*) \supseteq D(T)$ (so $T^*$ is 
      densely defined) and all eigenvalues of $T$ are
      real.
    \item For every $T$, its adjoint $T^*$ is closed.
    \item We have $(\ker T)^\perp = \overline{\Ran(T^*)}$
      and $(\ker T^*)^\perp = \overline{\Ran(T)}$.
    \item If $T$ is self-adjoint, i.e. $(T, D(T)) = (T^*, D(T^*))$, then \[\Spec T = \C \setminus \{z \in \C : \text{$(T - z)^{-1}$ exists as a bounded linear operator}\} \subseteq \R\]
      and $\|(T - z)^{-1}\|_{\mathcal{L}(\mathcal{H}, \mathcal{H})} \lesssim |{\im z}|^{-1}$.
  \end{enumerate}
\end{lemma}

\begin{proof}
  (1) One can check this by definition.

  (2) Take a sequence $f_n \in D(T^*)$ with
  $f_n \to f$, and suppose that $T^* f_n \to g$. By
  definition,
  \[
    \langle T \phi, f_n \rangle
    = \langle \phi, T^* f_n \rangle
    \quad \text{for every $\phi \in D(T)$}.
  \]
  Then $f_n \to f$ implies that
  $\langle T \phi, f_n \rangle \to \langle T \phi, f \rangle$,
  and $T^* f_n \to g$ implies that $\langle \phi, T^* f_n \rangle \to \langle \phi, g \rangle$.
  Then we have
  $\langle T \phi, f \rangle = \langle \phi, g \rangle$
  for every $\phi \in D(T)$, so
  $f \in D(T^*)$ and $T^* f = g$ by definition.

  (3) Note that in general, if $A \subseteq \mathcal{H}$, then
  $A^\perp$ is closed. To see this, let $f_n \in A^\perp$
  and $f_n \to f$ in $\mathcal{H}$. Then
  $\langle f_n, a \rangle = 0$ for every $a \in A$
  since $f_n \in A^\perp$. So letting
  $n \to \infty$, we have $\langle f, a \rangle = 0$,
  i.e. $f \in A^\perp$.

  We first show that $\overline{\Ran(T)} \subseteq (\ker T^*)^\perp$.
  For any $h \in \Ran(T)$, there exists $f$ such that
  $h = Tf$. Then for any $g \in \ker T^* \subseteq D(T^*)$,
  we have
  \[
    \langle Tf, g \rangle
    = \langle f, T^* g \rangle = 0,
  \]
  so $f \in (\ker T^*)^\perp$.
  Thus $\Ran(T) \subseteq (\ker T^*)^\perp$, and taking
  closures,
  $\overline{\Ran(T)} \subseteq \overline{(\ker T^*)^\perp} = (\ker T^*)^\perp$.

  Now we show that $(\ker T^*)^\perp \subseteq \overline{\Ran(T)}$.
  Let $f \in (\overline{\Ran(T)})^\perp$, so
  $f \in (\Ran(T))^\perp$. So
  \[
    \langle T^* f, g \rangle = \langle f, Tg \rangle = 0, \quad \text{for all $g \in D(T)$},
  \]
  so $T^* f = 0$. This gives $f \in \ker(T^*)$.
  So $(\overline{\Ran(T)})^\perp \subseteq \ker(T^*)$.
  Then taking orthogonal complements,
  \[
    (\ker T^*)^\perp \subseteq ((\overline{\Ran(T)})^\perp)^\perp
    = \overline{\Ran(T)}
  \]
  since $\overline{\Ran(T)}$ is closed. This
  gives $(\ker T^*)^\perp = \overline{\Ran(T)}$, and
  the other equality follows by duality.

  (4) Let $T = T^*$ and $z = x + iy$. Then we can write
  \[
    |\langle (T - z) f, f \rangle|
    = |\langle (T - (x + iy)) f, f \rangle|
    = |\langle (T - x) f, f \rangle
    - iy \langle f, f \rangle|
    \ge |y| \|f\|_{\mathcal{H}}^2,
  \]
  since $\langle (T - x) f, f \rangle$ is real and
  $iy \langle f, f \rangle$ is purely imaginary. On the
  other hand,
  \[
    \langle (T - z) f, f \rangle
    \le \|f\|_{\mathcal{H}} \|(T - z) f\|_{\mathcal{H}},
  \]
  by Cauchy-Schwarz, so $\|(T - z) f\|_{\mathcal{H}} \ge |y| \|f\|_{\mathcal{H}}$
  where $y = \im Z$. So $\ker(T - z) = 0$ if $|y| \ne 0$.
  Now
  \[
    \overline{\Ran(T - z)} = (\ker (T^* - \overline{z}))^\perp
    = (\ker (T - \overline{z}))^\perp
    = \mathcal{H}
  \]
  if $|y| \ne 0$ by $(3)$, so we just need to show that
  $\Ran(T - z)$ is closed. Assume $(T - z) f_n \to g$.
  Note that
  \[
    \|(T - z) f_n - (T - z) f_m\|_{\mathcal{H}}
    \ge |y| \|f_n - f_m \|_{\mathcal{H}},
  \]
  so $(T - z) f_n$  being Cauchy implies that
  $\{f_n\}$ is also Cauchy. So $f_n \to f$ for some
  $f \in \mathcal{H}$, i.e. $T = T^*$ is closed.
  This then implies that $T - z$ is closed, from which it
  follows that $f \in D(T - z)$ and $(T - z)f = g$, so
  $\Ran(T - z)$ is closed if $|y| \ne 0$. So
  $\ker(T - z) = \{0\}$ and $\Ran(T - z) = \mathcal{H}$
  if $|y| \ne 0$. So $T - z$ is invertible, hence
  $\Spec T \subseteq \R$ and
  $\|(T - z)^{-1}\|_{\mathcal{L}(\mathcal{H}, \mathcal{H})} \lesssim |{\im z}|^{-1}$.
\end{proof}

\section{Spectral Theory}

\begin{definition}
  The \emph{resolvent} of $T$, denoted $\rho(T) \subseteq \C$
  is defined as follows: We
  have $\lambda \in \rho(T)$ if we can find a bounded
  linear operator $R(\lambda)$ from $\mathcal{H}$ to
  $\mathcal{H}$ such that
  \[
    (T - \lambda) R(\lambda) = R(\lambda)(T - \lambda) = I.
  \]
  The \emph{spectrum} of $T$ is $\sigma(T) = \C \setminus \rho(T)$.
\end{definition}

\begin{definition}
  Let $\mathcal{B}$ be the Borel sets in $\R$.
  A \emph{spectral measure} $E$ on $\R$ is an orthogonal
  projection valued measure on $\mathcal{B}$, i.e.
  a map $E$ such that $E(A)$ is an orthogonal projection
  for every $A \in \mathcal{B}$, and
  \begin{enumerate}
    \item $E(\varnothing) = 0$ and $E(\R) = I$;
    \item for $A, B \in \mathcal{B}$, we have
      $E(A) E(B) = E(A \cap B)$;
    \item for disjoint $A_j \in \mathcal{B}$, we have
      $E(\bigcup_{j = 1}^\infty A_j) = \sum_{j = 1}^\infty E(A_j)$.
  \end{enumerate}
\end{definition}

\begin{remark}
  Given a spectral measure $E$, for $x, y \in \mathcal{H}$,
  we can define
  \[
    E_{x, y}(A) = \langle E(A) x, y \rangle \quad
    \text{for $A \in \mathcal{B}$}.
  \]
  For fixed $x, y$,
  $E_{x, y}$ is a complex-valued Borel measure. In particular,
  $E_{x, x}$ is a positive Borel measure.
\end{remark}

\begin{theorem}[Spectral theorem]
  Let $(T, D(T))$ be self-adjoint on $\mathcal{H}$. Then
  there is a unique spectral measure $E$ such that
  \[
    \langle Tx, y \rangle
    = \int \lambda \, dE_{x, y}(\lambda)
    \quad \text{for all $x \in D(T)$, $y \in \mathcal{H}$}.
  \]
\end{theorem}

\begin{remark}
  We make the following remarks:
  \begin{enumerate}
    \item We have $\supp E_{x, y} \subseteq \sigma(T)$.
    \item If $f$ is a Borel measurable function, then
      we can interpret
      \[
        \langle f(T) x, y \rangle
        = \int f(\lambda) \, dE_{x, y}(\lambda).
      \]
      In particular, we have a map
      $f \mapsto f(T)$.
  \end{enumerate}
\end{remark}
