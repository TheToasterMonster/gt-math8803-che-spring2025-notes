\chapter{Jan.~27 --- Kato Smoothing}

\section{Kato Smoothing}

\begin{theorem}[Kato $1 / 2$ smoothing estimate]
  Let $d \ge 2$ and $\chi \ge 0$ be a smooth
  cutoff function such that $\widehat{\chi}$ is
  compactly supported. Then we have the estimate
  \[
    \|\chi(x) (-\Delta)^{1 / 4} e^{i t \Delta / 2} f\|_{L^2_{x, t}} \le C\|f\|_{L^2}.
  \]
\end{theorem}

\begin{remark}
  The above theorem says that when we localize, we
  gain a smoothing effect ($1 / 2$ derivatives).
  Also note that this effect is not pointwise, but rather
  after integrating in time.
\end{remark}

\begin{remark}
  Let $f(t) = e^{it \Delta / 2} f_0$, where
  $f_0 = e^{-|x|^2 / 2} e^{-ix v}$. Think of this
  as a quantum particle at the origin with an initial
  velocity $v$, so
  that the particle will stay in
  $B(0, 1)$ for a period of order $O(1 / |v|)$. Then
  \[
    \left(\int_0^{1 / |v|} \int_{B(0, 1)} |f|^2\, dx dt\right)^{1 / 2}
    \sim \left(\frac{1}{|v|}\right)^{1 / 2}
    = \frac{1}{|v|^{1 / 2}}.
  \]
  Now $(-\Delta)^{1 / 2} f \sim |v|^{1 / 2}$, so these
  factors cancel each other out.
  This matches the above estimate.
\end{remark}

\begin{theorem}[1-D Kato smoothing estimate]
  For $d = 1$, we have
  \[
    \sup_x \|(-\Delta)^{1 / 4} e^{it \Delta / 2} f\|_{L^2_t}
    \le C \|f\|_{L^2}.
  \]
\end{theorem}

\begin{proof}
  For $d = 1$, by the Fourier transform we have
  \begin{align*}
    (-\Delta)^{1 / 4} e^{it \Delta / 2} f
    &= \frac{1}{2\pi} \int_\R e^{ix \xi} |\xi|^{1 / 2} e^{-it \xi^2 / 2} \widehat{f}(\xi)\, d\xi \\
    &= \frac{1}{2\pi} \int_{-\infty}^0 e^{ix \xi} |\xi|^{1 / 2} e^{-it \xi^2 / 2} \widehat{f}(\xi)\, d\xi
    + \frac{1}{2\pi} \int_{0}^\infty e^{ix \xi} |\xi|^{1 / 2} e^{-it \xi^2 / 2} \widehat{f}(\xi)\, d\xi.
  \end{align*}
  Since we want an $L^2_t$ bound, it indicates
  that we should apply Plancherel's theorem in time.
  We will prove the estimate for the latter integral, and
  the former integral is left as an exercise.
  Set $\eta = \xi^2$ with $d\eta = 2\xi\, d\xi = 2 \sqrt{\eta}\, d\xi$.
  Applying this change of variables, we obtain
  \[
    (-\Delta)^{1 / 4} e^{it \Delta / 2} f
    = \frac{1}{2\pi} \int_0^\infty e^{ix \sqrt{\eta}} |\eta|^{1 / 4} e^{-it \eta / 2} \widehat{f}(\sqrt{\eta}) \frac{1}{2\sqrt{\eta}} \, d\eta
    = \frac{1}{4\pi} \int_0^\infty e^{ix \sqrt{\eta}}
    |\eta|^{-1 / 4} e^{-it\eta / 2} \widehat{f}(\sqrt{\eta})\, d\eta.
  \]
  Fix $x$ and denote $h(\eta) = e^{ix \sqrt{\eta}} |\eta|^{-1 / 4} \widehat{f}(\sqrt{\eta})$. Then we have
  \[
    (-\Delta)^{1 / 4} e^{it \Delta / 2} f
    = \frac{1}{4\pi} \int_0^\infty e^{-it \eta / 2} h(\eta)\, d\eta = (\star).
  \]
  By Plancherel's theorem in time, we see that
  \[
    \| (\star) \|_{L^2_t}
    \le \|h(\eta)\|_{L^2_\eta}.
  \]
  Then we can estimate (setting $z = \sqrt{\eta}$ with
  $dz = d\eta / (2\sqrt{\eta})$)
  \[
    \int_0^\infty |h(\eta)|^2\, d\eta
    = \int_0^\infty |\eta|^{-1 / 2} |\widehat{f}(\sqrt{\eta})|^2\, d\eta
    = 2 \int_0^\infty |\widehat{f}(z)|^2\, dz
    = C \|f\|_{L^2}^2,
  \]
  where the last inequality follows by Plancherel's
  theorem in $z$. This yields
  $\|(\star)\|_{L^2_t} \le C \|f\|_{L^2}$, and since
  these estimates are all independent
  of $x$, we get that $\sup_x \|(\star)\|_{L^2_t} \le C\|f\|_{L^2}$.
  Putting this together with an identical estimate
  for the other integral, we obtain the desired
  bound.
\end{proof}

\begin{remark}
  The above 1-D version is stronger and implies the
  statement
  \[
    \| \chi (-\Delta)^{1 / 4} e^{it \Delta / 2} f\|_{L^2_{x, t}}
    \le C \|f\|_{L^2}
  \]
  for the 1-D case. Check this as an exercise.
\end{remark}

\section{Coarea Formula}

\begin{remark}
  In dimension $d$, suppose we have two nice functions
  $g$ and $u$ such that $u^{-1}(t)$ is a
  $(d - 1)$-dimensional hypersurface. Then the
  \emph{coarea formula} says that
  \[
    \int_{\R^d} g(x) |\nabla u(x)|\, dx
    = \int_\R \int_{\{u(x) = t\}} g(x)\, d\sigma(x) dt,
  \]
  where $\sigma(x)$ is the surface measure on
  $\{u(x) = t\}$.

  Note that for $d = 1$, this says that
  \[
    \int g(x) |\partial_x u|\, dx
    = \int_\R \left(\int_{\{u(x) = t\}} g(x)\, dx\right) dt
    = \int_\R g(u^{-1}(t))\, dt.
  \]
  In particular, this is the change of
  variables formula where $\eta = u^{-1}(t)$ (so
  $u(\eta) = t$ and $dt = \partial_\eta u\, d \eta$).
\end{remark}

\begin{lemma}
  Let $F \in C_0^\infty$ and $\phi$ be smooth. Then
  one has
  \[
    \int_\R \int_{\R^d} e^{i \lambda \phi(x)} F(x)\, dx d\lambda
    = (2\pi)^d \int_{\{\phi = 0\}} \frac{F(x)}{|\nabla \phi(x)|}\, d\sigma(x).
  \]
\end{lemma}

\begin{proof}
  By the coarea formula (using $g(x) = e^{i \lambda \phi(x)} F(x) / |\nabla \phi(x)|$ and $u = \phi$), we have
  \[
    \int_\R \int_{\R^d} e^{i\lambda \phi(x)} F(x)\, dx d\lambda
    = \int_\R \int_\R e^{i \lambda y} \int_{\{\phi = y\}} \frac{F(x)}{|\nabla \phi(x)|}\, d\sigma(x) dy d\lambda.
  \]
  Denote by $h(y)$ the integral
  \[
    h(y) = \int_{\{\phi = y\}} \frac{F(x)}{|\nabla \phi(x)|}\, d\sigma(x).
  \]
  Then we can see that
  \[
    \int_\R \int_\R e^{i \lambda y} \int_{\{\phi = y\}} \frac{F(x)}{|\nabla \phi(x)|}\, d\sigma(x) dy d\lambda
    = \int_\R \int_\R e^{i \lambda y} h(y)\, dy d\lambda
    = \int_\R \widehat{h}(\lambda)\, d\lambda
    = (2\pi)^d h(0).
  \]
  This gives the desired equality after plugging
  in the definition of $h(0)$.
\end{proof}
