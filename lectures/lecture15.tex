\chapter{Mar.~3 --- Dispersive Decay with Potential, Part 4}

\section{Low Energy Estimates, Continued}
\begin{lemma}
  Suppose $W(0) \ne 0$. Then we have
  \[
    \sup_{x < y} \left|\int_{-\infty}^\infty e^{it\lambda^2} \frac{\lambda \chi(\lambda^2)}{W(\lambda)} f_+(\lambda, y) f_-(\lambda, x)\, d\lambda\right|
    \le |t|^{-1 / 2}.
  \]
\end{lemma}

\begin{proof}
  We first consider the case $x < 0 < y$. Then
  $f_+(\lambda, y) \to e^{i y \lambda}$ as
  $y \to \infty$ and
  $f_-(\lambda, x) \to e^{-i x \lambda}$ as
  $x \to -\infty$. So we can write
  \begin{align*}
    \left|\int_{-\infty}^\infty e^{it\lambda^2} e^{i\lambda(y - x)} \frac{\lambda \chi(x^2)}{W(\lambda)} m_+(\lambda, y)m_-(\lambda, x)\, d\lambda\right|
    &= \left|e^{i\partial_x^2 t} \left( \frac{\lambda \chi(\lambda^2)}{W(\lambda)} m_+(\lambda, y) m_-(\lambda, x)\right)^\vee\right| \\
    &\lesssim |t|^{-1 / 2}
    \left\| \left(\frac{\lambda \chi(\lambda^2)}{W(\lambda)} m_+(\lambda, y) m_-(\lambda, x)\right)^\vee \right\|_{L^1}.
  \end{align*}
  Let $\tau$ be the dual variable of $\lambda$, and
  write the inner function as $G(x, y; \tau)$, where
  $x, y$ are fixed. Then we would like to bound
  $\|G(x, y; \tau)\|_{L^1_\tau}$, for which
  it suffices to show $|G(x, y; \tau)| \lesssim \langle \tau \rangle^{-2}$.
  By Fourier duality, it suffices to bound
  the derivatives of $G$ in $\lambda$:
  \[
    \partial_\lambda^2 \left(\frac{\lambda \chi(\lambda^2)}{W(\lambda)} m_+(\lambda, y) m_-(\lambda, x)\right), \quad
    \partial_\lambda \left(\frac{\lambda \chi(\lambda^2)}{W(\lambda)} m_+(\lambda, y) m_-(\lambda, x)\right), \quad
    \left(\frac{\lambda \chi(\lambda^2)}{W(\lambda)} m_+(\lambda, y) m_-(\lambda, x)\right).
  \]
  Recalling that $\chi$ is compactly supported, it
  is sufficent to bound the $L^\infty$ norms of the
  above quantities. So we just need to show that
  $\partial_{\lambda}^j W(\lambda)$ and
  $\partial_{\lambda}^j m_\pm$
  are bounded for $j = 0, 1, 2$. Recall that
  \[
    W(\lambda)
    = W[f_+(\lambda, x), f_-(\lambda, x)]
    = [e^{i\lambda x} m_+(\lambda, x), e^{-i\lambda x} m_-(\lambda, x)]
    = [m_+(\lambda, 0), m_-(\lambda, 0)],
  \]
  where we assumed $W(0) \ne 0$.
  Note that the Wronskian is independent of the point
  at which we evaluate, so that $W(\lambda)$
  does not depend on $x$. We previously
  estimated
  $\partial_\lambda^j m_\pm$ for $x \ge 0$ and
  $j = 0, 1, 2$, so we have $L^\infty$ bounds
  for $\partial_\lambda^j W(\lambda)$ and
  $\partial_\lambda^j m_\pm$ for $j = 0, 1, 2$.
  This proves pointwise decay in this case.

  Now we consider the case $0 \le x < y$. We need
  to analyze $f_+(\lambda, y) f_-(\lambda, x)$ (note
  that $f_-(\lambda, x)$ behaves badly as
  $x \to \infty$, we only know that $f_-(\lambda, x)$
  might grow linearly as $x \to \infty$). To deal
  with this, we use
  some $1$-D scattering theory. Note that
  $f_\pm(\lambda, x)$ solve the ODE
  $-\partial_x^2 f_\pm + V f_\pm = \lambda^2 f$,
  so we can write
  \[
    f_\pm(\lambda, x) = \alpha_{\pm}(\lambda) f_{\mp} f(\lambda, x)
    + \beta_{\pm}(\lambda) f_{\mp}(-\lambda, x)
  \]
  for $\lambda \ne 0$. Since
  $f_-(\lambda, x)$ and $f_-(-\lambda, x)$ are
  linearly independent ($W[f_-(\lambda, x), f_-(-\lambda, x)] = 2i\lambda$ and $\lambda \ne 0$).
  Since $0 \le x < y$, we can use this to write
  \begin{align*}
    f_+(\lambda, y) f_-(\lambda, x)
    &= f_+(\lambda, y)(\alpha_-(\lambda) f_+(\lambda, x) + \beta_-(\lambda) f_+(-\lambda, x)) \\
    &= \alpha_-(\lambda) f_+(\lambda, y) f_+(\lambda, x)
    + \beta_-(\lambda) f_+(\lambda, y) f_+(-\lambda, x),
  \end{align*}
  and we can now make the same argument as in the
  previous case to get the decay estimate.

  In the final case $x < y \le 0$, the idea is the same
  as when $x_0 \le x < y$:
  We can write $f_+(\lambda, y)$ as a linear combination of
  $f_-(-\lambda, y)$ and $f_-(\lambda, y)$ and
  proceed as before.
\end{proof}

\begin{remark}
  This completes the proof of the low energy part.
\end{remark}

\begin{remark}
  Consider the two \emph{scattering relations} from above
  \begin{align*}
    f_+(\lambda, x)
    &= \alpha_+(\lambda) f_-(\lambda, x) + \beta_+(\lambda) f_-(-\lambda, x) \\
    f_-(\lambda, x)
    &= \alpha_-(\lambda) f_+(\lambda, x) + \beta_-(\lambda) f_+(-\lambda, x).
  \end{align*}
  We can rewrite the first equation as
  \begin{align*}
    \frac{1}{\beta_+(\lambda)} f_+(\lambda, x)
    = \frac{\alpha_+(\lambda)}{\beta_+(\lambda)} f_-(\lambda, x) + f_-(-\lambda, x),
  \end{align*}
  and we find that
  \begin{align*}
    \frac{1}{\beta_+(\lambda)} f_+(\lambda, x)
    &\xrightarrow[x \to +\infty]{} \frac{1}{\beta_+(\lambda)} e^{i\lambda x} \\
    \frac{\alpha_+(\lambda)}{\beta_+(\lambda)} f_-(\lambda, x)
    &\xrightarrow[x \to -\infty]{} \frac{\alpha_+(\lambda)}{\beta_+(\lambda)} e^{-i\lambda x} \\
    f_-(-\lambda, x)
    &\xrightarrow[x \to -\infty]{} e^{i\lambda x}.
  \end{align*}
  We call $T(\lambda) = 1 / \beta_+(\lambda)$ the
  \emph{transmission coefficient} and
  $R_+(\lambda) = \alpha_+(\lambda) / \beta_+(\lambda)$
  the \emph{reflection coefficient}.
  One also has the relation
  $|T(\lambda)|^2 + |R_+(\lambda)|^2 = 1$.
\end{remark}

\section{Distorted Fourier Transforms}

\begin{remark}
  We give an alternate approach to proving
  the low energy estimates via ``distorted Fourier
  transforms.'' Recall that the usual Fourier transform
  diagonalizes differentiation. The ``distorted
  Fourier transform'' $\widetilde{\mathcal{F}}$
  will diagonalize $-\partial_x^2 + V$, i.e. it will
  satisfy
  \[
    \widetilde{\mathcal{F}}((-\partial_x^2 + V) g)(\xi)
    = \xi^2 \widetilde{\mathcal{F}}(g).
  \]
  In Fourier space, the scattering relations from
  before become:
  \begin{align*}
    T(\xi) f_+(x, \xi)
    &= f_-(x, -\xi) + R_-(\xi) f_-(x, \xi) \\
    T(\xi) f_-(x, \xi)
    &= f_+(x, -\xi) + R_+(\xi) f_+(x, \xi)
  \end{align*}
  Recall that for $H = -\partial_x^2 + V$, we have
  the spectral resolution
  \[
    (H - (\xi^2 + i 0))^{-1}(x, y)
    = R(\xi^2 + i 0)(x, y)
    = \frac{f_+(x, \xi) f_-(y, \xi) \mathbbm{1}_{\{x \ge y\}} + f_-(x, \xi) f_+(y, \xi) \mathbbm{1}_{\{x < y\}}}{W[f_+(\cdot, \xi), f_-(\cdot, \xi)]}
  \]
  for $x, y \in \R$ and $\xi \in \R$.
\end{remark}

\begin{lemma}
  The density of the spectral resolution
  $E(d\xi^2)$ on $[0, \infty)$ can be written as
  \[
    \frac{E(d\xi^2)}{d\xi}(x, y)
    = \frac{|T(\xi)|^2}{2\pi} [f_+(x, \xi) f_+(y, -\xi) + f_-(x, \xi) f_-(y, -\xi)].
  \]
  Alternatively, it can also be written as
  \[
    \frac{E(d\xi^2)}{d\xi}(x, y)
    = \frac{1}{\pi}
    \begin{cases}
      \re [T(\xi) f_+(x, \xi) f_-(y, \xi)], & \text{if $x \ge y$}, \\
      \re [T(\xi) f_+(y, \xi) f_-(x, \xi)], & \text{if $x \ge y$}.
    \end{cases}
  \]
\end{lemma}

\begin{proof}
  Recall that (make a change of variables $\lambda = \xi^2$
  in the usual formula for $E(d\lambda)$)
  \[
    \frac{E(d\xi^2)}{d\xi}(x, y)
    = \frac{\xi}{\pi i} [R(\xi^2 + i0) - R(\xi^2 - i0)](x, y).
  \]
  Then we can write
  \[
    \frac{E(d\xi^2)}{d\xi}(x, y)
    = \frac{\xi}{2\pi} [W(\xi)^{-1} f_+(x, \xi) f_-(y, \xi) - W(-\xi)^{-1} f_+(x, -\xi) f_-(y, -\xi)].
  \]
  Using the scattering relations, we have
  $T(\xi) W(\xi) = -2 \xi i$, so we have
  \[
    \frac{E(d\xi^2)}{d\xi}(x, y)
    = \frac{1}{2\pi}(T(\xi) f_+(x, \xi) f_-(y, \xi) + T(-\xi) f_+(x, -\xi) f_-(y, -\xi)).
  \]
  Write $f_-(y, \xi)$ in the first term
  using $f_+(y, \xi)$ and $f_-(y, -\xi)$, and
  write $f_+(x, -\xi)$ in the second term
  as $f_-(x, \xi)$ and $f_+(x, \xi)$ to obtain
  (note that $T(-\xi) = T(\xi)$ and
  $R_{\pm}(-\xi) = \overline{R_{\pm}(\xi)}$)
  \begin{align*}
    \frac{E(d\xi^2)}{d\xi}(x, y)
    &= \frac{1}{2\pi} [T(\xi) f_+(x, \xi)(T(-\xi) f_+(y, -\xi) - R_-(-\xi) f_-(y, \xi)) \\
    & \quad \quad + T(-\xi)(T(\xi)f_-(x, \xi) - R_+(\xi) f_+(x, \xi)) f_-(y, -\xi)] \\
    &= \frac{|T(\xi)|^2}{2\pi} [f_+(x, \xi) f_+(y, -\xi) + f_-(x, \xi) f_-(y, -\xi)].
  \end{align*}
  This proves the case $x > y$, and $x < y$ follows
  similarly.
\end{proof}

\begin{definition}
  Define the \emph{distorted Fourier basis}
  \[
    e(x, \xi) = \frac{1}{\sqrt{2\pi}}
    \begin{cases}
      T(\xi) f_+(x, \xi) & \text{if $\xi \ge 0$}, \\
      T(-\xi) f_-(x, -\xi) & \text{if $\xi < 0$}.
    \end{cases}
  \]
\end{definition}
