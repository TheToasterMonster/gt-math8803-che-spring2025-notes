\chapter{Mar.~12 --- Spectral Theory and Negative Eigenvalues}

\section{Spectral Theory for \texorpdfstring{$-\Delta + V$}{-Delta + V}}
\begin{lemma}
  Suppose $V \to 0$ as $|x| \to \infty$.
  Then the following are equivalent:
  \begin{enumerate}
    \item There exists an eigenvalue $\lambda < 0$
      of $H = -\Delta + V$.
    \item There exists $f \in H^2$
      such that $\langle H f, f \rangle < 0$.
  \end{enumerate}
\end{lemma}

\begin{proof}
  $(1 \Rightarrow 2)$ Let $f$ be the eigenfunction,
  so that $H f = \lambda f$. Then
  $\langle Hf, f \rangle  = \lambda \langle f, f \rangle < 0$.

  $(2 \Rightarrow 1)$
  If $\langle Hf , f \rangle < 0$, then by the
  spectral theorem, the spectrum of $H$ must extend
  below zero. But the essential spectrum of
  $H$ is $[0, \infty)$, so any part of the spectrum below
  must be an eigenvalue.
\end{proof}

\begin{theorem}[Birman-Schwinger]
  In $\R^3$, let $V^{-}$ denote the negative part
  of $V$, and suppose that
  \[
    \int_{\R^3} \int_{\R^3} \frac{|V^-(x)| |V^-(y)|}{|x - y|} \,dx dy < (4\pi)^2.
  \]
  Then $H = -\Delta + V$ has no negative eigenvalues.
\end{theorem}

\begin{remark}
  If $V \ge 0$, then integration by parts yields
  \[
    \langle (-\Delta + V) f, f \rangle
    = \langle \nabla f, \nabla f \rangle + \langle V f, f \rangle
    = \|\nabla f\|^2 + \langle V f, f \rangle \ge 0.
  \]
  So the previous lemma implies that $-\Delta + V$
  has no negative eigenvalues.
\end{remark}

\begin{proof}
  We argue by contradiction. Assume the exists an
  eigenvalue $\lambda < 0$. Then by the previous lemma,
  we can find $f \in H^2$ such that
  $\langle Hf, f \rangle < 0$. Write
  $V = V^+ + V^-$,
  where $V^+, V^-$ are the positive and negative parts
  of $V$, respectively. Since $V^+ \ge 0$, we have
  \begin{align*}
    \langle Hf, f \rangle
    = \langle (-\Delta + V) f, f \rangle
    &= \langle (-\Delta + V^+ + V^-) f, f \rangle \\
    &= \langle (-\Delta + V^-) f, f \rangle + \langle V^+ f, f \rangle
    \ge \langle (-\Delta + V^-) f, f \rangle.
  \end{align*}
  Let $\widetilde{H} = -\Delta + V^-$, so
  $\langle \widetilde{H} f, f \rangle < 0$.
  By the previous lemma, $\widetilde{H}$ has a
  negative eigenvalue $\widetilde{\lambda}$, and
  we can find $\phi$ such that
  $(-\Delta + V^-) \phi
    = \widetilde{H} \phi = \widetilde{\lambda} \phi$.
    Then $(-\Delta - \widetilde{\lambda}) \phi = -V^- \phi \in L^2$, and we can write
    \[
      \phi = -(-\Delta - \widetilde{\lambda})^{-1} V^- \phi,
    \]
    where $(-\Delta - \widetilde{\lambda}$ is invertible
    since $-\widetilde{\lambda} > 0$. Note that
    $V^- \le 0$, so
    \[
      \underbrace{\sqrt{|V^-|}}_{\psi} = \underbrace{\sqrt{|V^-|} (-\Delta - \widetilde{\lambda})^{-1}\sqrt{|V^-|}}_T \underbrace{(\sqrt{|V^-|}\phi)}_{\psi}.
    \]
    The above says $\psi = T \psi$, so
    $\|T\|_{L^2 \to L^2} \ge 1$ (recall that
    $\|T\|_{L^2 \to L^2} = \sup_{f \in L^2} (\|Tf\|_{L^2} / \|f\|_{L^2}))$.
    But
    \[
      Tf(x)
      = \sqrt{|V^{-}|(x)} \int_\R (-\Delta - \widetilde{\lambda})^{-1}(x, y) \sqrt{|V^{-}|(y)} f(y) \,dy.
    \]
    Such a $T$ is called a \emph{Hilbert-Schmidt operator}.
    These operators have kernels, so we have
    \begin{align*}
      \|T\|_{L^2 \to L^2}
      \le \|T\|_{\mathrm{H.S.}}
      &= \left(\int_{\R^3} \int_{\R^3} |\mathrm{kernel}(T)|^2(x, y)\, dxdy\right)^{1 / 2} \\
      &= \left(\int_{\R^3} \int_{\R^3} \sqrt{|V^{-}|}(x) ((-\Delta - \widetilde{\lambda})^{-1}(x, y) \sqrt{|V^-|}(y))^2\, dxdy\right)^{1 / 2} \\
      &= \left(\int_{\R^3} \int_{\R^3} |V^{-}|(x) \frac{e^{-i\sqrt{\widetilde{\lambda}}|x - y|}}{(4\pi|x - y|)^2} |V^-|(y) \, dxdy\right)^{1 / 2} \\
      &= \left(\int_{\R^3} \int_{\R^3} \frac{|V^-|(x) |V^-(y)|}{(4\pi |x - y|)^2} \, dxdy\right)^{1 / 2}
      < 1
    \end{align*}
    by assumption, which contradicts
    $\|T\|_{L^2 \to L^2} \ge 1$.
\end{proof}

\begin{remark}
  The full version of Birman-Schwinger says that
  \[
    \text{\# of negative eigenvalues of $H$}
    \le \frac{1}{(4\pi)^2} \int_{\R^3} \int_{\R^3} \frac{|V^-(x)| |V^-(y)|}{|x - y|} \,dx dy.
  \]
  A proof can be seen in Reed-Simon IV.
\end{remark}

\begin{lemma}
  In one dimension, if $V \in L^1(\R)$ and
  \[
    \int_{-\infty}^\infty V(x)\, dx < 0,
  \]
  then $H = -\partial_x^2 + V$ has a negative
  eigenvalue.
\end{lemma}

\begin{proof}
  By the first lemma from the day, it suffices
  to find $f \in H^2$ such that $\langle Hf, f \rangle < 0$.
  Let $\phi$ be a nice function, and
  set $f(x) = \phi(\epsilon x)$ for
  $\epsilon \ll 1$ to be determined. We have
  \begin{align*}
    \langle Hf, f \rangle
    = \langle f', f' \rangle + \langle Vf, f \rangle
    &= \epsilon^2 \int |\phi'(\epsilon x)|^2\, dx
    + \int V \phi^2(\epsilon x)\, dx \\
    &= \epsilon \int |\phi'|^2\, dx
    + \int V \phi^2(\epsilon x) \, dx
    \xrightarrow[\epsilon \to 0]{} 0 + |\phi^2(0)|^2 \int V\, dx < 0
  \end{align*}
  by the dominated convergence theorem.
  Thus taking $\epsilon$ small enough, we can get
  $\langle Hf, f \rangle < 0$.
\end{proof}

\begin{remark}
  Consider
  \[
    (*) = \iint \frac{|V(x)| |V(y)|}{|x - y|^2}\, dxdy
  \]
  in $\R^3$. We know that if $(*)$ is small enough,
  then $H$ behaves similarly to $-\Delta$.
  The \emph{Kato norm} of $V$ is
  \[
    \|V\|_{K}
    = \sup_x \int \frac{|V(y)|}{|x - y|}\, dy.
  \]
  This norm is scaling invariant. In 2004,
  Rodnianski-Schlag showed that
  if $\|V\|_K < 4\pi$, then
  \[
    \|e^{itH} P_{\mathrm{ac}} f\|_{L^\infty}
    \lesssim |t|^{-3 / 2} \|f\|_{L^1}.
  \]
  Another result by Beceanu-Goldberg in 2010 showed
  that if $\|V\|_{K} < \infty$ and $V$ is ``nice''
  such that $H$ has
  ``no zero resonance'' nor ``zero eigenfunctions,''
  then
  \[
    \|e^{itH} P_{\mathrm{ac}} f\|_{L^\infty}
    \lesssim |t|^{-3 / 2} \|f\|_{L^1}.
  \]
\end{remark}

\section{Agmon Bound}
\begin{remark}
  We consider the equation $(-\Delta + V) \psi = E\psi$
  with $E < 0$ and $\psi \in H^2$.
\end{remark}

\begin{theorem}[Agmon bound]
  Let $V \in C(\R^d)$ and $V \to 0$ as $|x| \to \infty$.
  Then
  \[
    \int_{\R^d} e^{\alpha |x|} |\psi(x)|^2\, dx < \infty
  \]
  for some $\alpha > 0$. If $V$ is smooth and
  $\|D^\gamma V\|_{L^\infty} < \infty$ for
  $|\gamma| \ge (n + 1) / 2$, then
  $|\psi(x)| \lesssim e^{-\alpha |x| / 2}$.
\end{theorem}

\begin{remark}
  Some intuition is the following:
  Consider $-\psi'' + V(x) \psi = E\psi$ in $1$-D,
  and suppose that
  \[
    \psi(x) \sim \exp\left(-\int_0^x\sqrt{(V - E)^+}\, dy\right).
  \]
  Then $\psi'(x) \sim -\sqrt{(V - E)^+} \psi(x)$ and
  $\psi''(x) \sim (V - E)^+ \psi(x) - (V'(x) / \sqrt{(V - E)^+}) \psi(x)$.
  In the limit $x \to \infty$, we have
  $V \to 0$ and $E < 0$, so
  $(V - E)^+ = V - E$. We also have $V' \to 0$, so
  $V'(x) / \sqrt{(V - E)^+}$ is small and so
  $\psi$ approximately solves
  $-\psi'' + V(x) \psi = E\psi$.
\end{remark}

\begin{proof}
  Define the \emph{Agmon metric} by
  \[
    P_E(x) = \inf_{\text{$\gamma$ a path $0 \to x$}}
    \int_0^1 \sqrt{(V(\gamma(t)) - E)^+} |\gamma'(t)|\, dt.
  \]
  Then one can check directly that $|\nabla P_E(x)| \le \sqrt{(V(x) - E)^+}$
  and $P_E(x) \le (\|V\|_{L^\infty} + |E|)^{1 / 2} |x|$.
  The crucial estimate is the following: Let
  $w(x) = \exp(\min\{2(1 - \epsilon P_E(x), N)\})$
  for some $0 < \epsilon \ll 1$ and $N \gg 1$, and let
  $\phi \in C^\infty$ with $\phi = 1$ for
  $|x| \gg 1$ and $\supp \phi \subseteq \{x : V(x) - E > 0 \}$.
  Then
  \[
    \int w(x) |\psi(x)|^2 \phi^2(x)\, dx \lesssim \|\psi\|_{L^2}^2.
  \]
  We will finish the proof next class.
\end{proof}

\begin{remark}
  In classic mechanics, the \emph{total energy} of
  a particle is given by
  \[
    E = V(x) + \frac{1}{2} mv^2,
  \]
  where $V$ is the \emph{potential energy} and $mv^2 / 2$
  is the \emph{kinetic energy}. Note that this means
  $V \le E$, so we cannot find a classical particle
  with $V > E$. For a quantum particle, however,
  we can find them in the region
  $R = \{x \in \R^n : V > E\}$. This region $R$ is
  called the \emph{classically-forbidden region}.
\end{remark}
