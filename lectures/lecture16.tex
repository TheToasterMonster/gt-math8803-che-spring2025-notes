\chapter{Mar.~5 --- Scattering Relations}

\section{Distorted Fourier Transform, Continued}

\begin{remark}
  Let $H = -\partial_x^2 + V$ and $P_{\mathrm{c}}$
  be the projection onto the continuous spectrum of
  $\mathcal{H}$. By spectral theory, we can write
  \[
    H P_{\mathrm{c}} f
    = \frac{1}{2\pi i} \int_0^\infty \lambda E(d\lambda) f,
  \]
  which gives
  \[
    P_c f
    = \frac{1}{2\pi i} \int_0^\infty E(d\lambda) f
    = \frac{1}{2\pi i} [R(\lambda + i 0) - R(\lambda - i 0)] f\, d\lambda
  .\]
  Setting $\lambda = \xi^2$, we find
  \begin{align*}
    P_{\mathrm{c}} f
    &= \frac{1}{2\pi i} \int_0^\infty 2\xi [R(\xi^2 + i 0) - R(\xi^2 - i 0)] f\, d\xi \\
    &= \int_{\R} \int_0^\infty
    \frac{|T(\xi)|^2}{2\pi} [f_+(x, \xi)f_+(y, -\xi) + f_-(x, \xi) f_-(y, -\xi)] f(y)\,
    d\xi dy \\
    &= 
    \int_{\R} \int_0^\infty
    \frac{|T(\xi)|^2}{2\pi} (f_+(x, \xi)f_+(y, -\xi)) f(y)\,
    d\xi dy
    +
    \int_{\R} \int_0^\infty
    \frac{|T(\xi)|^2}{2\pi} (f_-(x, \xi)f_-(y, -\xi)) f(y)\,
    d\xi dy.
  \end{align*}
  We keep the first piece and
  change variables $\xi \mapsto -\xi$ in the second
  piece to get
  \begin{align*}
    P_{\mathrm{c}} f
    &= \int_{\R} \int_0^\infty
    \frac{|T(\xi)|^2}{2\pi} (f_+(x, \xi)f_+(y, -\xi)) f(y)\,
    d\xi dy
    +
    \int_{\R} \int_{-\infty}^0
    \frac{|T(\xi)|^2}{2\pi} (f_-(x, -\xi)f_-(y, \xi)) f(y)\,
    d\xi dy \\
    &= \int_\R e(x, \xi) \int_\R \overline{e(y, \xi)} f(y)\, dy d\xi,
  \end{align*}
  where $e(x, \xi)$ is the distorted Fourier basis
  (note that
  $\overline{f_+}(y, \xi) = f_+(y, -\xi)$
  and $\overline{f_-}(y, -\xi) = f_-(y, \xi)$).
  Note when $V = 0$, we have
  $f_\pm = e^{\pm i x \xi}$ and
  $T(\xi) = 1$, so we recover the standard Fourier
  transform.
\end{remark}

\begin{corollary}
  Let $H = -\partial_x^2 + V$. Define the
  \emph{distorted Fourier transform} by
  \[
    \widetilde{F}(f) = \int_{\R} \overline{e(x, \xi)} f(x)\, dx \quad \text{and} \quad
    \widetilde{F}^{-1}(h) = \int_{\R} e(x, \xi) h(x)\, dx.
  \]
  Then for any Borel function $G$, we have
  the formula
  \[
    G(H) P_{\mathrm{c}} f
    = \widetilde{F}^{-1} [G(\xi) \widetilde{F}(f)(\xi)].
  \]
\end{corollary}

\begin{remark}
  Using the distorted Fourier transform, we can write
  \[
    e^{iHt} P_{\mathrm{c}} f
    = \int_\R e(x, \xi) e^{it \xi^2} \widetilde{\mathcal{F}}(f)(\xi)\, d\xi. \tag{$*$}
  \]
  Note that for $\xi \ge 0$, we have
  $e(x, \xi) = T(\xi) f_+(x, \xi) \sim e^{ix \xi}$
  as $x \to \infty$. So at least morally,
  \[
    \int_\R e(x, \xi) e^{it \xi^2} \widetilde{\mathcal{F}}(f)(\xi)\, d\xi
    = e^{-i\partial_x^2 t} [\mathcal{F}^{-1} [\widetilde{\mathcal{F}}(f)]].
  \]
  On the other hand, we do not know the precise
  behavior of $T(\xi) f_+(x, \xi)$ as $x \to -\infty$.
  But we can use the scattering relation and write
  $f_+$ in terms of $f_-$:
  \[
    T(\xi) f_+(x, \xi) = f_-(x, -\xi) + R_-(\xi) f_-(x, \xi),
  \]
  so we can write
  \[
    \int_{\R} e^{ix\xi} e^{it \xi^2} \widetilde{F}(f)(\xi)\, d\xi
    + \int_{\R} R_-(\xi) e^{-ix\xi} e^{it \xi^2} \widetilde{F}(f)(\xi)\, d\xi,
  \]
  which we can estimate as a $1$-D oscillatory
  integral. Then using $(*)$, we get
  $\|e^{iHt} P_{\mathrm{c}} f\|_{L^\infty} \lesssim |t|^{-1 / 2}$.
\end{remark}

\section{Scattering Relations}

\begin{remark}
  Recall that we used the scattering relations
  \begin{align*}
    T(\xi) f_+(x, \xi) = f_-(x, -\xi) + R_-(\xi) f_-(x, \xi), \\
    T(\xi) f_-(x, \xi) = f_+(x, -\xi) + R_+(\xi) f_+(x, \xi).
  \end{align*}
  Also recall that
  \[
    f_+(x, \xi) = \alpha_+(\xi) f_-(x, \xi) + \beta_+(\xi) f_-(x, -\xi) \tag{$\star$}
  \]
  since $f_+(x, \xi)$ and $f_-(x, -\xi)$ are
  linearly independent for $\xi \ne 0$. Similarly,
  we can write
  \[
    f_-(x, \xi)
    = \alpha_-(\xi) f_+(x, \xi)
    + \beta_-(\xi) f_+(x, -\xi). \tag{$\star\star$}
  \]
  The scattering relations then come from dividing
  these equations by $\alpha_{\pm}(\xi)$. We will now
  try to understand these coefficients in more detail.
  Recall that
  \begin{align*}
    W(\lambda)
    = W[f_+(x, \lambda), f_-(x, \lambda)]
    &= W[\alpha_+(\lambda) f_-(x, \lambda) + \beta_+(\lambda) f_-(x, -\lambda), f_-(x, \lambda)] \\
    &= \beta_+(\lambda) W[f_-(x, -\lambda), f_-(x, \lambda)].
  \end{align*}
  Now passing $x \to -\infty$ (since the Wronskian
  is independent of the point at which we evaluate),
  we have
  \[
    W(\lambda) = \beta_+(\lambda)
    \begin{vmatrix}
      e^{i\lambda x} & i\lambda e^{i\lambda x} \\
      e^{-i\lambda x} & -i\lambda e^{-i\lambda x} \\
    \end{vmatrix}
    = -2i\lambda \beta_+(\lambda).
  \]
  Using the same idea, we can compute that
  \[
    W[f_+(x, \lambda), f_-(x, -\lambda)]
    = \alpha_+(\lambda) W[f_-(x, \lambda), f_-(x, -\lambda)]
    = 2i\lambda \alpha_+(\lambda).
  \]
  Continuing in this manner, we also get
  $W[f_-(x, \lambda), f_+(x, -\lambda)] = -2i\lambda \alpha_-(\lambda)$
  and
  \[
    \beta_-(\lambda) = \beta_+(\lambda)
    = \frac{W(\lambda)}{-2i\lambda}.
  \]
  From this, we can also conclude
  \[
    \alpha_+ = \frac{W[f_+(\lambda), f_-(-\lambda)]}{2i\lambda} \quad \text{and} \quad
    \alpha_- = \frac{W[f_-(\lambda), f_+(-\lambda)]}{-2i\lambda}.
  \]
  Substituting $(\star\star)$ into $(\star)$, we get
  \begin{align*}
    f_+(x, \xi)
    &= \alpha_+(\xi) (\alpha_-(\xi) f_+(x, \xi) + \beta_-(\xi) f_+(x, -\xi))
    + \beta_+(\xi) (\alpha_-(-\xi) f_+(x, -\xi) + \beta_-(-\xi) f_+(x, \xi)) \\
    &= [\alpha_+(\xi) \beta_-(\xi) + \beta_+(\xi) \alpha_-(-\xi)] f_+(x, -\xi)
    + [\alpha_+(\xi) \alpha_-(\xi) + \beta_+(\xi) \beta_-(-\xi)] f_+(x, \xi).
  \end{align*}
  This then tells us that
  \[
    \alpha_+(\xi) \alpha_-(\xi) + \beta_+(\xi) \beta_-(-\xi) = 1
    \quad \text{and} \quad
    \alpha_+(\xi) \beta_-(\xi) + \beta_+(\xi) \alpha_-(-\xi) = 0.
  \]
  Using our formulas for $\alpha_{\pm}$ and $\beta_{\pm}$ in the second equation,
  we get
  \[
    2i \xi \beta(-\xi) = W(-\xi) = \overline{W(\xi)}
    = 2 i\xi \beta(\xi) \quad \text{and} \quad
    -2i \xi \alpha_+(\xi) = 2i\xi \overline{\alpha_-(\xi)},
  \]
  so $\beta(-\xi) = \beta(\xi)$, and
  $\alpha_+(\xi) = -\overline{\alpha_-(\xi)}$
  or $\alpha_-(\xi) = -\overline{\alpha_+(\xi)}$.
  By the first equation, this implies
  \[
    -|\alpha_+(\xi)|^2 + |\beta(\xi)|^2 = 1.
  \]
  Dividing by $|\beta(\xi)|^2$ and using some algebra,
  we find that
  \[
    \frac{1}{|\beta(\xi)|^2}
    + \frac{|\alpha_+(\xi)|^2}{|\beta(\xi)|^2} = 1.
  \]
  Setting $T(\xi) = 1 / \beta(\xi)$ and
  $R_{\pm}(\xi) = \alpha_{\mp}(\xi) / \beta(\xi)$,
  we get the relationss
  \[
    |T(\xi)|^2 + |R_{\pm}(\xi)|^2 = 1.
  \]
  We can also define
  \[
    S(\xi) =
    \begin{bmatrix}
      T(\xi) & R_-(\xi) \\
      R_+(\xi) & T(\xi)
    \end{bmatrix},
  \]
  which is called the \emph{scattering matrix}.
\end{remark}
