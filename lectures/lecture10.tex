\chapter{Feb.~12 --- Spectral Theory, Part 2}

\section{Examples for Spectral Theory}

\begin{example}
  Let $\mathcal{H} = \R^n$ with the standard inner product, and
  let $T$ be a Hermitian matrix. Then $T$ is self-adjoint, and
  the corresponding spectral measure $E$ is given by
  \[
    E(I) = \sum_{j : \lambda_j \in I} P_{L_j},
  \]
  where $\lambda_1 \le \lambda_2 \le \dots \le \lambda_k$ are 
  the eigenvalues of $T$ without multiplicity and
  $L_1, \dots, L_k$ are eigenspaces.
\end{example}

\begin{example}
  Let $T$ be a self-adjoint operator. Then
  \[
    e^{itA} f = \left(\int e^{it\lambda)}\, E(d\lambda)\right) f.
  \]
  From this we have
  \[
    \|e^{itA} f\|_{L^2}^2
    = \langle \int e^{it\lambda}\, E(d\lambda) f, \int e^{it\lambda}\, E(d\lambda) f\rangle
    = \iint e^{it \lambda} e^{-it\mu}\, \langle E(d\lambda) E(d\mu) f, f \rangle.
  \]
  Fixing $\lambda$ and integrating over $\mu$, the
  only point that survives is $\mu = \lambda$, so
  \[
    \|e^{itA} f\|_{L^2}^2
    = \int \langle Ed(\lambda) f, f \rangle
    = \langle E(\R) f, f \rangle
    = \|f\|_{L^2}^2.
  \]
  since $E(\R) = I$. Thus if $T$ is self-adjoint, then
  $\|e^{itT} f\|_{L^2} = \|f\|_{L^2}$, i.e. $e^{itT}$ is
  unitary. Also,
  \[
    e^{itT} e^{isT} f = e^{i(t+s)T} f \quad \text{and} \quad
    e^{-itT} e^{itT} = I = e^{itT} e^{-itT}.
  \]
  The above are group properties for $e^{itT}$ as
  $t$ varies.
\end{example}

\begin{example}
  Let $\mathcal{H} = L^2(X, \mu)$, where $\mu$ is a positive
  probability measure. Define $T = M_\phi$ to be the
  multiplication operator by $\phi$, i.e.
  $Tf = \phi f$ for $\phi : X \to \R$. We have
  \[
    D(T) = \{f \in L^2 : \phi f \in L^2\},
  \]
  which is dense in $\mathcal{H}$ since
  $\mu(|\phi| > M) \to 0$ as $M \to \infty$. So $T$ is
  densely defined. Then
  \[
    \langle Tf, g \rangle
    = \int \phi f \overline{g}\, \mu(dx)
    = \int f \overline{\phi g}\, \mu(dx)
    = \langle f, Tg \rangle
  \]
  since $\phi$ is real. So $T$ is symmetric, which implies
  $D(T) \subseteq D(T^*)$. To get the
  reverse inclusion, notice that for $g \in D(T^*)$, we have
  $\langle Tf, g \rangle = \langle f, h \rangle$ for
  some $h \in \mathcal{H}$, for all $f \in D(T)$. Then
  \[
    \int f \overline{\phi g}\, \mu(dx)
    \int \phi f \overline{g}\, \mu(dx)
    = \langle Tf, g \rangle
    = \langle f, h \rangle
    = \int f\overline{h} \mu(dx).
  \]
  Since $D(T)$ is dense, we get $\phi g = h \in \mathcal{H} = L^2(X, \mu)$, so
  $g \in D(T)$.
  This gives $D(T^*) \subseteq D(T)$, so we see that
  $D(T^*) = D(T)$ and $T$ is self-adjoint.
  The spectral theorem then gives the spectral resolution
  $E$:
  \[
    E(I) f = \chi_{\{\phi \in I\}} f,
  \]
  where $\chi$ denotes an indicator function.
\end{example}

\begin{example}
  Let $T = -\Delta$ on $L^2(\R^d)$. Clearly
  $D(T) = H^2(\R^d)$ is dense in $L^2(\R^d)$. Integration
  by parts shows that $-\Delta$ is symmetric.
  Now suppose $g \in D(T^*)$. Then
  \[
    \langle -\Delta f, g \rangle
    = \langle f, h \rangle \quad \text{for all $f \in D(T)$ and some $h \in L^2$}.
  \]
  Since
  $\langle -\Delta f, g \rangle = \langle f, -\Delta g \rangle = L f$
  (note that $-\Delta g \in \mathcal{S}'$, i.e. in the
  sense of distributions), we have
  \[
    |Lf| \le \|f \|_{L^2} \| h \|_{L^2}
    = C \|f \|_{L^2},
  \]
  where $C = \| h \|_{L^2}$. So by the Hahn-Banach
  theorem, $L$ can be extended to a continuous functional
  on $L^2$ such that $\|L\|_{B(L^2, L^2)} \le \|h \|_{L^2}$.
  We can identify  $Lf = \langle f, -\Delta g \rangle$, and thus
  \[
    \|\Delta g\|_{L^2} \le \|h \|_{L^2}.
  \]
  This implies $\Delta g \in L^2$, so by the elliptic theory,
  $g \in H^2 = D(T)$. This gives
  $D(T^*) \subseteq D(T)$ and thus $D(T) = D(T^*)$ (since
  $D(T) \subseteq D(T^*)$ because $T$ is symmetric), so
  $T$ is self-adjoint. To understand the spectral
  resolution, we can use the Fourier transform:
  $\widehat{Tf} = \xi^2 \widehat{f}$. Writing
  $\mathcal{F}(f) = \widehat{f}$, we have
  \[
    \mathcal{F} T \mathcal{F}^{-1} = M_{\xi^2}.
  \]
  Note that $\mathcal{F}$ is unitary on $L^2_x \to L^2_\xi$ by
  Plancherel's theorem. This says that $\mathcal{F}$
  diagonalizes $\Delta$. Letting $E_T$ be the
  spectral resolution for $T$ and $E_{\xi^2}$ be the
  spectral resolutoin for $M_{\xi^2}$, we have
  \[
    \mathcal{F} E_T \mathcal{F}^{-1} = E_{\xi^2}.
  \]
  Since $E_{\xi}^2(I) = \chi_{\{\xi^2 \in I\}}$, we can write
  \[
    E_T(I) f = (\chi_{\{\xi^2 \in I\}} \widehat{f})^\vee.
  \]
\end{example}

\begin{example}
  Let $\mathcal{H} = L^2(\R^d)$ and
  $T = - \Delta / 2 + V$, where $V \in L^\infty$ is
  real. We claim that $T$ is self-adjoint. It is easy to check
  that $T$ is symmetric, so $D(T) \subseteq D(T^*)$,
  where $D(T) = H^2(\R^d)$. For the reverse inclusion,
  take $g \in D(T^*)$, so $\langle Tf, g \rangle = \langle f, h \rangle$
  for some $h \in \mathcal{H}$. Then
  \[
    \langle -\Delta f / 2, g \rangle
    = \langle f, h \rangle - \langle Vf, g \rangle,
  \]
  so we have
  \[
    | \langle - \Delta f / 2, g \rangle | 
    \le (\| h \|_{L^2} + \| V \|_{L^\infty}\| g \|_{L^2}) \| f \|_{L^2}.
  \]
  As before, the Hahn-Banach theorem implies that
  $-\Delta g \in L^2$, and elliptic theory yields
  $g \in H^2 = D(T)$. Thus $D(T^*) \subseteq D(T)$, so
  we see that $T$ is self-adjoint.
\end{example}
